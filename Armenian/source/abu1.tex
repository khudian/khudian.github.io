
    \magnification=\magstep 1

\input arm
\input armkb-u8

\font\artmLARGE=artmr10 at 16pt
\font\artmlarge=artmr10 at 12pt
 
\font\artmbflarge=artmb10 at 12pt
\font\artmsmall=artmr10 at 9pt
\hyphenchar\artmsmall=-1
\artm

\parskip=5pt

\vglue 20pt

                                                                        
              {\bf 1}   ԱԲU ԼԱԼԱ ՄԱՀԱՐԻ



              {\bf 2}  Հռչակավոր բանաստեծը   Բաղդադի


             {\bf 3}   Տասնյակ տարիներ ապրեց  

    {\bf 4}   խալիֆաների հոյակապ քաղաքում



    
  {\bf 5}Ապրեց փաառքի և վայելքի մեջ


{\bf   6}Հզորների և մեծատւնեիր  հետ սեղան նստեց



{\bf 7} Գիտուների և իմաստuների հետ  վեճի մտավ 



                       {\bf 8} սիրեց և փորձեց ընկերներին
  


  {\bf 9} եղավ ուրիշ - uւրիշ ազգերի հայրենիքներում



   {\bf 10}տեսավ  և  դիտեց մարդկանց 
             



{\bf  110}
{\bf 11} մի  փոքրիկ քարավանը  պաշարով ու պարենով



{\bf 18} եվ մի գիշեր,երբ Բաղդադը քուն է մտել 


{\bf 19} Տիգրիսի նոճիածածկ ափերից


{\bf 2O} գաղտնի հեռացավ քաղաքից 


 


 \centerline   


\centerline{\bf  первая  сура}



\centerline{\bf I}

եՎ քարավանը  Աբuւ լալայի 


 աղբյուրի նման մեղմ կարկաչէլով 


 քայլում եր հանգիստ նիրհած գիշերով 


 հնչուն զանգերի  անուշ ղողանջով



\centerline  {\bf  II}

Հավասար քայլով չափում էր ճամփան


այն   քարավանը  ոլոր ոու   մոլոր

եվ ղողանջները  ծորում քաղցրալուր

օղողոոմ էին դաշտերը անդորր




\centerline   {\bf III}  VI} 



մեղկ փափքության մեջ Բաղդադն նիրհում 



ջեննաթի  շքեղ  վառ երազներով


գյուլստաններում բլբուլն էր երգում


գազելներն անուշ  սիրո   արցունքով

\centerline {\bf    IV}

շատրվանները   քրքջոմ էին պայծառ ծիծավիով ադանանդսվեն


բույր  համբոոյր էր  խնարկվրում չորս կողմ  խալիֆաների


քյոշքից լոուսեղեն



\centerline {\bf  VI}

մեխակի բույրով հովն էր շշնջում 



հեքիաթմերն հազար ու մի գիշերվա




արմավն ու նոճին անուշ քնի մեջ 

օրորվում էին ճամփեքի վրա

    \centerline{\bf  V}

Եվ քարավանը  ոլոր ու շորոր


քայլում էր առաջ ու ետ չեր նայuւմ,

Անհայտ  ուղինէր Աբու լալայի 


բյուր հրապույրով կանչում, փայփայում:


\centerline{\bf  VIII}



Գնա,  միշտ գնա,  իմ քարավանս,


գնա մինչև օրերիս վերջը

Այսպես  էր խոսում իր սրտի խորքում

Աբու Մահարին,    մեծ բանաստեղծը

\centerline {\bf IX  }             aaaaaaaaaaaaaaaaaaaaaaaaaaaaaaaaaaaaaaaaaaaIIX}

գնա մենավոր վայրերը թափուր,


ազատ,  կույս  և սուրբ զմբուխտյա հեռուն


դեպի սրևը սլացիր անդուլ,


և սիրտս այրիր արևի սրտում


            
\centerline {\bf X }


Ախ, մնաք բարև,չեմ ասում ես ձեզ, 


իմ հոր գերեզման, օրոցք մայրական,


 սիրտս  հավերժ խռով է ձեզ հետ


հայրենական հարկ,  հուշեր մանկական


\centerline {\bf  XI}

ատում եմ այն ինչ որ սիրել եմ հիմա


ինչ որ տեսել եմ  մարդկայն հոգում


Իժ դարձավ խայթող իմ սերը հիմա,

  թույն ատելությամբ  սիրտս է եռում

\centerline{\bf XII}

ատում եմ ինչ որ սիրել եմ առաջ


ինչ որ տեսել եմ մարդկային հոգում

մարդկային հոգում զազիր ու նանիր 


համրել են հազար գարշանք ու նեղկում  

            \centerline{\bf  XIII  }



բայց ամենից շատ ատում եմ 


հազար ու մեկ երրորրդը {\bf կսկիցծը } հոգու


որ զարդարում է անմեղ սիրելի լուսապսակով  երեսը մարդու




\centerline{\bf XIV}

մարդկային լեզու



դու  որ երկնային բուրով ու թույրով


շղարշով պայծառ,  ծածկում ես մարդու դժողքը հոգու



ոգելես արդյոք ճշմարիտ մի բառ



\centerline {\bf XV}



իմ սեգ քարավան,


գնա մխրճվիր 

անապատի մեջ վայրի ու բոցոտ


եվ  իջևանիր այն պղնձացած 


սեգ ժայրերի վրա գազաների մոտ


\centerline{\bf XVI}

խբեմ վրանըս օձ - կարիճների



բների գլխին վրանըս խփեմ



այնտեղ բյուր անգամ ես ապահովեմ 


քան թե  մարդկանց մոտ կեղծ ու  ժպտադրեմ



\centerline    {\bf XVII} 

  քան ընկերի մոտ, ախ որի կրծքին,


դնում էի ես գլուխս սիրով


գլուխը ընկերի որ շղարշում է


անդարձ կորստի անդունդը ստով



\centerline{\bf XVIII}

 այնքան ժամանակ որքան արնը



 կայրե  Սինայի գագաթները վես,


եվ անապատի դեղին շեղջերը հորձանքներ 



կտան ալիքիների պես




\centerline {\bf    XIX}

     ես չեմ կամենա ողջունել մարդդկանց



  նրանց սեղանից պատառ չեմ կտրի 


  գազանների հետ հացի կնստեմ,

  
  ողջույնը կ առնեմ բորենիների


\centerline {\bf XX}

Եվ գազանները թող ինձ հոշոտեն


վայրագ հողմերը շաչեն ինձ վրա


 և այսպես մինչև աշխարհի վերջը



 քարավանս անդարձ գնա,ու գնա...



\centerline {\bf XXI}

ն վերջին անգամ Աբու լալայի


ետ դարձավ նայեց նիրհած Բաղդադին


գարշանքով շրջեց ճակատը կնճրոտ

ու փարվեց ուղտի թավ պարանոցին


\centerline {\bf XXII}
 

սիրով գուրգուրեց, ջերմ շրթունքներով



համբուիեց ուղտի ճակատը վճիտ


ն թարթիչներից նրա կախվեցին



անզուսպ արցունքի երկու կախվող շիթ


\centerline {\bf XXIII}



 անուշ մրմունջով, նիրհած դաշտերով մեղմ օրորվում էր ձիգ քարավանը


 գնում  եր  առաջ դեպի անապատ անհայտ ափերը, կույս հեռաստանը


 անհայտ ափերը կuյս հեռաստանը

   
\centerline {\bf сура вторая }





\centerline {\bf первя сура}

Եվ օրորվում էր այն քարավանը


սեգ արմավների շարքերի միջով 

փոշի եր հանում հանում փոշու քարավանը


որ վարում էր լուռ խորշակն հուր շնչով



\centerline{\bf II}


քայլիր քարավանս  ինչ  ենք թողել մենք 


որ կարոտանալով ցանկամք մեր դարձը


այսպես էր խոսում իրեն սրտի հետ


Աբու  լալայի, մեծ բանաստեղծ                                                                                                                                                                                                                                                                                                                                                                                                                                          ը


\centerline{\bf  III}

թողել ենք այնտեղ կին ատվածային, 


սեր - երջանկություն,  անհուն երազանք


քայլիր, կանգ մի առ,


թողել ենք այնտեղ շղթա ու  կապանք {\bf կսկիծ} ն  տառփանք


\centerline  {\bf IV  }


եվ կինն ինչ է որ.....խորամանկ, խաբող



առնասանձ  մի սարդ հավերժ նանրամիտ


որհացդե դֆ\ ուտդ-ւոու ուտոտում;


համբույրի
մե............................................................................................ջ............................................շ





խորամանկ,խափող, 




սառնախաանձ մի սարդ,հավերժ  նանրամիտ




որ հացդ է սիրում համբույրի մեջմ  սուտ 





և քո գրկի մեջ գրկում ուրիշին 



\centerline  {\bf V}

խարխուլ մակույքով հանձնվիր ծովին


քան թե հավատաս կնոջ երդումին

նա կակազ վավար, մի չքնաղ դժոխք 


նրա բերանով Իբլիսն է խոսում

\centerline{\bf   VI}


դու երազել ես աստղը հեռավոր


հրեշտակաձն շուշանն ըսպիտակ

որ քո վերքերին բալասան լինի 


ոսկեշոջ  երազ կյանքի ցավի տակ


 \centerline  {\bf VII}

 դու տենչացել ես լույս ափերի մեջ


 քեզ իրեն կանչող աղբյուրի եզրին


 եվ անմահության ցողնես երազել


 եվ անուշ լացել երկնային կրծքին



\centerline {\bf VII}

բայց սերը կնոջ տոչորված հոգւդ



աղ - ջուր է տալիս  որ միշտ ծարվանաս


հուր տարփանքի  մեջ հաղթական կնոջ



մարմինը   լիզես   և չհագեմաս





\centerline {\bf IX}

ոհ  կնոջ մարմին, պաղղշոտ, օձեղեն, 




դիվական  անոթ ոճիտրների չար



դու, որ մսեղեն դառն



հաճոոյքով արևը հոգու  դարձնում ես խավար 







\centerline{\bf X}

ատում եմ սերը մահու պես անգութ


հավիտյան այրող, խոցող գաղտնաբար

այն քաղցր թույնը որով արփողը  


ստրուկ է դառնում  կամ բռնակալ 

\centerline {\bf  XI}

ով սեր  բնության  դու խոշտանգիչ կամք,
 

նենգ ու դավադիր ոգի աննահանջ

դու թոհուբոհի ընդերք մոլեգնած,


 արյուն ցավատանջ, արյան մղձավանջ 


\centerline{\bf XII}

  ատում եմ կնոջ տարերքը կրքի,


միշտ բերմնավորող եղեռնը անսանձ


  աղբյուր   անսպառ որ կուտակում է


  աշխարհի վրա տի
aaaaaaaaaa
                           
\centerline{\bf XIII}


ատում եմ  նորից սերն ու կնոջը



   իր համբույբրները շողոմ ու   դժնյա
 

փախչում եմ նրա ճահիճ - մահիՃից 


 ու անիծում եմ երկունքը նրա


\centerline{\bf ХIV}


երկունքը դաժան և հավերժական.


որ հեղեղում է վտառն իժերի.


որոնք խայթում են հոշոտում իրար


աստղերնեն պղծում տռփանքով ժահրի



\centerline{\bf XV}

սրիկա է նա որ հայր է լինում 
  















որ երանավետ  ծոցից ոչնչի


գոյության կոչում թշվառ հյուլեին 

և  գլխին վառում գեհենն այս կյանքի



 \centerline {\bf   XVI} 

                  

 իմ հայրը իմ դեմ մեղանչեց,


  սակայն չմեղանչեցի ես ոչ ոքի  դեմ, -- 


  այս իմ կտակը թող գրվի շիրմիս,


  եթե լուսնի տակ մի խորշ պիտ գտնեմ



\centerline{\bf XVII}


այնքան  ժամանակ որ ծովը


պիտի փարե Հեջազի ափերն զմբուխտյա 



ես ետ չեմ դառնա կնոջ մոտ երբեք, 



ես չեմ կարոտնա թովչանքին նըրա 

\centerline{\bf XVIII}

կգգվեմ վայրի տատասկը դժնի



 և կհամբուրեմ փշերը նրա


գլուխս կդնեմ այրվող ժայռերին 


ն կըլամ նրանց ջերմ կրծքի վրա  



\centerline{\bf XIХ}

 եվ քարավանը մեղմիկ կարկաչով 




չափում եր ճամփան ոլոր ու մոլոր



դեպի երազուն ն կապույտ հեռուն




հոսում եր առաջ հանգիստ ու անդորր


\centerline{\bf ХХ}




զանգակներն, ասես,հեկեկում  եին



և ծորում հատ - հատ հնչուն արցունքներ


քարավանն, ասես, լալիս եր անուշ


ինչ որ     Մահարին սիրել լքել եր





 

\centerline {\bf XXI}
սիրո վերքերի, վշտոտ  կարոտի 



ն երազական թաղծանքի քնքուշ


եվ Աբu  լալան խորհում եր մռայլ, 



և նրա վիշտը անհունի նման


ինչպես իր ուղին որ գալայրվում է, ձգվում է անծայր ու չունի վախճան



ողջույն չեր վերցնում, ողջույն չեր տալիս 


անցնող ու դարձնող  քարավաններին


{\bf   вторая  сура}


\centerline  {\bf I}

եվ քարավանը Աբու լալայի 




աղբյուրի նման մեղմ կարկաչելով



հանգիստ, միաչափ   քայլում եր առաջ



հեգ լուսնյակի  շողերի միջով:



 \centerline {\bf II} 



եվ լուսինն, ինչպես ջեննեթի մատաղ  



փերիի կուրծքը  չքնաղ, լուսավառ



մերթ ամանչելով  պահվումեր ամպում



ն մերթ թրթռուն փայլում  էր պաայծառ


\centerline{ \bf III}

նիրհ էին մտել ծաղկունքը բուրյան 



ադամանդերով շքեղ գինինդերով



ծիածանաթև  հավքերը   իրար



գուրգուրում    եին քնքուշ  մրմունջով



\centerline {\bf  IV}



մեխակի բույրով հովն եր շշնջում




հեքիաթներն հազար  ու մի գիշերվա 





արմավն ու նոճին անոուշ քնի 



օրորվում էին ճամփեքի վրա  


\centerline {\bf V}


հովի զրույցին ունկն դնելով  



Աբու Մահարին խոոում  էր  անձայն



աշխարհն էլ ասես մի  հեքիաթ լինի -- 



անսկիզբ, անվերջ հրա2ք     դյութական



\centerline {\bf  VI}



    եվ ով է հյուսել հեքիաթն այս վսեմ 



   հյուսել աստղերրով, բյուր հրաշքներով



   եվ ով է պատմում բյուր - բյուր ձևերով


  անդուլ և անխոնջ այսպես   թովչանքով 

\centerline {\bf VII}


ազգերեն եկել, ազգեր գնացել



և չեն ըմբռնել իմաստը նըրա


 բանաստեղծերն են հասկացել դույզն


   ինչ և թոթովում  են հնչուներն անմահ  


\centerline {\bf VIII}

ոչ ոք չի լսել սկիզբը նըրա 




ոչ ոք չի լսելու վախճանը նըրա




ամեն հնչյունը դարերը ապրում



ամեն հնչյունին վերջ, սկիզբ չկա





\centerline {\bf IX}  
բայց ամեն մի  նոր ծնվածի համար


նորից  պատմվում է հեքիաթն այս շքեղ

նորից սկսվում և  վերջանում է  



ամեն մի մարդու  կյանքի հետ մեկտեղ 



\centerline  {\bf X}



կյանքը երազ է, աշխարհը հեքիաթ, 

 
ազգեր սերուդներ անցնող քարավան


որ հեքիաթի մեջ վառ երազի հետ


չվում է անտես դեպի գերեզման


\centerline {\bf XI}
 
կույր ու գուլ  մարդիկ, առանց երազի

առանց լսելու հեքիաթն  այս վսեմ


իրար կոկորդից պատառեք եք հանում


և դարձնում  աշխարհն ահավորհն ահավոր  ջեհնեմ
       

\centerline{\bf XII}


ձեր օրենքները -- լուծ ու խարազան,


և անելք մի ցանց խոլական սարդի


եվ որոնց  ժահրով թունավորում   եք  երգը 



բլբuւլի անուրջը վարդի




\centerline{\bf  XIII}

եվ  ունայնաշունչ հողմը կշաչէ


ձեր ոսկորների քարերի վրա


Իսկ վայելելու դուք միշտ ապիկար


երազն այս չքնաղ հեքիաթն այս ոսկյա



\centerline{\bf    XIV           }



գոհար աստղերի քարավաաները 


 թափառում երկնի ճամփեքով



և ղողանջում էր ողջ երկինքն


պայծառ անշեջ  գընգոցով





\centerline{\bf   XV }   


եվ արար- աշխարհ լցված եր դյութական 
                       


բյուր նվագներով հավերժ երկնային


եվ անուրջներում նա վերասլաց 



լսում էր հոգով վսեմ երգերին 


\centerline    {\bf  XVI}


գնա քարավան. մեղմ հնչյուներով



հյուսելով երկնի լույս ղողանջի հետ



վիշտս տուր հովին 



քայլիր բնության ծոցը մայրական   և մի նայիր ետ



\centerline\bf {\bf   XVII}



տար ինձ  լուսազգեստ,   oտար մի եզերք

հեռու, հեռավոր  մենավոր   ափեր


սուրբ մենակության դու իմ օազիս,  դու երազների 



աղբյուր  զովաբեր


\centerline {\bf    XVIII}

լռության երկինք խոսիր դու ինձ հետ


աստղերիդ լեզվով և ամոքիր ինձ


\centerline{\bf  XIX}

լռության երկինք խոսիր դու ինձ հետ


գուրգուրիր  հոգիս աշխարհից խոցված.

մարդուց խայթըված վիրավոր  հոգիս

\centerline {\bf   XX  }



իմ մեջ այրվում է մի անհագ  կարոտ 

կարեկից մի սիրտ  լացող հավիտյան.




եվ իմ  հոգում  կա մի չքնաղ  երազ, 

և սուրբ արտասուք. և սեր անսահմման'

  Ոգիս  ազատ է ես  չեմ հանդուրժում իշխող իմ վրա



 ոչ  մի գոյդության




\centerline{\bf XIX}



ոչ օրենք;  սահման. ոչ ճակատագիր;



ոչ չար ու բարի և ոչ դատստան  


եվ  իմ կամքից դուրս ամեն ինչ բանտ է,



և ստրկացում;  և   բռնադատում






\centerline {XX}

ես կուզեմ լինել անսահման ազատ,



անպարտք  անիշխան այլև անաստված



հոգիս տենչում է միայն  միմիայն 



մեծ ազատության անհուն; անտարած: 




\centerline {\bf  XX}


եվ  քարավանը հյուսվում էր առաջ  




Ւինըրա վերև շողում էին վառ


մանկան ժպիտով աստղերը ազատ 


այն հավերժափայլ  աչքերը գոհար 



\centerline {\bf XXI}



 եվ կանչում էին նրան  կաթոգին լույս թարթումները    ոսկի աստղերի



\centerline {\bf   XXIII}



  եվ հոգին լցնում  վսեմ ղողանջով երկնքի հազար բյորեղ զանգերի






\centerline {\bf  XXIV    }


վճիտ գիշերին դյոթական ցոլքով փայլում էր ուղին  փիրուզյա հեռվում


եվ քարավանը օրոր ու շորոր   քայլոոմ էր անդորր փիրուզյա հեռուն ;


\centerline {\bf   ե}{



գիշերն ահարկoo;///////f 









\bye





























\centerline{\bf XXI}




\bye

:wq
  :AAAAAAAAAAAAAAAAAAAAA??"wq:::?<>?
