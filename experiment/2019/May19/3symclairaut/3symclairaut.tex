 

\magnification=1200
\baselineskip=14pt
\def\vare {\varepsilon}
\def\t {\tilde}
\def\a {\alpha}
\def\K {{\bf K}}
\def\N {{\bf N}}
\def\C {{\cal C}}
\def\L {{\cal L}}
\def\E {{\cal E}}
\def\s {{\sigma}}
\def\S {{\Sigma}}
\def\p{\partial}
\def\vare{{\varepsilon}}
\def\Q {{\bf Q}}
\def\D {{\cal D}}
\def\G {{\Gamma}}
\def\Z {{\bf Z}}
\def\R  {{\bf R}}
\def\l {\lambda}
\def\ll {{\bf l}}
\def\degree {{\bf {\rm degree}\,\,}}
\def \finish {${\,\,\vrule height1mm depth2mm width 8pt}$}
\def \m {\medskip}
\def\p {\partial}
\def\r {{\bf r}}
\def\pt {{\bf p}}
\def\v {{\bf v}}
\def\n {{\bf n}}
\def\t {{\bf t}}
\def\b {{\bf b}}
\def\c {{\bf c }}
\def\e{{\bf e}}
\def\f{{\bf f}}
\def\ac {{\bf a}}
\def \X   {{\bf X}}
\def \Y   {{\bf Y}}
\def \x   {{\bf x}}
\def \y   {{\bf y}}
\def\w {{\omega}}
\def \Tr  {{\rm Tr\,}}
\def\dim {{\rm dim\,\,}}
% I wrote this file on 16 January
\def\t {{\tilde}} 
\def\dist {{\hbox{\tt "distance"}}}


{\bf Symmetries of Clairaut equation}


It is well-known that the Clairaut equation
       $$
 y-xy'=f(y')
\eqno (1)
       $$
has the one-parametric family of solutions:
       $$
     y=kx+f(k)
       \eqno (2a)
       $$
and the special soluction, their envelope: $\varphi(x)$,
     $$
\varphi(x)=k(x)x+f(k(x))\colon\quad
    {\p\varphi\over k}=x+f'(k)=0\,.
      \eqno (2b) 
     $$
This is standard.


On the other hand the Clairaut equation (1)
can be approached in the following way:

\def\t {\tilde}

Consider in $3$-dimensional space  $E$ $=(p,x,y)$ ($p=y'$)
the following transformation
        $$
(x,p,y)\to (\t x,\t p,\t y)\quad
\cases {\t x=p\cr\t p=-x\cr \t y=y-px}
        $$
This is so called Legendre transformation of contact space
$E$.  Under this transformation the
characteristic $1$-form $pdx-dy$ is not changed
\footnote{}{the transformation (2a)  preserves the 
distribution of planes defined by
$1$-form $pdx-dy$, thus it can be applied to an arbitrary
differential equation $F(x,p,y)=0$ ($p=y'$).},
and  the Clairaut equation
transforms to the algebraic equation 
        $$
   \t y=f(\t x)
        $$,
every solution (2a) 
is transformed
to the generalised solution, the line
   $\cases {\t x=k\cr \t p=\hbox {arbitrary number}\cr
          \t y=f(k)\cr}$,
and the special solution (2b), $y=\varphi(x)$
is transformed to the solution, the curve
   $\cases {\t x= t \cr 
            \t p=  f'(t)\cr
             \t y=f(t)\cr}$,


Our aim to find the {\it one-parametric famiily
  of contact trasnformations which include the Legendre transformation and see how look the image of solutions (2a), (2b)
under this transformation.}

Consider the Hamiltnian $H={p^2\over 2}+{x^2\over 2}$
of harmonic oscillator.  This Hamiltonian 
defines the contact vector field
        $$
   \X_H={\p H\over \p p}{\p\over \p x}-
   {\p H\over \p x}{\p\over \p p}+
   \left(p{\p H\over \p p}-H\right){\p\over \p y}\,,
        $$
and it induces
the contact transformation
   $$
  \pmatrix {x\cr p\cr y\cr}\to
  \pmatrix {\t x_t\cr \t p_t\cr \t y_t\cr}\,,
  $$
such that 
        $$
\cases {
{d {\t x}\over dt}={\p H\over \p {\t p}}=\t p\cr
{d {\t p}\over dt}=-{\p H\over \p {\t x}}=-\t x\cr
{d {\t y}\over dt}= \left(
    {{\t p}^2\over 2}
    -{{\t x}^2\over 2}
          \right) 
        \cr
}
        $$
Solving this equation and taking in account the boudnary
conditions
     $$
  \pmatrix {\t x_t\cr \t p_t\cr \t y_t\cr}\big\vert_{t=0}=
  \pmatrix { x\cr  p\cr  y\cr}
     $$
we come to
     $$
 \cases
  {
  x_t=x\cos t+p\sin t\cr
  p_t=-x\sin t+p\cos t\cr
  y_t=y+
 {1\over 4}p^2\sin 2t
 -{1\over 4}x^2\sin 2t
 +{1\over 2}px\left(\cos 2t-1\right)\cr
       }
       $$
We see that for $t=0$ this is identity contact transformation,
and for $t={\pi\over 2}$ this the Legendre trasnformation
 $$
 \cases
  {
  x_{\pi\over 2}=p\cr
  p_{\pi\over 2}=-x\cr
  y_{\pi\over 2}=y-px\cr
       }
       $$
\bye


