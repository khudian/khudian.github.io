



\magnification=1200
\baselineskip=14pt
\def\vare {\varepsilon}
\def\t {\tilde}
\def\a {\alpha}
\def\K {{\bf K}}
\def\N {{\bf N}}
\def\C {{\bf C}}
\def\L {{\cal L}}
\def\E {{\bf E}}
\def\s {{\sigma}}
\def\S {{\Sigma}}
\def\p{\partial}
\def\vare{{\varepsilon}}
\def\Q {{\bf Q}}
\def\D {{\cal D}}
\def\G {{\Gamma}}
\def\Z {{\bf Z}}
\def\R  {{\bf R}}
\def\l {\lambda}
\def\ll {{\bf l}}
\def\degree {{\bf {\rm degree}\,\,}}
\def \finish {${\,\,\vrule height1mm depth2mm width 8pt}$}
\def \m {\medskip}
\def\p {\partial}
\def\r {{\bf r}}
\def\pt {{\bf p}}
\def\v {{\bf v}}
\def\n {{\bf n}}
\def\t {{\bf t}}
\def\b {{\bf b}}
\def\c {{\bf c }}
\def\e{{\bf e}}
\def\f{{\bf f}}
\def\ac {{\bf a}}
\def \X   {{\bf X}}
\def \Y   {{\bf Y}}
\def \x   {{\bf x}}
\def \y   {{\bf y}}
\def\w {{\omega}}
\def \Tr  {{\rm Tr\,}}
\def\dim {{\rm dim\,\,}}
% I wrote this file on 16 January
\def\t {{\tilde}} 
\def\dist {{\hbox{\tt "distance"}}}
\def  \dim {{\rm dim\,}}
\def  \Im  {{\rm Im\,}}
\def  \ker {{\rm ker\,}}


\def \Cl {\hbox{\tt Cliff}}

\centerline   {\bf Spinor group}


          
Let $\Cl (V,Q)$ be Clifford algebra of vector space
$V$  equipped with bilinear form  $Q$.
According the lemma that
           $$
\Cl \left(
 \left(Q_1,V_1\right)
   \right)
\oplus 
\left(Q_2,V_2\right)
=
\Cl (Q_1,V_1){\hat \otimes}
\Cl (Q_2)\,,
         $$
we come to theorem:

{\it Clifford algebra $\Cl(Q,V)$ is isomorphic to
wedge product of 
$p$ algberas of doube numbers
($a+b\vare$, $\vare^2=1$), 
$r-p$ algberas of complex  numbers
($a+b\vare$, $\vare^2=-1$), and 
$n-r$ algberas of dual numbers
($a+b\vare$, $\vare^2=0$), 
where   
$r$ is the rank of the form $Q$, 
$(p,r-p)$
is the signature of the form, 
$n$ is the dimension of  vector space $V$, 
i.e.  $p,q,r$ are integers such that  quadratic form $Q$
in $V$ can be reduced to the form  
          $$
Q=x_1^2+\dots+x_p^2-x_{p+1}^2-\dots x_r^2
          $$
in some linear coordinates
}

\smallskip

{\bf Example}  $\E^2$ with $Q(\x)=-x_1^2-x_2^2$,
then
     $$
\Cl(\E^2,Q=(-,-))=\C\hat\otimes\C={\bf H}\,\, \hbox
{(quaternions)}
     $$

\smallskip

One can view the algebra  $\Cl(V,Q)$ as unital algebra
such that vector space $V$ is subpsace in this algebra,
and all elements are generated by vectors of $V$  such
that $\x^2=Q(\x)\cdot 1$.

\smallskip

We denote vectors and their image in Clifford algebra by
the same letter


\bigskip

Now solve the following excises:


  {\bf Exercise} 
      $$
 \x^{-1}={\x\over Q(\x)}\,.
      $$ 

   {\bf Exercise}

        $$
    \v^{-1} \x\v=L_\v(\x)=2
     {(\v,\x)\over Q(\v)}\v-\x
      $$
and it is orthogonal operator (if $Q(\v)\not=0$).


   {\bf Exercise}




{\bf Definition}   Consider a set of all elements
          $$
    \v_1\cdot\v_2\dots \v_k\,
          $$
$k=0,1,2,\dots$, if $k=0$ then the element is a number
$c\not=0$,
where all $\v_i$ are vectors of non-zero length:
      $$
 Q(\v_1)\not=0\,\,\,
 Q(\v_2)\not=0\,\,\,\dots
 Q(\v_k)\not=0\,\,.
     $$
 
In other words we consider Clifford algebra forgetting 
the operation of $+$.

This set is a group. This group projects on the group
$SO(n, Q)$, the kernel of projection is the group $\R^*$
of non-zero constants. 

{The group  which we construct is generated by non-zero
constants and vectors of non-zero length }


{\bf Another way to say the same}


\centerline this is  a {\bf GROUP OF UNITS}
   of subspace $V$ in Clifford algebra, i.e. a group of
invertible elemeents belonging to the subspace $V$.

\bye


