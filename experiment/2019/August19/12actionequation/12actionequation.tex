

\magnification=1200
\baselineskip=14pt
\def\vare {\varepsilon}
\def\t {\tilde}
\def\a {\alpha}
\def\K {{\bf K}}
\def\N {{\bf N}}
\def\C {{\bf C}}
\def\L {{\cal L}}
\def\E {{\bf E}}
\def\s {{\sigma}}
\def\S {{\cal S}}
\def\SS {{\Sigma}}
\def\p{\partial}
\def\vare{{\varepsilon}}
\def\Q {{\bf Q}}
\def\D {{\cal D}}
\def\G {{\Gamma}}
\def\Z {{\bf Z}}
\def\R  {{\bf R}}
\def\l {\lambda}
\def\ll {{\bf l}}
\def\degree {{\bf {\rm degree}\,\,}}
\def \finish {${\,\,\vrule height1mm depth2mm width 8pt}$}
\def \m {\medskip}
\def\p {\partial}
\def\r {{\bf r}}
\def\pt {{\bf p}}
\def\v {{\bf v}}
\def\n {{\bf n}}
\def\t {{\bf t}}
\def\b {{\bf b}}
\def\c {{\bf c }}
\def\e{{\bf e}}
\def\f{{\bf f}}
\def\ac {{\bf a}}
\def \X   {{\bf X}}
\def \Y   {{\bf Y}}
\def \x   {{\bf x}}
\def \y   {{\bf y}}
\def\w {{\omega}}
\def \Tr  {{\rm Tr\,}}
\def\dim {{\rm dim\,\,}}
% I wrote this file on 16 January
\def\t {{\tilde}} 
\def\dist {{\hbox{\tt "distance"}}}
\def  \dim {{\rm dim\,}}
\def  \Im  {{\rm Im\,}}
\def  \ker {{\rm ker\,}}


\def \Cl {\hbox{\tt Cliff}}

\centerline {\bf Action and Hamiltonian}


Let $H=H(x,p)$  Hamiltonian , and
   $S_t(x,q)$  be its action:
        $$
S_0(x,q)=x^iq_i\,\,,\quad 
{\p S_t(x,q)\over \p t}=
  H\left(
{\p S_t(x,q)\over \p q},q
\right)\,,
     \eqno (0a)
        $$
i.e. $S_t(x,q)$ is its P-exponent.
We know (see also the previous blogs)  that
      $$
 p_a={\p S_t(x,q)\over \p x^a}\,,
 \quad
y^a={\p S_t(x,q)\over \p q_a}\,,
      $$
where $(x,p)$ are initial momenta and coordinates
and $(y,q)$--momenta and coordinates at the time $t$:
        $$
    \pmatrix {x\cr p\cr}\Rightarrow 
    \pmatrix {y\cr q\cr}\colon\quad
        \cases
      {
y=y(t,x,p)\cr
q=q(t,x,p)\cr
      } \,\,
    \hbox {canonical transformation induced by $H$ in
time $t$}\,\,.
         \eqno (0b)
      $$
         $$
 {\p y^a(t,x,p)\over \p t}={\p H(y,q)\over \p q_q}\,\,
 {\p q_a(t,x,p)\over \p t}=-{\p H(y,q)\over \p y^q}\,\,
         $$
One can see that
the following differential equations are obeyed

      $$
{\p p_a\over \p p_b}=\delta_a^b\,\,\,\quad {\rm
i.e.}\,\,\quad
{\p^2 S_t(x,q)\over \p x^a \p q_c}{\p q_c(t,x,p)\over \p
p_b}
=\delta_a^b\,,
  \eqno (1a)
      $$
      $$
{\p p_a\over \p x^b}=0\,\,\,\quad {\rm
i.e.}\,\,\quad
{\p^2 S_t(x,q)\over \p x^a \p x^b}
+
{\p^2 S_t(x,q)\over \p x_a \p q_c}{\p q_c(t,x,p)
\over \p x^b}
=0\,,
  \eqno (1b)
      $$
      $$
{\p p_a\over \p t}=0\,\,\,\quad {\rm
i.e.}\,\,\quad
{\p^2 S_t(x,q)\over \p x^a \p t}+
{\p S^2_t(x,q)\over \p x_a \p q_c}{\p q_c(t,x,p)
\over \p t}=
{\p^2 S_t(x,q)\over \p x_a \p t}-
{\p S^2_t(x,q)\over \p x_a \p q_c}
{\p H(y,q) \over \p y^c}=
0\,,
  \eqno (1c)
      $$
      $$
{\p y^a\over \p t}=
 {\p H(y,q)\over \p q_a}=
{\p^2 S_t(x,q)\over \p q_a \p t}
                          +
{\p^2 S_t(x,q)\over \p q_a \p q_c}{\p q_c(t,x,p)
\over \p t}=
{\p^2 S_t(x,q)\over \p q_a \p t}
                          -
{\p^2 S_t(x,q)\over \p q_a \p q_c}{\p H(y,q)
\over \p y^c}\,,
  \eqno (1d)
      $$
      $$
{\p y^a\over \p x^b}=
{\p^2 S_t(x,q)\over \p q_a \p x^b}
                          +
{\p^2 S_t(x,q)\over \p q_a \p q_c}{\p q_c(t,x,p)
\over \p x^b}\,,
  \eqno (1e)
      $$
      $$
{\p y^a\over \p p_b}=
{\p^2 S_t(x,q)\over \p q_a \p q_c}{\p q_c(t,x,p)
\over \p p_b}\,.
  \eqno (1e)
      $$
    $$
     $$

{\bf Claim}  

 Let $H$ be quadratic. Then 
 canonical transformtions (0b) are linear
and $S(x,q)$ is quadratic.


 \bigskip

{\bf Example}. {\tt Harmonic oscillator}

Let
     $$
H={1\over 2}(y^2+q^2)
      $$
Then action
       $$
S_t(x,q)={xq\over \cos t}+
    \left(
 {x^2\over 2}
+{q^2\over 2}
  \right){\rm tg\,}t
\eqno (2)
       $$

\medskip

 action for oscillator defines the group?

\medskip





Let 
       $$
H=H(p,q)=U_{ik}x^ix^k+L_i^kx^ip_k+T^{ik}p_ip_k
\eqno (3)      
 $$
be quadratic Hamiltonian.

   Our aim is to define its action  $S(x,q)$.

{\tt The claim says that it is
quadratic also}

One can see it just straightforwardly

First return to oscilator:

Look for 
  $$
S_t(x,q)=A(t)xq+
{1\over 2}B(t)x^2+
{1\over 2}C(t)q^2+
  $$
then  Hamilton-Jacobi equation
(0) give that
    $$
y={\p S\over \p q}=A(t)x+C(t)q
     $$
and
     $$
{\p S_t(x,q)\over \p t}=
{dA(t)\over dt}xq+
{1\over 2}{dB(t)\over dt}x^2+
{1\over 2}{dC(t)\over dt}q^2=
H\left(
{\p S_t(x,q)\over \p q},q
\right)=
      $$ 
        $$
{1\over 2}\left(
A(t)x+C(t)q\right)^2+q^2=
 A(t)C(t)xq+{1\over 2}A^2(t)x^2+
{1\over 2}\left(C^2+1\right)q^2\,,
        $$
thus we come to differential equations
      $$
 \cases
   {
 {dC(t)\over dt}
=1+C^2\cr
 {dA(t)\over dt}=A(t)C(t)\cr
 {dB(t)\over dt}=A^2(t)\cr
    }\,\,\hbox{with bundary conditions}\,\quad
\cases
   {
C(t)\big\vert_{t=0}=0\cr
A(t)\big\vert_{t=0}=1\cr
B(t)\big\vert_{t=0}=0\cr
}
      $$
 thus  we come to  solution
        $$
     \cases
       {
   C(t)={\rm tg\,}(t+C)\cr
     A(t)={1\cos (t+C)}\cr
   B(t)={\rm tg\,}(t+C)\cr
        }
        $$ 
       This is the answer (compare with example (2))   

Now consider the general quadratic case



\medskip

Now retunr to general case (3)
       $$
H=H(p,q)={1\over 2}U_{ik}x^ix^k+L_i^kx^ip_k+
{1\over 2}T^{ik}p_ip_k
 $$

  Look for
    $$
S(x,q)=q_iA^i_k(t)x^k+{1\over 2}B_{ik}(t)x^ix^k+
 {1\over 2}C^{ik}(t)(t)q_iq_k\,,\quad {\rm with}\,\,
\cases
  {     
 A^i_k(t)\big\vert_{t=0}=\delta^i_k\cr
 B_{ik}(t)\big\vert_{t=0}=0\cr
 C^{ik}(t)\big\vert_{t=0}=0\cr
       }\,.
    \eqno (5) 
    $$
One can find $S(x,a)$ in (5) just solving equation (0):
       $$
y^i={\p S\over \p q_i}=
A^i_k(t)x^k+
 C^{ik}(t)(t)q_k\,,
$$
and
          $$
{\p S\over \p t}=
        q_i{dA^i_k(t)\over dt}x^k+
     {1\over 2}{dB_{ik}(t)\over dt}x^ix^k+
 {1\over 2}{dC^{ik}(t)(t)\over dt}q_iq_k=
          $$
         $$
            \left(
{1\over 2}U_{ik}y^iy^k+L_i^ky^iq_k+
{1\over 2}T^{ik}q_iq_k
    \right)
     \big\vert_
  {y^i={\p S\over \p q_i}=
A^i_k(t)x^k+
 C^{ik}(t)(t)q_k}=
         $$
           $$
{1\over 2}U_{ij}
       \left(
A^i_k(t)x^k+
 C^{ik}(t)(t)q_k
         \right)
       \left(
A^j_r(t)x^r+
 C^{jr}(t)(t)q_r
         \right)
        +
      L_i^k
        \left(
A^i_k(t)x^k+
 C^{ik}(t)(t)q_k
         \right)
q_k+
{1\over 2}T^{ik}q_iq_k
        $$
Comparing tensors with coefficients
$x^ix^k$, $x^iq_k$ and $q_iq_k$
we come to first order differential equations. E.g. for
$x^ix^k$ we have  equation
     $$
{dB(t)\over dt}=A^+(t)UA(t)\,,\quad
B(t)\big\vert_{t=0}=0\,.
      $$
Another way to do it, 
 it is solve equations (1) just step
by step, and in this case we will  also  find the
linear canonical transformations.

 

\bye



