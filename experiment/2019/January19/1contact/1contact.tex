

\magnification=1200
\baselineskip=14pt
\def\vare {\varepsilon}
\def\t {\tilde}
\def\a {\alpha}
\def\K {{\bf K}}
\def\N {{\bf N}}
\def\C {{\cal C}}
\def\L {{\cal L}}
\def\E {{\cal E}}
\def\s {{\sigma}}
\def\S {{\Sigma}}
\def\p{\partial}
\def\vare{{\varepsilon}}
\def\Q {{\bf Q}}
\def\D {{\cal D}}
\def\G {{\Gamma}}
\def\Z {{\bf Z}}
\def\R  {{\bf R}}
\def\l {\lambda}
\def\ll {{\bf l}}
\def\degree {{\bf {\rm degree}\,\,}}
\def \finish {${\,\,\vrule height1mm depth2mm width 8pt}$}
\def \m {\medskip}
\def\p {\partial}
\def\r {{\bf r}}
\def\pt {{\bf p}}
\def\v {{\bf v}}
\def\n {{\bf n}}
\def\t {{\bf t}}
\def\b {{\bf b}}
\def\c {{\bf c }}
\def\e{{\bf e}}
\def\ac {{\bf a}}
\def \X   {{\bf X}}
\def \Y   {{\bf Y}}
\def \x   {{\bf x}}
\def \y   {{\bf y}}
\def\w {{\omega}}
\def \Tr  {{\rm Tr\,}}
% I wrote this file on 1-st January  2019
 

   \centerline{\bf Geometry of diff.equations}


  \centerline 
{\bf ${\cal x} 1$ Necessary mathematics from Arnold's book}



{\tt I began to understand the pages in Arnold on diff. equations.....}


Here it is:

{\bf Definition}   Let $\w$ be $1$-form on $M$ which does not vanish.
. We say that it is
{\it contact} form if 
       $$
\hbox{$2$-form $d\w$ is non-degenerate on the plane $\w=0$ in $TM$}
       $$

Since $d\w$ is not degenerate on $\w=0$ and $\w\not\equiv 0$
them ${\rm dim}\, M=2k+1$.

{\bf Theorem}  Contact form is defined up to a function
(Valya had a talk on it!)  If $\w$ is contact form and
  $f\not=0$ then $f\a$ is contact also.

\m

   Let $\cal K$ be a distributions of $2n$-dimensional planes in $TM$
such that $\w$ vanishes on these planes.


We say that this distribution is a contct structure
\footnote{$^*$}{One can say  that distribution of hyperplanes
 defines constant structure if an $1$-form which vanishes this dstribution is non-degenerated on it}



{\bf Theorem}  Let $N$ be a submanifold of $M$
which is an integral submanifold 
(not necessarily maximal)  of contact distribution $\cal K$,
i.e. for every point on $N$ the tangent  vectors belong to
this distribution.   Then ${\rm dim}\, N\leq n$,
where ${\rm dim}\,M \leq 2n+1$ 

{\tt Proof}  Let $\w$ be an arbitrary non-zero 
form which vanishes at $\cal K$.   
Since a form $\w$ vanishes on vectors tangent to the manifold
$N$, the form $d\w$ vanishes on these vectors also:
               $$
\iota\colon N\subset M, \qquad
    d\left(\iota^* \w\big\vert_N\right)=
     \iota^* dw\big\vert_N=0\,.
               $$
Hence two arbitrary vectors are orthogonal to each other with
respect to this form.  If dimension of tangent plane is bigger
than $n$ then there exist at least two vectors which are not orthogonal,
since $d\w$ is not degenerate.
Now we apply this mathematics to the differential equations.

\m

    \centerline {${\cal x}2\,\,$  
\bf Geometry of first order equation}

  Let  $J^1M$  be a space of first jets of functions on 
manifold $M$. Cooridnates on $J^1M$ are $(p_i,q^j,u)$,
where $q^j$ are coordinates on $M$.  Jet of every function  $u=u(x)$
has coordinates $\left(p_i={\p u\over \p x q^i},q^i,u\right)$.

Consider  $\C$ , the Cartan 
 distribution of $2n$-dimensional planes in $J^1M$
defined by the form $\w=p_idq^i-du$
                      $$
\C_\pt\subset T_\pt J^1M=\{
T_\pt (J^1M)\ni\X\colon\,\,\w(\X)=0 
                    \}\,,
                      $$
Vector field  
     $$
M^i{\p\over \p q^i}+
N_i{\p\over \p p_i}+A{\p\over \p u}\,
\hbox{belongs to Cartan distribution $\C$ if $A=p_iM^i$}\,.
      $$
$\C$ is non-integrable distribution.  It is a 
{\it contact structure} and
the form  $\w_C=p_idq^i=du$  is a contact form since
              $$
          d\w\big\vert_{\w=0}=dp_i\wedge dq^i
              $$
is non-degenerte form.
Consider differential equation,  
      $$
\E\colon       F(p,q,u)=0\,.
      $$
Differential equation is sumbmanifold of codimension $1$
in the space $J^1(M)$.

The Cartan distribution  $\C$ of hyperplanes on $J^1M$ 
defines distribution  $\C(\E)$ in $T\E$:
                  $$
\C(E)=\C\cap T\E\,.
                   $$
         $$
\X=
M^i{\p\over \p q^i}+
N_i{\p\over \p p_i}+A{\p\over \p u}\in \C(\E)\,\, {\rm if}\,\,
 A=p_iM^i \& 
\left(M^i{\p\over \p q^i}+
N_i{\p\over \p p_i}+A{\p\over \p u}\right)F(p,q,u)\big\vert_{F=0}=0\,.
          $$

This distribution is not integrable.


The solution of differential equation (1) is the maximal integral
  of the distribution.

What is the dimension of $N$?


The considerations of the first paragraph say
that 
    $$
 \dim N\leq n
     $$

If $N$ is the $n$-dimensional
 submanifold in $M$.  
(We come in this case to the $n$-parametric family of solutions?)



Let  $N$ be an arbitrary solution, 
$N$ is the surface of dimension
  at least $n$. Any tangent plane to $N$  belongs to Cartan distribution
and is tangent to $\E$.   Consider an arbitrary
point $\pt \in N$. Let $\a=\a_\pt$  be the tangent plane.  The vectors
in tangent plane belong to distribution $\C_\E=\C\cap\E$,
i.e. they are orthogonal to the covector ($1$-form) 
in $2n+1$-dimensional
space
            $$
  \w_C=(1,0,\dots,0,-p_1,\dots, -p_n)\,,
  \quad\hbox{ (the vector belongs to Cartan distribution $\C$)} 
            $$
and to the covector
     $$
dF=
         \left(
      F_u,
  {\p F\over \p q^1}, \dots ,{\p F\over \p q^n};
  {\p F\over \p p_1}, \dots {\p F\over \p p_n}
          \right)\,,
\qquad
  \hbox {(the vector is tangent to the differential equation $\E=0$)} 
     $$
    $$
\a_\pt\ni\X\Leftrightarrow  \w_C(\X)=dF(\X)=0\,. 
    $$

{\bf Definition}  The point $\pt$ is not singular point if
tnhe rank of the matrix $\pmatrix {\w_C\cr  dF\cr}$  is equal to $2$.
IN this case the dimension of 


 \bigskip


\bye
