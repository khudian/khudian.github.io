

\magnification=1200 


\baselineskip=14pt
\def\vare {\varepsilon}
\def\A {{\bf A}}
\def\t {\tilde}
\def\a {\alpha}
\def\K {{\bf K}}
\def\N {{\bf N}}
\def\V {{\cal V}}
\def\s {{\sigma}}
\def\S {{\Sigma}}
\def\s {{\sigma}}
\def\p{\partial}
\def\vare{{\varepsilon}}
\def\Q {{\bf Q}}
\def\D {{\cal D}}
\def\G {{\Gamma}}
\def\C {{\bf C}}
\def\M {{\cal M}}
\def\Z {{\bf Z}}
\def\U  {{\cal U}}
\def\H {{\cal H}}
\def\R  {{\bf R}}
\def\S  {{\bf S}}
\def\E  {{\bf E}}
\def\l {\lambda}
\def\ll {{\bf l}}
\def\degree {{\bf {\rm degree}\,\,}}
\def \finish {${\,\,\vrule height1mm depth2mm width 8pt}$}
\def \m {\medskip}
\def\p {\partial}
\def\r {{\bf r}}
\def\pt {{\bf p}}
\def\v {{\bf v}}
\def\n {{\bf n}}
\def\t {{\bf t}}
\def\b {{\bf b}}
\def\c {{\bf c }}
\def\e{{\bf e}}
\def\ac {{\bf a}}
\def \X   {{\bf X}}
\def \Y   {{\bf Y}}
\def \x   {{\bf x}}
\def \y   {{\bf y}}
\def \G{{\cal G}}

% I began this file on 15 August 2018


\centerline {Some amasing formulas}

 {\it These  days I study Hilbnert Theorem of non-immersion of Lobachevsky plane  in $\E^3$. To my surprise it has so many mathphysical applications.
   Here I just give some very preliminary results. (see the file in 
Etudes/Geometry/Lobachevsky)}

Consider on $\R^2$  Riemannian metric
                  $$
    G=dx^2+2cos\Theta(x,y)dxdy+dy^2
                  $$
This metric naturally appears in the proof of Hilbert Theorem.



   Performing  routine  calculations we come to
                       $$
\Gamma_{xx}^x=\Theta_x{\rm cotan\,}\Theta\,,\quad 
\Gamma_{xx}^y=-\Theta_x{1\over \sin \Theta}\,,
                       $$
                       $$
\Gamma_{yy}^y=\Theta_y{\rm cotan\,}\Theta\,,\quad 
\Gamma_{yy}^x=-\Theta_y{1\over \sin \Theta}\,,
                       $$
and
               $$
\Gamma^x_{xy}=
\Gamma^x_{yx}=
\Gamma^y_{xy}
\Gamma^y_{yx}=0
               $$

and  
                $$
   K={R\over 2}={R_{1212}\det g}={-\Theta_{xy}sin\Theta\over 1-\cos^2\Theta}=
              {-\Theta_{xy}\over \sin \Theta}\,,
                $$
and in particular
         the condition
                   $$
           \Theta_{xy}=\sin \Theta
                   $$
means that this manifold is (whole?)  Lobachesky plane
  (this function has to exist)

E.g. function
                $$
     \theta={arctan }Ce^(x+y)
                $$
\bye
