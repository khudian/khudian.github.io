:\magnification=1200
\baselineskip 17pt
\def\a {\alpha}
\def\Z {{\bf Z}}
\def\e {{\bf e}}
\def\f {{\bf f}}
\def \E {{\bf E}}
 \centerline {Continuous fractions: review}

4 August 18

   Let  $\e,\f$  be standard basis in  ${\bf R}^2$.
$\e=(1,0)$, and $\f=(0,1)$.


   Let   $\a$  be an arbitrary 
non-negative real number and 
  let  $[a_0,a_1,\dots,]$ be its continuous fraction;
and  let ${p_k\over q_k}$  be the $k$-th approximation
corresponding to this fraction:
            $$
  [a_0,\dots,a_k]={p_k\over q_k}
          \eqno (2b)
            $$
(We assume that $p_k,q_k$ are coprime)

Consider vectors $\{\E_{-2},\E_{-1},\E_0.\E_1,\dots\}$
in $\Z^2$   defined by real number $\a$ in the following way:
            $$
\E_{-2}=\e,\quad  \E_{-1}=\f\,,
     \eqno (3)
            $$

{\bf Proposition} {``metod vytiagivanie nosov''}:  
for arbitrary $k$:
        $$
\E_{k+1}=\E_{k-1}+a_{k+1}\E_k\,. (k=-1,0,1,2,\dots)
        \eqno (Prop)
        $$

This Proposition immediately implies
 the statement about good approximmation

{\bf Corollary}
 Area of parallelgrom $\Pi_{\E_k \E_{k+1}}$ is 
equal to $1$:
         $$
\omega(\E_k,\E_{k+1})=
(-1)^k
\omega(\e,\f)\,,\quad i.e.
     \left|
      {p_{k+1}\over q_{k+1}}
         - 
      {p_{k}\over q_{k}}
     \right|={1\over q_{k+1}q_k}
         $$
Indeed  by definition of these vectors
  $$
\omega(\E_k,\E_{k+1})=
\omega(\E_k,\E_{k-1}+a_{k+1}\E_k)=-
\omega(\E_{k-1},\E_{k-1})\,,
    $$
thus we come by induction to the statement.

   
(Compare wth proofs in the previous files)


\bigskip


\centerline {$\cal x 2$ Convex span for continuous fraction.}

The following remarkable property is obeyed.


{\it Convex span of vectors $\{\E_2,\E_0,\E_2,\dots\}$
in the quadrant $x>0,y>0$
is the set $\Pi_-(\a)$,
and
Convex span of vectors $\{\E_1,\E_1,\E_3,\dots\}$
in the quadrant $x>0,y>0$
is the set $\Pi_+(\a)$.}

  As usuall we
 define  $\Pi_-$ as the set of all points in the quadrant
   $x>0,y>0$  which have integer coordinates and they
are below the line $y=\a x$, respectively 
$\Pi_-$ is the set of all points in the quadrant
   $x>0,y>0$  which have integer coordinates and they
are over  the line $y=\a x$, respectively:
           $$
\Pi_-=\{(m,n)\colon m,n\in\Z\,,\quad 0\leq m\leq n\a\}\,,
           $$ 
           $$
\Pi_+=\{(m,n)\colon m,n\in\Z\,,\quad 0\leq n\a\leq m\}\,.
           $$ 
(As usual we suppose that number $\a$ is not negative.)




\bye
