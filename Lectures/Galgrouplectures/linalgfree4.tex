
 

\magnification=1200
\baselineskip=14pt
\def\vare {\varepsilon}
\def\t {\tilde}
\def\a {\alpha}
\def\K {{\bf K}}
\def\N {{\bf N}}
\def\C {{\cal C}}
\def\L {{\cal L}}
\def\E {{\bf E}}
\def\s {{\sigma}}
\def\S {{\Sigma}}
\def\p{\partial}
\def\vare{{\varepsilon}}
\def\Q {{\bf Q}}
\def\D {{\cal D}}
\def\G {{\Gamma}}
\def\Z {{\bf Z}}
\def\R  {{\bf R}}
\def\l {\lambda}
\def\ll {{\bf l}}
\def\degree {{\bf {\rm degree}\,\,}}
\def \finish {${\,\,\vrule height1mm depth2mm width 8pt}$}
\def \m {\medskip}
\def\p {\partial}
\def\r {{\bf r}}
\def\pt {{\bf p}}
\def\v {{\bf v}}
\def\n {{\bf n}}
\def\t {{\bf t}}
\def\b {{\bf b}}
\def\c {{\bf c }}
\def\e{{\bf e}}
\def\f{{\bf f}}
\def\ac {{\bf a}}
\def \X   {{\bf X}}
\def \Y   {{\bf Y}}
\def \x   {{\bf x}}
\def \y   {{\bf y}}
\def\w {{\omega}}
\def \Tr  {{\rm Tr\,}}
\def\dim {{\rm dim\,\,}}
% I wrote this file on 16 January
\def\t {{\tilde}} 
\def\dist {{\hbox{\tt "distance"}}}





\centerline 
{\bf On one  theorem on quadratic forms and free particles.}

\centerline {H.M. Khudaverdian}

\centerline {\tt Galois Lecture }

\centerline {20 III  2019}


{\it 
  The talk is devoted to the following fact:
{\it Every mechanical system in a vicinity of stability point
can be described as a system of non-interacting oscillators.}
To explain it we  
         consider a  system of 
interacting $N$  particles,
      Every particle has three degrees of freedom 
(three coordinates).
      This system can be described by second order Newton
      differential equations on $3N$ coordinates.
   Suppose that this system is 'almost'  in 
equilibrium position.
      Using the 
linear algebra
    one can 
introduce new  $3N$ coordinates  
$y^1,\dots, y^{3N}$ so
      called  {\it collective  coordinates} 
in this $3N$-dimensional dynamical
      system,
      such that every coordinate $y^i$ will obey
       the differential equation,
      $$
{d^2 y^i(t)\over dt^2}+\omega^2 y^i=0\,.
      \eqno (0.1)
     $$
This is the equation for harmonic oscillator.
Thus we see
that in these new
      coordinates the system is described by $3N$  
free oscilators, the
      osillators, which do not 
interact with each other.
These free harmonic osillators, may be interpreted
      as non-relativistic version ('phonons')
      of photons\footnote{$^{1)}$}{ The material of the lecture
will appear in my personal homepage "khudian.net"
in the subdirectory Lectures/Galois group lectures}.


}


{\tt Physiscts consider Harmonic osillator as a fundamental physical system.
Why?
  Because every system in linear approximmation behaves as a collection
of free oscillators.}

\bigskip
         Consider an arbitrary system of 
interacting $N$  particles $\r_1,\r_2,\dots,\r_N$.
      Every particle has three degrees of freedom 
(three coordinates).  
      This system can be described by second order Newton
      differential equations on $3N$ coordinates:
                $$
   m_i{d^2\r_i(t)\over dt^2}={\bf F}_i.
                \eqno (1.1)
                $$
(for every $i=1,\dot,N$ we have three second order equations.)

  Suppose that all forces are potential, i.e. there exist
a function $U=U(\r_1,\r_2,\dots,\r_n)$ such that
       the force acting on $i$-th particle is equal to
       $$
{\bf F}_i=-{\p U(\r_1,\dots,\r_n)\over \p \r_i}
           \eqno (1.2)
        $$
A function $U=U(\r_1,\dots,\r_N)$ is called {\it potential energy} function.

   Suppose that the  system is 'almost' 
in equilibrium position. What it means? 
   Consider a  set of vectors,
     $\{
 \r_1^{(0)},
 \r_2^{(0)},
  \dots
 \r_{N-1}^{(0)},
 \r_N^{(0)},
      \}$, 
such that if  $i$-th particle is at the point $\r_i^{(0)}$ 
for all   $i=1,2,\dots,N-1,N$  then 
all the forces acting on every particle vanish, i.e.
this becomes an {\it equilibrium configuration}:
all the particles are in the equilibrium position. This means
that the configuration 
              $\{ 
   \r_1^{(0)}, \r_2^{(0)}, \dots \r_{N-1}^{(0)},\r_{N}^{(0)}
           \}
           $ is a stationary point of potential energy function:
       $$
    {\bf F}_i=-{\p U\over \p \r_i}\big\vert_{
\r_1=\r_1^{(0)}, \r_2=\r_2^{(0)}, \dots,
     \r_{N-1}=\r_{N-1}^{(0)},
     \r_{N}=\r_{N}^{(0)}
                           }=0\,.
           \eqno (1.3)
       $$ 
every particle $\r_i(t)$
oscillates around the stability point $\r_{(0)}$\footnote{$^{2)}$}
{is it stable, or instable equilibrium point? This depends on the Hessian
 matrix ${\p^2 U\over \p \r_i \p \r_k}$ in stationary point (see further.) 
 }. 


Our aim is to study the solutions $\r_i(t)$ of equation (1) in the case
of every particle $\r_i(t)$ is almost in equilibrium position (3).

('Almost' means that oscillations are small.)

It is useful to rewrite equations (1.1) as 
equations for Lagrangian:
                    $$
   m_i{d^2\r_i(t)\over dt^2}={\bf F}_i=
-{\p U(\r_1,\dots,\r_n)\over \p \r_i}
 \Leftrightarrow
     {\p L(\r_i,\v_j)\over \p \r_i}=
{d\over dt}\left({\p L\over \p \v_i}\right)
                          \eqno (1.4)  
                    $$
where
           $$
  L=L(\r_i,\v_i)=\hbox {Kinetic energy}-\hbox{Potential energy}=
           \sum_{k=1}^N{m_k \v^2_k\over 2}-U(\r_1,\dots,\r_N)\,.
    \eqno (1.4a)
             $$
%(We denote $\v={d\r\over dt}$.)

{Equations (1.1) are  invariant only with respect to
orthogonal trasnformations. Euler Larange equations (1.1L)
on Lagrangian (5) 
accept arbitrary transformations of coordinates.}.

Change little bit notations

   Following traditions denote the set of all coordinates
  by $q^i$ (where $i=1,\dots, 3N$)
                            $$
              \left(
 \underbrace {\r_1}_{\hbox {first particle}},
 \underbrace {\r_2}_{\hbox {second particle}}\,,
\ldots
 \underbrace {\r_N}_{\hbox {$N$-th  particle}}
               \right)=
                            $$


                            $$
              =\left(
 \underbrace {\pmatrix {x_1\cr y_1\cr z_1\cr }}_{\hbox {first particle}},
 \underbrace {\pmatrix {x_2\cr y_2\cr z_2\cr }}_{\hbox {second particle}}\,,
\ldots
 \underbrace {\pmatrix {x_N\cr y_N\cr z_N\cr }}_{\hbox {$N$-th particle}}
               \right)=
                            $$
                             $$
              \left(
 \underbrace {\matrix {q^1, q^2, q^3\cr }}_{\hbox {first particle}},
 \underbrace {\matrix {q^4, q^5, q^6, }}_{\hbox {second particle}}\,,
\ldots
 \underbrace {\matrix {q^{3N-2}, q^{3N-1}, q^{3N }}}_{\hbox {$N$-th particle}}
               \right)\,,
  \eqno (1.5)
                            $$

Respectively we will rewrite Lagrangian and motion equations
               $$
 L(\r_1,\dots\r_N, \v_1,\dots,\v_N)\longrightarrow
   L(q^1,\dots q^{3N}, \dot q_1,\dots q_{3N})=
\sum_{k=1}^{3N} {m_k \dot q_k^2\over 2}-U(q_1,\dots,q_{3N})\,.
         \eqno (1.4c)
               $$
One can consider the system of $3N$ `one-dimensional particles'
which oscillate around the stability point.

 {\bf Remark} one can see from equation (?)
  that  for every $k=1,\dots N$ coordinates $(q_{3k},q_{3k+1},q_{3k+1})$
are coordinates of the $k$-th particle, i.e. in equation
 (7) 'masses'  $m_{3k}=m_{3k+1}=m_{3k+2}$.

{\bf Remark}  In geometry usually we use upper indices for coordinates.
Beacuse of technical reasons we use lower and upper indices...  

{\bf Remark}  A good example of such a system is the system of
particles whic are loclaised in the nodes of integer lattice.


    For potential energy $U=U(q_1,\dots,q_{3N})$ we have according to (3)
that
      $$
  {\p U(q_1,\dots,q_N)\over \p q_i}
\big\vert_
{q_i=q_i^{(0)}}=0\,,
    \eqno (8)
       $$
where we  denote by $q_i^{(0)}$ the stationary
value of the coordinate $q_i$  (see considerations in equation (6));
  E.g. $\pmatrix {q^{(0)}_1\cr q^{(0)}_2\cr q^{(0)}_3\cr}=
   \r^{(0)}_1$.

        
{\bf Remark}  If we suppose that these are stable 
equilibrium points, this means
that corresponding hessian is non-degenerate and it is positive defnite.

Consider the  Taylor expansion for the 
function $U$ in a vicinity of stationary point.
          $$
U(q_1,\dots q_{3N})=U(q_1^{(0)},\dots q_{3N}^{(0)})+
  \sum_{k=1}^{3N} {\p U(q_1,\dots,q_{3N})\over \p q^k}
       \big\vert_{q_i=q_i^{(0)}}
     \cdot (q_k-q_k^{(0)})+
          $$
           $$
+  {1\over 2}
\sum_{k,m=1}^{3N} {\p^2 U(q_1,\dots,q_{3N})\over \p q_k\p q_m}
       \big\vert_{q_i=q_i^{(0)}}
     \cdot (q_k-q_k^{(0)})+
     \cdot (q_m-q_m^{(0)})+o\left(\left(q-q^{(0)}\right)^2\right)
           \eqno (1.6a)
           $$
Since we are in stability point then due to (8)
this formula can be simplified:
         $$
U(q_1,\dots q_{3N})=U(q_1^{(0)},\dots q_{3N}^{(0)})+
                  {1\over 2}
\sum_{k,m=1}^{3N} 
                \underbrace{
           {\p^2 U(q_1,\dots,q_{3N})\over \p q_k\p q_m}
       \big\vert_{q_i=q_i^{(0)}}
               }_{H_{ik}}
     (q_k-q_k^{(0)})
         (q_m-q_m^{(0)})+o\left(\left(q-q^{(0)}\right)^2\right)
          \eqno (1.6b)
          $$


Now we introduce new coordinates
        $$
\{x^i\}\colon x^i=q_i-q_i^{(0)}
    \eqno (1.7)
         $$
This is just translation, which does not affect differential equations
(5).   We consider small oscillations, thus we will omit terms
of order bigger than $2$ in equation (11);
we omit also constant term in Lagrangian, and we will come to
Lagrangian:
      $$
L(x^1,x^{3N},\dot x^1,\dot x^{3N})=
 \sum_{k=1}^{3N}{m_k \dot x^k \dot x^k\over 2} 
            -
{1\over 2}\sum_{k,m=1}^{3N}H_{km}x^kx^m
    \eqno (1.8)
      $$
In these coordinates differential equations for coordinates
  $x^k$ look almost the same as for coordinates $q_k$:
                 $$
    m_k{d^2 x^k\over dt^2}+\sum_{m=1}^{3N} H_{km}x^m=0\,.
            \eqno (1.8b)
                 $$ 
(we omit higher order terms)
Our aim is to see can we come to new coordinates
such that in these coordinates thiese
equations look simpler.

It is a time to switch on the linear algebra.

\bigskip

  

Consider vector space $\R^{3N}$ of $3N$-tuples.

To every point  with coordinates $(x^1,\dots,x^{3N})$
correspond a vector $\x$. 
We suppose that the canonical basis $\{\e_i\}$ is defined
in $\R^{3N}$, ($\e_i$, $i=1,\dots,3N$  
is $3N$-tuple such that the $i$-th component
of this vector is eual to $1$, and
 all other components are equal to zero. 

  The dynamicla system defines  two bilinear symmetric forms in
the vector space $\R^{3N}$:
First form induced by kinetic energy of the Lagrangian (1.8):
         $$
  T(\x,\y)\colon \quad     T(\x,\x)=\sum_{k=1}^{3N} 
{m_k x^kx^k\over 2}
        \eqno (2.1a)
        $$
and the second form induced by the Hessian of potential 
energy at the equilibrium point:
         $$
   H(\x,\y)\colon\quad    H(\x,\x)=\sum_{k=1}  H_{km}x^kx^m\,,
     \eqno (2,1b)
        $$



{\bf Theorem}  {\it Let $T=T(\x,\y)$, $H=H(\x,\y)$  be two symmetric blinear
forms in (finite-dimensional) 
vector space $V$, and the first form
is positive definite:
                      $$
  T(\x,\x)\geq 0\,, \quad T(\x,\x)=0\Rightarrow \x=0\,.
           \eqno (2.2a)
                      $$
This form defines Euclidean structure 
in vector space $V$, i.e. the
scalar product:
                   $$
    (\x,\y)=T(\x,\y)\,.
             \eqno (2.2b)
                   $$
There exist an orthonormal basis $\{\f_i\}$ 
(with respect to the scalar product (T2)) such that 
                   $$
     H(\e_i,\e_j)=\cases {
        0 \,\,\hbox {if $i\not =j$}\cr
                         }
                $$
i.e. in this basis matrix of bilinear form $T$
is identity matrix, and matrix of the bilinear 
form $H$ is the matrix
                  $$         
||H||={\rm diag}\,\, \{\lambda_1,\lambda_2,\dots, \lambda_n\}\,.
            $$
}





Use this Theorem to the Lagrangian. Consider the new coordinates
$\{y^i\}$ corresponding to the basis $\{\f_i\}$. We come to 
Lagrangian         
                $$
         L=\sum_{k=1}^{3N}{\dot y_k^2\over 2}
         -\sum_{k=1}^{3N}{\lambda_k y_k^2\over 2}\,,
           \eqno (2.3)
                 $$
This is the Lagrangian of $3N$ non-interacting particles.
Every particle obeys the equation of motion
         $$
{d^2 y^i(t)\over dt^2}+\omega^2 y^i=0\,.
     \eqno (2.3b)
         $$
We come to the equation that we declared above (see 
equation (0.1)).

If all $\lambda_i>0$, i.e. (the bilinear form is
$H$ positive definite also) then the equilibrium position is
stable, and  


Now we prove the Theorem 

\centerline {\bf Proof of the Theorem}

  Formulate 

{\bf Lemma} 
A symmetric operator $P$ on Euclidean
space $\E^n$ has at least one eigenvector.
(Operator $P$ is  a symmetric operator
on $\E^n$ if $(P\x,\y)=(\x,P\y)$. 


{Proof of the Theorem ({\tt based on Lemma})}

Symmetric positive-definite bilinear form  $T(\x,\y)$
defines in $V$ the Euclidean structure, the 
scalar product  (2.1a).
    Symmetric bilinear form $H=H(\x,\y)$ defines on this
Euclidean space a symmetric linear operator 
$P_H$ such that  for arbitrary vectors $\x,\y$
    $\left(P_H(\x),\y\right)=H(\x,\y)$.

Using lemma consider the unit vector
   $\e_1$ which is proportional to eigenvector of 
this symmetric operator, and 
  consider the subspace  $V_1$ of vectors orthogonal to the vector $\e_1$: $V_1=\{\x\colon\,\, (\x,\e_1)=0\}$.
The subspace $V_1$ is invariant subspace of operator  $P_H$, 
and this operator is a 
symmetric operator on this subspace. 
We can by induction continue the process
on $V_1$.  Thus we will come to the statement 
of the Theorem.

{\bf Remark} It is very easy to see that in fact we deduced from 
the lemma  the 

{\bf Theorem $'$}  A  symmetric operator on $\E^n$.
is diagonalisable, and all its eigenvalues may be chosen
to be orthogonal to each other.

\medskip


It remains prove the lemma.

   We will present two proves which look different.

       {\tt PROOF OF THE LEMMA}



\centerline  {\tt First proof}

 Let $P$ be a symmetric operator 
on Euclidean space $\E^n$. Consider the function
  $f(\x)=(P\x,\x)$
on the the unit sphere  $S^{n-1}=\{\x\colon, (\x,\x)=1$.The compactness of the sphere implies the existence of a  vector $\x_0$ on the sphere,
      such that this
{function atteints the maximum on this vector.
  This  is an eigenvector of operator $P$.

\m


\centerline {\tt Second proof}


Consider complexification $V_{\E^n}=\E^n\times \C$
  of initial Euclidean vector space $\E^n$.
   
Let $\lambda$ be a root (maybe complex)}
of polynomial  $P(z)=\det (P_H-z)$,
and let $\x$ be a corresponding  eigenvector (may be complex).
Then  the condition $(P\x,\x)=\lambda(\x,\x)=
(\x,P\x)=\bar {\lambda}(\x,\x)$ implies that $\lambda$ is real,
respectively the eigenvector $\x$ maybe chosen also real.

\m
         
It seems that the second proof is purely algebraic, on the other 
hand it uses the fundamental theorem of algebra, hence it uses
more or less the continuity.





 \bye

