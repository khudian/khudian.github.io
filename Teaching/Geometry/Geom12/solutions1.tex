% I begin this version on 20 january 2007
%\magnification=1200
%\baselineskip=14pt
\def\vare {\varepsilon}
\def\A {{\bf A}}
\def\t {\tilde}
\def\a {\alpha}
\def\K {{\bf K}}
\def\N {{\bf N}}
\def\V {{\cal V}}
\def\s {{\sigma}}
\def\S {{\Sigma}}
\def\s {{\sigma}}
\def\p{\partial}
\def\vare{{\varepsilon}}
\def\Q {{\bf Q}}
\def\D {{\cal D}}
\def\G {{\Gamma}}
\def\C {{\bf C}}
\def\M {{\cal M}}
\def\Z {{\bf Z}}
\def\U  {{\cal U}}
\def\H {{\cal H}}
\def\R  {{\bf R}}
\def\E  {{\bf E}}
\def\l {\lambda}
\def\degree {{\bf {\rm degree}\,\,}}
\def \finish {${\,\,\vrule height1mm depth2mm width 8pt}$}
\def \m {\medskip}
\def\p {\partial}
\def\r {{\bf r}}
\def\v {{\bf v}}
\def\n {{\bf n}}
\def\t {{\bf t}}
\def\b {{\bf b}}
\def\e{{\bf e}}
\def\ac {{\bf a}}
\def \X   {{\bf X}}
\def \Y   {{\bf Y}}
\def \x   {{\bf x}}
\def \y   {{\bf y}}
\def \z   {{\bf z}}


\centerline  {\bf Homework 1. Solutions}

\bigskip


{\bf 1} {\it Show that the set of vectors $\{\ac_1,\ac_2\dots,\ac_m\}$ in vector space $V$ is linearly dependent
if at least one of these vectors is equal to zero.}


WLOG suppose that $\ac_1=0$. Then
                  $$
          \lambda\ac_1+0\cdot \ac_2+\dots+0\cdot\ac_n=0
                  $$
where $\lambda$ is an arbitrary real number.   We see that there
exists a linear combinations of vectors $\{\ac_1,\ac_2\dots,\ac_m\}$
which is equal to zero and one of the  coefficients
$\{\lambda,0,\dots,0\}$ could be  equal to non-zero. Hence vectors
$\{\ac_1,\ac_2\dots,\ac_m\}$ are linearly dependent.



\m

{\bf 2}  {\it Show that any three vectors $\{\x_1,\x_2,\x_3\}$ in $\R^2$ are linearly dependent.}
   We will show it straightforwardly here.
   \m

    Let three vectors
               $$
               \matrix
                 {
               \x_1=(a^1,a^2)\cr
               \x_2=(b^1,b^2)\cr
               \x_3=(c^1,c^2)\cr
               }
               $$
be linearly independent.
If vector $\x_1=(a_1,a_2)=0$ then nothing to prove.(See exercise 1). Let $\x_1\not=0$.WLOG suppose $a_1\not=0$.
Consider
              $$
                          \matrix
                 {
               \x'_2=\x_2-{b_1\over a_1}\x_1=(b^1,b^2)-{b_1\over a_1}(a_1,a_2)=(0,b'_2)\cr
                \x'_3=\x_3-{c_1\over a_1}\x_1=(c^1,c^2)-{c_1\over a_1}(a_1,a_2)=(0,c'_2)\cr
               }
                 $$
We see that vectors  $\x'_2,\x'_3$ are proportional---i.e. they are linearly dependent:
there exist  $\mu_2\not=0$ or $\mu_3\not=0$ such that $\mu_2\x'_2+\mu_3\x'_3=0$
E.g. we can take $\mu_2=c'_2$, $\mu_3=-b'_2$ if $c'_2\not=0$ or $b'_2\not=0$ (if $c'_2=b'_2\not=0$ then we can take
coefficients $\mu_1,\mu_2$ any real numbers. )  We have:
                 $$
              0=\mu_2\x'_2+\mu_3\x'_3=\mu_2\left(\x_3-{c_1\over a_1}\x_1\right)+
              \mu_3\left(\x_3-{c_1\over a_1}\x_1\right)=\mu_2\x_2+\mu_3\x_3-
              \left({\mu_2b_1\over a_1}+{\mu_3c_1\over a_1}\right)\x_1=0,
                 $$
where $\mu_2\not=0$ or $\mu_3\not=0$. Hence vectors $\x_1,\x_2,\x_3$ are linearly dependent.\finish

   (Compare with the solution of general statement in the next exercise.)




 {\bf 3}
 {\it Let $3$ vectors $\{\x_1,\x_2,\x_3\}$ in vector space $V$
 can be expressed as a linear combination of $2$ vectors
 $\{\ac,\b\}$ of this vector space, i.e. 3 vectors   $\{\x_1,\x_2,\x_3\}$ belong to the span of 2
 vectors $\{\ac,\b\}$. Prove that three vectors $\{\x_1,\x_2,\x_3\}$ are linearly dependent.}

 Let
                $$
                \cases
                {
                \x_1=\lambda_1\ac+\mu_1\b\cr
                \x_2=\lambda_2\ac+\mu_2\b\cr
                \x_3=\lambda_3\ac+\mu_3\b\cr
                }
                \eqno (1)
                $$
If one of vectors is equal to zero then nothing to prove (See previous exercise).

$\x_1\not=0$. WLOG suppose that  $\lambda_1\not=0$. Thus vector $\ac$ can be expressed as a linear combination
of vectors  $\x_1$ and $\b$: $$
\ac={1\over \lambda_1}\x_1-{\mu_1\over \lambda_1}\b
\eqno (2)
                               $$.
(If $\lambda_1=0$ then $\mu\not=0$ and we express the vector $b$ as a linearly combination
of vectors  $\x_1$ and $\ac$).  Then using the relation (2) we
 express vector $\x_2$ as linear combinations of vectors $\ac$ and $\x_1$:
                           $$
                \x_2=\lambda_2\left({1\over \lambda_1}\x_1-{\mu_1\over \lambda_1}\b\right)+
                \mu_2\b=\lambda'_2\x_1+\mu'_2\b
                \eqno (3)
                           $$
If $\mu'_2=0$ then everything is proved: vectors $\x_1,\x_2$ are linearly dependent, hence
vectors $\x_1,\x_2,\x_3$ are linearly dependent too.
If $\mu'_2\not=0$ we express vector $\b$ via vectors $\x_1$ and $\x_2$:
                    $$
                \b=-{1\over \mu_1}\x_2-{\lambda'\over\mu'}\x_1
                \eqno (4)
                    $$
and using relations (4) and (2) we express vector $\x_3$ in (1) as a linear combinations of vectors $\x_1$ and $\x_2$,
thus proving that vectors $\{\x_1,\x_2,\x_3\}$ are linearly dependent.
                  $$
     \x_3=\lambda_3\ac+\mu_3\b=\lambda_3\left(-{1\over \lambda_1}\x_1-{\mu_1\over \lambda_1}\b
\right)+\mu_3\left({1\over \mu_1}\x_2-{\lambda'\over\mu'}\x_1\right)=
                  $$
                  $$
    \lambda_3\left(-{1\over \lambda_1}\x_1-{\mu_1\over \lambda_1}\left({1\over \mu_1}\x_2-{\lambda'\over\mu'}\x_1\right)
\right)+\mu_3\left({1\over \mu_1}\x_2-{\lambda'\over\mu'}\x_1\right)=
\lambda_3'' \x_1+\mu_3''\x_2
                  $$
Vector $\x_3$ is a linear combination of vectors  $\x_2,\x_3$. Hence vectors $\x_1y,\x_2,\x_3$ are linearly dependent.

 In a similar way one can prove that any $m+1$ vectors are linearly dependent if they belong to the span
of $m$ vectors


\m

{\bf 4} {\it Let $\{\ac,\b\}$ be two vectors in the linear space $V$ such that

    i) these vectors are linearly independent

    ii) for an arbitrary vector $\x\in V$ vectors $\{\ac,\b,\x\}$ are linearly dependent.

  What is a dimension of the vector space $V$?

    Is an ordered set $\{\ac,\b\}$ a basis in the vector space $V$?}


\m

   Recall that the dimension of vector space $V$ is equal to $n$ if there exist $n$ linearly independent vectors
   and any $n+1$ vectors are linearly dependent.
 Show that the dimension of the vector space under consideration is equal to $2$.

   On one hand there exist two linearly dependent vectors  $\ac$ and $\b$.
    This means that dimension of $V$ is greater  or equal than 2:
${\rm dim\,} V\geq 2$.

   To prove that ${\rm dim\,} V= 2$ it remains to prove that any
three vectors are linearly dependent.

   Show first that arbitrary vector  $\x\in V$ can be expressed
   via vectors  $\ac, \b$.  Indeed vectors $\{\ac,\b,\x\}$ are linearly dependent, hence
                 $$
              \mu_1\ac+\mu_2\b+\mu_3\x=0,
             \quad
             \hbox{where  $\mu_1\not=0$, or $\mu_2\not=0$ or $\mu_3\not=0$
                 }
                 $$
   If $\mu_3=0$ then $\mu_1\not=0$, or $\mu_2\not=0$ and  $\mu_1\ac+\mu_2\b=0$, i.e. vectors
   $\ac,\b$ are linearly dependent. Contradiction. Hence $\mu_3\not=0$, that is a vector $\x$ can be expressed
   as a linear combination of vectors $\ac,\b$, i.e. it belongs to the span of the vectors $(\ac,\b)$.



    Let $\{\x_1,\x_2,\x_3\}$ be a set of
   arbitrary $3$ vectors.
   We just proved that any of these vectors belong to the span of the vectors $\{\ac,\b\}$.
    Hence according to previous exercise these three vectors
     $\{\x_1,\x_2,\x_3\}$ are linearly dependent.  Thus we proved that
     any three vectors
     $\{\x_1,\x_2,\x_3\}$ are linearly dependent.

     Hence the dimension of the space $V$ is equal to $2$.

     The vectors $\{\ac,\b\}$ are two linearly independent vectors in $2$-dimensional vector space $V$.
       Hence it is a basis.


     \m



  {\bf 5}  {\it Let $\{\e_1,\e_2,\e_3\}$ be a basis  in $3$-dimensional vector space $V$.
  Show that


  a) all vectors $\e_1,\e_2,\e_3$ are not equal to zero.


  b)  an arbitrary vector $\ac\in V$ can be expressed as a
  linear combination of the basis vectors $\{\e_1,\e_2,\e_3\}$ in a unique way, i.e. if
                          $$
          \ac=a^1\e_1+a^2\e_2+a^3\e_3=a^{1'}\e_1+a^{2'}\e_2+a^{3'}\e_3\,\,\,{\rm then}\,\,\,
           a^1=a^{1'},\,a^2=a^{2'},\,a^3=a^{3'}\,.
           \eqno (5)
                           $$
}
 a)  Suppose one of these vectors is equal to zero: $\e_1=0$. Then the vectors
   $\{\e_1,\e_2,\e_3\}$ are linearly dependent. (See the exercise 1).

   b)
   First prove the uniqueness of expansion (5) then the existence.
   Let $\ac$ be an arbitrary vector in $V$.
   Suppose
                $$
     \ac=a^1\e_1+a^2\e_2+a^3\e_3=a^{1'}\e_1+a^{2'}\e_2+a^{3'}\e_3\,.
                $$
   Then
                 $$
     0=\ac-\ac=(a^1-a{1'})\e_1+(a^2-a^{2'})\e_2+(a^3-a^{3'})\e_3\,.
                 $$
   On the other hand vectors $\{\e_1,\e_2,\e_3\}$ are linearly independent.
   Hence
  all  coefficients  $(a^1-a^{1'}), (a^2-a^{2'}),(a^3-a^{3'})$
  are equal to zero:
                $$
       a^1-a^{1'}=a^2-a^{2'}=a^3-a^{3'}=0,\,\,
    {\rm i.e.}\,\, a^1=a^{1'},a^2=a^{2'},a^3=a^{3'}\,.
                $$
   According to definition of basis
   $4$ vectors $\{\e_1,\e_2,\e_3,\ac\}$ are linearly dependent. Hence
   vector $\ac$ can be expressed via the vectors $\{\e_1,\e_2,\e_3\}$. Indeed
   there exist coefficients  $\lambda_1,\lambda_2,\lambda_3,\lambda_4$ such that
              $$
   \lambda_1\e_1+\lambda_2\e_2+\lambda_3\e_3+\lambda_4\ac=0
   \eqno (6)
              $$
   and at least one of these coefficients is not equal to zero.

   Prove that $\lambda_4\not=0$. Suppose $\lambda_4=0$. Then it follows from
   (6) that vectors $\{\e_1,\e_2,\e_3\}$ are linearly dependent. Contradiction. Hence
    $\lambda_4\not=0$
   and  $\ac$ can be expressed via $\e_1,\e_2,\e_3$:
                    $$
       \ac=-{\lambda_1\over \lambda_4}\e_1-{\lambda_2\over \lambda_4}\e_2-
       {\lambda_3\over \lambda_4}\e_3
                    $$


\m


{\bf 6$^{\dagger}$}  {\it Show that the ordered set
  $\{\e_1,\e_2,\e_3,\e_4,\dots,\e_n\}$
of vectors is a basis in $\R^n$ in the case if}
                  $$
                  \matrix
                    {
    \e_1&=(1,&2,&3,&4,\dots,&n)\cr
    \e_2&=(0, &1,&2,&3,\dots,&n-1)\cr
    \e_3&=(0, &0, &1,&2,\dots,&n-2)\cr
        \ldots \cr
    \e_n&=(0,&0, &0,&0,\dots,&1)\cr
                    }
                    $$
If $\sum\lambda_i\e_i=0$ then one can see that $\lambda_1=0$. This implies
that $\lambda_2=0$ and so on all coefficeints $\lambda_i$ vanish.
 We proved that these $n$  vectors in $n$-dimensional space $\R^n$
are linear independent. Hence
  $\{\e_1,\e_2,\e_3,\e_4,\dots,\e_n\}$
is a basis.
 \m
{\bf 7}  Let $\{\e_1,\e_2,\e_3\}$ be a basis of $3$-dimensional vector space $V$.

   Is a set of vectors $\{\e'_1,\e'_2,\e'_3\}$ a basis of $V$ in the case if

a) $\e'_1=\e_2$, $\e'_2=\e_1$, $\e'_3=\e_3$;

b)  $\e'_1=\e_1$, $\e'_2=\e_1+3\e_3$, $\e'_3=\e_3$;


c) $\e'_1=\e_1-\e_2$, $\e'_2=3\e_1-3\e_2$, $\e'_3=\e_3$;

d) $\e'_1=\e_2$, $\e'_2=\e_1$, $\e'_3=\e_1+\e_2+\lambda \e_3$ (where $\lambda$ is an arbitrary coefficient)?



To analyse the cases we use the definition of basis:
3 vectors in $3$-dimensional space form a basis if and only if these vectors
 are linearly independent.

Case a) Vectors $\e'_1=\e_2, \e_2'=\e_1, \e'_3=\e_3$ are linearly independent, since
$\{\e_1,\e_2,\e_3\}$ is a basis. Hence $\{\e'_1,\e'_2,\e'_3\}$ is a basis too.

\m

Case b) Vectors  $\e'_1=\e_1, \e_2'=\e_1+3\e_3, \e'_3=\e_3$ are linearly dependent. Indeed
          $$
\e'_1-\e'_2+3\e'_3=\e_1-(\e_1+3\e_3)+3\e_3=0\,.
          $$
Hence it is not a basis.


Case c)  First two vectors  $\e'_1=\e_1-\e_2, \e_2'=3\e_1-3\e_2$
are already linearly dependent: $\e'_1=3\e_2'$. Hence these three vectors  do not form a basis.




\m
Case d)  Check are vectors linearly independent or not. Let
        $c_1 {\e'_1}+c_2 \e'_2+c_3\e'_3=0$, i.e.
                $$
c_1 {\e'_1}+c_2 \e'_2+c_3\e'_3=c_1\e_2+c_2\e_1+c_3(\e_1+\e_2+\lambda\e_3)=(c_2+c_3)\e_1+(c_1+c_3)\e_2+c_3\lambda\e_3=0\,.
                $$
I-st case $\lambda\not=0$. We have  $c_2+c_3=c_1+c_3=\lambda c_3=0$. Hence $c_3=0,c_1=0,c_2=0$. These three vectors are
linearly independent. This means that ordered triple $\{\e'_1,\e'_2,\e'_3\}$ is a basis.


II-nd case $\lambda=0$. We have  $c_2+c_3=c_1+c_3=0c_3=0$. Hence $c_3$ can be an arbitrary number and
$c_1=-c_3,c_2=-c_3$. $c_3$  These three vectors are
linearly dependent. This means that ordered triple $\{\e'_1,\e'_2,\e'_3\}$ is not a basis.


\bye
