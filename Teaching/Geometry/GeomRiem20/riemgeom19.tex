%It is a new course. I began this document on 31-st January 2010.
% This is version of 2013
% I began to write the file on 31-st January
%on the base of the previous year
% I began to write it on the base of previous year  23 January 2015

% I began to write it on the base of previous year  04 February  2016
% I began to write it on the base of previous year  24 January  2017
% I began to write it on the base of previous year  11 January  2018
% here I have to cut 4-th year material and 
%add shape operator

% I began it on 2 February 2019 and do some 
%changings

\def\vare {\varepsilon}
\def\A {{\bf A}}
\def\B {{\bf B}}
\def\t {\tilde}
\def\a {\alpha}
\def\K {{\bf K}}
\def\N {{\bf N}}
\def\V {{\cal V}}
\def\s {{\sigma}}
\def\S {{\bf S}}
\def\s {{\sigma}}
\def\p{\partial}
\def\vare{{\varepsilon}}
\def\Q {{\bf Q}}
\def\D {{\cal D}}
\def\L {{\cal L}}
\def\G {{\Gamma}}
\def\C {{\bf C}}
\def\M {{\cal M}}
\def\Z {{\bf Z}}
\def\U  {{\cal U}}
\def\H {{\cal H}}
\def\R  {{\bf R}}
\def\E  {{\bf E}}
\def\l {\lambda}
\def\degree {{\bf {\rm degree}\,\,}}
\def \finish {${\,\,\vrule height1mm depth2mm width 8pt}$}
\def \m {\medskip}
\def\p {\partial}
\def\r {{\bf r}}
\def\v {{\bf v}}
\def\n {{\bf n}}
\def\t {{\bf t}}
\def\b {{\bf b}}
\def\e{{\bf e}}
\def\f{{\bf f}}
\def\ac {{\bf a}}
\def \X   {{\bf X}}
\def \Y   {{\bf Y}}
\def\diag {\rm diag\,\,}
\def\pt {{\bf p}}
\def\w {\omega}
\def\la{\langle}
\def\ra{\rangle}
\def\x{{\bf x}}

\documentclass[12pt]{article}
\usepackage{amsmath,amsthm}



\usepackage{amsmath,amssymb,amsfonts,amsthm}


\theoremstyle{theorem}
\newtheorem{thm}{Khimera}

\numberwithin{equation}{section}


\title{Riemannian Geometry}
\date{}
\begin{document}
\maketitle

  \centerline {it is a draft of Lecture Notes of H.M. Khudaverdian.}

  \centerline { Manchester, 12 April,  2019}





\tableofcontents
\pagenumbering{roman}
\newpage
\pagenumbering{arabic}




\section {Riemannian manifolds}


\subsection { Manifolds. Tensors. (Recollection)}

\subsubsection{Manifolds}



I recall briefly basics of manifolds 
and tensor fields on manifolds.

An $n$-dimensional manifold $M=M^n$ is a 
space\footnote
{A space $M$ is a topological space, i.e. it 
is covered by a collection $\cal F$ of sets,
which are called {\it open} sets.  This collection
obeys the following axioms

i) the union of an arbitrary set of open sets
is an open set

ii) the intersection of finite number of open sets
is an open set

iii) the whole space $M$ and the empty set 
$\emptyset$ are open sets
} 

such that in a vicinity of an
arbitrary  point
one can consider local coordinates 
$\{x^1,\dots,x^n\}$.
(We say that in a vicinity of this point
a manifold $M$ is covered by local 
coordinates $\{x^1,\dots,x^n\}$).
One can consider different local coordinates.
If coordinates $\{x^1,\dots,x^n\}$ and 
$\{x^{1'},\dots,x^{n'}\}$ 
both are defined in a vicinity of the given point
then they are related by  {\it bijective transition functions} 
which are  defined on domains in $\R^n$ and 
taking values also in $\R^n$:
              \begin{equation}
        \label{changingofcoordinates}
             \begin{cases}
             x^{1'}=x^{1'}(x^1,\dots,x^n)\cr
             x^{2'}=x^{2'}(x^1,\dots,x^n)\cr
                \dots\cr
            x^{{n-1}'}=x^{{n-1}'}(x^1,\dots,x^n)\cr
              x^{{n}'}=x^{{n}'}(x^1,\dots,x^n)\cr
             \end{cases}
              \end{equation}
 We say that $n$-dimensional
manifold is {\it differentiable} or {\it smooth} 
if all transition functions are diffeomorphisms,
i.e. they are smooth.
  Invertability implies that  Jacobian matrix is non-degenerate:
              \begin{equation}\label{jacobiannonzero}
      \det
          \begin{pmatrix}
          {\p x^{1'}\over \p x^1} &{\p x^{1'}\over \p x^2}\dots &{\p x^{1'}\over \p x^n}\cr
         {\p x^{2'}\over \p x^1} &{\p x^{2'}\over \p x^2} \dots &{\p x^{2'}\over \p x^n}\cr
                           \dots\cr
      {\p x^{n'}\over \p x^1} &{\p x^{n'}\over \p x^2}\dots &{\p x^{n'}\over \p x^n}\cr
          \end{pmatrix}\not=0\,.
              \end{equation}
 (If bijective function $x^{i'}=x^{i'}(x^i)$ is smooth function,
and its inverse, the transition function $x^{i}=x^{i}(x^i)$  
is also smooth function, then
matrices  $||{\p x^{i'}\over \p x^i}||$ and
$||{\p x^i\over \p x^{i'}}||$  
are both well defined, hence condition
\eqref{jacobiannonzero} is obeyed.
 
\m

{\bf Example}
    
  \centerline {open domain in $\E^n$}

    A good example of manifold is an open domain  $D$ in  
$n$-dimensional vector space $\R^n$.
    Cartesian coordinates on $\R^n$ define global coordinates on $D$.
    On the other hand one can consider an arbitrary 
local coordinates in different domains in $\R^n$. E.g. 
one can consider polar coordinates $\{r,\varphi\}$ 
in a domain $D=\{x,y\colon y>0\}$ of $\R^2$
    defined by standard formulae:
              \begin{equation}\label{exampleofpolarcoordinates}
                \begin{cases}
                x=r\cos\varphi\cr y=r\sin\varphi\cr
                \end{cases}\,,
              \end{equation}
            \begin{equation}\label{jacobianofpolartransform}
                \det
          \begin{pmatrix}
          {\p x\over \p r} &{\p x\over \p \varphi}\cr
          {\p y\over \p r} &{\p y\over \p \varphi}\cr
          \end{pmatrix}=
                \det
          \begin{pmatrix}
          \cos\varphi &-r\sin\varphi\cr
          \sin\varphi &r\cos\varphi\cr
          \end{pmatrix}=r
            \end{equation}
             or one can consider spherical 
coordinates  $\{r,\theta,\varphi\}$ in a domain
  $D=\{x,y,z\colon x>0,y>0, z>0\}$ of $\R^3$ (or in other domain of $\R^3$)
  defined by standard formulae
  \begin{equation*}
                \begin{cases}
                x=r\sin\theta\cos\varphi\cr 
            y=r\sin\theta\sin\varphi\cr z=r\cos\theta
                \end{cases}\,,
              \end{equation*}\,,
            \begin{equation}\label{jacobianofspherictransform}
                \det
          \begin{pmatrix}
          {\p x\over \p r} &{\p x\over \p \theta} &{\p x\over \p \varphi}\cr
          {\p y\over \p r} &{\p y\over \p \theta} &{\p y\over \p \varphi}\cr
           {\p z\over \p r} &{\p z\over \p \theta} &{\p z\over \p \varphi}\cr
          \end{pmatrix}=
                \det
          \begin{pmatrix}
          \sin\theta\cos\varphi &r\cos\theta\cos\varphi &-r\sin\theta\sin\varphi\cr
          \sin\theta\sin\varphi &r\cos\theta\sin\varphi &r\sin\theta\cos\varphi\cr
           \cos\theta &-r\sin\theta &0\cr
                 \end{pmatrix}=r^2\sin\theta
            \end{equation}
 Choosing domain where polar (spherical) 
coordinates are well-defined
we have to be aware
that coordinates have to be well-defined and
transition functions \eqref{changingofcoordinates}
have to obey condition \eqref{jacobiannonzero}, i.e. they
  have to be diffeomorphisms. E.g. for domain $D$
in example \eqref{exampleofpolarcoordinates} Jacobian
\eqref{jacobianofpolartransform} does not vanish if and only if
   $r>0$ in $D$.
\m

Consider another examples of manifolds,
and local coordinates on manifolds.

 
{\bf Example}  

 \centerline {\it  Circle  $S^1$ in $\E^2$}

Consider circle $x^2+y^2=R^2$ of 
radius $R$ in $\E^2$.

One can consider on the circle 
different local coordinates

i) {\it polar coordinate $\varphi$}:
    \begin{equation*}
      \begin{cases}
    x=R\cos\varphi\cr
    y=R\sin\varphi\cr
      \end{cases}\,,\qquad 0<\varphi<2\pi
     \end{equation*}
(this coordinate is defined on all the circle 
except a point $(R,0)$),

ii) {\it another polar coordinate $\varphi'$}:
    \begin{equation*}
      \begin{cases}
    x=R\cos\varphi\cr
    y=R\sin\varphi\cr
      \end{cases}\,,\qquad -\pi<\varphi<\pi\,,
     \end{equation*}
this coordinate is defined on all the circle 
except a point $(-R,0)$,

iii)  stereographic coordinate $t$
with respect to north pole of the circle
     \begin{equation}\label{stereogrcoordoncircle}
      \begin{cases}
    x={2R^2t\over t^2+R^2}\cr
    y=R{t^2-R^2\over t^2+R^2} \cr
      \end{cases}\,,\quad
       t={Rx\over R-y}\,,
    \end{equation}
this coordinate is defined at all the 
circle except the north pole,

iiii)  stereographic coordinate $t'$
with respect to south pole of the circle
     \begin{equation*}
      \begin{cases}
    x={2R^2t'\over t'^2+R^2}\cr
  z=R{R^2-t'^2\over t'^2+R^2} \cr
      \end{cases}\,,\quad
       t'={Rx\over R+y}\,,
     \end{equation*}
this coordinate is defined at all the points except
the south pole.

We considered four different local coordinates
on the circle $S^1$. Write down some
 transition 
functions \eqref{changingofcoordinates}
between these coordinates
 
\begin{itemize}
\item
 polar
coordinate $\varphi$ coincide with
 polar coordinate
  $\varphi'$ in the domain $x^2+y^2>0$,
and in the domain $x^2+y^2<0$
        $\varphi'=\varphi-2\pi$. 

\item
Transition function
from polar coordinate  $\varphi$ 
to stereographic coordinates
  $t$ is  $t=\tan \left(
{\pi\over 4}+{\varphi\over 2}\right)$,

\item
transition function from stereographic
coordinate $t$ to stereographic coordinate
$t'$ is

      \begin{equation*}
       t'={R^2\over t}\,,
     \end{equation*}

\end{itemize}

(see Homework 0.)


\m

{\bf Example}   

    \centerline {\it Sphere  $S^2$ in $\E^3$}

Consider sphere $x^2+y^2+z^2=R^2$ of 
radius $a$ in $\E^3$.

One can consider on the sphere 
different local coordinates

i) {\it spherical coordinates on domain of sphere}
  $\theta,\varphi$:
    \begin{equation*}
      \begin{cases}
    x=a\sin\theta\cos\varphi\cr
    y=a\sin\theta\sin\varphi\cr
    z=a\cos\theta\cr
      \end{cases}\,,\quad
           0<\theta<\pi\,,
     -\pi<\varphi<\pi
     \end{equation*}

ii)  stereographic coordinates $u,v$
with respect to north pole of the sphere
     \begin{equation*}
      \begin{cases}
    x={2a^2u\over a^2+u^2+v^2}\cr
    y={2a^2v\over a^2+u^2+v^2}\cr
    z=a{u^2+v^2-a^2\over a^2+u^2+v^2} \cr
      \end{cases}\,,\quad
       {x\over u}=      
       {y\over v}=      
       {a-z\over a}\,,\quad
    \begin{cases}
      u={ax\over a-z}\cr
      v={ay\over a-z}\cr
     \end{cases}\,.     
    \end{equation*}

iii)  stereographic coordinates $u',v'$
with respect to south pole of the sphere
     \begin{equation*}
      \begin{cases}
    x={2a^2u'\over a^2+u'^2+v'^2}\cr
    y={2a^2v\over a^2+u'^2+v'^2}\cr
    z=a{a^2-u'^2-v'^2\over a^2+u^2+v^2} \cr
      \end{cases}\,,\quad
       {x\over u'}=      
       {y\over v'}=      
       {a+z\over a}\,,\quad
    \begin{cases}
      u'={ax\over a+z}\cr
      v'={ay\over a+z}\cr
     \end{cases}\,.     
     \end{equation*}
(see also Homework 0)

Spherical coordinates are defined elsewhere except
 poles and  the  meridians  $y=0,x\leq 0$.

Stereographical coordinates $(u,v)$ are defined elsewhere
except north pole;  

stereographic 
coordinates $(u',v')$ are defined elsewhere
except south pole. 

One can consider transition function
between these different coordinates.E.g. transition functions
from spherical coordinates i) 
to stereographic coordinates
  $(u,v)$ are
     \begin{equation*}
       \begin{cases}
     u={ax\over a-z}=
{a\sin\theta\cos\varphi\over 1-\cos\theta}=
  a{\rm cotan\,}{\theta\over 2}\cos\varphi\cr
     v={ay\over a-z}=
{a\sin\theta\sin\varphi\over 1-\cos\theta}=
  a{\rm cotan\,}{\theta\over 2}\sin\varphi\cr
        \end{cases}\,,
     \end{equation*}
and transition function from stereographic
coordinates $u,v$ to stereographic coordinates
$(u',v')$
are
      \begin{equation*}
       \begin{cases}
     u'={a^2u\over u^2+v^2}\cr
     v'={a^2v\over u^2+v^2}\cr
        \end{cases}\,,
     \end{equation*}
(see Homework 0.)

{\bf Remark}

{\footnotesize $^\dagger$ One very important property of stereographic projetion
which we do not use in this course but it is too beautiful not to mention it:
under stereographic projection  all points
of the circle of radius $R=1$
with rational coordinates $x$ and $y$ and 
only these points
transform to rational points on line. 
Thus we come to
Pythagorean triples $a^2+b^2=c^2$.
The same is for unit sphere: the stereographic
projection establishes one-one correspondence
between points on the unit sphere with rational
coordinates and rational points on the plane. }






  \subsubsection{ Tensors on Manifold}


{\it tangent vector and tangent vector space}

\smallskip

 Tangent vector at the given point
can be considered as a derivation of function at this point.
  For an arbitrary (smooth)
function $f$ defined in a vicinity of a given 
 point $\pt$ a tangent vector $\A(x)=A^i(x){\p\over \p x^i}$
defines the directional derivative of this function 
      \begin{equation*} 
 \A\colon \quad 
   f \mapsto  \p_\A f\big\vert_\pt=A^i(x)
{\p f\over \p x^i}\big\vert_{\pt}\,.    
      \end{equation*} 
Using the chain rule 
one can see that
under changing of coordinates 
it transforms as follows: 
             $$
         \A=A^i(x){\p\over \p x^i}=
A^i(x){\p x^{i'}(x)\over \p x^i}{\p\over \p x^{i'}}=
         A^{i'}(x'(x)){\p\over \p x^{i'}}\,,
             $$
i.e.
            \begin{equation}\label{transformofvector}
                   A^{i'}(x')={\p x^{i'}\over \p x^i}A^i(x)\,.
            \end{equation}
     This leads as to the following equivalent definition
of the tangent vector.

{\bf Definition}   Let $M=M^n$ be $n$-dimensional
manifold, and $\pt$ the point on it.
To define a  vector $\A$ tangent to the 
manifold  at the point $\pt$
we assign to an
arbitrary   
given local coordinates $\{x^i\}$
the array $\{A^i\}$ ($i=1,\dots,n$)
of numbers (components) 
such that under changing of local coordinates
this array  transforms  according to equation
\eqref{transformofvector}: 
         \begin{equation}\label{definitionofvector}
  \begin{matrix}
&\hbox{{\tt {\footnotesize coordinates}}}
 &            
&\hbox{{\tt {\footnotesize components of vector}}}\cr
&\{x^i\}   &\rightarrow
     &   \{A^i\}\cr
 &&&\cr
&\{x^{i'}\}   &\rightarrow
     &   \{A^{i'}\}\cr
      \end{matrix}\hbox{such that}\, 
      A^{i'}=
{\p x^{i'}(x)\over \p x^i}\big\vert_\pt  A^i\,.  
        \end{equation}


Tangent vector space $T_\pt M$  at the point $\pt$ is
the space of vectors tangent to the manifold at 
the point $M$.


%\end{document}
% 2 February 2019

{\it $1$ -form (covector) in a given point}

We defined above  vectors of tangent space $T_\pt M$.
Now we consider dual obejcts: we consider cotangent space $T^*_\pt M$
(for every point $\pt$ on manifold $M$)---
    space of linear functions on tangent vectors, i.e. space 
of $1$-forms which sometimes are called
              {\it covectors.}:



 Linear function, $1$-form  $\w=\w_idx^i$ is a function on tangent vectors:
               $$
 T_\pt M\ni\A=A^i{\p \over \ \p x^i}\,, \w\left(\A\right)
=\w_mdx^m\left(A^i{\p \over \ \p x^i}\right)=\w_mA^i dx^i
\underbrace{\left({\p\over \p x^m}\right)}_{\delta_i^m}=\w_mA^m\,.
               $$ 
If we consider new coordinates $x^{i'}=x^{i'}(x)$, then
                  $$
\w=\w_idx^i=\w_i\left({\p x^i\over \p x^{i'}}dx^{i'}\right)=
           \underbrace{\w_i{\p x^i\over \p x^{i'}}_{\w_{i'}}} dx^{i'}
                  $$
                   i.e., nne-form (covector) $\w=\w_i(x)dx^i$ 
transforms as follows
         \begin{equation}\label{transformofcovector}
            \w_{m'}(x')={\p x^{m}(x')\over \p x^{m'}}\w_m(x)\,.
         \end{equation}

Differential form sometimes is called {\it covector}.


In the same way as for vectors we may give definition
of covectors in the following way:

{\bf Definition}   Let $M=M^n$ be $n$-dimensional
manifold, and $\pt$ the point on it.
To define a  {\it covector $\A$} at the point $\pt$, 
(the linear function
on tangent vectors at $\pt$)
we assign to an
arbitrary   
given local coordinates $\{x^i\}$
the collection $\{\w_i\}$ ($i=1,\dots,n$)
of numbers (components) such that under 
changing of local coordinates
this collection transforms  according to equation
\eqref{transformofcovector}: 
         \begin{equation}\label{definitionofcovector}
  \begin{matrix}
&\hbox{{\tt {\footnotesize coordinates}}}
 &            
&\hbox{{\tt {\footnotesize components of covector}}}\cr
&\{x^i\}   &\rightarrow
     &   \{\w_i\}\cr
 &&&\cr
&\{x^{i'}\}   &\rightarrow
     &   \{\w_{i'}\}\cr
      \end{matrix}\hbox{such that}\, 
      \w_{i'}=
{\p x^{i}(x')\over \p x^{i'}}\big\vert_\pt \w_i \,.  
        \end{equation}

{\bf Remark}  Notice the difference between formulae
\eqref{transformofvector} and \eqref{transformofcovector}.
In formulae \eqref{transformofvector}, \eqref{definitionofvector} 
transformation is performed
by matrix of derivatives ${\p x^{i'}\p x^i}$ from coordinates
    $x^i$ to the new coordinates $x^{i'}$, and
in formula \eqref{transformofcovector} transformation is performed
by the {\it inverse }matrix, matrix  of derivatives ${\p x^{i}\p x^{i'}}$ 
from new coordinates
    $x^{i'}$ to the initial coordinates $x^{i}$.

\m


          {\it Tensors}:




\smallskip

{\bf Definition}  
   Consider geometrical object
    such that in arbitrary local coordinates
   $(x^i)$
it is given by components
    $$
 Q=\left\{
    Q^{i_1i_2\dots i_p}_{j_1j_2\dots j_q}
      {\p\over \p x^{i_1}}(x)
        \right\}\,, i_1,\dots, i_p;j_1,\dots,j_q=1,2,\dots,n\,,
        $$
and under changing of coordinates this object is transformed in the following
way:
      \begin{equation}\label{ruleoftransformationofarbitrarytensors}
    Q^{i'_1i'_2\dots i'_p}_{j'_1j'_2\dots j'_q}(x')=
    {\p x^{i'_1}\over \p x^{i_1}}
    {\p x^{i'_2}\over \p x^{i_2}}
    \dots
    {\p x^{i'_p}\over \p x^{i_p}}
    {\p x^{j_1}\over \p x^{j'_1}}
    {\p x^{j_2}\over \p x^{j'_2}}
    \dots
    {\p x^{j_q}\over \p x^{j'_q}}
    Q^{i_1i_2\dots i_p}_{j_1j_2\dots j_q}(x)\,.
\end{equation}
We say that  this is
   {\it $p$-times contravariant, $q$-times covariant
tensor of valence $\begin{pmatrix} p\cr q\cr\end{pmatrix}$,
or shorter, 
 tensors of the type  $\begin{pmatrix} p\cr q\cr\end{pmatrix}$}.

{\tt Caution: this tensor possess $n^{p+q}$ components.}
 
\smallskip

Sometimes it is useful
to view  
 $\begin{pmatrix} p\cr q\cr\end{pmatrix}$-tensor as
       \begin{equation*}
 Q= Q^{i_1i_2\dots i_p}_{j_1j_2\dots j_q}
      {\p\over \p x^{i_1}}\otimes
      {\p\over \p x^{i_2}}\otimes\dots
      \otimes{\p\over \p x^{i_p}}
             dx^{j_1}\otimes
             dx^{j_2}\otimes\dots
             \otimes dx^{j_q}
          \end{equation*}       

(Compare with definition of vector: $\A=A^i{\p\over \p x^i}$
and covector ($1$-form) $\w=\w_idx^i$).



\centerline {\bf Examples}

\smallskip


    Note that vector field \eqref{transformofvector} is nothing but
 tensor field of valency 
$\begin{pmatrix}1 \cr 0\cr\end{pmatrix}$,
and $1$-form \eqref{transformofcovector} is nothing but
tensor field of valency 
$\begin{pmatrix}0 \cr 1\cr\end{pmatrix}$,

\medskip


One can consider {\it contravariant} tensors  of the rank $p$
                   $$
             T=  T^{i_1i_2\dots i_p}(x){\p\over \p x^{i_1}}\otimes
             {\p\over \p x^{i_2}}\otimes \dots \otimes {\p\over \p x^{i_k}}
                   $$
with components $\{T^{i_1i_2\dots i_k}\}(x)$.
Under changing of coordinates 
$(x^1,\dots,x^n)\to (x^{1'},\dots,x^{n'})$ (see
\eqref{changingofcoordinates}) they transform as follows:
            \begin{equation}\label{transformofcontrtensors}
T^{i'_1i'_2\dots i'_p}(x')={\p x^{i'_1}\over \p x^{i_1}}
                       {\p x^{i'_2}\over \p x^{i_2}}\dots
                       {\p x^{i'_p}\over \p x^{i_p}}
                     T^{i_1i_2\dots i_p}(x)\,.
                         \end{equation}

\medskip

One can consider {\it covariant} tensors  of the rank $q$
                   $$
             S=  S_{j_1j_2\dots j_q}dx^{j_1}\otimes
              dx^{j_2}\otimes \dots dx^{j_q}
                   $$
with components $\{S_{j_1j_2\dots j_q}\}$.
Under changing of coordinates 
$(x^1,\dots,x^n)\to (x^{1'},\dots,x^{n'})$ 
they transform as follows:
            \begin{equation*}\label{transformofcotensors}
S_{j'_1j'_2\dots j'_q}(x')={\p x^{i_1}\over \p x^{i'_1}}
                       {\p x^{i_2}\over \p x^{i'_2}}\dots
                       {\p x^{i_p}\over \p x^{i'_p}}
                     S_{j_1j_2\dots j_q}(x)\,.
                         \end{equation*}


E.g. if $S_{ik}$ is a covariant tensor
of rank $2$ then
            \begin{equation}\label{ruleoftransformfor2tensor}
                S_{i'k'}(x')=
   {\p x^i(x')\over \p x^{i'}}
   {\p x^k(x')\over \p x^{k'}}
             S_{ik}(x)\,.
            \end{equation}
 If $A^i_k$ is a tensor of rank $\begin{pmatrix} 1\cr 1\cr\end{pmatrix}$ 
(linear operator on $T_\pt M$)
 then

             $$
    A^{i'}_{k'}(x')=        
{\p x^{i'}(x)\over \p x^{i}}{\p x^k(x')\over \p x^{k'}}A^i_k(x)\,.
             $$





{\bf Remark} {\it Einstein  summation rules}


 In our lectures we always use so called {\it Einstein summation convention}.
 it  implies that when an index occurs twice
in the same expression in upper and in lower postitions, then
 the expression is implicitly summed over all possible values
for that index.
  Sometimes it is called dummy indices summation rule.
 

   Using Einstein summation rules we avoid to write
bulky expressions. 
Later we will see that these notations are really very effective.
E.g. equation \eqref{transformofvector}
in `standard' notations will appear as
         $$
  \hbox {for every $i'=1,\dots,n$}\,\,
A^{i'}(x')=\sum_{i=1}^n{\p x^{i'}\over \p x^i}A^i(x)\,.
         $$ 

\subsection {Riemannian manifold}

    \subsubsection {Riemannian manifold---
manifold equipped with Riemannian metric}

{\bf Definition} The Riemannian manifold $(M,G)$ is a
manifold equipped with a Riemannian metric.



  The Riemannian metric $G$ on the manifold $M$ defines the
  length of the tangent vectors and the length of the curves.

{\bf Definition}
  Riemannian metric $G$ on n-dimensional manifold $M^n$
  defines for every point $\pt\in M$ the scalar product
  of tangent vectors in the tangent space $T_\pt M$
  smoothly depending on the point $\pt $.

  It means that in every coordinate system $(x^1,\dots,x^n)$
  a metric $G=g_{ik}dx^idx^k$ is defined by a matrix valued smooth function $g_{ik}(x)$ ($i=1,\dots,n;k=1,\dots n$)
  such that for any two vectors
       $$
  {\bf A}=A^i(x){\p\over \p x^i},\,\, {\bf B}=B^i(x){\p\over \p x^i},
      $$
tangent to the manifold $M$ at the point $\pt$ with coordinates $x=(x^1,x^2,\dots,x^n)$ ($\A,{\bf B}\in T_{\pt}M$)
the scalar product is equal to:
              $$
              \langle\A,{\bf B}\rangle_G\big\vert_\pt= G({\bf A},{\bf B})\big\vert_\pt=
A^i(x)g_{ik}(x)B^k(x)=
            $$
            \begin{equation}\label{scalarproduct}
  \begin{pmatrix}
   A^1 \dots A^n\\
   \end{pmatrix}
  \begin{pmatrix}
     g_{11}(x)&\dots &g_{1n}(x)\\
      \dots &\dots & \dots \\
         g_{n1}(x)&\dots &g_{nn}(x)\\  \\
   \end{pmatrix}
\begin{pmatrix}
   B^1\\
     \cdot \\
   \cdot\\
   \cdot\\
   B^n\\
   \end{pmatrix}
\end{equation}

where
\begin{itemize}

  \item  $G({\bf A},{\bf B})=G({\bf B},{\bf A})$, i.e.  $g_{ik}(x)=g_{ki}(x)$ (symmetricity condition)

    \item
       $G({\bf A},{\bf A})>0$ if $\bf A\not=0$, i.e.

    $g_{ik}(x)u^iu^k\geq 0$, $g_{ik}(x)u^iu^k=0$ iff $u^1=\dots=u^n=0$  (positive-definiteness)

   \item  $G({\bf A},{\bf B})\big\vert_{\pt=x}$, i.e. $g_{ik}(x)$ are smooth functions.


\end{itemize}


 The matrix $||g_{ik}||$ of components of 
the metric $G$ we also sometimes denote by $G$.

%\end{document} %6 February 2018
%\end{document} %31 January 2015


%\end{document} %6 February 2019



Now we establish rule of 
transformation for entries of matrix $g_{ik}(x)$, of metric $G$.


  Notice that an arbitrary matrix entry $g_{ik}$ is nothing but scalar product
of vectors $\p_i,\p_k$ at the given point:  
   \begin{equation}\label{importantremark2}
 g_{ik}(x)=\left\langle{{\p \over \p x^i},{\p\over \p x^k}}\right\rangle\,,
\quad \hbox{in coordinates $(x^1,\dots,x^n)$}
     \end{equation}
Use this formula for establishing rule of transformations of $g_{ik}(x)$.
  In the new coordinates $x^{i'}=(x^{1'},\dots,x^{n'})$
according this formula we have that
   \begin{equation*}\label{innewcoordinates}
 g_{i'k'}(x')=\left\langle{{\p \over \p x^{i'}},{\p\over \p x^{k'}}}
                \right\rangle\,,
\quad \hbox{in coordinates $(x^1,\dots,x^n)$}\,.
      \end{equation*}
Now using chain rule, linearity of scalar product and formula
 \eqref{importantremark2} we see that
               $$ 
g_{i'k'}(x')=\left\langle
        {{\p \over \p x^{i'}},{\p\over \p x^{k'}}}
          \right\rangle
 =\left\langle
     {{\p x^i\over \p x^{i'}}{\p \over \p x^{i}},
     {\p x^k\over\p x^{k'}}{\p\over \p x^{k}}}
   \right\rangle
         $$
       \begin{equation}\label{transformofmetric}
%\label{newcoordinates2}
 ={\p x^i\over \p x^{i'}}
        \underbrace
             {
          \left\langle
     {{\p \over \p x^{i}},
     {\p\over \p x^{k}}}
   \right\rangle
           }_{g_{ik}(x)}
          {\p x^k\over\p x^{k'}}=
   {\p x^i\over \p x^{i'}}
          g_{ik}(x)          
{\p x^k\over\p x^{k'}}
         \end{equation}
This transformation law implies  that $g_{ik}$ entries
of matrix $||g_{ik}||$ are components of {\it covariant
tensor field
 $G=g_{ik}dx^idx^k$ of rank 2}(see equation 
\eqref{ruleoftransformfor2tensor}).

{\it One can say that Riemannian metric is defined by  
symmetric covariant smooth tensor field $G$ of the rank 2
which defines scalar product in the tangent spaces $T_{\pt}M$ 
  smoothly depending on the point $\pt$.
 Components of tensor field  $G$ in coordinate system are 
functions $g_{ik}(x)$}:
\begin{equation*}\label{symb}
    G=g_{ik}(x)dx^i\otimes dx^k\,,
\end{equation*}
%for arbitrary vector fields  ${\bf A},{\bf B}$
             \begin{equation}\label{geommeaning}
 \langle {\bf A},{\bf B}\rangle=G({\bf A},{\bf B})=
  g_{ik}(x)dx^i\otimes dx^k
             \left(
  {\bf A},{\bf B}
          \right)\,.
                 \end{equation}
In practice it is more convenient to perform transformation
of metric $G$ under changing of coordinates in the following way:
       \begin{equation*}
        G=g_{ik}dx^i\otimes dx^k=g_{ik}
        \left({\p x^i\over \p x^{i'}}dx^{i'}\right)
        \otimes
        \left({\p x^k\over \p x^{k'}}dx^{k'}\right)=
    \end{equation*}
   \begin{equation}\label{transformationlaw}
         {\p x^i\over \p x^{i'}}g_{ik}{\p x^k\over \p x^{k'}}
        dx^{i'}
        \otimes dx^{k'}=
        g_{i'k'}
        dx^{i'}
        \otimes dx^{k'}\,,\,{\rm hence}\,\,
g_{i'k'}={\p x^i\over \p x^{i'}}g_{ik}{\p x^k\over \p x^{k'}}\,.
            \end{equation}
%   Hence
%    \begin{equation}\label{transformationlaw}
%     g_{i'k'}={\p x^i\over \p x^{i'}}g_{ik}{\p x^k\over \p x^{k'}}\,.
%       \end{equation}
We come to transfomation rule \eqref{transformofmetric}.




Later by some abuse of notations we sometimes omit the sign of tensor product
and write a metric just as
\begin{equation*}\label{symb1}
    G=g_{ik}(x)dx^idx^k\,.
\end{equation*}


                               
\subsubsection{Examples}


   \begin{itemize}
\item


   $\R^n$ with canonical coordinates $\{x^i\}$ and with metric
               $$
           G=(dx^1)^2+(dx^2)^2+\dots+(dx^n)^2
               $$
           $G=||g_{ik}||=\diag [1,1,\dots,1]$

Recall that this is a basis example of $n$-dimensional 
Euclidean space $\E^n$, where scalar product
is defined by the formula:
              $$
     G(\X,\Y)=\la\X,\Y\ra=g_{ik}X^iY^k=X^1Y^1+X^2Y^2+\dots+X^nY^n\,.
              $$
In the general case if $G=||g_{ik}||$ is an
arbitrary symmetric positive-definite metric then
                $
            G(\X,\Y)=\la\X,\Y\ra=g_{ik}X^iY^k$.
One can show that there exists a new basis $\{\e_i\}$ such that in this basis
              $
              G(\e_i,\e_k)=\delta_{ik}$.
This basis is called orthonormal basis. (See the Lecture notes in Geometry)



Scalar product in vector space defines the {\it same}
scalar product at all the points. In general case
for Riemannian manifold scalar product depends on a point.
In Riemannian manifold we consider arbitrary transformations from
local coordinates to new local coordinates.


\item  Euclidean space $\E^2$ with polar coordinates in the domain $y>0$
($x=r\cos\varphi, y=r\sin\varphi$):

   $dx=\cos\varphi dr-r\sin\varphi d\varphi, dy=\sin\varphi dr+r\cos\varphi d\varphi$.
  In new coordinates the Riemannian metric $G=dx^2+dy^2$ will
have the following appearance:
             $$
        G=(dx)^2+(dy)^2=(\cos\varphi dr-r\sin\varphi d\varphi)^2+(\sin\varphi dr+r\cos\varphi d\varphi)^2=
         dr^2+r^2(d\varphi)^2
            $$
   We see that for matrix    $G=||g_{ik}||$
             \begin{equation*}
                 \underbrace{G=
           \begin{pmatrix}        g_{xx} &g_{xy}\cr  g_{yx} &g_{yy} \end{pmatrix}=
      \begin{pmatrix}    1 &0\cr  0 &1\cr
\end{pmatrix}}_{\hbox{in Cartesian coordinates}},\qquad
                 \underbrace
                 {G=
      \begin{pmatrix}g_{rr} &g_{r\varphi}\cr g_{\varphi r}
&g_{\varphi\varphi}\end{pmatrix}=
      \begin{pmatrix}1 &0\cr 0 &r^2\end{pmatrix}}_ {\hbox{in polar coordinates}}
             \end{equation*}



    \item   Circle

     Interval $[0,2\pi)$ in the line $0\leq x< 2\pi$ with Riemannian  metric
           \begin{equation}\label{circle1}
          G=  a^2dx^2
           \end{equation}
Renaming $x\mapsto \varphi $ we come to habitual formula for
metric
for circle of the radius $a$: $x^2+y^2=a^2$ embedded in
the Euclidean space $\E^2$:
           \begin{equation}\label{circle2}
          G= a^2d\varphi^2\qquad
          \begin{cases}
          x=a\cos\varphi\cr
          y=a\sin\varphi
          \end {cases},
          0< \varphi <2\pi,
\quad {\rm or}\,\, -\pi<\varphi<\pi\,.
           \end{equation}
  Rewrite this metric in stereographic coordinate
   $t$:
                \begin{equation}\label{circlestereograph} 
             G=a^2d\varphi^2=4a^4dt^2(a^2+t^2)^2\,,
\quad {\rm where}\,\,\,
       t={ax\over a-y}=
      {a^2\cos\varphi\over a-a\sin\varphi}=
\tan\left({\pi\over 4}+{\varphi\over 2}\right)\,.
                  \end{equation}
  (See \eqref{stereogrcoordoncircle} and Homeworks 0 and 2.)

%\item Domain in $\R^2$ with metric $G=du^2+u^2dv^2$
% (Compare with $\E^2$ with polar coordinates).


    \item Cylinder surface

    Consider domain in $\R^2$,
 $D=\{(x,y)\colon,\,\, 0\leq x< 2\pi$ with Riemannian  metric
           \begin{equation}\label{domainascylinder1}
          G=  a^2dx^2+dy^2
           \end{equation}
We see that renaming variables $x\mapsto \varphi $, $y\mapsto h$ we come to habitual, familiar formulae for
metric in standard polar coordinates
for cylinder surface of the radius $a$ embedded in the Euclidean space $\E^3$:
           \begin{equation}\label{domainascylinder2}
          G= a^2d\varphi^2+dh^2\qquad
          \begin{cases}
          x=a\cos\varphi\cr
          y=a\sin\varphi\cr
          z=h\cr
          \end {cases},
          0< \varphi <2\pi, -\infty<h<\infty
           \end{equation}
(Coordinate  $\varphi$ is well defined for $-\pi<\varphi<\pi$
also.)

 \item  Sphere

   Consider domain in $\R^2$,  $0<x<2\pi$, $0<y<\pi$ 
with metric $G=dy^2+\sin^2 y dx^2$
We see that renaming variables $x\mapsto \varphi $, $y\mapsto h$ we come to habitual, familiar formulae for
metric in standard spherical coordinates
for sphere $x^2+y^2+z^2=a^2$ of the radius $a$ embedded in the Euclidean space $\E^3$:
           \begin{equation}\label{domainascylinder2}
          G= a^2d\theta^2+a^2\sin^2\theta
          d\varphi^2\qquad
          \begin{cases}
          x=a\sin\theta\cos\varphi\cr
          y=a\sin\theta\sin\varphi\cr
          z=a\cos\theta\cr
          \end {cases},
      \quad 0<\theta<\pi\,,\,
          0< \varphi <2\pi\,.
           \end{equation}

\end{itemize}
(See examples also in the Homeworks.)


{\footnotesize If we omit the
condition of positive-definiteness for Riemannian metric we come to so
called  {\it Pseudoriemannian metric}.
Manifold equipped with pseudoriemannian metric is called
pseudoriemannian manifold.  Pseudoriemannian manifolds
appear in applications in the special and general
relativity theory.

In pseudoriemanninan space scalar product $(\X,\X)$ may take an arbitrary real
values: it can be positive, negative, it can be equal to zero. Vectors
$\X$ such that $(\X,\X)=0$ are called null-vectors.


For example consider $4$-dimensional
linear space $\R^4$ with pseudometric
           $$
    G=(dx^0)^2-(dx^1)^2-(dx^2)^2-(dx^3)^2\,.
            $$
 For an arbitrary vector $\X=(a^0,a^1,a^2,a^3)$
scalar product $(\X,\X)$ is positive if
$(a^0)^2>(a_1)^2+(a_2)^2+(a_3)^2$, and it is negative if
$(a^0)^2<(a_1)^2+(a_2)^2+(a_3)^2$, and
$\X$ is null-vector if  $(a^0)^2=(a_1)^2+(a_2)^2+(a_3)^2$.
It is so called Minkovski space.
The coordinate $x^0$ plays a
role of the time: $x^0=ct$, where $c$ is
the value of the speed of the light. 
Vectors $\X$ such that $(\X,\X)>0$
are called time-like vectors and they called space-like vectors
if $(\X,\X)<0$.} 




%10 February  2016 

%\end{document} % 8 February



\subsubsection {Scalar product $\to$
Length of tangent vectors
and angle between them }\label{lengthandangle}

The Riemannian metric defines scalar product of tangent vectors attached at the given point.
Hence it defines the length of tangent vectors and angle between them.
  If $\X=X^m{\p \over \p x^m}, \Y=Y^m{\p \over \p x^m}$ are two tangent vectors at the given point
  $\pt$ of Riemannian manifold with coordinates $x^1,\dots,x^n$, 
  then we have that lengths of
  these vectors equal to
   \begin{equation}\label{lengthandangle1}
    |\X|=\sqrt {\langle \X,\X\rangle}=\sqrt {g_{ik}(x)X^iX^k},\quad
    |\Y|=\sqrt {\langle \Y,\Y\rangle}=\sqrt {g_{ik}(x)Y^iY^k},\quad
\end{equation}
and an `angle' $\theta$ between these vectors is defined by the relation
         \begin{equation}\label{lengthandangle1}
    \cos\theta={{\langle \X,\Y\rangle}\over |\X|\cdot|\Y|}={g_{ik}X^iY^k\over \sqrt {g_{ik}(x)X^iX^k}
    \sqrt {g_{ik}(x)Y^iY^k}}
\end{equation}
{\bf Remark} We say `angle' but we calculate just cosinus
    of angle.


%\end{document}   % 9 February 2018

{\bf Example} Let $M$ be $3$-dimensional Riemannian manifold,
and $\pt\in M$ a point in it. Suppose that
the manifold $M$ is equipped with local coordinates
$x,y,z$ in a vicinity of this point, and
the expression of Riemannian
metric in these local  coordinates is 
      \begin{equation}\label{exampleofconfeuclid}
G={dx^2+dy^2+dz^2\over (1+x^2+y^2+z^2)^2}\,.
           \end{equation}
 Consider the vectors $\X=a\p_x+b\p_y+c\p_z$ 
and $\Y=p\p_x+q\p_y+r\p_z$,
attached at the point $\pt$,
with  coordinates
 $x=2,y=2,z=1$.
Find the lengths of vectors $\X$ 
and $\Y$ and find cosinus of the angle between
these vectors.

\smallskip

We see that matrix of Riemannian metric is
           \begin{equation*}
        ||g_{ik(x)}||=
    \begin{pmatrix}
     {1\over (1+x^2+y^2+z^2)^2} & 0 &0\cr
      0 &{1\over (1+x^2+y^2+z^2)^2}  &0\cr
    0& 0& {1\over (1+x^2+y^2+z^2)^2} \cr
    \end{pmatrix}\,{\rm i.\,e.}\,
    g_{ik}(x,y,z)={\delta_{ik}\over (1+x^2+y^2+z^2)^2}\,,
           \end{equation*}
where $g_{ik}(x)$ are entries of matrix:
$G=g_{ik}(x)dx^idx^k$,
($\delta_{ik}$ is Kronecker symbol: $\delta_{ik}=1$ if $i=k$
and it vanishes otherwise).


According to formulae above
    $$
    |\X|=\sqrt {\langle \X,\X\rangle}=
  \sqrt {g_{ik}(x,y,z)X^iX^k}\big\vert_{\pt}=
  \sqrt {{\delta_{ik}X^iX^k\over (1+x^2+y^2+z^2)^2}}
\big\vert_{x=2,y=2,z=1}=
    $$
       $$
   {\sqrt {a^2+b^2+c^2\over (1+2^2+2^2+1^2)^2}}=
{\sqrt {a^2+b^2+c^2}\over 10}\,,
       $$
    $$
    |\Y|=\sqrt {\langle \Y,\Y\rangle}=
  \sqrt {g_{ik}(x,y,z)Y^iY^k}\big\vert_{\pt}=
  \sqrt {{\delta_{ik}Y^iY^k\over (1+x^2+y^2+z^2)^2}}
\big\vert_{x=2,y=2,z=1}=
    $$
       $$
  \sqrt {{p^2+q^2+r^2\over (1+2^2+2^2+1^2)^2}}=
{\sqrt {p^2+q^2+r^2}\over 10}\,,
       $$
and
          $$
\cos\theta={\langle\X,\Y\rangle\over |\X| |\Y|}
        =
  {g_{ik}(x,y,z)X^iY^k\big\vert_{\pt}\over
\sqrt{g_{pq}(x,y,z)X^pX^q}\sqrt{g_{rs}(x,y,z)Y^rY^s} }
                =
                {
    {\delta_{ik}X^iY^k\over (1+x^2+y^2+z^2)^2}
           \over
      |\X||\Y|
              }
             $$
              $$
   ={{ap+bq+cr\over (1+2^2+2^2+1)^2}\over
 {\sqrt {a^2+b^2+c^2}\over 10}{\sqrt {p^2+q^2+r^2}\over 10}}=     
   ={ap+bq+cr
          \over
 \sqrt {a^2+b^2+c^2}\sqrt {p^2+q^2+r^2}} \,.    
          $$
 This example is related with the notion of so called 
{\it conformally euclidean metric} 
(see the next paragraph, \ref{conformally}).


\subsubsection {Conformally Euclidean metric}\label{conformally}


   Let $(M,G)$ be a Riemannian manifold.

\smallskip

  {\bf Definition}  We say that metric 
$G$ is locally conformally Euclidean 
in a vicinity of the point $\pt$
if in a vicinity of this point
there exist local coordinates $\{x^i\}$
such that in these coordinates metric has an appearance
                \begin{equation}\label{conformaleuclideandef}
                  G=\s(x)\delta_{ik}dx^idx^k=
\s(x)\left((dx^1)^2+\dots+(dx^n)^2\right)\,,
                \end{equation}
i.e. it is proportional to `Euclidean metric'.
 We call coordinates  $\{x^i\}$ {\it conformall} coordinates
or {\it isothermic} coordinates  if condition  
                \eqref{conformaleuclideandef}  holds.  


 
We say that metric is conformally Euclidean if
 it is locally conformally 
Euclidean in the vicinity of every point.
  We say that Riemannian manifold $(M.G)$ is conformally
Euclidean if the metric $G$ on it is conformally Euclidean  

\m

%  It is evident that coefficient $\s(x)$ in 
%\eqref{conformaleuclideandef} has to be positive.
%  It is convenient sometimes to denote it as $\s(x)=e^{\l(x)}$,
%                \begin{equation}\label{conformaleuclideandef1}
%                  G=\s(x)\delta_{ik}
%dx^idx^k=e^{\l(x)}\delta_{ik}dx^idx^k\,,
%                \end{equation}
 
 One can see that  if metric is
 conformally Euclidean in a vicinity of some point $\pt$,
then
 the angle between vectors, more precisely the 
cosinus of the angle between vectors  attached at this point
(see equation \eqref{lengthandangle1})
 is the same as for Eucldean metric.
Indeed, let $G$ be conformally Euclidean metric and let
 $x^i$ be local coordinates such that the metric has
an appearance \eqref{conformaleuclideandef} in these coordinates.
Let $\X,\Y$ be two non-vanishing vectors 
 $\X=X^m(x){\p \over \p x^m}, \Y=Y^m(x){\p\over \p x^m}$
  ($|\X|\not=0,|\Y|\not=0$) attached at a same point.
  Then
                $$
 \cos\theta=
{{\langle \X,\Y\rangle}\over |\X|\cdot|\Y|}=
{g_{ik}X^iY^k\over \sqrt {g_{ik}(x)X^iX^k}
    \sqrt {g_{ik}(x)Y^iY^k}}=
                         $$
         \begin{equation}\label{scalarproductforconformal}
{\sigma(x)\delta_{ik}X^iY^k\over \sqrt {\sigma(x)\delta_{ik}(x)X^iX^k}
    \sqrt {\sigma(x)\delta_{ik}(x)Y^iY^k}}=
{\sum_kX^kY^k\over \sqrt {\sum_k(x)X^kX^k}
    \sqrt {\sum_kY^kY^k}}\,.
                                  \end{equation}
(Note that coefficient $\sigma$ in equation
\eqref{conformaleuclideandef}  has to be positive.)

{\bf Remark}  One can show that the condition of 
`preserving the angles' is not only necessary condition but
it is also sufficient condition for metric
to be conformally Euclidean (see the problem 1 in Homework 2).

Now   consider examples.

\smallskip
  First It is instructive to recall the example    
considered in previous subsection \ref{lengthandangle}),
where Riemannian metric in a vicinity of a point
had an appearance \eqref{exampleofconfeuclid}
This is example of Riemannian manifold which is
locally conformally Euclidean in a vicinity of a point
$\pt$. 

\smallskip

   Another 

{\bf Example}    Consider the surface of cylinder with the metric
                 \begin{equation}
           G=a^2d\varphi^2+dh^2
                  \end{equation}
(see equation \eqref{domainascylinder2} ).  
In a vicinity of every point one
can conisder coordinates $\begin{cases}u=a\varphi\cr v=h\cr\end{cases}$.
It is evident that in these coordinates 
$G=du^2+dv^2$, i.e. this Riemannian
manifold is conformally Euclidean. 

  {\bf Remark}. In fact we proved more: for metric of cylinder in
coordinates $u,v$, the coefficient $\sigma(x)\equiv 1$, i,e. in these coordinates  metric is  not only {\it locally conformally Euclidean}, 
but also it is  {\it locally Euclidean}.  
We will study this question later
in details. (see  
paragraph "Locally Euclidean Riemannian manifold" later).

Later we consider also another important examples.

It is important, that 
the following Theorem takes  a place:

   {\bf Theorem} {(\tt Gauss)}   {\it Every $2$-dimensional 
Riemannian manifold is locally
  conformally Euclidean, i.e. for arbitrary $2$-dimensional Riemannian 
manifold, in a vicinity of arbitrary point, there exist coordinates
$u',v'$  such that in these coordinates
 Riemannian metric
                \begin{equation}\label{gaussconformally}
     G=A(u,v)du^2+2B(u,v)dudv+C(u,v)dv^2=
         \sigma(u',v')\left({du'}^2\right)
                \end{equation}


}
We will not prove this theorem
footnote{the proof is easy and almost evident for anaylitical manifolds,
and it is hard for smooth manifolds}, but consider many examples
of $2$-dimensional Riemannian manifolds, with suitable
conformall coordinates.


%\end{document}  % 8 February 2019

\subsubsection {Length of  curves}

Let $\gamma\colon\,\, x^i=x^i(t), (i=1,\dots,n))$
 $(a\leq t\leq b)$ be a curve on the Riemannian manifold $(M,G)$.

  At the every point of the curve 
the velocity vector (tangent vector)
  is defined:
\begin{equation*}\label{velvector}
  \v(t)=\begin{pmatrix}
       \dot x^1 (t)\\
             \cdot\\
             \cdot\\
             \cdot\\
             \dot x^n(t)
         \end{pmatrix}
        ={dx^i(t)\over dt}{\p\over \p x^i}
\end{equation*}

{\bf Remark}  Note that $\v(t)$  is a vector; check transformation rules: 
          \begin{equation*}
       {dx^i(t)\over dt}{\p\over \p x^i}=
       {dx^i(t)\over dt}
  {\p x^{i'}\over \p x^i}{\p\over \p x^{i'}}=
{\p x^{i'}\over \p x^i}{dx^i(t)\over dt}
  {\p\over \p x^{i'}}=
       {dx^{i'}(t)\over dt}{\p\over \p x^{i'}}\,.
  \end{equation*}
        


The length of velocity vector $\v\in T_xM$
(vector $\v$ is tangent to the manifold $M$ at the point $x$)
equals to
     \begin{equation*}\label{speedforaunt}
       |\v|_x=\sqrt {\la \v,\v\ra_G\big\vert_{x}}=
       \sqrt{g_{ik}v^iv^k}\big\vert_{x}=
       \sqrt{g_{ik}{dx^i(t)\over dt}
     {dx^k(t)\over dt}}\big\vert_{x}\,.
     \end{equation*}
For an arbitrary curve its length is equal
 to the integral of the length of velocity vector:
\begin{equation}\label{lengthofthecurve}
  L_\gamma=\int_a^b \sqrt {\langle\v,\v\rangle_G\big\vert_{x(t)}}dt=
  \int_a^b \sqrt {g_{ik}(x(t))\dot x^i(t) \dot x^k(t)}dt\,.
\end{equation}

Bearing in mind that metric \eqref{geommeaning} defines the length
we often write metric in the following form
\begin{equation*}\label{metric}
  G=ds^2=g_{ik}dx^idx^k
\end{equation*}

\smallskip


{\bf Example 1}
Consider $2$-dimensional Riemannian manifold with metric
                    $$
                 ||g_{ik}(u,v)||=
                     \begin{pmatrix}
                     g_{11}(u,v) &g_{12}(u,v)\cr
                     g_{21}(u,v) &g_{22}(u,v)\cr
                    \end{pmatrix}\,.
                    $$
        Then
                           $$
 G=ds^2=g_{ik}du^idv^k=g_{11}(u,v)du^2+2g_{12}(u,v)dudv+g_{22}(u,v)dv^2\,.
                         $$
     The length of the curve
$\gamma\colon u=u(t),v=v(t)$, where $t_0\leq t\leq t_1$
                according to \eqref{lengthofthecurve} is equal to
$L_\gamma=\int_{t_0}^{t_1} \sqrt {\langle\v,\v\rangle}=
  \int_{t_0}^{t_1} \sqrt {g_{ik}(x)\dot x^i \dot x^k}=$
  \begin{equation}
\int_{t_0}^{t_1}
\sqrt {{g_{11}}\left(u\left(t\right),v\left(t\right)\right)u_t^2
+2{g_{12}}\left(u\left(t\right),v\left(t\right)\right)u_tv_t+
{g_{22}}\left(u\left(t\right),v\left(t\right)\right)v_t^2}dt\,.
\end{equation}

\smallskip


{\bf Example}   Consider  Lobachevsky
 (hyperbolic)  plane.  We consider upper-half model of Lobachevsky 
(hyperbolic) plane:  
   $$
G={dx^2+dy^2\over y^2}\,,\quad (y>0)
   $$  
Consider in Lobachevsky plane the curve
     $
C\colon\quad \begin{cases}x=x_0\cr y=t\end{cases}\,, a<t<b
     $
and claculate its length:
      $$
      L_C=\int_{a}^{b} \sqrt {\langle\v,\v\rangle}=
  \int_{a}^{b} \sqrt {g_{ik}(x)\dot x^i \dot x^k}=
        $$
       $$
\int_{a}^{b}
\sqrt {{g_{11}}\left(x\left(t\right),y\left(t\right)\right)x_t^2
+2{g_{12}}\left(x\left(t\right),y\left(t\right)\right)x_ty_t+
{g_{22}}\left(x\left(t\right),y\left(t\right)\right)y_t^2}dt=
      $$
       $$
\int_{a}^{b}
\sqrt {{1\over y^2}\left(x_t^2
+y_t^2\right)}dt=
\int_{a}^{b}
\sqrt {{1\over t^2}\left(0
+1\right)}dt=\int_a^b{dt\over t}=
  \left|\log {a\over b}\right|\,.
      $$

\m

{\footnotesize
The length of  curves defined by the
formula\eqref{lengthofthecurve} obeys the following natural conditions

\begin{itemize}

\item It coincides with the usual length in the Euclidean space $\E^n$
($\R^n$ with standard metric  $G=(dx^1)^2+\dots+(dx^n)^2$
in Cartesian coordinates). E.g. for $3$-dimensional Euclidean space
\begin{equation*}\label{coincidewithlength}
      L_\gamma=
  \int_a^b \sqrt {g_{ik}(x(t))\dot x^i(t) \dot x^k(t)}dt=
  \int_a^b \sqrt{(\dot x^1(t))^2+(\dot x^2(t))^2+(\dot x^3(t))^2}dt
\end{equation*}


\item It does not depend on parameterisation of the curve

\begin{equation*}\label{independenceon parameterisation}
    L_\gamma=  \int_a^b \sqrt {g_{ik}(x(t))\dot x^i(t) \dot x^k(t)}dt=
        \int_{a'}^{b'} \sqrt {g_{ik}(x(\tau))\dot x^i(\tau) \dot x^k(\tau)}
                 d\tau\,,
\end{equation*}
($x^i(\tau)=x^i(t(\tau))$,
$a'\leq \tau\leq b'$ while $a\leq t\leq b$) since under
changing of parameterisation
  $$
  \dot x^i(\tau)={dx(t(\tau))\over d\tau}=
  {dx(t(\tau))\over d t}{dt\over d\tau}=
  \dot x^i(t){dt\over d\tau}\,.
  $$


\item It {\it does not depend on coordinates on Riemannian manifold $M$}
 \begin{equation*}\label{independence on coordinates}
    L_\gamma=  \int_a^b \sqrt {g_{ik}(x(t))\dot x^i(t) \dot x^k(t)}dt=
    \int_a^b \sqrt {g_{i'k'}(x'(t))\dot x^{i'}(t) \dot x^{k'}(t)}dt\,.
 \end{equation*}
This immediately follows from transformation rule
\eqref{transformationlaw}  for Riemannian metric:
       $$
g_{i'k'}\dot x^{i'}(t) \dot x^{k'}(t)=
   g_{ik}
 \left({\p x^i\over \p x^{i'}(t)}\dot x^{i'}(t)\right)
 \left({\p x^k\over \p x^{k'}}\dot x^{k'}(t)\right)
    g_{ik}\,\dot x^{i}(t) \dot x^{k}(t)\,.
        $$
\item  It is additive: length of the sum of two curves is equal to the sum of
their lengths.
If a curve  $\gamma=\gamma_1+\gamma$, i.e.
$\gamma\colon x^i(t), a\leq t\leq b$,
$\gamma_1\colon x^i(t), a\leq t\leq c$ and
$\gamma_2\colon x^i(t), c\leq t\leq b$ where
a point $c$ belongs to the interval $(a,b)$ then
$L_{\gamma}=L_{\gamma^1}+L_{\gamma^2}$.

\end{itemize}

}
{\footnotesize One can show that formula \eqref{lengthofthecurve}
for length is defined
uniquely by these conditions.
More precisely one can show under some technical conditions
one may show that any local additive functional on curves which does not
depend on coordinates and parameterisation, and depends on
derivatives of curves of order $\leq 1$  is equal
to \eqref{lengthofthecurve} up to a constant multiplier. To feel the taste of this statement
you may do the following exercise:

 {\bf Exercise} Let
$A=A\left(x(t), y(t),
 {dx(t)\over dt},{dy(t)\over dt}\right)$ be a function such that
an integral $L=\int A\left(x(t), y(t),
  {dx(t)\over dt},{dy(t)\over dt}\right)dt$ over an arbitrary curve
$\gamma$
in $\E^2$ does not change
under reparameterisation of this curve and under an arbitrary isometry,
i.e. translation and rotation of the curve.
Then one can easy show (show it!) that
        $$
   A\left(x(t), y(t),
 {dx(t)\over dt},{dy(t)\over dt}\right)=
  c\sqrt {\left({dx(t)\over dt}\right)^2+
   \left({dy(t)\over dt}\right)^2}\,,
    $$
where $c$ is a constant, i.e. it is a usual length up to a multiplier
}


%\end{document}  % 6 February 2015

\subsection{Riemannian structure on the surfaces embedded in Euclidean space}


Let $M$ be a surface embedded in Euclidean space. Let $G$ be Riemannian structure on the manifold $M$.

  Let $\X, \Y$ be two vectors tangent to the surface
$M$ at a point $\pt\in M$. An External Observer calculate this scalar product viewing
these two vectors as vectors in $\E^3$ attached at the point $\pt\in \E^3$
using scalar product in   $\E^3$.  An Internal Observer will calculate the scalar product
viewing these two vectors as vectors  tangent to the surface $M$
using the Riemannian
metric $G$ (see the formula \eqref{scalarproduct}).  Respectively


If $L$ is a curve in $M$ then an External Observer consider this curve as a curve in $\E^3$,
calculate the modulus of velocity  vector (speed) and the length of the curve using Euclidean scalar
product of ambient space. An Internal Observer ("an ant") will define the modulus of the velocity vector and
the length of the curve using Riemannian metric.


 {\bf Definition}  Let $M$ be a surface embedded in the Euclidean space. 
We say that metric $G_M$ on the surface is
 induced by the Euclidean metric
if the scalar product of arbitrary two vectors ${\bf A,B}\in T_\pt M$ calculated in terms of the metric $G$
equals to Euclidean scalar product of these two vectors:
          \begin{equation}\label{inducedmetric1}
      \langle{\bf A,B}\rangle_{G_M}=\langle{\bf A,B}\rangle_{G_{\rm Euclidean}}
          \end{equation}
In other words  we say that Riemannian metric on the embedded surface is induced by the Euclidean structure of the ambient space
    if External and Internal Observers come to the same results calculating scalar product of vectors tangent to the surface.

    In this case modulus of velocity vector (speed) and the length of the curve is the same for
    External and Internal Observer.



\subsubsection {Internal and external observers}



\centerline {\it Tangent vectors, coordinate 
tangent vectors}

Here we recall basic notions from the course of 
Geometry which we will need here.

%\end{document}  %14 February 2019

    Let $\r=\r(u,v)$ be parameterisation 
of the surface $M$ embedded in the Euclidean space:
                           \begin{equation*}
                            \r(u,v)=\begin{pmatrix}x(u,v)\cr
                              y(u,v)\cr
                              z(u,v)\cr
                              \end{pmatrix}
                           \end{equation*}
Here as always $x,y,z$ are Cartesian coordinates in $\E^3$.



Let $\pt$ be an arbitrary point on the surface $M$.
Consider the plane formed by the vectors 
which are adjusted to the point $\pt$
and tangent to the surface $M$. We call this plane
{\it plane tangent to $M$ at the point $\pt$ } 
and denote it by $T_\pt M$.

For a point  $\pt\in M$ one can consider a basis
in the tangent plane $T_pM$ adjusted to the parameters $u,v$.

       Tangent basis vectors at any point $(u,v)$
   are
              \begin{equation*}
                \r_u={\p \r(u,v)\over \p u}=\begin{pmatrix}
                              {\p x(u,v)\over \p u}\cr
                              {\p y(u,v)\over \p u}\cr
                              {\p z(u,v)\over \p u}\cr
                              \end{pmatrix}=
                              {\p x(u,v)\over \p u}{\p\over \p x}+
                              {\p y(u,v)\over \p u}{\p\over \p y}+
                              {\p z(u,v)\over \p u}{\p\over \p z}
              \end{equation*}
and            \begin{equation*}
                \r_v={\p \r(u,v)\over \p v}=\begin{pmatrix}
                              {\p x(u,v)\over \p v}\cr
                              {\p y(u,v)\over \p v}\cr
                              {\p z(u,v)\over \p v}\cr
                              \end{pmatrix}=
                              {\p x(u,v)\over \p v}{\p\over \p x}+
                              {\p y(u,v)\over \p v}{\p\over \p y}+
                              {\p z(u,v)\over \p v}{\p\over \p z}
              \end{equation*}

{\bf Definition}  We call basis vectors $\r_u,\r_v$
adjusted to parameters (coordinates) $u,v$ 
{\it coordinate basis vectors}



 Every vector $\X\in T_pM$
can be expanded over the basis of coordinate basis vectors:
\begin{equation*}\label{expansion}
  \X=X_u\r_u+X_v\r_v,
\end{equation*}
where $X_u, X_v$ are coefficients, components of the vector $\X$.



Internal Observer views  the basis 
vector $\r_u\in T_pM$ as the vector
$\p_u$.  Why?
The vector $\r_u$ attached at the point $\pt$
is a  velocity vector
for the curve $\gamma_{\r_u} (t)\colon\, \begin{cases}
u=u_0+t\cr v=v_0\cr \end{cases}$ starting at the point $\pt$ 
($(u_0,v_0)$  are coordinates of the point $\pt$).
If $f=f(u,v)$ is a function on the surface $M$, then
one can see that directional derivative of this function
along a vector $\r_u$ is defined by ${\p\over \p u}$:
                \begin{equation*}
\p_uf(u,v)\big\vert_\pt=
{d\over dt} f\left(\gamma_{\r_u}(t)\right)=
{d\over dt} f\left(u_0+t,v_0\right)\,.
                \end{equation*}
Respectively the basis vector $\r_v\in T_pM$ 
for an Internal Observer, is velocity vector
for the curve $\gamma_{\r_v} (t)\colon\, \begin{cases}
u=u_0\cr v=v_0+t\cr \end{cases}$ starting at the point $\pt$   
and Internal Observer denotes this vector $\p_v$: 
  \begin{equation*}
\p_vf(u,v)\big\vert_\pt=
{d\over dt} f\left(\gamma_{\r_v}(t)\right)=
{d\over dt} f\left(u_0,v_0+t\right)\,.
                \end{equation*}


For an  arbitrary vector $\X$ which is tangent to surface $M$ at the point
$\pt$, ($\X\in T_pM$) 
               \begin{equation*}
                \begin {matrix}
  \hbox{\tt External observer}    & \hbox{\tt Internal observer}\cr
   \X= a\r_u+b\r_v             &          \X=a\p_u+b\p_v\cr
              \end{matrix}
                  \end{equation*}


{\bf Example}  Consider sphere of radius $R$ in
$\E^3$, $x^2+y^2+z^2=R^2$. In spherical coordinates
          \begin{equation*}
      \E^3 \ni\r=\r(\theta,\varphi)\,\,
         \begin{cases}
          x=R\sin\theta\cos\varphi\cr
          y=R\sin\theta\sin\varphi\cr
          x=R\cos\theta\cr
         \end{cases}\,,
          \end{equation*}
these coordinates are well-defined for
$0<\theta<{\pi\over 2}$ and $0<\varphi<2\pi$\,.
For coordinate basis  vectors
$\r_\theta$ and $\r_\varphi$ we have:
          $$
   \r_\theta=
    {\p \r(\theta,\varphi)\over \p \theta}=
       {\p\over \p \theta}
           \begin{pmatrix}
              x(\theta,\varphi)\cr
              y(\theta,\varphi)\cr
              z(\theta,\varphi)\cr
           \end{pmatrix}
             =
       {\p\over \p \theta}
           \begin{pmatrix}
              R\sin\theta\cos\varphi\cr
              R\sin\theta\sin\varphi\cr
              R\cos\theta\cr
           \end{pmatrix}=
                 $$
\begin{equation*}
            \begin{pmatrix}
              R\cos\theta\cos\varphi\cr
              R\cos\theta\sin\varphi\cr
              -R\sin\theta\cr
           \end{pmatrix}=
     R\cos\theta\cos\varphi {\p\over \p x}
           +
     R\cos\theta\sin\varphi {\p\over \p y}
       -R\sin\theta{\p\over \p z}\,,
  \end{equation*}
and respectively        $$
   \r_\varphi=
    {\p \r(\theta,\varphi)\over \p \theta}=
       {\p\over \p \varphi}
           \begin{pmatrix}
              x(\theta,\varphi)\cr
              y(\theta,\varphi)\cr
              z(\theta,\varphi)\cr
           \end{pmatrix}
             =
       {\p\over \p \varphi}
           \begin{pmatrix}
              R\sin\theta\cos\varphi\cr
              R\sin\theta\sin\varphi\cr
              R\cos\theta\cr
           \end{pmatrix}=
                 $$
\begin{equation}\label{coordinatebasisvectorsforsphere}
            \begin{pmatrix}
              -R\sin\theta\sin\varphi\cr
              R\sin\theta\cos\varphi\cr
                    0\cr
           \end{pmatrix}=
     -R\sin\theta\sin\varphi {\p\over \p x}
           +
     R\sin\theta\cos\varphi {\p\over \p y}\,.
  \end{equation}
\smallskip
  Here is a table how observers
look at the objects on sphere:
 \begin{equation*}
  \begin{matrix}
& \hbox{\tt INTERNAL OBSERVER} &
\hbox{\tt EXTERNAL OBSERVER} &\cr
\hbox {point on $S^2$} 
&2\,\,\hbox {coordinates $\theta,\varphi$} 
&3\,\,\hbox {coordinates $\r=\r(\theta,\varphi)$}&\cr
\cr
 \hbox {curve on $S^2$} 
&\theta(t),\varphi(t) 
&\r(t)=\r(\theta(t,\varphi(t)))&\cr
\cr
\hbox {\footnotesize coordinate tangent vectors to $S^2$}
        & 
{\p\over \p \theta}\,,
{\p\over \p \varphi}
&
\r_\theta\,,\quad
   \r_\varphi
\cr
\cr 
\hbox {tangent vector to $S^2$}
        & 
a{\p\over \p \theta}+
b{\p\over \p \varphi}
&
A{\p\over \p x}+
B{\p\over \p y}+
C{\p\over \p z}=
  a\r_\theta+b\r_\varphi
\cr 
    \end{matrix}
 \end{equation*}

\bigskip

  \centerline{\it 
Explicit formulae for induced Riemannian metric 
(First Quadratic form)}\label
  {inducedriemannianmetric}

Now we are ready to write down the explicit 
formulae for the Riemannian metric on the surface
induced by metric (scalar product) in ambient Euclidean 
space (see the Definition \eqref{inducedmetric1}.
 We will return to induced metric again
in next paragraph\ref{formulaeforembeddingsurfaces}.

     Let $M\colon \r=\r(u,v)$ be a surface embedded in $\E^3$.

The formula \eqref{inducedmetric1} means that 
scalar products of basic vectors  $\r_u=\p_u, \r_v=\p_v$
has to be  the same calculated on the surface
or in the ambient space, i.e. 
calculated by Internal  observer,
or by External  observer.
For example scalar product  
$\langle\p_u,\p_v\rangle_M=g_{uv}$ 
calculated by the Internal Observer
is the same as a scalar product  
$\langle\r_u,\r_v\rangle_{\E^3}$ 
calculated by the External Observer,
scalar product  $\langle\p_v,\p_v\rangle_M=g_{uv}$ 
calculated by the Internal Observer
is the same as a scalar product  
$\langle\r_v,\r_v\rangle_{\E^3}$ 
calculated by the External Observer and so on:
               \begin{equation}\label{inducedmetric3}
   G=\begin{pmatrix}g_{uu} &g_{uv}\cr g_{vu} &g_{vv}\cr\end{pmatrix}=
    \begin{pmatrix}\langle\p_u,\p_u\rangle &\langle\p_u,\p_v\rangle\cr
     \langle\p_v,\p_u\rangle &\langle\p_v,\p_v\rangle\cr\end{pmatrix}=
     \begin{pmatrix}\langle\r_u,\r_u\rangle_{\E^3} &\langle\r_u,\r_v\rangle_{\E^3}\cr
     \langle\r_v,\r_u\rangle_{\E^3} &\langle\r_v,\r_v\rangle_{\E^3}\cr\end{pmatrix}
     \end{equation}
where as usual we denote 
by $\langle\,\,,\,\,\rangle_{\E^3}$ the scalar 
product in the ambient Euclidean space.

  {\bf Remark}   It is convenient sometimes to denote
    parameters $(u,v)$ as $(u^1,u^2)$ or $u^\a$ ($\a=1,2$)
   and to write $\r=\r(u^1,u^2)$ or $\r=\r(u^\a)$ ($\a=1,2$)
   instead $\r=\r(u,v)$

In these notations:
\begin{equation*}
    G_M=
\begin{pmatrix}
   g_{11} & g_{12} \\
   g_{12}& g_{22} \\
   \end{pmatrix}=
   \begin{pmatrix}
   \la\r_u,\r_u\ra_{\E^3} & \la\r_u,\r_v\ra_{\E^3} \\
   \la\r_u,\r_v\ra_{\E^3} & \la\r_v,\r_v\ra_{\E^3} \\
   \end{pmatrix},\quad g_{\a\beta}=\la\r_\a,\r_\beta\ra\,,
\end{equation*}
             \begin{equation}\label{firstquadraticform}
          G_M=g_{\a\beta}du^\a du^\beta=g_{11}du^2+2g_{12}dudv+g_{22}dv^2
             \end{equation}

where $(\,,\,)$ is a scalar product in Euclidean space.


The formula \eqref{firstquadraticform} 
is the formula for induced Riemannian
metric on the surface $\r=\r(u,v)$ \footnote {
it is called sometimes First Quadratic Form of this surface.}.

  If ${\X,\Y}$ are two tangent vectors in the 
tangent plane $T_pC$ then $G(\X,\Y)$
  at the point $p$ is equal to scalar product of vectors $\X,\Y$:
\begin{equation}\label{scalarproduct}
(\X,\Y)=(X^1\r_1+X^2\r_2, Y^1\r_1+Y^2\r_2)=
\end{equation}
                $$
X^1 (\r_1,\r_1)Y^1+X^1 (\r_1,\r_2)Y^2+X^2 (\r_2,\r_1)Y^1+X^2 (\r_2,\r_2)Y^2=
$$
$$X^\a (\r_\a,\r_\beta)Y^\beta=
X^\a g_{\a\beta}Y^\beta=G(\X,\Y)
$$



\subsubsection{Formulae for induced metric}
\label{formulaeforembeddingsurfaces}

We obtained \eqref{firstquadraticform}
from equation  \eqref{inducedmetric1}.

 We can do these calculations in a 
little bit other way. 

The  Riemannian structure of Euclidean space---
standard Euclidean metric
in Euclidean coordinates is given by
           \begin{equation}\label{RiemEuclid}
            G_{\E^3}=(dx)^2+(dy)^2+(dz)^2\,.
           \end{equation}
Then the induced metric \eqref{inducedmetric1} on the surface
$M$ defined by equation $\r=\r(u,v)$
 is equal to  
       \begin{equation}\label{intermsofdifferentials1}
   G_M= G_{\E^3}\big\vert_{\r=\r(u,v)}=\left((dx)^2+(dy)^2+(dz)^2\right)\big\vert_{\r=\r(u,v)}=
        G_M=g_{\a\beta}du^\a du^\beta
       \end{equation}
i.e.  $\left((dx)^2+(dy)^2+(dz)^2\right)\big\vert_{\r=\r(u,v)}=$
          $$
\left({\p x(u,v)\over \p u}du+{\p x(u,v)\over \p v}dv\right)^2+
\left({\p y(u,v)\over \p u}du+{\p y(u,v)\over \p v}dv\right)^2+
\left({\p z(u,v)\over \p u}du+{\p z(u,v)\over \p v}dv\right)^2=
          $$
          $$
(x_u^2+y_u^2+z_u^2)du^2+2(x_ux_v+y_uy_v+z_uz_v)dudv+(x_v^2+y_v^2+z_v^2)dv^2
          $$
          We see that
\begin{equation}\label{firstquadraticform1}
G_M=g_{\a\beta}du^\a du^\beta=
g_{11}du^2+2g_{12}dudv+g_{22}dv^2\,,
\end{equation}
where for matrix $||g_{\a\beta}||$, ($\a,\beta=1,2$),  
            $$
||g_{\a\beta}||=
\begin{pmatrix}
 g_{11}   &g_{12}\cr
 g_{21}   &g_{22}\cr
\end{pmatrix}
          =
\begin{pmatrix}
 g_{uu}   &g_{uv}\cr
 g_{vu}   &g_{vv}\cr
\end{pmatrix}
       =
              $$
 \begin{equation}\label{firstquadraticform2}
\begin{pmatrix}
 (x_u^2+y_u^2+z_u^2)   &(x_ux_v+y_uy_v+z_uz_v)\cr
    (x_ux_v+y_uy_v+z_uz_v)&(x_v^2+y_v^2+z_v^2) \cr
\end{pmatrix}=
      \begin{pmatrix}
 \langle\r_u,\r_u\rangle_{\E^3}   
&\langle\r_u,\r_v\rangle_{\E^3}\cr
  \langle\r_v,\r_u\rangle_{\E^3}
&\langle\r_v,\r_v\rangle_{\E^3} \cr
\end{pmatrix}\,.
\end{equation}
We come to same formula \eqref{firstquadraticform}.

\smallskip
{\bf Example}
Consider again sphere of radius $R$  in $\E^3$,
$x^2+y^2+z^2=R^2$ in stereographic coordinates.
 We calculated coordinate tangent vectors to this sphere
in \eqref{coordinatebasisvectorsforsphere}. Now
calculate induced Riemannian metric:
            $$
     G_{S^2}=\left(
    dx^2+dy^2+dz^2\right)
        \big\vert_
{x=R\sin\theta\cos\varphi,
  y=R\sin\theta\sin\varphi,
   x=R\cos\theta\cos\theta
      }=
             $$
             $$
  \left[d(R\sin\theta\cos\varphi)\right]^2+
  \left[d(R\sin\theta\sin\varphi)\right]^2+
  \left[d(R\cos\theta)\right]^2=
             $$
             $$
  \left[
  R\cos\theta\cos\varphi d\theta
   -R\sin\theta\sin\varphi d\varphi\right]^2+
  \left[
  R\cos\theta\sin\varphi d\theta
   +R\sin\theta\cos\varphi d\varphi\right]^2+
  \left[-R\sin\theta d\theta\right]^2=
            $$
       $$
    (R^2\sin^2\theta\sin^2\varphi+
  R^2\sin^2\theta\cos^2\varphi)d\varphi^2+
   (R^2\cos^2\theta\cos^2\varphi+R^2\sin^2\theta)
      d\theta^2=R^2d\theta^2+R^2\sin^2\theta d\varphi^2\,. 
      $$
We see that
 \begin{equation}\label{riemmetriconsphere}
  G_{S^2}=R^2d\theta^2+R^2\sin^2\theta d\varphi^2\,,
             ||g_{\a\beta}||=
       \begin{pmatrix}
        g_{11}&
        g_{12}\cr
        g_{21}&
        g_{22}\cr
       \end{pmatrix}=
       \begin{pmatrix}
        g_{\theta\theta}&
        g_{\theta\varphi}\cr
        g_{\varphi\theta}&
        g_{\varphi\varphi}\cr
       \end{pmatrix}=
       \begin{pmatrix}
        R^2&
        0\cr
        0&
        R^2\sin^2\theta\cr
       \end{pmatrix}\,.
 \end{equation}

\medskip

{\bf Remark}  Sometimes it is useful to use 
the following ``condensed'' notations.
We denote Cartesian coordinates $(x,y,z)$ 
of Euclidean space
by $x^i$, ($i=1,2,3$).   Let surface $M$ 
be given in local  parameterisation $x^i=x^i(u^\a)$.
Riemannian metric of Euclidean space \eqref{RiemEuclid} 
has appearance
                  \begin{equation}
           \label{goodnotations1}
                   G_\E=dx^i\delta_{ik}dx^k\,.
                    \end{equation}
and calculations
\eqref{intermsofdifferentials1}
---\eqref{firstquadraticform2}
 for 
induced metric \eqref{intermsofdifferentials1} 
           has appearance
                  \begin{equation}
\label{goodnotations2}
                 G_M=
    dx^i\delta_{ik}dx^k\big\vert_{x^i=x^i(u^\a)}=
                 {\p x^i(u)\over \p u^\a}
                 \delta_{ik}
                 {\p x^k(u)\over \p u^\beta}
                 du^\a du^\beta=g_{\a\beta}(u)du^\a du^\beta\,
                    \end{equation}
(See also remark above before 
equation \eqref{firstquadraticform}).
One can rewrite \eqref{goodnotations2} in the following way:
              \begin{equation*}\label{inducedmetric4}
              g_{\a\beta}=
      {\p x^i(u)\over \p u^\a}
             \delta_{ij}
      {\p x^j(u)\over \p u^\beta}\,.
         \quad (\a,\beta=1,2,)\,.
              \end{equation*}
It is instructive to come to this equation
 straightforwardly 
from equation \eqref{inducedmetric1}
and definition \eqref{geommeaning}.
We have that due to \eqref{inducedmetric1}
                   $$
   g_{\a\beta}=
g_{\pi\rho}dx^\pi dx^\rho
                     \left(
{\p\over \p u^\a},{\p\over \p u^\beta}
                \right)=
G_M         
                 \left(
{\p\over \p u^\a},{\p\over \p u^\beta}
                \right)=
G_{\E^3}         
                 \left(
\r_\a,\r_\beta
                \right)=
                  $$
         \begin{equation*}
\delta_{pq}dx^pdx^q
        \left(
   {\p x^i\over \p u^\a}{\p \over \p x^i}\,,
   {\p x^j\over \p u^\beta}{\p \over \p x^j}
         \right)=
{\p x^i\over \p u^\a}
          \left[
   \delta_{pq}dx^pdx^q
            \left(
   {\p \over \p x^i}\,,
   {\p \over \p x^j}
         \right)
            \right]
{\p x^j\over \p u^\beta}
          =
      {\p x^i\over \p u^\a}
   \delta_{ij}
{\p x^j\over \p u^\beta}\,.
                    \end{equation*} 

{\footnotesize
Representation \eqref{goodnotations2} in condenced notations
is very useful. It is easy to see that this
formula 
 works for arbitrary dimensions, i.e. if we have
$m$-dimensional manifold embedded 
in $n$-dimensional Euclidean space.
We just have to suppose that in this case $i=1,\dots, n$ and $\a=1,\dots,m$;
manifold is given by parameterisation $x^i=x^i(u^\a)$ ($\a=1,\dots,m$).
Moreover in the case if manifold is embedded not in Euclidean space but in an arbitrary
Riemannian space then one can see 
that we come to the induced metric

   \begin{equation*}\label{RiemanianmetricofEuclideanspace3}
                 G_M=dx^i g_{ik}\left(\left(x(u)\right)\right)dx^k\big\vert_{x^i=x^i(u^\a)}=
                 {\p x^i(u)\over \p u^\a}
                 g_{ik}\left(\left(x(u)\right)\right)
                 {\p x^k(u)\over \p u^\beta}
                 du^\a du^\beta=g_{\a\beta}(x(u))du^\a du^\beta\,
                    \end{equation*}



}


%\end{document}  % 17 February 2018 
%here are three weeks lectures


Check explicitly again that  length of the tangent vectors and curves on the surface
calculating by External observer (i.e. using Euclidean metric \eqref{RiemEuclid})
{\it is the same} as calculating by Internal Observer, ant
(i.e. using the induced Riemannian metric \eqref{firstquadraticform}, \eqref{firstquadraticform1}).
 Let $\X=X^\a\r_\a=a\r_u+b\r_v$ be a vector tangent to the surface $M$.
  The square of the length $|\X|$ of this vector calculated by External observer
  (he calculates using the scalar product in $\E^3$) equals to
\begin{equation}\label{lengthforexternalobserver}
 |\X|^2=\la\X,\X\ra=\la a\r_u+b\r_v,a\r_u+b\r_v\ra=a^2\la\r_u,\r_u\ra+
   2ab\la\r_u,\r_v\ra+b^2\la\r_v,\r_v\ra
\end{equation}
where $\la\,,\,\ra$ is a scalar product in $\E^3$.
The internal observer will calculate the length using Riemannian metric \eqref{firstquadraticform}
\eqref{firstquadraticform1}:
\begin{equation}\label{thevalueoffirstquadraticform}
  G(\X,\X)=
  \begin{pmatrix}
   a, &b
  \end{pmatrix}
   \cdot
   \begin{pmatrix}
   g_{11} & g_{12} \\
   g_{21}& G_{22} \\
   \end{pmatrix}\cdot
  \begin{pmatrix}
   a\\ b\\
  \end{pmatrix}=
  g_{11}a^2+2g_{12}ab+g_{22}b^2
\end{equation}





{\it External observer (person living in ambient space $\E^3$) calculates
 the length
of the tangent vector  using formula
\eqref{lengthforexternalobserver}. An ant living on the surface
calculates length of this vector in internal coordinates using formula
\eqref{thevalueoffirstquadraticform}. External observer deals
with external coordinates of the vector, ant on the surface 
with internal coordinates.
 They come to the same answer.}


\smallskip



  Let $\r(t)=\r(u(t),v(t))$ $a\leq t\leq b$ be a curve on the surface.


  Velocity of this curve at the  point $\r(u(t),v(t))$ is equal to
  $$
  \v=\X=\xi \r_u+\eta\r_v
  \hbox
  {where
  $\xi=u_t,\eta=v_t\colon\quad \v={d\r(t)\over dt}=
    u_t\r_u+v_t\r_v$}
  \,.
  $$


The length of the curve is equal to
\begin{equation}\label{lengthofthecurveexternal}
L=\int_a^b |\v(t)|d t= \int_a^b\sqrt
{\la \v(t),\v(t)\ra_{\E^3}}d t = \int_a^b\sqrt
{\la u_t\r_u+v_t\r_v,u_t\r_u+v_t\r_v\ra_{\E^3}}dt=
\end{equation}
           $$
\int_a^b\sqrt {\la\r_u,\r_u\ra_{\E^3}u_t^2+2\la\r_u,\r_v\ra_{\E^3}u_tv_t+\la\r_v,\r_v\ra_{\E^3}v_t^2}d\tau=
           $$
\begin{equation}\label{lengthofthecurveinternal}
  \int_a^b\sqrt {g_{11} u_t^2+2g_{12}u_tv_t+g_{22}v_t^2}dt
\end{equation}

{\it An external observer will calculate the length of the curve using
\eqref{lengthofthecurveexternal}.  An ant living on the surface calculate
length of the curve using
\eqref{lengthofthecurveinternal} using  Riemannian
metric on the surface. They will come to the same answer.}


%\end{document}

\subsubsection{Induced Riemannian metrics. Examples.}
We consider already an example of induced Riemannian
metric on sphere in spherical coordinates.
Now 
we consider here other
examples of induced Riemannian metric
on some surfaces in $\E^3$.
using calculations for tangent vectors (see \eqref{firstquadraticform}) or
explicitly in terms of differentials (see \eqref{intermsofdifferentials1} and \eqref{firstquadraticform1}).




\m

First of all  consider  the general case when
 a surface $M$ is defined by the
 equation $z-F(x,y)=0$. One can consider the following parameterisation
 of this surface:
\begin{equation}\label{surface}
  \r(u,v)\colon\quad
  \begin{cases}
  x=u\\
  y=v\\
  z=F(u,v)
  \end{cases}
\end{equation}

Then  coordinate tangent vectors $\r_u,\r_v$ are
  \begin{equation}\label{c}
  \r_u=\begin{pmatrix}
        1\\
        0\\
        F_u\\
   \end{pmatrix}
\quad
  \r_v=\begin{pmatrix}
        0\\
        1\\
        F_v\\
   \end{pmatrix}
 \end{equation},
            $$
     (\r_u,\r_u)=1+F_u^2,\quad
     (\r_u,\r_v)=F_uF_v,\quad
     (\r_v,\r_v)=1+F_v^2
            $$
and induced Riemannian metric (first quadratic form) \eqref{firstquadraticform} is equal to
\begin{equation}\label{formula forfirstform}
   ||g_{\a\beta}||=
\begin{pmatrix}
   g_{11} & g_{12} \\
   g_{12}& g_{22} \\
   \end{pmatrix}=
   \begin{pmatrix}
   (\r_u,\r_u) & (\r_u,\r_v) \\
   (\r_u,\r_v) & (\r_v,\r_v) \\
   \end{pmatrix}=   \begin{pmatrix}
   1+F_u^2 & F_uF_v \\
   F_uF_v& 1+F_v^2 \\
   \end{pmatrix}
\end{equation}
\begin{equation}\label{formula forfirstformgeneral1}
   G_M=ds^2=(1+F_u^2)du^2+2F_uF_vdudv+(1+F_v^2)dv^2
\end{equation}
  and the length of the curve $\r(t)=\r (u(t), v(t))$ on $C$  $(a\leq t\leq b)$
  can be calculated by the formula:
               \begin{equation*}
             L=\int
             \int_a^b\sqrt{(1+F_u^2)u_t^2+2F_uF_vu_tv_t+(1+F_v)^2v^2_t}dt
               \end{equation*}
One can calculate \eqref{formula forfirstformgeneral1} explicitly using \eqref{intermsofdifferentials1}:
                     $$
                                    G_M=\left(dx^2+dy^2+dz^2\right)\big\vert_{x=u,y=v,z=F(u,v)}=
               (du)^2+(dv)^2+(F_udu+F_vdv)^2=
                     $$
               \begin{equation}\label{formula forfirstformgeneral2}
=(1+F_u^2)du^2+2F_uF_vdudv+(1+F_v^2)dv^2\,.
               \end{equation}



\medskip

         \medskip


       \centerline  {\it Cylinder}


  Cylinder is given by the equation $x^2+y^2=a^2$. One can consider the following
parameterisation
 of this surface:
\begin{equation}\label{surface1}
  \r(h,\varphi)\colon\quad
  \begin{cases}
  x=a\cos\varphi\\
  y=a\sin\varphi\\
  z=h\\
  \end{cases}
\end{equation}

\medskip

We have   $G_{cylinder}=\iota^*G_{\E^3}=\left(dx^2+dy^2+dz^2\right)
       \big\vert_{x=a\cos\varphi,y=a\sin\varphi,z=h}=$
        \begin{equation}\label{firstquadraticformcylinder}
               =(-a\sin\varphi d\varphi)^2+(a\cos\varphi
              d\varphi)^2+dh^2=a^2d\varphi^2+dh^2
        \end{equation}

The same formula in terms of scalar product of tangent vectors:



  \begin{equation}\label{cyl1}
\hbox {coordinate basis vectors\,\,}
  \r_h=\begin{pmatrix}
        0\\
        0\\
        1\\
   \end{pmatrix}
\,,
\quad
  \r_\varphi=\begin{pmatrix}
        -a\sin\varphi\\
        a\cos\varphi\\
          0\\
   \end{pmatrix}
 \end{equation},
            $$
     (\r_h,\r_h)=1,\quad
     (\r_h,\r_\varphi)=0,\quad
     (\r_\varphi,\r_\varphi)=a^2
            $$
and
\begin{equation*}\label{formula forfirstformcyl}
||g_{\a\beta}||=
   \begin{pmatrix}
   (\r_u,\r_u) & (\r_u,\r_v) \\
   (\r_u,\r_v) & (\r_v,\r_v) \\
   \end{pmatrix}=   \begin{pmatrix}
   1 & 0 \\
   0& a^2 \\
   \end{pmatrix}\,,
\end{equation*}
\begin{equation}\label{formula forfirstformcyl}
   G=dh^2+a^2d\varphi^2
\end{equation}
  and the length of the curve $\r(t)=\r (h(t), \varphi(t))$ on the cylinder
    $(a\leq t\leq b)$
  can be calculated by the formula:
               \begin{equation}
             L=
             \int_a^b\sqrt{h_t^2+a^2\varphi_t}dt
               \end{equation}



\medskip

  \centerline {\it Cone}
 Cone is given by the equation $x^2+y^2-k^2z^2=0$.
One can consider the following
parameterisation
 of this surface:
\begin{equation}\label{surfacecone}
  \r(h,\varphi)\colon\quad
  \begin{cases}
  x=kh\cos\varphi\\
  y=kh\sin\varphi\\
  z=h\\
  \end{cases}
\end{equation}

\medskip

   Calculate induced Riemannian metric:

We have               $$
             G_{conus}=\iota^*G_{\E^3}=\left(dx^2+dy^2+dz^2\right)
\big\vert_{x=kh\cos\varphi,y=kh\sin\varphi,z=h}=
                      $$
                      $$
                      (k\cos\varphi dh-kh\sin\varphi d\varphi)^2+
             (k\sin\varphi dh+kh\cos\varphi d\varphi)^2+dh^2
                      $$
        \begin{equation}\label{firstquadraticformforconus}
            G_{conus} =k^2h^2d\varphi^2+(1+k^2)dh^2,\,\,
                        ||g_{\a\beta}||=
   \begin{pmatrix}
   1+k^2 & 0 \\
   0&  k^2h^2 \\
   \end{pmatrix}
                       \end{equation}
The length of the curve $\r(t)=\r (h(t), \varphi(t))$ on the
  cone
    $(a\leq t\leq b)$
  can be calculated by the formula:
               \begin{equation}
             L=\int_a^b
             \sqrt{(1+k^2)h_t^2+k^2h^2\varphi_t^2}dt
               \end{equation}

\medskip
 \centerline {\it Circle {\tt (again)}}
  Circle of radius $R$ is given by the equation $x^2+y^2=R^2$. Consider standard parameterisation
  $\varphi$
 of this surface:
\begin{equation*}\label{surfacecircle}
  \r(\varphi)\colon\quad
  \begin{cases}
  x=R\cos\varphi\\
  y=R\sin\varphi\\
  \end{cases}
\end{equation*}
   Calculate induced Riemannian metric (first quadratic form)
              $$
              G_{S^1}=\iota^*G_{\E^3}=
    \left(dx^2+dy^2\right)\big\vert_{x=R\cos\varphi,y=R\sin\varphi}=
                      $$
                      $$
                      (-R\sin\varphi d\varphi)^2+
                      (a\cos\varphi d\varphi)^2=
          (R^2\cos^2\varphi+R^2\sin^2\varphi)d\varphi^2=
             R^2 d\varphi^2\,.
             $$

One can consider stereographic coordinates on the circle (see Example in the subsection 1.1)
A point $x,y\colon x^2+y^2=R^2$ has stereographic coordinate $t$ if points $(0,1)$ (north pole),
the point $(x,y)$ and the point $(t,0)$ belong to the same line, i.e.
     ${x\over t}={R-y\over R}$, i.e.
              $$
     t={Rx\over R-y}, \qquad \begin{cases}
      x={2tR^2\over R^2+t^2}\cr
      y={t^2-R^2\over t^2+R^2}R\cr
      \end{cases}\,. \quad{\rm since\,\,} x^2+y^2=R^2\,.
                   $$
 Induced metric in coordinate $t$ is
               $$
                       G=(dx^2+dy^2)\big\vert_{x=x(t),y=y(t)}=
\left(d\left(2tR^2\over R^2+t^2\right)\right)^2+
         \left(d\left({t^2-R^2\over R^2+t^2}R\right)\right)^2=
         $$
         \begin{equation}\label{circleinstereographiccoordinates}
       \left({2R^2dt\over R^2+t^2}-{4t^2R^2dt\over (R^2+t^2)^2}\right)^2+
         \left(-{4R^2 tdt\over (t^2+R^2)^2}\right)^2=
         {4R^4dt^2\over (R^2+t^2)^2}\,.
               \end{equation}
(See for detail Homework 2\footnote{ One can also 
obtain this formula in a very beautiful 
way using inversion (see Appendices)}.

{\bf Remark} Stereographic coordinates very often are preferable since
 they define birational equivalence between circle and line.


\m

 \centerline {\it Sphere {\tt (again...)}} 
  Sphere of radius $R$ is given by the
equation $x^2+y^2+z^2=R^2$.
Consider first stereographic coordinates 
 \begin{equation}\label{surfacesphere}
  \r(\theta,\varphi)\colon\quad
  \begin{cases}
  x=R\sin\theta\cos\varphi\\
  y=R\sin\theta\sin\varphi\\
  z=R\cos\theta\\
  \end{cases}
\end{equation}

 We already calculated the coordinate basis
in \eqref{coordinatebasisvectorsforsphere}
and we calculated induced Riemannian metric
in \eqref{riemmetriconsphere}:
   \begin{equation}
 \label{firtsquadraticformforsphere(diff)}\,,\qquad
      G_{S^2}=
R^2d\theta^2+R^2\sin^2\theta d\varphi^2\,,\quad
                        ||g_{\a\beta}||=
   \begin{pmatrix}
   R^2 & 0 \cr
   0&  R^2\sin^2\theta \cr
   \end{pmatrix}
                       \end{equation}


      
                     
      One comes to the same answer calculating scalar product of coordinate tangent vectors:
  \begin{equation*}\label{ca}
\hbox{coordinate tangent vectors are}\,
  \r_\theta=\begin{pmatrix}
        R\cos\theta\cos\varphi\\
        R\cos\theta\sin\varphi\\
        -R\sin\theta\\
   \end{pmatrix}
     \,,
\quad
  \r_\varphi=\begin{pmatrix}
        -R\sin\theta\sin\varphi\\
        R\sin\theta\cos\varphi\\
          0\\
   \end{pmatrix}
 \end{equation*},
            $$
     (\r_\theta,\r_\theta)=R^2,\quad
     (\r_h,\r_\varphi)=0,\quad
     (\r_\varphi,\r_\varphi)=R^2\sin^2\theta
            $$
and
\begin{equation*}\label{formula forfirstform3*}
   ||g||=
   \begin{pmatrix}
   (\r_u,\r_u) & (\r_u,\r_v) \\
   (\r_u,\r_v) & (\r_v,\r_v) \\
   \end{pmatrix}=
\end{equation*}
\begin{equation*}\label{formula forfirstformsphere}
   \begin{pmatrix}
   R^2 & 0 \\
   0&  R^2\sin^2\theta \\
   \end{pmatrix}, \quad
   G_{S^2}=ds^2=R^2d\theta^2+R^2\sin^2\theta d\varphi^2
\end{equation*}
  The length of the curve $\r(t)=\r (\theta(t), \varphi(t))$ on the
  sphere of the radius $a$
    $(a\leq t\leq b)$
  can be calculated by the formula:
               \begin{equation}
             L=\int_a^b
             R\sqrt{\theta_t^2+\sin^2\theta\cdot \varphi_t^2}dt
               \end{equation}

One can consider on sphere as well as on a circle stereographic coordinates:
                 \begin{equation}
                 \label{stereographiccoordinatesonsphere}
                           \begin{cases}
                    u={Rx\over R-z}
                 \cr
                 v={Ry\over R-z}\cr
                 \end{cases},
                 \qquad \begin{cases}
                 x={2uR^2\over R^2+u^2+v^2}\cr
                 y={2vR^2\over R^2+u^2+v^2}\cr
                 z={u^2+v^2-R^2\over u^2+v^2+R^2}R
                    \end{cases}
                    \end{equation}
In these coordinates Riemannian metric is
                     $$
G=(dx^2+dy^2+dz^2)\big\vert_{x=x(u,v),y=y(u,v),z=z(u,v)}=
                        $$
                        $$
  \left(d\left(2uR^2\over R^2+u^2+v^2\right)\right)^2+
                     \left(d\left(2vR^2\over R^2+u^2+v^2\right)\right)^2+
         \left(d\left(1-{2R^2\over R^2+u^2+v^2}\right)R\right)^2=
                        $$
                       \begin{equation}\label{sphereinstereographiccoordinates}
                        =
         {4R^4(du^2+dv^2)\over (R^2+u^2+v^2)^2}\,.
            \end{equation}



(See for detail Homework 2\footnote
{Another beautiful deduction of this formula
see in Appendices (Inversion)}.

 %\end{document}   % 15 February 2019


\m
% 18  february 2016

%\end{document}
Notice that we showed  that metric on sphere is 
conformally Euclidean.


\centerline {\it Saddle (paraboloid)}
%\footnote{{\it This example was not
%considered on lectures. 
%It could be useful for learning purposes.}
%}}

Consider paraboloid $z=x^2-y^2$.
It can be rewritten as $z=axy$ and it is 
called sometimes ``saddle''
(rotation on the angle $\varphi=\pi/4$ 
transforms $z=x^2-y^2$ onto $z=2xy$.)
We considered this example in homework 3
{\footnotesize Paraboloid and saddle they are  ruled surfaces which are formed by lines.}

%  Consider the following
%(standard ) parameterisation
% of this surface:
%\begin{equation}\label{surface3}
%  \r(u,v)\colon\quad
%  \begin{cases}
%  x=u\\
%  y=v\\
%  z=uv\\
%  \end{cases}
%\end{equation}



%Calculate induced metric:
%                       $$
% G_{saddle}=\left(dx^2+dy^2+dz^2\right)
%\big\vert_{x=u\cos\varphi,y=v\sin\varphi,z=uv}=
%                      du^2+dv^2+(udv+vdu)^2=
%                      $$
%        \begin{equation*}\label{firtsquadraticformforsaddle(diff)}
%            G_{saddle} =(1+v^2)du^2+2uvdudv+(1+u^2)dv^2\,.
%            \end{equation*}


  Examples of other quadratic 
surfaces see in in Appendix.


%\end{document}   %12 February 2015
%\end{document} %12 February   2018   19 February  2017

\subsection {Isometries of Riemannian manifolds.}

\subsubsection {Riemannian metric induced by map}

Let $M$ be a manifold, and let $(N,G(N)^{})$ 
be a Riemannian manifold.
Let $F$ be a map from $M$ to $N$,
         $$
F\colon \qquad  M\longrightarrow\, N
         $$
We do not suppose that $F$ is diffeomorphism, we 
even do not suppose that manifolds $M$ and $N$
have the same dimension\footnote{We just suppose that
$F$ is differentiable map, i.e. local expressions
for $F$ are smooth functions}.

  We can define $F^*G$,  a {\it Riemannian metric on
manifold $M$ induced by the map $F\colon\, M\to N$}.

 Describe the metric $F^*G$ on $M$ in local coordinates.
  
Let  $x^a$, ($a=1,\dots,m$) be local coordinates on 
$m$-dimensional  manifold
$M$  in a vicinity of some point
$\pt_M$ on $M$. Consider a point $\pt_N=F(\pt_M)$
on manifold $N$  and let $y^i$,
($i=1,\dots,n$)  
be local coordinates on $n$-dimensional manifold $N$
in a vicinity of point  $\pt_N$.
If in local coordinates $y^i$,
 Riemannian metric $G^{(N)}$ on $N$ has appearance 
       $$G^{(N)}
       =g^{(N)}_{ij}(y)dy^idy^j\,,
       $$
then in local coordinates $x^a$,
Riemannian metric $F^*(G^{(N)})$ on $M$ has appearance 
           \begin{equation*}
            F^*(G^{(N)})=
g^{(M)}_{ab}(x)dx^adx^b=
     dx^a{\p y^i(x)\over \p x^a}
       g^{(N)}_{ij}(y(x)) 
     {\p y^j(x)\over \p x^b}dx^b\,,\quad`
          \end{equation*}
i.e.      \begin{equation}\label{generalembedding}
g^{(M)}_{ab}(x)=
     {\p y^i(x)\over \p x^a}
       g^{(N)}_{ij}(y(x)) 
     {\p y^j(x)\over \p x^b}\,,\quad
          \end{equation}
where $y^i=y^i(x^a)$ is expression of map $F$ in local 
coordinates  $x^a$ and $y^i$,       

{\bf Remark}  The metric $F^*(G^{(N)})$ on $M$
is called {\it pull-back} of metric $G^{(N)}$
under the map $F\colon\, M\to N$.

\medskip
 
{\bf Example}  The induced metric
on surfaces in $\E^3$ is a special example of this
general construction. Indeed embedding
   $\iota\colon M\to \E^3$ is a map from points
of $2$-dimensional 
manifold $M$ to points of $3$-dimensional 
Euclidean space $\E^3$.  Applying
\eqref{generalembedding} to this map we come
to formulae  \eqref{firstquadraticform1}  
for induced metric: 
              $$
   G_M= \iota^*G_{\E^3}=\iota^*(dx^2+dy^2+dz^2)=
              $$
            \begin{equation*}
      G_{\E^3}\big\vert_{\r=\r(u,v)}=\left((dx)^2+(dy)^2+(dz)^2\right)\big\vert_{\r=\r(u,v)}=
        G_M=g_{\a\beta}du^\a du^\beta
                 \end{equation*}
(or another manifestation of this formula, equation
  \eqref{goodnotations2}).


\subsubsection {Diffeomorphism, which is an isometry}

  Let $(M_1,G_{(1)})$, $(M_2,G_{(2)})$ be 
two Riemannian manifolds---
 manifolds equipped
with Riemannian metric $G_{(1)}$ and $G_{(2)}$ respectively.

Loosely  speaking isometry is the
diffeomorphism of Riemannian manifolds which preserves the distance.



{\bf Definition}
Let $F$ be a diffeomorphims (one-one smooth map with smooth inverse)
 of manifold $M_1$ on manifold $M_2$.


We say that diffeomorphism $F$ is an isometry of
Riemannian manifolds
 $(M_1,G_{(1)})$ and $(M_2,G_{(2)})$ 
if it  preserves the metrics, i.e.
$G_{(1)}$ is pull-back of $G_{(2)}$:
              \begin{equation}\label{pullback1}
            F^*G_{(2)}=G_{(1)}\,.
               \end{equation}
%  In local coordinates this means the following:
%  Let $\pt_1$ be an arbitrary point on manifold $M_1$
%and $\pt_2\in M_2$ be its image:$F(\pt_1)=\pt_2$.
%      Let $\{x^i\}$ be arbitrary
%oordinates in a vicinity of a point $\pt_1\in M_1$
%      and $\{y^a\}$ be arbitrary
% coordinates in a vicinity of a point $\pt_2\in M_2$.
%      Let  Riemannian  metrics  $G_{(1)}$ on $M_1$
%  has local expression
%      $G_{(1)}=g_{(1)ik}(x)dx^idx^k$ in coordinates $\{x^i\}$
% and respectively
%      Riemannian  metrics  $G_{(2)}$ has local expression
%      $G_{(2)}=g_{(2)ab}(y)dy^ady^b$ in coordinates $\{y^a\}$ on $M_2$.
%                 Then the formula \eqref{pullback1} 
%has the following
%appearance in these local coordinates:
According \eqref{generalembedding} this means that  
                  $$
      F^*\left(g_{_{(2)}ab}(y)dy^a dy^b\right)=
       g_{_{(2)}ab}(y)dy^a dy^b\big\vert_{y=y(x)}=
                 $$
                \begin{equation*}\label{transformationlaw}
               g_{_{(2)}ab}(y(x))
   {\p y^a(x)\over \p x^i}dx^i{\p y^b(x)\over \p x^k}dx^k
   =g_{_{(1)}ik}(x)dx^idx^k\,,
               \end{equation*}
    i.e.  \begin{equation}\label{transformationlaw}
   g_{_{(1)}ik}(x)={\p y^a(x)\over \p x^i}g_{_{(2)}ab}(y(x))
        {\p y^b(x)\over \p x^k}\,,
                 \end{equation}
 where $y^a=y^a(x)$ is local expression for diffeomorphism  $F$.
We say that diffeomorphism $F$ is {\it isometry}
 of Riemanian manifolds
$(M_1, G_{(1)})$ and $M_2,G_{(2)}$.
The difference of this equation with equation
 \eqref{generalembedding} is that $F$ in 
\eqref{generalembedding}
was just a differentiable map, which is not a diffeomorphism.
  In \eqref{transformationlaw}   
diffeomorphism $F$ establishes one-one 
correspondence between local
coordinates on manifolds $M_1$ and $M_2$.
The left hand side of equation \eqref{transformationlaw} can be
considered as a local expression of metric $G_{(2)}$ in coordinates
$x^i$ on $M_2$ and the right hand side of this equation is local
expression of metric $G_{(1)}$ in coordinates $x^i$ on $M_1$.
Diffeomorphism $F$ identifies manifolds $M_1$ and $M_2$
and it can be considered as changing of coordinates.



\m



{\bf Example} Consider surface of cylinder
$C$, $x^2+y^2=a^2$ in $\E^3$
with induced Riemannian metric
 $G_{C}=a^2 d\varphi^2+dh^2$ (see equations
\eqref{surface1} and \eqref{firstquadraticformcylinder}).
If we remove the line $l\colon\,\,x=a,y=0$
from the cylinder surface $C$ we come to
surface $C'=C\backslash l$.
Consider a map $F$ of this surface
in Euclidean space $E^2$ with Cartesian coordinates $u,v$
(with standard Euclidean metric $G_{Eucl}=du^2+dv^2$):
         \begin{equation}\label{diffofcylinderondomain}
        F\colon\qquad
        \begin{cases}
           u=a\varphi\cr
           v=h\cr
          \end{cases}\,\quad 0<\varphi < 2\pi\,.
           \end{equation}
One can see that $F$ is the diffeomorphism  of $C'$ on the domain
$0<u<2\pi a$ in $\E^2$ and this diffeomorphism is an isometry:
 it transforms the metric
$G_{Eucl}$ on Euclidean space  in metric $G_{C}$ on cylinder, i.e.
pull-back condition \eqref{pullback1} is obeyed:
       $$
    F^*G_{Eucl}=F^*\left(du^2+dv^2\right)=
    \left(du^2+dv^2\right)\big\vert_{u=a\varphi,v=h}=
     a^2d\varphi^2+dh^2=G_1\,.
       $$
We see that cylinder surface with removed line is isometric to domain
in $\E^2$ and the map $F$ establishes this isometry.

{\footnotesize
{\bf Remark} Notice that
if  $F$ is diffeomorphism of manifold $M_1$
on a Riemannian manifold $(M_2,G_{(2)}$, 
then  it defines Riemannian structure,
the pull-back $G_{1}=F^*(G(2)_{})$ on $M_1$,
and   $F$ is isometry of Riemannian manifold
 $(M_1,G_{(1)})$
on Riemannian manifold $(M_2,G_{(2)})$.
}

\subsubsection {Isometries of Riemannian manifold on itself}
  
{\bf Definition} Let $(M,G)$ be a Riemannian manifold. We say that
a diffeomorphism $F$ is an isometry of Riemannian manifold
on itself if it preserves the metric, i.e. $F^*G=G$.
In local coordinates this means that
              \begin{equation}\label{isometry1}
        g_{ik}(x)=g_{pq}(x'(x)){\p x^p(x')\over \p x^i}
                       {\p x^q(x')\over \p x^k}\,,
               \end{equation}
where $x'=x'(x)$ is a local expression for diffeomorphism $F$.
{\bf Example}  Let $\E^2$ be Euclidean plane with metric $dx^2+dy^2$
in Cartesian coordinates $x,y$. Consider the transformation
       \begin{equation*}
   \begin{cases}
      x'=p+ax+by\cr
     y'=q+cx+dy\cr
    \end{cases}
          \end{equation*}
is isometry if and only if the matrix 
$A=\begin{pmatrix}a &b\cr c &d\cr\end{pmatrix}$
is an orthogonal matrix, i.e. if the trasformation above
is combination of translation, rotation and reflection.

\medskip

Another example
  {\bf Example} Consider Lobachevsky (hyperbolic) plane:
     an upper half-plain ($y>0$) 
in $\R^2$ equipped with 
Riemannian metric
            $$
     G={dx^2+dy^2\over y^2}\,,
            $$
One can see that  the map 
$$
\begin{cases} 
x=\l x'\cr y=\l y'\cr
 \end{cases}\,, (\l>0)
     $$
($\l>0$)  is an isometry of the Lobachevsky plane on itself.
Are there other isometries? Yes  there are 
(See the disucssion of these questions in Homeworks.) 



\subsubsection {Locally Euclidean Riemannian 
 manifolds }
It is useful to formulate the  local isometry condition between
Riemannian manifold and Euclidean space.
A neighbourhood of every point of $n$-dimensional
 manifold is diffeomorphic to $\R^n$.
Let as usual $\E^n$ be $n$-dimensional
Euclidean space, i.e. $\R^n$  with standard Riemannian metric
$G=dx^i\delta_{ik}dx^k=(dx^1)^2+\dots+(dx^n)^2$
in Cartesian coordinates $(x^1,\dots,x^n)$.

{\bf Definition} We say that $n$-dimensional Riemannian manifold
$(M,G)$ is locally isometric to Euclidean space $\E^n$,
i.e.  it is locally Euclidean Riemannian manifold,
if for every point $\pt\in M$ there exists an  open
neighboorhood $D$ (domain) containing this point, $\pt\in D$
such that $D$ is isometric to a domain in Euclidean plane. In other words
in a vicinity of every point $\pt$ there exist local coordinates
$u^1,\dots,u^n$ such that Riemannian metric $G$ in
these coordinates has an appearance
            \begin{equation}\label{defofisometrytoEuclid}
           G=du^i\delta_{ik}du^k=(du^1)^2+\dots+(du^n)^2\,.
             \end{equation}
The coordinates $(u^1,\dots,u^n)$  are called
{\it locally Euclidean coordinates}.

\m






   Consider examples.

{\bf Example} Consider again cylinder surface..



 We know that cylinder is not diffeomorphic to plane
(cylinder surface is $S^1\times \R$, $\E^2=\R\times \R$,
 and circle is not diffeomorphic to line ).
 In the previous subsection we cutted the line from cylindre.
Thus we came to surface diffeomorphic to plane. We established that
this surface is isometric to Euclidean plane. (See equation
 \eqref{diffofcylinderondomain} and considerations above.)
Local isometry of cylinder to the Euclidean plane, i.e. the fact
that it is locally Euclidean Riemannian surface
 immediately follows from the fact
that under changing of local coordinates $u=a\varphi, v=h$
  in equation \eqref{diffofcylinderondomain},
the standard Euclidean metric
$du^2+dv^2$ transforms to the metric
$G_{cylinder}=a^2d\varphi^2+dh^2$ on cylinder.

{\bf Remark} Strictly speaking we consider all the points
except the points on the cutted line (with coordinate $\varphi-0$).
On the other hand for the points on cuttng line we can consider
 instead coordinate $\varphi$ another coordinate $\varphi'=\varphi-\pi$,
  $-\pi<\varphi'<\pi$,
and we will come to the same answer. In this case the cutted line
will be the line $\varphi/=\pi$.

  \m

{\bf Example}
Now show that cone is locally Euclidean Riemannian surface, i.e, it is
locally isometric to the Euclidean  plane.
This means that we have to find local coordinates $u,v$ on the cone such that in these coordinates
  induced metric $G\vert_c$ on cone would have the appearance $G\vert_c=du^2+dv^2$.  Recall calculations of the metric on cone in
coordinates $h,\varphi$ where
             $$
          \r(h,\varphi)\colon
          \begin{cases}
          x=kh\cos\varphi\cr
          y=kh\sin\varphi\cr
          z=h\cr
          \end{cases},
             $$
$x^2+y^2-k^2z^2=k^2h^2\cos^2\varphi+k^2h^2\sin^2\varphi-
k^2h^2=k^2h^2-k^2h^2=0$.
 We have that metric $G_c$ on the cone in coordinates
$h,\varphi$  induced with
the Euclidean metric $G=dx^2+dy^2+dz^2$ is equal to
                $$
            G_c=\left(dx^2+dy^2+dz^2\right)
\big\vert_{x=kh\cos\varphi, y=kh\sin\varphi, z=h}=
            (k\cos\varphi dh-kh\sin\varphi d\varphi)^2+
            $$
            $$
            (k\sin\varphi dh+kh\cos\varphi d\varphi)^2+dh^2=
            (k^2+1)dh^2+k^2h^2d\varphi^2\,.
                $$
In analogy with polar coordinates try to find new local coordinates $u,v$
such that $\begin{cases} u=\a h\cos \beta\varphi\cr v=\a h\sin \beta\varphi\end{cases}$,
where $\alpha, \beta$ are parameters. We come to  $du^2+dv^2=$
             $$
    \left(\a\cos\beta\varphi dh-\a\beta h\sin\beta\varphi d\varphi\right)^2+
  \left(\a\sin\beta\varphi dh+\a\beta h\cos\beta\varphi d\varphi\right)^2=
  \alpha^2 dh^2+\a^2\beta^2 h^2d\varphi^2.
             $$
Comparing with the metric on the cone 
$G_c=(1+k^2)dh^2+k^2h^2d\varphi^2$  
we see that if we put $\alpha=\sqrt {k^2+1}$ and
$\beta={k\over \sqrt{1+k^2}}$
then $du^2+dv^2=\alpha^2 dh^2+\a^2\beta^2 h^2d\varphi^2=(1+k^2)dh^2+k^2h^2d\varphi^2$.

Thus in new local coordinates
                  $$
                  \begin{cases}
             u= \sqrt {k^2+1}h\cos {k\over \sqrt {k^2+1}}\varphi\cr
             v=  \sqrt {k^2+1} h\sin {k\over \sqrt {k^2+1}}\varphi\cr
                  \end{cases}
                  $$
induced metric on the cone becomes
$G\vert_c= du^2+dv^2$, i.e. cone locally is isometric to the Euclidean plane \finish

Of course these coordinates are local.---  Cone and plane are not homeomorphic, thus they are not globally isometric.

\m

{\bf Example and counterexample}

  Consider domain $D$ in Euclidean plane with two metrics:
              \begin{equation}\label{twoexamples}
    G_{(1)}=du^2+\sin^2v dv^2\,,\quad {\rm and}\quad
    G_{(2)}=du^2+\sin^2u dv^2
              \end{equation}
Thus we have two different Rimeannian manifolds $(D,G_{(1)})$
and $(D, G_{(2)})$.
Metrics in \eqref{twoexamples} look similar. But....
It is easy to see that the first one is locally
isometric to Euclidean plane, i.e. it is locally Euclidean Riemannian manifold
since $\sin^2 vdv^2=d(-\cos v)^2$: in new coordinates
$u'=u,v'=\cos v$ Riemannian metric $G_{(1)}$ has appearance of standard
Euclidean metric:
            $$
(du')^2+(dv')^2=(du)^2+(d(\cos v))^2=du^2+\sin^2 vdv^2=G_{(1)}\,.
            $$
This is not the case for second metric $G_{(2)}$. If we change notations
$u\mapsto \theta$, $v\mapsto \varphi$ then
 $G_{(2)}=d\theta^2+sin\theta^2 d\varphi^2$.
This is local
expression for Riemannian metric induced on the sphere of radius $R=1$.
Suppose that there exist coordinates $u'=u'(\theta,\varphi)$
 $v'=v'(\theta,\varphi)$ such that in these coordinates
metric has Eucldean appearance. This means that locally geometry of
sphere is as  a geometry of Euclidean plane.
On the other hand we know from the course of Geometry
that this is not the case:
sum of angles of triangels on the sphere is not equal to $\pi$,
sphere cannot be bended without shrinking. Later in this course we will
return to this question....

     \m

   There are plenty other examples:

      2) Plane with metric $4R^4(dx^2+dy^2)\over (R^2+x^2+y^2)^2$
is isometric to the sphere with radius $R$.

    3) Disc with metric $du^2+dv^2\over (1-u^2-v^2)^2$ is isometric to half plane with metric $dx^2+dy^2\over 4y^2$.

(see also exercises in Homeworks and Coursework.)


% 20 February
%\end{document} 20 February 2015


%\end{document} %22 February




    \subsection {Volume element in Riemannian manifold}




   The volume element
in $n$-dimensional Riemannian manifold with metric $G=g_{ik}dx^i
    dx^k$ is defined by the formula
     \begin{equation}\label{volumelement}
  \sqrt {\det g}\,dx^1dx^2\dots dx^n\,.
\end{equation}
    If $D$ is a domain in the $n$-dimensional Riemannian manifold with metric
    $G=g_{ik}dx^i$
    then its volume is equal to to the integral of volume element over this domain.
          \begin{equation}\label{volumeofriemannianmanifold}
  V(D)=\int_D \sqrt {\det g}\,dx^1dx^2\dots dx^n\,.
\end{equation}


     Note that in the case of $n=1$ volume is just the length, in the case
     if $n=2$ it is area.

{\footnotesize Why this formula for  volume form?
One can see that  volume form \eqref{volumelement} is invariant with respect 
to changing of coordinates
 i.e. if $y^1,\dots,y^n$ are new coordinates:
    $x^1=x^1(y^1,\dots,y^n), x^2=x^2(y^1,\dots,y^n)...$,
                      $$
          x^i=x^i(y^p), i=1,\dots,n\quad, p=1,\dots,n
                      $$
and   $\tilde g_{pq}(y)$ matrix of the metric in new coordinates:
                       \begin{equation}\label{changingofcoord2}
    \tilde g_{pq}(y)={\p x^i\over \p y^{p}}g_{ik}(x(y))
             {\p x^k\over \p y^{q}}\,.
\end{equation}
Then
                    \begin{equation}\label{invarianceofvolumeelement}
 \sqrt {\det g_{ik}(x)}\,dx^1 dx^2 \dots dx^n=
 \sqrt {\det \tilde g_{pq}(y)}\,dy^1 dy^2 \dots dy^n
\end{equation}
This follows from \eqref{changingofcoord2}. Namely
                             $$
\sqrt {\det g_{ik}(y)}\,dy^1 dy^2 \dots dy^n=
\sqrt {\det \left({\p x^i\over \p y^{p}}g_{ik}(x(y))
             {\p x^k\over \p y^{q}}\right)}\,dy^1 dy^2 \dots dy^n
                             $$
Using the fact that $\det (ABC)=\det A\cdot \det B\cdot \det C$
and  $\det \left({\p x^i\over \p y^{p}}\right)=
\det \left({\p x^k\over \p y^{q}}\right)$\footnote{determinant of matrix does not change
if we change the matrix on the adjoint, i.e. change columns on rows.} we see that
from the formula above follows:
           $$
           \sqrt {\det g_{ik}(y)}\,dy^1 dy^2 \dots dy^n=
\sqrt {\det \left({\p x^i\over \p y^{p}}g_{ik}(x(y))
             {\p x^k\over \p y^{q}}\right)}
          dy^1 dy^2 \dots dy^n=
           $$
           $$
\sqrt {\left(\det
        \left(
    {\p x^i\over \p y^{p}}
    \right)\right)^2}
     \sqrt {\det g_{ik}(x(y))}
     dy^1 dy^2 \dots dy^n=
           $$
\begin{equation}\label{transformofvolumeform4}
\sqrt {\det g_{ik}(x(y))}
         \det
        \left(
    {\p x^i\over \p y^{p}}
    \right)
    dy^1 dy^2 \dots dy^n=
\end{equation}


Now note that   $$
\det \left({\p x^i\over \p y^{p}}\right)dy^1 dy^2 \dots dy^n=dx^1\dots dx^n
                $$
according to the formula for changing coordinates in $n$-dimensional integral
\footnote{Determinant of the matrix $\left({\p x^i\over \p y^{p}}\right)$ of changing of coordinates
is called sometimes Jacobian. Here we consider the case if Jacobian is positive.
If Jacobian is negative then formulae above remain valid just the symbol of modulus appears.}.
Hence
\begin{equation}\label{transformofvolumeform5}
\sqrt {\det g_{ik}(x(y))}\det
      \left(
 {\p x^i\over \p y^{p}}
 \right) dy^1 dy^2 \dots dy^n=
\sqrt {\det g_{ik}(x(y))}dx^1 dx^2 \dots dx^n
\end{equation}
Thus we come to \eqref{invarianceofvolumeelement}.


{\footnotesize
{\bf Remark} 
Students who know the concept of exterior forms
 can read the volume element as $n$-form
     $\sqrt {{\det g}}\,
dx^1\wedge dx^2\wedge \dots \wedge dx^n$

}

 }


In the next paragraph we will give another motivation of
this formula from linear algebra.
 


\subsubsection {Motivation: Gram formula for
  volume of parallelepiped}

   In  this short paragraph
we consider formulae for volume of $n$-dimensional
parallelepiped, and we  
 explain how formulae \eqref{volumelement},
 \eqref{volumeofriemannianmanifold} are related with 
basic formulae in geometry.
For simplicity one can  consider just the case
if $n=2,3$.


 Let $\E^n$ be Euclidean
  vector space equipped with orthonormal 
basis $\{\e_i\}$. 
 
Let
  $\{\ac_i\}=\{\ac_1,\dots,\ac_n\}$ be an arbitrary row
of $n$-vectors in $\E^n$.
Consider $n$-dimensional parallelepiped 
$\Pi_{\{\ac_i\}}$ formed 
by these vectors:  
 $\Pi_{\{\ac_i\}}\colon  \r=t^i\v_i, 0\leq t^i\leq 1$.
The volume of this parallelepiped 
is equal to 
    \begin{equation}\label{parallelepipinn-dimspace1}
       Vol(\Pi_{\{\ac_i\}})=\det A\,,
             \end{equation}
where the matrix $A=||\ac^m_i||$ 
is defined by expansion of vectors $\{\ac_i\}$
over orthonormal basis $\{\e_i\}$:
$\ac_i=\e_ma_m^i$
(Volume vanishes ($Vol(\Pi_{\ac_i})=\det A=0$) 
$\Leftrightarrow$ if $\{\ac_i\}$ is not a basis.)

 Now consider the scalar product (Riemannian metric) 
in $\E^n$
in the basis ${\ac_1,\dots,\ac_n}$:
        \begin{equation}\label{gramm1}
    g_{ik}=\langle\ac_i,\ac_k\rangle\,,
        \end{equation}
    where $\langle\,\,,\,\,\rangle$ is 
scalar product in $\E^n$:
   $\langle\e_i,\e_j\rangle=\delta_{ij}$.  We see that
in \eqref{gramm1}
             $$
  g_{ij}=\langle\ac_i,\ac_k\rangle=
\langle \sum _m a^m_i\e_m,\sum_n a^n_j\e_n\rangle=
      a^m_i\delta_{mn}a^n_i=(A^T\cdot A)_{ij}\Rightarrow \det G=(\det A)^2\,,
            $$                     
where $G=||g_{ij}||$. Comparing with formula 
\eqref{parallelepipinn-dimspace1}
we come to formula:
          \begin{equation}\label{grammformula}
           Vol(\Pi_{\ac_i})=\sqrt{\det g_{ik}}
             \end{equation}
This formula is called Gram formula,
and   the matrix $G=||g_{ik}||$ is called 
Gram matrix for the vectors
$\{\ac_i\}$.  Gram  formula justifies
equations \eqref{volumelement} and
 \eqref{volumeofriemannianmanifold}
\footnote{
  We see that  $n$-dimensional parallelepiped
    $\Pi_{\{\ac_i\}}$ in new coordinates $t^i$
corresponding to the basis $\{\ac_i\}$
 becomes $n$-dimensional cube,
  Standard Euclidean metric $G=dx^i\delta_{ik}dx^k$
(in orthonormal basis $\{e_i\}$) transforms to
       $
  G=dx^i\delta_{ik}dx^k=(a^i_m dt^m)\delta_{ik}(a^k_ndt^n)=(A^TA)_{mn}dt^mdt^n
       $
and 
       $$
{\rm Volume}\Pi_{\{\ac_i\}}=\int_{\x\in \Pi}dx^1\dots dx^n=
       \int_{0\leq t_i\leq 1}\sqrt {G}dt^1\dots dt^n=
   \sqrt {\det G}=\sqrt{\det{A^TA}}\,.
       $$
}.

{\footnotesize
{\bf Remark} One can easy see that 
formula \eqref{grammformula}
works for arbitrary $n$-dimensional parallelogramm in $m$-dimensional space.
Indeed if ${\a_1,\dots,\a_n}$ are just arbitrary $n$ vectors
 in $m$-dimensional Euclidean space
 then if $n<m$, the formula \eqref{parallelepipinn-dimspace1} 
s failed (matrix $A$ is $m\times n$ matrix), but formula \eqref{grammformula}
works. For example the area  of parallelogram formed by arbitrary
vectors $\ac_1,\ac_2$ in $\E^n$ is equal to
           $$
        \sqrt{
 \det \begin{pmatrix}
    g_{11}& g_{12}\cr g_{21}& g_{22} 
          \end{pmatrix}
            }=
        \sqrt{
 \det \begin{pmatrix}
    \langle\ac_1,\ac_1\rangle&\langle\ac_1,\ac_1\rangle \cr
   \langle\ac_1,\ac_1\rangle &\langle\ac_1,\ac_1\rangle  
          \end{pmatrix}
            }\,.
           $$
}






\subsubsection { Examples of calculating volume element}


Consider first very simple example: Volume element of plane in Cartesian coordinates,
metric $g=dx^2+dy^2$. Volume element  is equal to
                      $$
         \sqrt {\det g}dxdy=
         \sqrt
           {
         \det
         \left(
    \begin{array}{cc}
  1 & 0 \\
  0&  1 \\
\end{array}
\right)}dxdy=dxdy
                      $$
  Volume of the domain $D$ is equal to
              $$
          V(D)=\int_D\sqrt {\det g}dxdy=\int_D dxdy
              $$

If we go to polar coordinates:
 \begin{equation}\label{polarcoord1}
    x=r\cos\varphi, y=r\sin\varphi
\end{equation}
Then we have for metric:
              $$
G=dr^2+r^2d\varphi^2
              $$
because
               \begin{equation}\label{polarknow}
       dx^2+dy^2=(dr\cos\varphi-r\sin\varphi d\varphi)^2+
    (dr\sin\varphi+r\cos\varphi d\varphi)^2
       =dr^2+r^2 d\varphi^2
              \end{equation}
Volume element in polar coordinates is equal to
  $$
           \sqrt {\det g}drd\varphi=
         \sqrt
           {
         \det
         \left(
    \begin{array}{cc}
  1 & 0 \\
  0&  r^2 \\
\end{array}
\right)}drd\varphi=rdrd\varphi\,.
  $$

\medskip
\m

{\it Lobachesvky plane.}

 In coordinates $x,y$ ($y>0$) metric $G={dx^2+dy^2\over y^2}$,
 the corresponding matrix $G=\begin{pmatrix}
      {1/y^2} & _0 \\
   _0 & {1/y^2} \
 \end{pmatrix}$. Volume element is equal to
 $\sqrt {\det g} dxdy={dxdy\over y^2}$\,.



 {  \it Sphere in stereographic coordinates}
 In stereographic coordinates 
                     \begin{equation}\label{stereographic}
    G={4R^4(du^2+dv^2)\over (R+u^2+v^2)^2}
\end{equation}

(It is isometric to the sphere of the radius $R$ without 
North pole in stereographic coordinates (see the Homeworks.))

   Calculate its volume element and volume.
            It is easy to see that:
                    \begin{equation}\label{}
    G=\left(
    \begin{array}{cc}
  {4R^4\over (R^2+u^2+v^2)^2} & 0 \\
  0&  {4R^4\over (R^2+u^2+v^2)^2} \\
\end{array}
\right)
\qquad
  \det g= {16R^8\over (R^2+u^2+v^2)^4}
\end{equation}
and volume element is equal to  $\sqrt {\det g}dudv= 
{4R^4dudv\over (R^2+u^2+v^2)^2}$

One can calculate volume in coordinates $u,v$ but it is better to
consider homothety  $u\to Ru, v\to Rv$ and polar coordinates:
$u=Rr\cos\varphi, v=Rr\sin\varphi$.
Then volume form is equal to
$\sqrt {\det g}dudv= 
{4R^4dudv\over (R^2+u^2+v^2)^2}={4R^2rdr d \varphi\over (1+r^2)^2}$.

Now calculation of integral becomes easy:
                  $$
    V=\int   {4R^2rdr d \varphi\over (1+r^2)^2}=8\pi R^2 \int_0^\infty {rdr\over (1+r^2)^2}=
                  4\pi  R^2 \int_0^\infty {du\over (1+u)^2}=4\pi R^2\,.
                  $$
\medskip

   Domain in Lobachevsky plane.

    Consider in Lobachevsky plane the domain $D_a$ such that
                            \begin{equation}\label
          {triangleinlobachevskyplane}
               D_a=\{x,y\colon \quad   x^2+y^2\geq 1, |y|\leq a\}\,, 
                 (|a|\leq 1)\,.
                            \end{equation}
 {\bf Remark} Note that vertical lines and half-circle are geodesics.
   One can see that the distance between these lines tends to zero.
   (We will studly it later). If we 
denote by $A$ a point $(a,\sqrt {1-a^2})$ and by $B$ the point 
 $(-a,\sqrt {1-a^2})$, then the domain $D_a$ can be cinsidered
as a `triangle' with vertices at the point $A,B,C$ where $C$
is a point at infinity. The meaning of this remark we will study later.


   One can calculate the area of this domain, using area
form on Lobachevsky plane
                                \begin{equation}\label{areaoftirangle}
        V(D_a)=\int_{-a\leq y\leq a\,, x^2+y^2\geq 1}{dxdy\over y^2}=
                  2\arcsin{a}
                   \end{equation}
(See in detail Homework)
  We will discuss later the geometrical meaning of this formula.


                  \medskip
  {\it Segment of the sphere.}

    Consider sphere of the radius $a$ in Euclidean space with standard Riemannian metric
                       $$
            a^2 d\theta^2+a^2\sin^2\theta d\varphi^2
                       $$
  This metric is nothing but first quadratic form   on the sphere (see \eqref{formula forfirstformsphere}).
   The volume element is
                             $$
           \sqrt {\det g}d\theta  d\varphi=
         \sqrt
           {
         \det
         \left(
    \begin{array}{cc}
  a^2 & 0 \\
  0&  a^2\sin \theta \\
\end{array}
\right)}d\theta d\varphi=a^2\sin \theta d\theta d\varphi
                             $$
 Now calculate the volume of the segment of the sphere between two parallel planes,
 i.e. domain restricted  by parallels $\theta_1\leq \theta \leq \theta_0$:
   Denote by  $h$ be the height of this segment. One can see that
                 $$
               h=a\cos\theta_0-a\cos\theta_1=a(\cos\theta_0-a\cos \theta_1)
                $$
   There is remarkable formula which express the area of segment via the height $h$:
       $$
   V= \int_{\theta_1\leq \theta \leq \theta_0}\left (a^2\sin \theta \right)d\theta d\varphi=
         \int_{\theta^0}^{\theta^1}\left(\int_0^{2\pi}\left(a^2\sin \theta\right) d\varphi\right)d\theta=
                $$
\begin{equation}\label{areaoftheseqment}
       \int_{\theta^1}^{\theta^0} 2\pi a^2\sin\theta d\theta=2\pi a^2 (\cos \theta_0-\cos\theta_1)=
            2\pi a (a\cos \theta_0-a\\cos \theta_1)=2\pi a h
\end{equation}
 E.g. for all the sphere $h=2a$. We come to $S=4\pi a^2$.
It is remarkable formula: area of the segment is a polynomial function of radius of the sphere
 and height (Compare with formula for length of the arc of the circle)

%\end{document} 28 Bebruary  2018

%\end{document} 26 February   2017
%\end{document} % 13 february 2019


\section {Covariant differentiaion. Connection. Levi Civita  Connection on Riemannian manifold}

\subsection {Differentiation of vector field along the vector field.---Affine connection}

How to differentiate vector fields on a (smooth )manifold  $M$?


Recall  the differentiation  of functions on a (smooth )manifold  $M$.







Let $\bf X=X^i(x){\bf e}_i(x)={\p\over \p x^i}$ be a vector field on $M$.
Recall that vector field
\footnote{here like always we suppose by default the summation over repeated indices.
E.g.$\X=X^i{\bf e}_i$ is nothing but
$\X=\sum_{i=1}^nX^i{\bf e}_i$}
 $\bf X=X^i{\bf e}_i$ defines at the
every point $x_0$ an infinitesimal curve: $x^i(t)=x^i_0+tX^i$
(More exactly the equivalence class $[\gamma(t)]_\X $
of curves $x^i(t)=x^i_0+tX^i+\dots$).


Let $f$ be an arbitrary (smooth) function on $M$ and $\X=X^i{\p\over \p x^i}$.
 Then derivative
of function $f$ along vector field $\X=X^i{\p\over \p x^i}$ is equal to
                              $$
             \p_{\bf X}f= \nabla_{\bf X}f
                  =X^i{\p f\over \p x^i}
                                $$
The geometrical meaning of this definition is following:
If $\X$ is a velocity vector of the curve $x^i(t)$ at the point $x^i_0=x^i(t)$ at the "time"
$t=0$ then the value of the derivative $\nabla_{\bf X}f$ at the point $x^i_0=x^i(0)$
is equal just to the derivative by $t$ of the function $f(x^i(t))$ at the "time" $t=0$:
\begin{equation}\label{meaningofderivative}
{\rm if}\quad
    X^i(x)\big\vert_{x_0=x(0)}={dx^i(t)\over dt}\big\vert_{t=0},\quad
    {\rm then}\quad
\nabla_\X f\big\vert_{x^i=x^i(0)}=
{d\over dt}f\left(x^i\left(t\right)\right)\big\vert_{t=0}
\end{equation}


{\bf Remark} In the course of Geometry and Differentiable Manifolds the operator
 of taking derivation of function along the vector field
was denoted by "$\p_{\bf X}f$". In this course we prefer to denote it by "$\nabla_{\bf X}f$"
to have the uniform notation for both operators of taking derivation of functions and vector fields
along the vector field.


One can see that the operation $\nabla_X$ on the 
space $C^\infty(M)$ (space of smooth functions on the manifold)
 satisfies the following conditions:
\begin{itemize}
\item
   $\nabla_{\X}\left(\lambda f+\mu g\right)=
   \lambda\nabla_{\X}f+\mu\nabla_{\X}g$
 where $\lambda,\mu\in \R$ (linearity over numbers )

\item
 $\nabla_{h\X+g\bf Y}(f)=h\nabla_{\X}(f)+g\nabla_{\bf Y}(f)$
                   (linearity over the space of functions)


\item

  $\nabla_\X(\lambda fg)=f\nabla_\X(\lambda g)+g\nabla_\X(\lambda f)$                       (Leibnitz rule)

\begin{equation}\label{conditionsforderivative}
\end{equation}

\end{itemize}

{\bf Remark}
One can prove that these properties characterize  vector fields:operator on smooth functions
obeying the conditions above is a vector field.
(You will have a detailed analysis of this statement in the course of Differentiable Manifolds.)

\m

How to define differentiation of vector fields along vector fields.

The formula \eqref{meaningofderivative} cannot be generalised straightforwardly because
vectors at the point $x_0$ and $x_0+t X$ are vectors from different vector spaces.
(We cannot substract the vector from one vector space from the
vector from the another vector space, because {\it apriori}
we cannot compare vectors from different vector space.
One have to define an operation of transport of vectors from the space
$T_{x_0}M$ to the point $T_{x_0+tX}M$ defining the transport
from the point $T_{x_0}M$ to the point $T_{x_0+tX}M$).

Try to define the operation $\nabla$ on vector fields such
that conditions \eqref{conditionsforderivative} above  be satisfied.



\subsubsection{Definition of connection. Christoffel symbols of connection}
\label{defintionofconnection}

{\bf Definition} Affine connection on $M$ is the {\it operation} $\nabla$
which assigns to every vector field $\bf X$ a linear map, (but not necessarily $C(M)$-linear map!)
(i.e. a map which is linear over numbers not necessarily over functions)
 $\nabla_{\bf X}$
 on the space  of
vector fields on $M$:
\begin{equation}\label{defofconnection}
  \nabla_{\bf X}\left(\lambda{\bf Y}+\mu{\bf Z}\right)=
   \lambda\nabla_{\bf X}{\bf Y}+\mu\nabla_{\bf X}{\bf Z},\qquad
   \hbox{for every $\lambda,\mu\in \R$}
\end{equation}
(Compare the first condition in \eqref{conditionsforderivative}).

\noindent which satisfies the following conditions:


\begin{itemize}

\item

for arbitrary (smooth) functions $f,g$ on $M$
\begin{equation}\label{linearityonfunctions}
  \nabla_{f\bf X+g\bf Y}\left({\bf Z}\right)=
   f\nabla_{\bf X}\left({\bf Z}\right)+
   g\nabla_{\bf Y}\left({\bf Z}\right)\qquad
   \hbox {($C^\infty(M)$-linearity)}
\end{equation}


(compare with second condition in \eqref{conditionsforderivative})

\item
for arbitrary function $f$


\begin{equation}\label{leibnitzrule}
  \nabla_{\bf X} \left( f\bf Y \right)=
   \left(\nabla_{\bf X}f\right){\bf Y}+
   f\nabla_{\bf X}\left({\bf Y}\right)\qquad
   \hbox {(Leibnitz rule)}
\end{equation}
Recall that $\nabla_\X f$ is just usual derivative of a function $f$ along vector field:
$\nabla_\X f=\p_\X f$.

(Compare with Leibnitz rule in \eqref{conditionsforderivative}).



{\it The operation $\nabla_\X\Y$ is called  covariant derivative
of vector field $\Y$ along the vector field $\X$.}


\end{itemize}



  Write down explicit formulae in a given local coordinates $\{x^i\}$ ($i=1,2,\dots,n$) on manifold $M$.

Let
         $$
      \X=X^i\e_i=X^i{\p\over \p x^i}\,\quad \Y=Y^i\e_i=Y^i{\p\over \p x^i}\,\quad
         $$
The basis vector fields ${\p\over \p x^i}$ we denote sometimes 
by $\p_i$ sometimes by $\e_i$


  Using properties above one can see that
\begin{equation}\label{explicitexpression}
  \nabla_{\bf X}{\bf Y}=\nabla_{X^i\p_i}{Y^k \p_k}
  =X^i\left(\nabla_i\left(Y^k\p_k\right)\right),\qquad
  \hbox{where $\nabla_i=\nabla_{\p_i}$}
\end{equation}
Then  according to \eqref{linearityonfunctions}
                 $$
           \nabla_i
            \left(
             Y^k \p_k
            \right)=
              \nabla_i
              \left(Y^k\right)\p_k+
            Y^k \nabla_i \p_k
                   $$

 Decompose the vector field  $\nabla_i \p_k$ over the basis $\p_i$:
             \begin{equation}\label{cristoffelinlocalcoordiantes1}
                \nabla_i \p_k=\Gamma_{ik}^m\p_m
              \end{equation}
             and
\begin{equation}\label{covariantderivative}
    \nabla_i\left({Y^k \p_k}\right)=
    {\p Y^k(x)\over \p x^i}{\p}_k+Y^k\Gamma_{ik}^m\p_m,
\end{equation}
   \begin{equation}\label{covariantderivative2}
    \nabla_{\bf X} {\bf Y}=
    X^i{\p Y^m(x)\over \p x^i}\p_m+X^iY^k\Gamma_{ik}^m\p_m,\quad
    \end{equation}

    In components
           \begin{equation}\label{covderivincomponents}
             \left( \nabla_{\bf X} {\bf Y}\right)^m=
    X^i\left({\p Y^m(x)\over \p x^i}+Y^k\Gamma_{ik}^m\right)
           \end{equation}
    Coefficients $\{\Gamma_{ik}^m\}$ are called {\it Christoffel symbols} in coordinates $\{x^i\}$.
These coefficients define covariant derivative---{\bf connection}.


If operation of taking covariant derivative is given we say that the connection is given on the manifold.
Later it will be explained why we us the word "connection"


We see from the formula above that to define covariant derivative of vector fields, connection,
we have to define Christoffel symbols in local coordinates.

\subsubsection {Transformation of Christoffel symbols for an arbitrary connection}

 Let $\nabla$ be a connection on manifold $M$.
  Let $\{\Gamma^i_{km}\}$ be Christoffel symbols of this connection in  given local coordinates $\{x^i\}$.
  Then according \eqref{cristoffelinlocalcoordiantes1} and \eqref{covariantderivative} we have
                 $$
              \nabla_\X\Y=X^m{\p Y^i\over \p x^m}{\p\over \p x^i}+X^m\Gamma^i_{mk}Y^k{\p\over \p x^i},
                 $$
and in particularly
               $$
               \Gamma^i_{mk}\p_i=\nabla_{\p_m}\p_k
               $$
  Use this relation to calculate Christoffel symbols in new coordinates $x^{i'}$
               $$
            \Gamma^{i'}_{m'k'}\p_{i'}=
            \nabla_{\p_m'}\p_{k'}
               $$
  We have that $\p_{m'}={\p\over \p x^{m'}}={\p x^m\over \p x^{m'}}{\p\over \p x^{m}}= {\p x^m\over \p x^{m'}}\p_m $.
  Hence due to properties \eqref{linearityonfunctions}, \eqref{leibnitzrule} we have
    $$
          \Gamma^{i'}_{m'k'}\p_{i'}=
            \nabla_{\p_{m'}}\p_{k'}=\nabla_{\p_m'}\left({\p x^k\over \p x^{k'}}\p_{k}\right)=
            \left({\p x^k\over \p x^{k'}}\right)\nabla_{\p_m'}\p_{k}+
            {\p\over \p x^{m'}}\left({\p x^k\over \p x^{k'}}\right)\p_{k}=
    $$
              $$
    \left({\p x^k\over \p x^{k'}}\right)\nabla_{{\p x^m\over \p x^{m'}}\p_m}\p_{k}+
            {\p^2  x^k\over \p x^{m'}\p x^{k'}}\p_{k}=
                      {\p x^k\over \p x^{k'}}
          {\p x^m\over \p x^{m'}}
          \nabla_{\p_m}\p_{k}+
           {\p^2  x^k\over \p x^{m'}\p x^{k'}}\p_{k}
          $$
          $$
     {\p x^k\over \p x^{k'}}
          {\p x^m\over \p x^{m'}}
          \Gamma_{mk}^i\p_i+{\p^2  x^k\over \p x^{m'}\p x^{k'}}\p_{k}=
          {\p x^k\over \p x^{k'}}
          {\p x^m\over \p x^{m'}}\Gamma_{mk}^i{\p x^{i'}\over \p x^i}\p_{i'}+
          {\p^2  x^k\over \p x^{m'}\p x^{k'}}{\p x^{i'}\over \p x^k}\p_{i'}
          $$
Comparing the first and the last term in this formula we come to the transformation law:

\m

If $\{\Gamma^i_{km}\}$ are Christoffel symbols of the connection $\nabla$ in  local coordinates $\{x^i\}$
and $\{\Gamma^{i'}_{k'm'}\}$ are Christoffel symbols of this connection  in  new local coordinates $\{x^{i'}\}$
then
 \begin{equation}\label{formualfortransformationofconnection}
    \Gamma^{i'}_{k'm'}=
          {\p x^k\over \p x^{k'}}
          {\p x^m\over \p x^{m'}}
          {\p x^{i'}\over \p x^i}
          \Gamma_{km}^i+
    {\p^2  x^r\over \p x^{k'}\p x^{m'}}{\p x^{i'}\over \p x^r}
 \end{equation}


{\bf Remark}  Christoffel symbols do not transform as tensor.
If the second term is equal to zero, i.e. transformation of coordinates are linear
(see the Proposition on flat connections)  then the transformation rule above is the the same as a
transformation rule for tensors of the type $\begin{pmatrix} 1\cr 2\cr\end{pmatrix}$
(see the formula \eqref{ruleoftransformationofarbitrarytensors}).
 In general case this is not true. Christoffel symbols do not transform as tensor
 under arbitrary non-linear coordinate transformation: see the second term in the formula above.

%\end{document}  % 3 March


{\footnotesize
{\bf Remark} On the other hand note that {\it difference of two arbitrary connections is a tensor}.
If $\Gamma^{i}_{km}$ and $\tilde \Gamma^i_{km}$ are corresponding Chrstoffel symbols then it follows
from \eqref{ruleoftransformationofarbitrarytensors}) that their difference
$T^i_{km}=\Gamma^{i}_{km}-\tilde\Gamma^{i}_{km}$ transforms as a tensor:
                        \begin{equation*}
T^{i'}_{k'm'}=\Gamma^{i'}_{k'm'}-\tilde\Gamma^{i'}_{k'm'}=
{\p x^k\over \p x^{k'}}
          {\p x^m\over \p x^{m'}}
          {\p x^{i'}\over \p x^i}
          \left(
          \Gamma_{km}^i-\tilde\Gamma_{km}^i\right)=
             {\p x^k\over \p x^{k'}}
          {\p x^m\over \p x^{m'}}
          {\p x^{i'}\over \p x^i}T^i_{km}
                                  \end{equation*}
                        (See for detail the Homework 5.)

}
%28 February (2015)

%end{document}  % 3 March

% 4 March (2016)

\subsubsection {Canonical flat affine connection }\label{canonicalflatconnection}

  It follows from the properties of connection that it is suffice
   to define connection at vector fields which form basis at the every point
   using \eqref{cristoffelinlocalcoordiantes1}, i.e. to define Christoffel symbols of this connection.

   {\bf Example} Consider $n$-dimensional Euclidean space $\E^n$ with Cartesian coordinates
   $\{x^1,\dots,x^n\}$.

   Define connection such that all Christoffel symbols
    are equal to zero in these Cartesian coordinates $\{x^i\}$.
\begin{equation}\label{flatconnection1}
  \nabla_{\e_i}\e_k=\Gamma_{ik}^m\e_m=0,\quad   \Gamma_{ik}^m=0
\end{equation}
Does  this mean that Christoffel symbols are equal to zero in
an arbitrary Cartesian coordinates if they equal to zero in given Cartesian coordinates?

 Does this mean that  Christoffel symbols of this connection 
equal to zero in arbitrary coordinates system?


  it follows from transformation rules 
\eqref{formualfortransformationofconnection} for Christoffel symbols
that Christofel symbols vanish also in new coordinates $x^{i'}$
if and only if
                    \begin{equation}\label{additionalterm}
                    {\p^2 x^{i}\over \p x^{m'}\p x^{i'}}=0, \,\,{\rm i.e.}\,\,
                    x^i=b^i+a^i_kx^k
                    \end{equation}
i.e. the relations between new and old coordinates are linear.
We come to simple but very important
\m

 {\bf Proposition} {\it
   Let all Christoffel symbols of a given connection be equal to zero
 in a given coordinate system $\{x^i\}$.
  Then   all Christoffel symbols of this connection are equal to zero in an arbitrary
 coordinate system  $\{x^{i'}\}$ such that the relations between new and old coordinates are linear:
               \begin{equation}\label{additionalterm1}
                   x^{i'}=b^i+a^i_kx^k
                    \end{equation}
 If transformation to new coordinate system is not linear, i.e.  ${\p^2 x^{i}\over \p x^{m'}\p x^{i'}}\not=0$
 then Christoffel symbols of this connection in general are not equal to zero in new
 coordinate system  $\{x^{i'}\}$.}

\m

{\bf Definition} We call connection $\nabla$ flat if there exists coordinate system such that
all Christoffel symbols of this connection are equal to zero in a given coordinate system.

\m

In particular connection \eqref{flatconnection1} has zero Christoffel symbols in arbitrary Cartesian coordinates.



{\bf Corollary} Connection has zero Christoffel symbols in arbitrary Cartesian coordinates
if it has zero Christoffel symbols in a given Cartesian coordinates.


Hence the following definition is correct:

\m


{\bf Definition}  A connection on the Euclidean space
$\E^n$ which Christoffel symbols vanish in Cartesian coordinates
is called {\it canonical flat connection.}


{\bf Remark} {\small   Canonical flat connection in Euclidean space is uniquely defined,
since Cartesian coordinates
are defined globally.  On the other hand
on arbitrary manifold one can define flat connection locally just choosing any arbitrary {\it local}
coordinates and define {\it locally flat connection} by condition that Christoffel symbols
vanish in these local coordinates. This does not mean that one can define flat connection {\it globally.}
 We will study this question after learning transformation law for Christoffel symbols.}


\m
{\bf Remark}   One can see that flat connection is symmetric connection.


\m


  {\bf Example} Consider a connection \eqref{flatconnection1} in $\E^2$. It is a flat connection.
            Calculate Christoffel symbols of this connection in polar coordinates
                       \begin{equation}\label{polarcoordinates}
                        \begin{cases}
                        x=r\cos\varphi\cr
                        y=y\sin\varphi\cr
                        \end{cases}
                        \qquad
                        \begin{cases}
                        r=\sqrt {x^2+y^2}\cr
                        \varphi={\rm arctan\,} {y\over x}\cr
                           \end{cases}
                                \end{equation}
%\end{document} %28 February 2015

Write down Jacobians of transformations---matrices of partial derivatives:
                      \begin{equation}\label{polarcoordinates2}
                      \begin{pmatrix}
                      x_r &y_r\cr
                      x_\varphi &y_\varphi\cr
                      \end{pmatrix}=
                      \begin{pmatrix}
                      \cos\varphi &\sin\varphi\cr
                      -r\sin\varphi & r\cos\varphi\cr
                      \end{pmatrix},\qquad
                      \begin{pmatrix}
                      r_x &\varphi_x\cr
                      r_y &\varphi_y\cr
                      \end{pmatrix}=
                      \begin{pmatrix}
                      {x\over \sqrt{x^2+y^2}} &-{y\over {x^2+y^2}}\cr
                      {y\over \sqrt{x^2+y^2}} & {x\over {x^2+y^2}}\cr
                      \end{pmatrix}
                      \end{equation}
According \eqref{formualfortransformationofconnection} and since Chrsitoffel symbols are equal to zero in Cartesian coordinates
$(x,y)$ we have
             \begin{equation}\label{christoffelsymbolsinpolarcoordinates}
\Gamma^{i'}_{k'm'}=
    {\p^2  x^r\over \p x^{k'}\p x^{m'}}{\p x^{i'}\over \p x^r},
                             \end{equation}
           where $(x^{1},x^{2})=(x,y)$ and $(x^{1'},x^{2'})=(r,\varphi)$. Now using \eqref{polarcoordinates2}
           we have
            $$
         \Gamma^r_{rr}={\p^2 x\over \p r\p r}{\p r\over \p x}+{\p^2 y\over \p r\p r}{\p r\over \p y}=0
            $$
              $$
        \Gamma^r_{r \varphi}=\Gamma^r_{\varphi r}={\p^2 x\over \p r\p \varphi}{\p r\over \p x}+
        {\p^2 y\over \p r\p \varphi}{\p r\over \p y}=-\sin\varphi\cos\varphi+\sin\varphi\cos\varphi=0\,.
              $$
               $$
           \Gamma^r_{\varphi \varphi}={\p^2 x\over \p \varphi\p \varphi}{\p r\over \p x}+
        {\p^2 y\over \p r\p \varphi}{\p r\over \p y}=-x{x\over r}-y{y\over r}=-r\,.
               $$
                $$
                 \Gamma^\varphi_{rr}=
        {\p^2 x\over \p r\p r}{\p \varphi\over \p x}+{\p^2 y\over \p r\p r}{\p \varphi\over \p y}=0\,.
                $$
                $$
     \Gamma^\varphi_{\varphi r}=
     \Gamma^\varphi_{r\varphi}=
        {\p^2 x\over \p r\p \varphi}{\p \varphi\over \p x}+{\p^2 y\over \p r\p \varphi}{\p \varphi\over \p y}=
        -\sin\varphi{-y\over r^2}+\cos\varphi{x\over r^2}={1\over r}
                $$
             \begin{equation}\label{clhristoffelsymbolsinpolarcoordinates2}
                \Gamma^\varphi_{\varphi\varphi}=
        {\p^2 x\over \p \varphi\p \varphi}{\p \varphi\over \p x}+
        {\p^2 y\over \p \varphi\p \varphi}{\p \varphi\over \p y}=
        -x{-y\over r^2}-y{x\over r^2}=0\,.
                     \end{equation}
       Hence  we have that the covariant derivative \eqref{flatconnection1}
        in polar coordinates has the following appearance
                  \begin{equation*}
                  \nabla_r \p_r=
                  \Gamma_{rr}^r\p_r+
                  \Gamma_{rr}^\varphi\p_\varphi=0\,,
                  , \qquad
                  \nabla_r\p_\varphi=\Gamma_{r\varphi}^r\p_r+
                  \Gamma_{r\varphi}^\varphi\p_\varphi={\p_\varphi\over r}
                  \end{equation*}
                  \begin{equation}\label{covariantderivativeinpolarcoordinates}
                  \nabla_\varphi \p_r=
                  \Gamma_{\varphi r}^r\p_r+
                  \Gamma_{\varphi r}^\varphi\p_\varphi={\p_\varphi\over r}, \qquad
                  \nabla_\varphi\p_\varphi=\Gamma_{\varphi\varphi}^r\p_r+
                  \Gamma_{\varphi\varphi}^\varphi\p_\varphi=-r\p_r
                  \end{equation}
{\bf Remark}  Later when we study geodesics we will learn a very quick method to calculate
Christoffel symbols.

% 4 March
%\end{document}  % 6 March 2018


\subsection {Connection induced on the surfaces}

   Let $M$ be a manifold embedded in 
 Euclidean space%\footnote{We know that every $n$-dimensional 
%manifold  can be embedded
 %   in $2n+1$-dimensional Euclidean space}.
    Canonical flat connection on $\E^N$  induces the 
connection on surface in the following way.

    Let $\X,\Y$ be tangent vector fields to the surface $M$ and $\nabla^{\rm can.flat}$ a canonical flat
    connection in $\E^N$.
    In general
            \begin{equation}\label{nottangent!}
                \Z=    \nabla^{\rm can.flat}_\X\Y\quad \hbox{is not tangent to manifold $M$}
                \end{equation}
                Consider its decomposition on two vector fields:
                          \begin{equation}\label{nottangent!2}
             \Z=\Z_{tangent}+\Z_\bot,       \nabla^{\rm can.flat}_\X,\Y=
                    \left(\nabla^{\rm can.flat}_\X\Y\right)_{tangent}+
                    \left(\nabla^{\rm can.flat}_\X\Y\right)_{\bot}\,,
                \end{equation}
where $\Z_{\bot}$ is a component of vector which is orthogonal to the surface $M$ and $\Z_{||}$ is a component which
is tangent to the surface. Define an induced connection $\nabla^{M}$ on the surface $M$ by the following formula
\begin{equation}\label{inducedconnection1}
\nabla^{M}\colon\qquad \nabla^{M}_\X\Y \colon =
\left(\nabla^{\rm can.flat}_\X\Y\right)_{tangent}
\end{equation}

One can see that this formula really defines the connection on surface
  $M$, i.e. the operation defined by this relation obeys all 
axioms of connection.  Indeed it is easy to see that
for arbitrary vector fields $\X$ and $\Y$, the vector field
$\nabla^M_\X \Y$ is tangent vector field,
and this operation obeys relations
\eqref{defofconnection}, 
\eqref{linearityonfunctions}
and \eqref{leibnitzrule}. For example check Leibnitz rule:
             $$
  \nabla^M_\X(f\Y)=\left(\nabla^{\rm can.flat}_\X(f\Y)\right)_{tangent}=
  \left(\p_\X f\Y+f\nabla^{\rm can.flat}_\X\Y\right)_{tangent}=
            $$
            \begin{equation*}
     \p_\X f\Y+f\nabla^{\rm can.flat}_\X\Y_{tangent}=
     \p_\X f\Y+f\nabla^M_\X\Y\,.
             \end{equation*}


    We mainly apply this construction for $2$-dimensional manifolds
(surfaces) in $\E^3$.


\subsubsection {Calculation of induced connection on surfaces in $\E^3$.}

Let $\r=\r(u,v)$ be a surface in $\E^3$. Let $\nabla^{\rm can.flat}$ be a flat connection in $\E^3$.
   Then
\begin{equation}\label{inducedconnection1}
\nabla^{M}\colon\qquad \nabla^{M}_\X\Y \colon =\left(\nabla^{\rm can.flat}_\X\Y\right)_{||}=
\nabla^{\rm can.flat}_\X\Y-
\n(\nabla^{\rm can.flat}_\X\Y,\n),
\end{equation}
where $\n$ is normal unit vector field to $M$. Consider a special example

  {\bf Example} (Induced connection on sphere)
   Consider a sphere of the radius $R$ in $\E^3$:
              $$
           \r(\theta,\varphi)\colon \begin{cases}
           x=R\sin\theta \cos\varphi\cr
           y=R\sin\theta \sin\varphi\cr
           z=R\cos\theta\cr
           \end{cases}
              $$
then
           $$
      \r_{\theta}=\begin{pmatrix}
            R\cos\theta\cos\varphi\cr
            R\cos\theta\sin\varphi\cr
            -R\sin\theta\cr
                  \end{pmatrix},
       \r_{\varphi}=\begin{pmatrix}
            -R\sin\theta\sin\varphi\cr
            R\sin\theta\cos\varphi\cr
            0\cr
                  \end{pmatrix},
                  \n=\begin{pmatrix}
            sin\theta\cos\varphi\cr
            sin\theta\sin\varphi\cr
            \cos\theta\cr
                  \end{pmatrix},
           $$
where $\r_\theta={\p \r(\theta,\varphi)\over \p \theta}$,
$\r_\varphi={\p \r(\theta,\varphi)\over \p \varphi}$ are basic tangent vectors and $\n$ is normal unit vector.

Calculate an induced connection  $\nabla$ on the sphere.

First calculate $\nabla_{\p_\theta}\p_\theta$.
\begin{equation*}
    \nabla_{\p_\theta}\p_\theta=\left({\p \r_\theta\over \p\theta}\right)_{tangent}=
    \left(\r_{\theta\theta}\right)_{tangent}.
\end{equation*}
On the other hand one can see that $\r_{\theta\theta}=\begin{pmatrix}
            -Rsin\theta\cos\varphi\cr
            -Rsin\theta\sin\varphi\cr
            -R\cos\theta\cr
                  \end{pmatrix}=-R\n$ is proportional to normal vector, i.e. $\left(\r_{\theta\theta}\right)_{tangent}=0$.
                  We come to
\begin{equation}\label{inducedonsphere4}
    \nabla_{\p_\theta}\p_\theta=\left(\r_{\theta\theta}\right)_{tangent}=0 \Rightarrow
    \Gamma^\theta_{\theta\theta}=\Gamma^\varphi_{\theta\theta}=0\,.
\end{equation}

{\bf Remark}  Notice that equation \eqref{inducedonsphere4}
follows from teh fact that $\r_{\theta\theta}$
is centripetal acceleration which is directed along 
$\r\sim \n$.

\smallskip 



Now calculate $\nabla_{\p_\theta}\p_\varphi$ and $\nabla_{\p_\varphi}\p_\theta$.
\begin{equation*}
    \nabla_{\p_\theta}\p_\varphi=\left({\p \r_\varphi\over \p\theta}\right)_{tangent}=
    \left(\r_{\theta\varphi}\right)_{tangent},\,
    \nabla_{\p_\varphi}\p_\theta=\left({\p \r_\theta\over \p\varphi}\right)_{tangent}=
    \left(\r_{\varphi\theta}\right)_{tangent}
\end{equation*}
 We have
           $$
           \nabla_{\p_\theta}\p_\varphi=\nabla_{\p_\varphi}\p_\theta=\left(\r_{\varphi\theta}\right)_{tangent}=
           \begin{pmatrix}
            -R\cos\theta\sin\varphi\cr
            R\cos\theta\cos\varphi\cr
            0\cr
                  \end{pmatrix}_{tangent}.
           $$
We see that the vector $\r_{\varphi\theta}$ is orthogonal to $\n$:
\begin{equation*}
     \langle\r_{\varphi\theta},\n\rangle=-R\cos\theta\sin\varphi \sin\theta\cos\varphi+
     R\cos\theta\cos\varphi\sin\theta\sin\varphi=0.
\end{equation*}
Hence
\begin{equation*}
 \nabla_{\p_\theta}\p_\varphi=\nabla_{\p_\varphi}\p_\theta=
 \left(\r_{\varphi\theta}\right)_{tangent}=\r_{\varphi\theta}=
 \begin{pmatrix}
            -R\cos\theta\sin\varphi\cr
            R\cos\theta\cos\varphi\cr
            0\cr
                  \end{pmatrix}=
                  {\rm cotan\,}\theta  \r_\varphi\,.
\end{equation*}
We come to

\begin{equation}\label{inducedonsphere8}
    \nabla_{\p_\theta}\p_\varphi=\nabla_{\p_\varphi}\p_\theta={\rm cotan\,}\theta  \p_\varphi \Rightarrow
    \Gamma^\theta_{\theta\varphi}=\Gamma^\theta_{\varphi\theta}=0,\,\,
    \Gamma^\varphi_{\theta\varphi}=\Gamma^\varphi_{\varphi\theta}={\rm cotan\,}\theta
\end{equation}

Finally calculate $\nabla_{\p_\varphi} \p_\varphi$
                $$
            \nabla_{\p_\varphi} \p_\varphi=\left(\r_{\varphi\varphi}\right)_{tangent}=
            \left(
            \begin{pmatrix}
            -R\sin\theta\cos\varphi\cr
            -R\sin\theta\sin\varphi\cr
            0\cr
                  \end{pmatrix}
                  \right)_{tangent}
                $$
Projecting on the tangent vectors to the sphere (see \eqref{inducedconnection1}) we have
     $$
\nabla_{\p_\varphi} \p_\varphi=\left(\r_{\varphi\varphi}\right)_{tangent}=
\r_{\varphi\varphi}-\n\langle\n,\r_{\varphi\varphi}\rangle=
         $$
         $$
     \begin{pmatrix}
            -R\sin\theta\cos\varphi\cr
            -R\sin\theta\sin\varphi\cr
            0\cr
                  \end{pmatrix}-
                  \begin{pmatrix}
            \sin\theta\cos\varphi\cr
            \sin\theta\sin\varphi\cr
            \cos\theta\cr
                  \end{pmatrix}\left(-R\sin\theta\cos\varphi\sin\theta\cos\varphi-
                  R\sin\theta\sin\varphi\sin\theta\sin\varphi\right)=
     $$
         $$
      -\sin\theta\cos\theta
      \begin{pmatrix}
            R\cos\theta\cos\varphi\cr
            R\cos\theta\sin\varphi\cr
            -R\sin\theta\cr
                  \end{pmatrix}=-\sin\theta\cos\theta\r_\theta,
         $$
i.e.
\begin{equation}\label{connectiononsphere12}
   \nabla_{\p_\varphi}\p_\varphi=-\sin\theta\cos\theta\r_\theta \Rightarrow
    \Gamma^\theta_{\varphi\varphi}=-\sin\theta\cos\theta,\,\,
    \Gamma^\varphi_{\varphi\varphi}=\Gamma^\varphi_{\varphi\varphi}=0\,.
\end{equation}

%\end{document} % 7 March

\subsection {Levi-Civita connection}


  \subsubsection {Symmetric connection}

 {\bf Definition}. We say that connection is symmetric if its Christoffel symbols $\Gamma^i_{km}$ are symmetric with respect to
  lower indices
  \begin{equation}\label{symmetricconnection}
    \Gamma^i_{km}=\Gamma^i_{mk}
  \end{equation}
  The canonical flat connection and induced connections considered above are symmetric connections.

  \m

  {\small

   \centerline {\it Invariant definition of symmetric connection}

  A connection $\nabla$ is symmetric if for an arbitrary vector fields $\X,\Y$

  \begin{equation}\label{symmetricconnectioninvariant}
    \nabla_\X \Y-\nabla_\Y \X-[\X,\Y]=0
  \end{equation}
  If we apply this definition to basic fields $\p_k,\p_m$ which commute: $[\p_k,\p_m]=0$
  we come to the
  condition
          $$
     \nabla_{\p_k}\p_m-\nabla_{\p_m}\p_k=\Gamma_{mk}^i\p_i-\Gamma_{km}^i\p_i=0
          $$
 and this is the condition  \eqref{symmetricconnection}.
  }



  \subsubsection {Levi-Civita connection. 
Theorem and Explicit formulae}\label{levicivitaconnection1}
Let $(M, G)$ be a Riemannian manifold.

{\bf Definition-Theorem}

{\it A symmetric connection $\nabla$ is called Levi-Civita connection if it is compatible with metric, i.e.
if it preserves the scalar product:
        \begin{equation}\label{connpreservingmetric}
  \p_\X\langle\Y,\Z\rangle=\langle\nabla_\X\Y,\Z\rangle+\langle\Y,\nabla_\X\Z\rangle
        \end{equation}
        for arbitrary vector fields $\X,\Y,\Z$.

  There exists unique levi-Civita connection on the Riemannian manifold.

  In local coordinates Christoffel symbols of Levi-Civita connection are given by the following formulae:
 \begin{equation}\label{levi-civitaformula}
    \Gamma^i_{mk}={1\over 2}g^{ij}\left({\p g_{jm}\over\p x^k}+
    {\p g_{jk}\over\p x^m}-{\p g_{mk}\over \p x^j}\right)\,.
 \end{equation}
 where $G=g_{ik}dx^idx^k$ is Riemannian metric in local coordinates and
 $||g^{ik}||$ is the matrix inverse to the matrix $||g_{ik}||$.

 }

%\end{document}

% 11 March 2016

 {\sl Proof}

{\small Suppose that this connection exists and $\Gamma^i_{mk}$ are its Christoffel symbols.
 Consider vector fields $\X=\p_m,\Y=\p_i$ and $\Z=\p_k$ in \eqref{connpreservingmetric}.
  We have that
            \begin{equation}\label{conditionforallindices}
         \p_mg_{ik}=\langle\Gamma_{mi}^r\p_r,\p_k\rangle+\langle\p_i,\Gamma^r_{mk}\p_r\rangle=
              \Gamma^r_{mi}g_{rk}+g_{ir}\Gamma^r_{mk}\,.
            \end{equation}
   for arbitrary indices $m,i,k$.

    Denote by $\Gamma_{mik}=\Gamma^r_{mi}g_{rk}$ we come to
                $$
          \p_mg_{ik}=\Gamma_{mik}+\Gamma_{mki}, \,\,{\rm i.e.}
                $$
    Now using the symmetricity $\Gamma_{mik}=\Gamma_{imk}$ since $\Gamma_{mi}^k=\Gamma_{im}^k$ we have
                $$
    \Gamma_{mik}=\p_mg_{ik}-\Gamma_{mki}=\p_mg_{ik}-\Gamma_{kmi}=\p_mg_{ik}-\left(\p_kg_{mi}-\Gamma_{kim}\right)=
                $$
                $$
\p_mg_{ik}-\p_kg_{mi}+\Gamma_{kim}=\p_mg_{ik}-\p_kg_{mi}+\Gamma_{ikm}=
\p_mg_{ik}-\p_kg_{mi}+\left(\p_ig_{km}-\Gamma_{imk}\right)=
                $$
                $$
       \p_mg_{ik}-\p_kg_{mi}+\p_ig_{km}-\Gamma_{mik}\,.
                $$
 Hence
                \begin{equation}\label{levi-vivitaforlowerindices}
   \Gamma_{mik}={1\over 2}(\p_mg_{ik}+\p_ig_{mk}-\p_kg_{mi})\Rightarrow \Gamma^k_{im}=
   {1\over 2}g^{kr}\left(\p_mg_{ir}+\p_ig_{mr}-\p_rg_{mi}\right)
                \end{equation}
  We see that if this connection exists then it is given by the formula\eqref{levi-civitaformula}.

  On the other hand one can see that \eqref{levi-civitaformula} obeys the condition
  \eqref{conditionforallindices}. We prove the uniqueness and existence.

since $\nabla_{\p_i}\p_k=\Gamma_{ik}^m\p_m$.}

\m

%\end {document}   % 11 March

Consider examples.

\subsubsection{Levi-Civita connection of $\E^n$}\label{canonicalflat=levicivita}

For Euclidean space $\E^n$ in standard Cartesian coordinates
               $$
               G_{_{\rm Eucl}}=(dx^1)^2+\dots+(dx^n)^2=\delta_{ik}dx^idx^k
               $$
    Components of metric are constants (they are equal to $0$ or $1$). Hence obviously
    Christoffel symbols of Levi-Civita connection in Cartesian coordinates according
    formula \eqref{levi-civitaformula} vanish:
               $$
              \Gamma^I_{km}=0 \hbox {\,\,in Cartesian coordiantes}
               $$
Recalling canonical flat connection (see \ref{canonicalflatconnection})
we come to simple but important observation:

{\bf Observation} Levi-Civita connection coincides with canonical flat connection
in Euclidean space $\E^n$. They have
 vanishing Cristoffel symbols in Cartesian coordinates.

%\end{document} % 6 March 2015

\subsubsection {Levi-Civita connection on $2$-dimensional Riemannian
manifold with metric $G=adu^2+bdv^2$.}

{\bf Example} Consider $2$-dimensional manifold with Riemannian metrics
                  $$
                  G=a(u,v)du^2+b(u,v)dv^2, \qquad
                  G=\begin{pmatrix}
                       g_{11} &g_{12}\\
                       g_{21}  &g_{22}\\
                     \end{pmatrix}=
         \begin{pmatrix}
                       a(u,v) &0\\
                       0  &b(u,v)\\
                     \end{pmatrix}
                  $$
Calculate Christoffel symbols of Levi Civita connection.

 Using  \eqref{levi-vivitaforlowerindices} we see that:
                       \begin{equation}\label{explicitcalculationofchristoffel}
                       \begin{matrix}
        \Gamma_{111}&={1\over 2}\left(\p_{1} g_{11}+\p_{1} g_{11}-\p_{1} g_{11}\right)=
        {1\over 2}\p_{1} g_{11}=&{1\over 2}a_u\\
             &\\
        \Gamma_{211}=\Gamma_{121}&={1\over 2}\left(\p_{1} g_{12}+
        \p_{2} g_{11}-\p_{1} g_{12}\right)=
        {1\over 2}\p_{2} g_{11}=&{1\over 2}a_v\\
           &\\
\Gamma_{221}&={1\over 2}\left(\p_{2} g_{12}+\p_{2} g_{12}-\p_{1} g_{22}\right)=
        -{1\over 2}\p_{1} g_{22}=-&{1\over 2}b_u\\ &\\
        \Gamma_{112}&={1\over 2}\left(\p_{1} g_{12}+\p_{1} g_{12}-\p_{2} g_{11}\right)=
        -{1\over 2}\p_{2} g_{11}=-&{1\over 2}a_v\\&\\
        \Gamma_{122}=\Gamma_{212}&=
        {1\over 2}\left(\p_{2} g_{21}+\p_{1} g_{22}-\p_{2} g_{21}\right)=
        {1\over 2}\p_{1} g_{22}=&{1\over 2}b_u\\&\\
        \Gamma_{222}&={1\over 2}\left(\p_{2} g_{22}+\p_{2} g_{22}-\p_{2} g_{22}\right)=
        {1\over 2}\p_{2} g_{22}=&{1\over 2}b_v\\
                           \end {matrix}
                       \end{equation}
To calculate $\G^i_{km}=g^{ir}\Gamma_{kmr}$ note that for the metric
$a(u,v)du^2+b(u,v)dv^2$
        $$
          G^{-1}=\begin{pmatrix}
                       g^{11} &g^{12}\\
                       g^{21}  &g^{22}\\
                     \end{pmatrix}=
         \begin{pmatrix}
                       {1\over a(u,v)} &0\\
                       0  &{1\over b(u,v)}\\
                     \end{pmatrix}
              $$
Hence
\begin{equation}\label{christophelfortwodimensionalmetric}
      \begin{matrix}
  \Gamma^{1}_{11}=&g^{11}\Gamma_{111}=&{a_u\over 2a},\qquad
  \Gamma^1_{21}=\Gamma^{1}_{12}=&g^{11}\Gamma_{121}=&{a_v\over 2a},\qquad
  \Gamma^{1}_{22}=&g^{11}\Gamma_{221}=&{-b_u\over 2a}\\&&&&&&\\
  \Gamma^{2}_{11}=&g^{22}\Gamma_{112}=&{-a_v\over 2b},\qquad
  \Gamma^2_{21}=\Gamma^{2}_{12}=&g^{22}\Gamma_{122}=&{b_u\over 2b},\qquad
  \Gamma^{2}_{22}=&g^{22}\Gamma_{222}=&{b_v\over 2b}\\
  \end{matrix}
\end{equation}




\subsubsection {Example of the sphere again}

 {Calculate Levi-Civita connection on the sphere.

 On the sphere first quadratic form (Riemannian metric)
 $G=R^2d\theta^2+R^2\sin^2\theta d\varphi^2$.  Hence we
 use calculations from the previous example
 with $a(\theta,\varphi)=R^2, b(\theta,\varphi)=R^2\sin^2 \theta$
  ($u=\theta, v=\varphi$).
 Note that $a_\theta=a_\varphi=b_\varphi=0$.
  Hence only non-trivial components
 of $\Gamma$ will be:
\begin{equation}\label{Christoffelforsphere:}
  \Gamma^{\theta}_{\varphi\varphi}={-b_\theta\over 2a}={-\sin 2\theta\over 2},
  \qquad
  \left(\Gamma_{\varphi\varphi\theta}={-R^2\sin 2\theta\over 2}\right),
       \end{equation}
  \begin{equation}\label{connectionforsphere}
  \Gamma^{\varphi}_{\theta\varphi}=\Gamma^{\varphi}_{\varphi\theta}=
  {b_\theta\over 2b}={\cos\theta \over \sin\theta}\,
  \qquad
  \left(\Gamma_{\theta\varphi\varphi}={R^2\sin 2\theta\over 2}\right)
\end{equation}
All other components are equal to zero:
                 $$
  \Gamma^{\theta}_{\theta\theta}=\Gamma^{\theta}_{\theta\varphi}=\Gamma^{\theta}_{\varphi\theta}=
  \Gamma^{\varphi}_{\theta\theta}=\Gamma^{\varphi}_{\varphi\varphi}=0
             $$

{\bf Remark} Note that Christoffel symbols of Levi-Civita connection on the sphere coincide
with  Christoffel symbols of induced connection calculated
in the subsection "Connection induced on surfaces".
later we will understand the geometrical meaning of this fact.


\subsection {Levi-Civita connection = induced connection on surfaces in $\E^3$}




We know already that
{\it canonical flat connection of Euclidean space is the Levi-Civita connection
of the standard metric on Euclidean space.}
(see section \ref{canonicalflat=levicivita}.)
Now we show that Levi-Civita connection on surfaces in Euclidean space coincides with
 the connection induced on the surfaces by canonical flat connection.
We perform our analysis for surfaces in $\E^3$.

Let $M\colon \r=\r(u,v)$ be a surface in $\E^3$. Let  $G$ be induced Riemannian metric on $M$
and $\nabla$ Levi--Civita connection of this metric.

We know that the induced connection $\nabla^{(M)}$ is defined
    in the following way: for arbitrary vector fields  $\X,\Y$ tangent to the surface $M$,
    $\nabla^{M}_\X\Y$ equals to the projection on the tangent space of the vector field $\nabla^{{\rm can.flat}}_\X\Y$:
                   $$
                   \nabla^{M}_\X\Y =
                   \left(\nabla^{\rm can.flat}_\X\Y\right)_{\rm tangent}\,,
                   $$
where  $\nabla^{\rm can.flat}$ is canonical flat connection
   in $\E^3$ (its Christoffel symbols
   vanish in Cartesian coordinates).
   We denote by $\A_{\rm tangent}$ a projection of the vector $A$ attached at the point of the surface on the
tangent space:
             $
          \A_{\perp}=\A-\n(\A,\n)\,,
             $
($\n$ is normal unit vector field to the surface.)

\m


{\bf Theorem}    {\it Induced connection on the surface $\r=\r(u,v)$ in $\E^3$ coincides with
  Levi-Civita connection of Riemannian metric induced by the canonical metric on Euclidean space $\E^3$.}

\m

{\sl Proof}

Let $\nabla^{M}$ be induced connection on a surface $M$  in $\E^3$ given by equations
$\r=\r(u,v)$. Considering this connection on
the basic vectors $\r_h,\r_v$ we see that it is symmetric connection. Indeed
          $$
      \nabla^{M}_{\p_u}\p_v=\left(\r_{uv}\right)_{\rm tangent}=\left(\r_{vu}\right)_{\rm tangent}=
          \nabla^{M}_{\p_v}\p_u\,. \Rightarrow \Gamma^{u}_{uv}=\Gamma^{u}_{vu},
          \Gamma^{v}_{uv}=\Gamma^{v}_{vu}\,.
          $$
Prove that this connection preserves scalar product on $M$.
For arbitrary tangent vector fields  $\X,\Y,\Z$ we have
         $$
         \p_\X\langle \Y,\Z\rangle_{\E^3}=\langle \nabla_\X ^{\rm can.\,flat}\Y,\Z\rangle_{\E^3}+
    \langle \Y,\nabla_\X ^{\rm can.\,flat}\Z\rangle_{\E^3}\,.
         $$
since canonical flat connection in $\E^3$
preserves Euclidean metric in $\E^3$ (it is evident in Cartesian coordinates).
Now project the equation above on the surface $M$.
If $\A$ is an arbitrary vector attached to the surface and $\A_{\rm tangent}$ is its
 projection on the tangent space to the surface, then
 for every tangent vector $\bf B$ scalar product $\langle\A,{\bf B}\rangle_{\E^3}$ equals to
 the scalar product $\langle\A_{\rm tangent},{\bf B}\rangle_{\E^3}=
 \langle\A_{\rm tangent},{\bf B}\rangle_{M}$
  since vector $\A-\A_{\rm tangent}$
 is orthogonal to the surface.  Hence we deduce from  (2) that  $ \p_\X\langle \Y,\Z\rangle_{M}=$
                    $$
   \langle \left(\nabla_\X ^{\rm can.\,flat}\Y\right)_{\rm tangent},\Z\rangle_{\E^3}+
    \langle \Y,\left(\nabla_\X ^{\rm can.\,flat}\Z\right)_{\rm tangent}\rangle_{\E^3}=
                    \langle \nabla_\X ^{M}\Y,\Z\rangle_{M}+
    \langle \Y,\nabla_\X ^{M}\Z\rangle_{M}\,.
                    $$
 We see that induced connection is symmetric connection which preserves the induced metric.
 Hence due to Levi-Civita Theorem it is unique and is expressed as in the formula \eqref{levi-civitaformula}.

\m




{\bf Remark} {\footnotesize One can easy to reformulate and prove more general statement:
Let $M$ be a submanifold in Riemannian manifold $(E,G)$. Then Levi-Civita connection
of the metric induced on this submanifold coincides with the connection induced on the manifold
by Levi-Civita connection of the metric $G$.}


   \section {Parallel transport and geodesics}

   \subsection{Parallel transport}\label{paralleltransport}

%\end{document}  % 16 March 2018


\subsubsection {Definition}

   Let $M$ be a manifold equipped with affine connection $\nabla$.

   {\bf Definition} Let $C\colon \x(t), t_0\leq t\leq t_1$ 
     be a curve on the manifold $M$,
   starting at the point $\pt_0=x(t_0)$ and ending at the point
    $\pt_1=x(_1)$ ((with coordinates $x^i=x^i(t)$)).
   Let $\X=\X(t_0)$ be an arbitrary tangent  vector 
attached at the initial point $\pt_0=\x_0$ 
(with coordinates $x^i(t_0)$) of the curve $C$, i.e.
  $\X(t_0)\in T_{\pt_0}M$ is a vector tangent to the manifold $M$
  at the point $\pt_0$ with coordinates $x^i(t_0)$.
  (The vector $\X$ is not necessarily tangent to the curve $C$)

  We say that $\X(t)$, $t_0\leq t\leq t_1$ is a parallel transport of the
  vector $\X(t_0)\in T_{\p_0}M$ along the curve $C\colon x^i=x^i(t), t_0\leq t\leq t_1$ if

\begin {itemize}

\item For an arbitrary $t$, $t_0\leq t\leq t$, vector $\X=\X(t)$, ($\X(t)\vert_{t=t_0}=\X(t_0)$)
 is a vector attached at the point
  $\x(t)$ of the curve $C$,
 i.e. $\X(t)$ is a vector tangent to the manifold $M$
  at the point $\x(t)$  of the curve $C$.

  \item The covariant derivative of $\X(t)$ along the curve $C$ equals to zero:

  \begin{equation}\label{conditofparalleltransport}
    {\nabla \X\over dt}=\nabla_\v \X=0\,.
  \end{equation}
 In components: if $X^m(t)$ are components of the vector field $\X(t)$
 and   $v^m(t)$ are components of the velocity vector $\v$ of the curve  $C$ ,
                $$
                \X(t)=X^m(t){\p \over \p x^m}\vert_{\x(t)}\,,\quad
                \v={d\x(t)\over dt}={dx^i\over dt}{\p \over \p x^m}\vert_{\x(t)}
                $$
  then the condition \eqref{conditofparalleltransport} can be rewritten as
  \begin{equation}\label{conditofparalleltransportcomponents}
    {dX^i(t)\over dt}+v^k(t)\Gamma^i_{km}(x^i(t))X^m(t)\equiv 0\,.
  \end{equation}


\end{itemize}

\m

{\bf Remark}    We say sometimes that $\X(t)$ is 
{\it covariantly constant along the curve $C$}
   if $\X(t)$ is parallel transport of the 
vector $\X$ along the curve $C$.
  If we consider Euclidean space with canonical flat 
connection then in Cartesian coordinates
  Christoffel symbols vanish and parallel transport is nothing but
   ${d \X\over dt}=\nabla_\v \X=0$, i.e.
$\X(t)$ is a constant vector.



%  \end{document} %14 March 

%{\bf Remark} Compare this 
%definition of parallel 
%transport with the definition which we consider
%in the course of "Introduction to Geometry" 
%where we consider parallel transport of the vector along the
%curve on the surface embedded in $\E^3$ and 
%define parallel transport by the condition, 
%that only orthogonal component
%of the vector changes during parallel transport, i.e.
%${d\X(t)\over dt}$ is a vector orthogonal to the surface  
%(see the Exercise in the Homework 7).



\subsubsection {Parallel transport is a linear map.
Parallel transport with respect to Levi-Civita connection}
\label{paralleltransport2}




%{\it Dear Students!  The end of the lecture on Tuesday 15 March 
%will appear in Lecture notes the next week
% since it was made essentiall changes.}

%\end{document}

 We usualy consider parallel 
transport on Riemannian manifold with respect to Levi-Civita connection.
If $(M,G)$ is Riemannian manifold
 then we consider parallel transport with respect to 
connection $\nabla$ which is Levi-Civita connection
of the  Riemannian metric $G$.




Consider again curve $C\colon \x=\x(t), t_0\leq t\leq t_1 $ 
on manifold $M$ starting at the point
$\pt_0$ and ending at the point $\pt_1$ (see above).
Let $\X\in T_{\pt_0}$ be an arbitrary tangent 
vector at the point $\pt_0$,
and $\X(t)$ be parallel transport 
\eqref{conditofparalleltransport} of this vector 
along the curve $C$:
              $$
  \X(t)\big\vert_{t=t_0}=\X\,,\quad {\nabla \X(t)\over dt}=0\,.
              $$ 
 Taking value of $\X(t)$ at the final point $\pt_1$ of the curve $C$
we come to the new vector $\X'=\X(t)\big\vert_{t=t_1}$
tangent to the manifold $M$ at the point $\pt_1$. Thus we define
the map between tangent vectors 
at the initial point $\pt_0$ of the curve $C$ 
and tangent vectors at the ending
point $\pt_1$ of this curve:
  \begin{equation}\label{paralleltransportalongthecurve}
 P_{C}\colon \quad   T_{\pt_0}M \ni\X
   \longrightarrow P_C(\X)=\X' \in T_{\pt_1}M\,.
               \end{equation}
Sure this map depends on the curve $C$
which joins starting and ending points
(if we are not in Euclidean space).


\m

{\bf Proposition}  

  Let $C$ be a an arbitrary curve with starting point $\pt_0$
and ending point  $\pt_1$. Then the map 
\eqref{paralleltransportalongthecurve} defines
 linear operator $P_C$ which does not depend on parameterisation
of the curve, providing the initial and ending 
points of the curve are not swapped; i.e.
 operator is not changed under reparameterisations of the
which do not change the orientation of the curve.
(One can say that linear operator $P_C$ is an operator
defined for oriented curve, since we fix initial and ending poitns
of the curve.):
         \begin{equation}
\label{parallelransportlienaroperator} 
         P_C(\l\X_1+\mu\X_2)=\l P_C(\X_1)+\mu P_C(\X_2)\,. 
         \end{equation}
In the case if connection $\nabla$ is Levi-Civita connection,
then $P_C$ is an orthogonal operator:
 for two arbitrary vectors $\X,\Y\in T_{\p_0}M$
   \begin{equation}\label{paralleltransportorthogonaloperator} 
      \langle\X,\Y\rangle_{\pt_0}=
      \langle \X',\Y'\rangle_{\pt_1}\,\quad \X'=P_C(\X)\in T_{p_1}M\,,
                                      \quad \Y'=P_C(\Y)\in T_{p_1}M\,,
         \end{equation}
where as usual
$\langle\,\,,\,\,\rangle_{\pt}$ is the scalar product at the point $\pt$.

In particular the length of the vector is preserved
during parallel transport.


\smallskip

{\sl Proof}
The fact that it is a linear map follows immediately 
from the fact that
differential equations
\eqref{conditofparalleltransport}
 is linear equation.
If  vector fields $\X(t),\Y(t)$ are
 covariantly constant along the curve $C$, i.e. they obey  differential
equation \eqref{conditofparalleltransportcomponents}, 
then their linear
combination $\l \X(t)+\mu \Y(t)$ obeys this equation, also,
This implies \eqref{parallelransportlienaroperator}. 

The fact that the map \eqref{paralleltransportalongthecurve} 
does not depend on the parameterisation
(if it does not change the orinetation)
follows from  differential equation
\eqref{conditofparalleltransport} 
(or the same equation in components,
the equation \eqref{conditofparalleltransportcomponents}.)
Indeed let $t=t(\tau)$, $\tau_0\leq \tau\leq \tau_1$, 
$t(\tau_0)=t_0, 
t(\tau_1)=\tau_1$ be another parameterisation
of the curve $C$, which does not change orientation,
i.e. initial and ending  points
of the curve do not interchange.
 Then multiplying the equation 
\eqref{conditofparalleltransport} 
on ${dt\over d\tau}$
and using the fact that velocity $\v'(\tau)=t_\tau\v(t)$
we come to the differential equation
in new parameterisation: 
  $$
{\nabla \X(t(\tau))\over d\tau}=
  \nabla_{\v'}\X(t(\tau))=
  \nabla_{t_\tau\v}\X(t(\tau))=
  {dt\over d\tau}\nabla_\v\X(t)=
  {dt\over d\tau}{\nabla \X(t)\over dt}=0\,,
    $$
or in components
         \begin{equation}
\label{conditofparalleltransportcomponentsnewparameter}
    {dX^i(t(\tau))\over d\tau}+v'^k(t(\tau))\Gamma^i_{km}(x^i(t(\tau)))X^m(t(\tau))\equiv 0\,.
  \end{equation}
The functions $X^(t(\tau))$ with the 
same initial conditions are the solutions of this equation.



It remains to prove that $P_C$ is orthogonal operator.

It follows immediately from the definition 
\eqref{conditofparalleltransport} of a parallel transport
and the definition  \eqref{connpreservingmetric} 
of Levi-Civita connection that during parallel transport
  the scalar product $\langle\X(t),\Y(t)\rangle_{\x(t)}$
is preserved:
\begin{equation}\label{lengthdoesnotchangeduringparalleltransport}
{d\over dt}\langle\X(t),\Y(t)\rangle=
 \p_\v\langle\X(t),\Y(t)\rangle=
\left\langle\nabla_\v \X(t), \Y(t)\right\rangle+
\left\langle\X(t), \nabla_\v \Y(t)\right\rangle=
 \langle 0, \Y(t)\rangle+
 \langle \X(t), 0\rangle
    =0\,.
     \end{equation}
This implies \eqref{paralleltransportorthogonaloperator}. 


\subsection {Geodesics}

\subsubsection {Definition.  Geodesics on 
Riemannian manifold}\label{geodesicsofriem1}


%\end{document} % 8 April 2016

%\end{document} % 16 March 2019

    We are going to define geodesics in Riemannian manifold.

        Geodesic is generalisation of straight line.

           Straight line is

   \begin {itemize}
  
  \item the shortest

   \item the straightest: i.e. the velocity vector at all the points
is the 'same'

   \item trajectory of free particle
  
 \end {itemize}

Any of these properties may be generalised. We will 
  focus attention on the second one.


Let $M$ be Riemannian manifold equipped with 
Levi-Civita connection $\nabla$.

{\bf Definition} A parameterised curve  
$C\colon\quad x^i=x^i(t)$ in Rieamannian manifold
is called geodesic if velocity vector
$\v(t)\colon v^i(t)={dx^i(t)\over dt}$ is covariantly constant along 
this curve, i.e. parallel transport of velocity vector along the curve
preserves the velocity vector:
\begin{equation}\label{geodesdef1}
    \nabla_\v \v={\nabla\v\over dt}={dv^i(t)\over dt}+v^k(t)\Gamma^i_{km}(x(t))v^m(t)=0,\,\,\,{i.e.}
\end{equation}
\begin{equation}\label{geodesdef2}
    {d^2x^i(t)\over dt^2}+{d x^k(t)\over dt}\Gamma^i_{km}(x(t)){d x^m(t)\over dt}=0\,.
\end{equation}
  These are linear second order differential equations. One can prove that
  this equations have solution and it is unique\footnote{this is true under additional technical conditions which we do not
  discuss here} for an arbitrary initial data
  ($x^i(t_0)=x^i_0, \dot x^i(t_0)=\dot x^i_0$. )



In other words the curve $C\colon x(t)$ is a geodesic 
if parallel transport of velocity vector 
along the curve is  a velocity vector at any point of the curve.
\m

{\bf Remark}  One can see that our definition of geodesic works 
for arbitrary connection. However we will consider here
%{\tt Ici changer: make definition only for Levi-Civita}
only geodesics on Riemannian manifold,
defined only with Levi-Civita connection.

\m

Since velocity vector of the geodesics on Riemannian manifold  
at any point is a parallel transport
with the Levi-Civita connection, hence due to Proposition 
above (see equation \eqref{parallelransportlienaroperator},
\eqref{paralleltransportorthogonaloperator}  and
\eqref{lengthdoesnotchangeduringparalleltransport})
the length of the velocity vector remains constant:

{\bf Proposition}   {\it If $C \colon\,\, \x(t)$ is a 
geodesics on Riemannian manifold then the length of
velocity vector is preserved along the geodesic.}


{\sl Proof}   Since the connection is Levi-Civita connection then it
preserves scalar product of tangent vectors,
(see \eqref{connpreservingmetric}) in particularly the length
of the velocity vector $\v$:

\m

{\bf Example 1}  {\it Geodesics of Euclidean space.}
 In Cartesian coordinates
Christoffel symbols of Levi-Civita connection vanish, and differential equation
\eqref{geodesdef1}, \eqref{geodesdef2} are reduced to equation
  \begin{equation}\label{geodesicstraightlines2}
    {d^2x^i(t)\over dt^2}=0\,,\Rightarrow {dx^i(t)\over dt}=v^i \Rightarrow x^i=x^i_0+v^it\,.
  \end{equation}
We come to straight lines.

  \m



{\bf Example 2}  {\it Geodesics of  cylindrical surface}
One can see that if Riemannian metric $G=G_{ik}du^idv^k$ have constant coefficients
  in  coordinates $u^i$ then Christoffel symbols of Levi-Civita connection vanish in these coordinates,
  (see formula \eqref{levi-civitaformula}) and according to \eqref{geodesicstraightlines2}
   geodesics are  ``straight lines'' in coordinates $u^i$. In particular this is a case for cylinder:
  If surface of cylinder is given by equation
  $\begin{cases} x=a\cos\varphi\cr y=a\sin\varphi \cr z=h\cr
  \end{cases}$
  then Riemannian metric is equal to $G=a^2d\varphi^2+dh^2$ and we come to equations:
    \begin{equation}\label{geodesicstraightlines4}
    \begin {cases}
    {d^2 \varphi(t)\over dt^2}=0 \cr
    {d^2 h(t)\over dt^2}=0 \cr
    \end{cases}\Rightarrow
         \begin {cases}
    {d \varphi(t)\over dt}=\Omega \cr
    {d h(t)\over dt}=c \cr
    \end{cases} \Rightarrow
     \begin {cases}
    \varphi(t)=\varphi_0+\Omega t\cr
    h(t)=h_0+tc \cr
    \end{cases}\,.
       \end{equation}
    In general case we come to helix:
     \begin{equation}\label{geodesicofcylinder2}
      \begin{cases} x=a\cos\varphi (t)=a\cos \left(\varphi_0+\Omega t\right)\cr
       y=a\sin\varphi(t)=a\sin \left(\varphi_0+\Omega t\right) \cr z=h(t)=h_0+ct\cr
  \end{cases}
  \end{equation}
  If $c=0$ then geodesics are circles $x^2+y^2=a^2, z=h_0$. If angular velocity $\Omega=0$
  then geodesics are vertical lines $x=x_0, y=y_0, z=h_0+ct$.


%\end{document} % 12 march 2015

   \subsubsection {Geodesics and Lagrangians of "free" particle on Riemannian manifold.}


     \centerline {\it Lagrangian and Euler-Lagrange equations}

   A function $L=L(x,\dot x)$ on points and velocity vectors on manifold $M$
   is a {\it Lagrangian} on manifold $M$.


   We assign  to  Lagrangian $L=L(x,\dot x)$ the following second order differential equations
\begin{equation}\label{ELequations1}
    {d\over dt}\left({\p L\over \p \dot x^i}\right)={\p L\over \p  x^i}
\end{equation}
In detail
\begin{equation}\label{ELequationsindetail}
    {d\over dt}\left({\p L\over \p \dot x^i}\right)={\p^2 L\over \p x^m \p \dot x^i}\dot x^m+
    {\p^2 L\over \p \dot x^m \p \dot x^i}{\buildrel \cdot\cdot\over x^m}
    =
    {\p L\over \p  x^i}\,.
\end{equation}

  These equations are called {\it Euler-Lagrange equations} 
of the Lagrangian $L$. E.g. for particle in
external field with potential energy
 $U=U(\r)$ in Cartesian coordinates in $\E^n$,
               \begin{equation}\label{LagrangianinCartesiancoordinates}
         L(\r,\dot \r)=\hbox {kinetic energy}-\hbox{potential energy}
 ={m\v^2\over 2}-U(\r)\,.
               \end{equation}\label{LagangianinCartesiancoordinates}
                


  We will explain later the variational origin of these equations
    
 {\bf Remark} {\footnotesize To every mechanical system one can put 
in correspondence  a Lagrangian on configuration space.
      The dynamics of the system is described by Euler-Lagrange equations.
      The advantage of Lagrangian approach is that it works in an arbitrary coordinate system:
      Euler-Lagrange equations are invariant
      with respect to changing of coordinates since 
they arise from variational principe.  E.g. one can easy to see that
  Euler-Lagrange equations  
                \begin{equation}\label{NewtonequationsinCartesiancoordinates}
  m{d\v(t)\over dt}=-{\p U(\r)\over \p \r_i}
               \end{equation}\label{LagangianinCartesiancoordinates}
                  
 for Lagrangian \eqref{LagrangianinCartesiancoordinates}
coincide with standard Newton equations
\eqref{NewtonequationsinCartesiancoordinates}
 in Cartesian coordinates.
However Newton equations 
 \eqref{NewtonequationsinCartesiancoordinates} 
do not survive under arbitrary coordinate transformations
contrary to Euler-Lagrange equations\label{ELequations1}.}

\m

    \centerline {\it Lagrangian of "free" particle}

\smallskip

     Let $(M,G)$, $G=g_{ik}dx^idx^k$ be a Riemannian manifold.


     {\bf Definition} We say that {\it Lagrangian} $L=L(x,\dot x)$ 
 is the Lagrangian of a a `free' particle on the
     Riemannian manifold $M$ if
      \begin{equation}\label{frepaticlelagrangian}
        L={g_{ik}\dot x^i\dot x^k\over 2}
      \end{equation}

{\bf Example} "Free" particle in Euclidean space.
    Consider $\E^3$ with standard metric $G=dx^2+dy^2+dz^2$
      \begin{equation}\label{freeparticleineuclideanspace}
        L={g_{ik}\dot x^i\dot x^k\over 2}={\dot x^2+\dot y^2+\dot z^2\over 2}
      \end{equation}
Note that this is the Lagrangian that describes the dynamics 
of a free particle.

\m

{\bf Example} A `free' particle on a sphere.

The metric on the sphere of radius $R$ is
$G=R^2d\theta^2+R^2\sin^2\theta d\varphi^2$.
Respectively for the Lagrangian of "free" particle we have
\begin{equation}\label{freeparticleonthesphere}
        L={g_{ik}\dot x^i\dot x^k\over 2}={R^2\dot\theta^2+R^2\sin^2\theta\dot \varphi^2\over 2}
      \end{equation}


\m

\centerline {\it Equations of geodesics and Euler-Lagrange equations}

\m


{\bf Theorem}. {\it Euler-Lagrange equations of the 
Lagrangian of a free particle are equivalent
to the second order differential equations
for geodesics.}

This Theorem makes very easy calculations for Christoffel indices.

\m

 This Theorem can be proved by direct calculations.

Calculate Euler-Lagrange equations \eqref{ELequations1} for the Lagrangian \eqref{frepaticlelagrangian}:
               $$
    {d\over dt}\left({\p L\over \p \dot x^i}\right)=
    {d\over dt}\left({\p \left({g_{mk}\dot x^m\dot x^k\over 2}\right)\over \p \dot x^i}\right)=
     {d\over dt}\left(g_{ik}\dot x^k\right)=
    g_{ik}{\buildrel \cdot\cdot\over x{^k}}+{\p g_{ik}\over \p x^m}\dot x^m\dot x^k
               $$
and
            $$
            {\p L\over \p  x^i}={\p \left({g_{mk}\dot x^m\dot x^k\over 2}\right)\over \p  x^i}=
            {1\over 2}{\p g_{mk}\over \p x^i}\dot x^m\dot x^k.
            $$
Hence we have
           $$
 {d\over dt}\left({\p L\over \p \dot x^i}\right)=
 g_{ik}{\buildrel \cdot\cdot\over x{^k}}+{\p g_{ik}\over \p x^m}\dot x^m\dot x^k=
    {\p L\over \p  x^i}={1\over 2}{\p g_{mk}\over \p x^i}\dot x^m\dot x^k\,,
           $$
  i.e.
       $$
 g_{ik}{\buildrel \cdot\cdot\over x{^k}}+\p_mg_{ik}\dot x^m\dot x^k=
 {1\over 2}\p_i g_{mk}\dot x^m\dot x^k\,.
       $$
 Note that    $\p_mg_{ik}\dot x^m\dot x^k={1\over 2}\left(\p_mg_{ik}\dot x^m\dot x^k+\p_kg_{im}\dot x^m\dot x^k\right)$.
Hence we come to equation:
               $$
               g_{ik}{d^2 x^k\over dt^2}+
               {1\over 2}\left(
 \p_mg_{ik}+\p_kg_{im}-\p_ig_{mk}
                  \right)\dot x^m\dot x^k
               $$
 Multiplying on the inverse matrix $g^{ik}$ we come
\begin{equation}\label{ELequationforgeodesic1}
    {d^2 x^i\over dt^2}+{1\over 2}g^{ij}\left({\p g_{jm}\over\p x^k}+
    {\p g_{jk}\over\p x^m}-{\p g_{mk}\over \p x^j}\right){dx^m\over dt}{dx^k\over dt}=0\,.
\end{equation}
We recognize here Christoffel symbols of Levi-Civita connection (see \eqref{levi-civitaformula}) and we rewrite
this equation as
   \begin{equation}\label{ELequationforgeodesic2}
    {d^2 x^i\over dt^2}+{dx^m\over dt}\Gamma^i_{mk}{dx^k\over dt}=0\,.
\end{equation}
This is nothing but the equation \eqref{geodesdef1}.

Applications of this Theorem: calculation of Christoffel symbols of Levi-Civita connection.



\subsubsection { Calculations of Christoffel 
symbols and geodesics using the Lagrangians of a free particle.}
It turns out that 
  equation \eqref{ELequationforgeodesic2}
is the very effective tool to  caluclatie Christoffel symbols
of Levi-Civita connection.

{\it Examples in this subsection will be calculated in detail on tutorial
  (see Homework 7)}

Consider two examples: We calculate Levi-Civita connection on sphere in $\E^3$ and on Lobachevsky plane
using Lagrangians and find geodesics.

1) {\it Sphere of the radius $R$ in $\E^3$}:


   Lagrangian of "free" particle on the sphere is given by \eqref{freeparticleonthesphere}:
    $$
    L={R^2\dot \theta^2+R^2\sin^2\theta\dot \varphi^2\over 2}
    $$
Euler-Lagrange equations defining geodesics are
          \begin{equation}\label{EL eqforsphere}
  {d\over dt}\left({\p L\over \p  \dot\theta}\right)-
  {\p L\over \p \theta}=
  {d\over dt} \left(R^2{\buildrel \cdot\over \theta}\right)-
  R^2\sin\theta\cos\theta\dot \varphi^2\Rightarrow  {\buildrel \cdot\cdot\over \theta}-
  \sin\theta\cos\theta\dot \varphi^2=0\,,
          \end{equation}
              $$
{d\over dt}\left({\p L\over \p  \dot\varphi}\right)-
  {\p L\over \p \varphi}=
  {d\over dt} \left(R^2{\sin^2\theta\dot\varphi}\right)=0\Rightarrow
  {\buildrel \cdot\cdot\over \varphi}+
  2{\rm cotan\,}\theta\dot\theta\dot\varphi=0\,.
                            $$
Comparing Euler-Lagrange equations with equations for geodesic in 
terms of Christoffel symbols:
         $$
   {\buildrel \cdot\cdot\over \theta}+\Gamma^\theta_{\theta\theta}\dot\theta^2+
   2\Gamma^\theta_{\theta\varphi}\dot\theta\dot\varphi+\Gamma^\theta_{\varphi\varphi}\dot\varphi^2=0,
         $$
    $$
  {\buildrel \cdot\cdot\over \varphi}+\Gamma^\varphi_{\theta\theta}\dot\theta^2+
   2\Gamma^\varphi_{\theta\varphi}\dot\theta\dot\varphi+\Gamma^\varphi_{\varphi\varphi}\dot\varphi^2=0
    $$
    we come to
 \begin{equation}\label{christsymbolsthroughlagrangiansforsphere1}
    \Gamma^\theta_{\theta\theta}=
    \Gamma^\theta_{\theta\varphi}=\Gamma^\theta_{\varphi\theta}=0\,,
    \Gamma^\theta_{\varphi\varphi}=-\sin\theta\cos\theta\,,
 \end{equation}
 \begin{equation}\label{christsymbolsthroughlagrangiansforsphere2}
    \Gamma^\varphi_{\theta\theta}=\Gamma^\varphi_{\varphi\varphi}=0,\,
    \Gamma^\varphi_{\theta\varphi}=\Gamma^\varphi_{\varphi\theta}={\rm cotan\, }\theta\,.
 \end{equation}
 (Compare with previous calculations for connection in subsections 2.2.1 and 2.3.4)

%\end{document} % 20 March 2019

{\footnotesize  We know already and we will prove
later in a elegant way that geodesics on the sphere are great circles.
(see subsection \ref{greatcircles} below).
 Consider another technically more difficult but 
straightforward proof of this fact.
To find geodesics one have to solve second order 
differential equations \eqref{ELequationforgeodesic2}

One can see that the great circles: $\varphi=\varphi_0$, $\theta=\theta_0+t$ are solutions of
second order differential equations \eqref{EL eqforsphere} with initial conditions
\begin{equation}\label{initialconditions}
  \theta(t)\big\vert_{t=0}=\theta_0, \dot\theta(t)\big\vert_{t=0}=1,\quad
  \varphi(t)\big\vert_{t=0}=\varphi_0, \dot\theta(t)\big\vert_{t=0}=0\,.
\end{equation}
The rotation of the sphere is isometry, which does not change Levi-Civta connection.
Hence an arbitrary great circle is geodesic.

Prove that an arbitrary geodesic is an arc of great circle.
Let the curve $\theta=\theta(t),\varphi=\varphi(t)$,
$0\leq t\leq t_1$ be geodesic. Rotating the sphere
we can come to the curve $\theta=\theta'(t),\varphi=\varphi'(t)$, $0\leq t\leq t_1$
such that  velocity vector at the initial time is direccted along meridian, i.e.
initial conditions are
\begin{equation}\label{initialconditionsnew}
  \theta'(t)\big\vert_{t=0}=\theta_0, \dot\theta'(t)\big\vert_{t=0}=a,\quad
  \varphi'(t)\big\vert_{t=0}=\varphi_0, \dot\varphi'(t)\big\vert_{t=0}=0\,.
\end{equation}
(Compare with initial conditions \eqref{initialconditions})
Second order differential equations with boundary conditions for coordinates and velocities at $t=0$ have unique
solution. The solutions of second order differential equations \eqref{EL eqforsphere} with initial conditions
\eqref{initialconditionsnew} is a curve
                $\theta'(t)=\theta_0+at$, $\varphi'(t)=\varphi_0$. It is great circle.
                Hence initial curve the geodesic $\theta=\theta(t),\varphi=\varphi(t)$,
$0\leq t\leq t_1$ is an arc of great circle too.

This is another proof that geodesics are great circles.}


\m


2) {\it Lobachevsky plane.}


   Lagrangian of "free" particle on the Lobachevsky plane with metric $G={dx^2+dy^2\over y^2}$ is
      $$
   L={1\over 2}{\dot x^2+\dot y^2\over y^2}.
      $$
Euler-Lagrange equations are

  $$
  {\p L\over \p x}=0={d\over dt}{\p L\over \p \dot x}={d\over dt}\left({\dot x\over y^2}\right)=
            {{\buildrel \cdot\cdot\over x}\over y^2}-{2\dot x\dot y\over y^3}, {\rm i.e.}\quad
            {\buildrel \cdot\cdot\over x}-{2\dot x\dot y\over y}=0\,,
             $$
             $$
              {\p L\over \p y}=-{\dot x^2+\dot y^2\over y^3}=
              {d\over dt}{\p L\over \p \dot y}={d\over dt}\left({\dot y\over y^2}\right)=
            {{\buildrel \cdot\cdot\over y}\over y^2}-{2\dot y^2\over y^3}, {\rm i.e.}\quad
            {\buildrel \cdot\cdot\over y}+{\dot x^2\over y}-{\dot y^2\over y}=0\,.
            $$
Comparing these equations with equations for geodesics:
${\buildrel \cdot\cdot\over x^i}-\dot x^k\Gamma^i_{km}\dot x^m=0$
($i=1,2$, $x=x^1,y=x^2$) we come to
                $$
\Gamma^{x}_{xx}=0,
\Gamma^{x}_{xy}=\Gamma^{x}_{yx}=-{1\over y},\,
\Gamma^{x}_{yy}=0,\,\Gamma^{y}_{xx}={1\over y}, \Gamma^{y}_{xy}=\Gamma^{y}_{yx}=0, \Gamma^{y}_{yy}=-{1\over y}\,.
              \hbox{\finish}  $$
{\footnotesize In  a similar way as for a sphere one can find geodesics on Lobachevsky plane.
First we note that vertical rays are geodesics.  Then using the inversions with centre on the absolute
one can see that arcs of the circles with centre at the absolute ($y=0$) are geodesics too.
}

See also examples in Homework 6 

%\end{document} % 11 April
%\end{document}


\subsubsection {Un-parameterised geodesic}

We defined a geodesic as a parameterised curve such that the velocity vector
is covariantly constant along the curve.

\m

What happens if we change the parameterisation of the curve?


\m

  Another  question:  Suppose a tangent vector to the curve remains tangent to the curve during parallel transport.
  Is it true that this curve (in a suitable parameterisation) becomes geodesic?

\m

 {\bf Definition}  We call un-parameterised curve geodesic 
if under suitable parameterisation it obeys
   the equation \eqref{geodesdef1} for geodesics.

\m


   Let $C$--be un-parameterised geodesic.
   Then the following statement is valid.

\m

   {\bf Proposition} {\it A curve $C$ (un-parameterised) is geodesic
   if an only if a non-zero vector tangent to the curve 
remains tangent to the curve
   during parallel transport.}



   \m
   {\sl Proof}.  Let $\A$ be tangent vector at the point $\pt\in C$ of the curve.
  Parallel transport does not depend on parametersiation of the curve
  (see subsection 3.1.2, equation \eqref{paralleltransportalongthecurve}).  
Choose a suitable parameterisation $x^i=x^i(t)$  such that
 $x^i(t)$ obeys the equations \eqref{geodesdef1} for geodesics,
 i.e. the velocity vector $\v(t)$ is covariantly constant along the curve:
 $\nabla_\v\v=0$. If $\A(t_0)=c\v(t_0)$
 at the given point $\pt$ ($c$ is a scalar coefficient) then due to linearity $\A(t)=c\v(t)$ is a parallel
 transport of the vector $\A$. The vector $\A(t)$ is tangent to the curve since it is proportional to velocity vector.
  We proved that any tangent vector remains tangent during parallel transport.

Now prove the converse: Let $\A(t)$ be a parallel 
transport of non-zero vector and it 
is proportional to velocity, i.e.
    $\A(t)=c(t)\v(t)$. Thus
      $$
{\nabla \A(t)\over dt}=\nabla_{\v}\A=0\,,\,
      $$
  Choose a reparameterisation $t=t(\tau)$ such that
  ${dt(\tau)\over d\tau}=c(t)$. 
In the new parameterisation the velocity vector
  $\v'(\tau)={dt(\tau)\over d\tau}\v(t(\tau))=
  c(t)\v(t)=\A(t(\tau))$ and
     $$
{d \v'\over d\tau}=\nabla_{\v'}\v'=
\nabla_{t_\tau\v}(t_\tau\v)=
t_\tau\nabla_{\v}(t_\tau\v)=
t_\tau\nabla_{\v}(c \v)=
t_\tau\nabla_\v\A=0\,.
     $$
We come to parameterisation such that 
velocity vector remains covariantly constant,
i.e. it is parameterised geodesic.
(This is 
reparameterisation invariance of parallel transport
\eqref{conditofparalleltransport}
(see \eqref{conditofparalleltransportcomponentsnewparameter}
).  
}
  Thus we come to parameterised geodesic.
  Hence $C$ is a geodesic.

\m

{\bf Remark} In particularly it follows from the Proposition above the following important observation:

Let  $C$ is un-parameterised geodesic, $x^i(t)$ be its  arbitrary parameterisation
and $\v(t)$ be velocity vector in this parameterisation. Then the velocity vector remains parallel to the curve
since it is a tangent vector.

In spite of the fact that velocity vector is not covariantly constant along the curve,
i.e. it will not remain velocity vector during parallel transport, since
it will be remain tangent to the curve during parallel transport.




{\footnotesize
{\bf Remark} One can see
that if $x^i=x^i(t)$ is geodesic in
 an arbitrary parameterisation and
$s=s(t)$ is a natural parameter (which defines the length of the curve) then
$x^i(t(s))$ is parameterised geodesic.
}

\subsubsection {Parallel transport of vectors along geodesics}

We already now that during parallel transport along curve with respect to Levi-Civita connection
scalar product of vectors, i.e. lengths of vectors and angle between them does not change
(see subsection 3.1.2, equations
\eqref{paralleltransportalongthecurve}
and \eqref{paralleltransportorthogonaloperator}). 
This remark makes easy to calculate parallel transport
of vectors along geodesics in Riemannian manifold. Indeed let $C$ a geodesic (in general un-parameterised)
and a vectors $\X(t)$ is attached to the point $\pt_1\in C$ on the curve $C$.   In the special case if $\X$
is a tangent vector to geodesic $C$ then during parallel transport it remains tangent, i.e. proportional
to velocity vector:
                   \begin{equation}\label{paralleltransalonggeodesic2}
                    \X(t)=a(t)\v(t)\,.
                    \end{equation}
                   Here $\v(t)={d\r(t)\over dt}$ and $\r=\r(t)$ is
                   an {\it arbitrary parameterisation} of geodesic $C$.
Note that in general $t$ is not parameter such that $\r=\r(t)$ is parameterised geodesic; $t$
is an arbitrary parameter. In the special case if $t$ is a parameter such that $\r=\r(t)$
is parameterised geodesic then velocity vector remains velocity vector during parallel transport,
i.e.  $\X(t)=a\v(t)$ where $a$ is not dependent on $t$.

To calculate  the dependence of coefficient $a$ on $t$ 
in \eqref{paralleltransalonggeodesic2}
we note that the length of the vector is not changed 
(see equation \eqref{paralleltransportorthogonaloperator} 
in the section \ref{paralleltransport2},
i.e.
             \begin{equation}\label{paralleltransalonggeodesic2}
\langle\X(t),\X(t)\rangle=\langle a(t)\v(t),a(t)\v(t)\rangle=a^2(t)|\v(t)|^2 ={\rm constant }
                    \end{equation}

%\end{document} % 22 March (16 March 2015)


%\end{document} % 24 March 2017

\subsubsection  {Geodesics on surfaces in $\E^3$}\label{greatcircles}

 Let $M\colon \r=\r(u,v)$ be a surface in $\E^3$.  Let $G_M$ be induced Riemannian metric
 and $\nabla$ a Levi-Civita connection on $M$. We consider on $M$ Levi-Civita connection of the metric $G_M$.

  Let $C$ be an arbitrary  geodesic and $\v(t)={d\r(t)\over dt}$ the velocity vector.
  According to the definition of geodesic
        $\nabla_\v\v=0$. On the other hand we know that Levi-Civita connection
coincides with the connection induced on the surface by canonical flat connection in $\E^3$
 (see the Theorem in subsection 2.4). Hence
                 \begin{equation}\label{proofofcoicidence ofconnections}
    \nabla_\v\v=0=\nabla^M_\v\v=\left(\nabla^{\rm can.flat}_\v\v\right)_{\rm tangent}
\end{equation}
In Cartesian coordinates  $\nabla^{\rm can.flat}_\v\v=\p_\v \v={d\over dt}\v(u(t),v(t))={d^2\r(t)\over dt^2}=\ac$.

Hence according to \eqref{proofofcoicidence ofconnections} the tangent component of acceleration equals to zero.

Converse if for the curve $\r(t)=\r(u(t),v(t))$ the acceleration vector $\ac(t)$ is orthogonal to the surface
then due to  \eqref{proofofcoicidence ofconnections} $\nabla_\v\v=0$.

We come to very beautiful observation:

{\bf Theorem} {\it The acceleration vector of an curve  $\r=\r(u(t),v(t))$ on $M$
 is orthogonal to the surface $M$ if and only if this curve is geodesic.

\m

In other words due to Newton second law particle moves along along geodesic on the surface if and only if
the force is
orthogonal to the surface.}

%\end{document} %22 March
\m


One can very easy using this Proposition to calculate geodesics of cylinder and sphere.


{\it Geodesic on the cylinder}

Let $\r(h(t),\varphi(t))$ be a geodesic on the cylinder $\begin{cases}x=a\cos\varphi\cr y=a\sin\varphi \cr z=h\cr
\end{cases}$.
    We have  $\v={d\r\over dt}=\begin{pmatrix}-a\dot \varphi\sin\varphi\cr a\dot
    \varphi\cos\varphi\cr \dot h\end{pmatrix}$
and for acceleration:
                       \begin{equation*}\label{}
    \ac={d\v\over dt}= \underbrace
              {
    \begin{pmatrix}-a{\buildrel \cdot\cdot\over \varphi\,}\sin\varphi\cr
    a{\buildrel \cdot\cdot\over \varphi\,}\cos\varphi\cr
    {\buildrel \cdot\cdot\over h\,}\cr
    \end{pmatrix}
         }_{\hbox {tangent acceleration}}+
         \underbrace
            {
    \begin{pmatrix}-a\dot \varphi^2\cos\varphi\cr
     -a\dot\varphi^2\sin\varphi\cr
         0\cr\end{pmatrix}
    }_{\hbox {normal acceleration}}
\end{equation*}
Since tangential acceleration equals to zero hence
 ${d^2\varphi\over dt^2}=0$, $\varphi(t)=\varphi_0+\Omega t$, and
 ${d^2h\over dt^2}=0$, $h(t)=h_0+ct$.
Normal acceleration is centripetal acceleration of the rotation over circle with constant speed
(projection on the plane $OXY$). The geodesic is helix.
(Compare these calculations with calculations of geodesics of cylinder in the last example of
section \ref{geodesicsofriem1}: see \eqref{geodesicofcylinder2}.)

\m

 \centerline  {\it Geodesics on sphere}

\smallskip

\noindent Let $\r=\r(\theta(t),\varphi(t))$ be a geodesic on the sphere  of the radius $a$:
$\r(\theta,\varphi)\colon \begin{cases}x=a\sin\theta\cos\varphi\cr y=a\sin\theta\sin\varphi \cr z=a\cos\theta\cr
\end{cases}$


Consider the vector product of the vectors $\r(t)$ and velocity vector $\v(t)$
${\bf M}(t)=\r(t)\times \v(t)$.
Acceleration vector $\ac(t)$ is proportional to the $\r(t)$
since due to Proposition it is orthogonal to the surface of the sphere. This implies that
$\M(t)$ is constant vector:

               \begin{equation}\label{momentofmotion}
{d\over dt}{\bf M}(t)={d\over dt}\left(\r(t)\times \v(t)\right)=
\left(\v(t)\times \v(t)\right)+\left(\r(t)\times \ac(t)\right)=0
              \end{equation}
         We have ${\bf M}(t)={\bf M}_0$.  $\r(t)$ is orthogonal to ${\bf M}=\r(t)\times \v(t)$.
         We see that $\r(t)$ belongs to the sphere and to the plane orthogonal to the vector ${\bf M}_0=\r(t)\times \v(t)$.
         The intersection of this plane with sphere is a great circle.
         We proved that if $\r(t)$ is geodesic hence it belongs to great circle (as un-parameterised curve).

         The converse is evident since if particle moves along the great circle with constant velocity
         then obviously acceleration vector is orthogonal to the surface.

{\bf Remark} The vector $\M=\r(t)\times \v(t)$ is the torque. The torque is integral of motion
 in isotropic space.---This is the core of the 
considerations for geodesics on the sphere.



\subsubsection { Geodesics and shortest distance.}
{\footnotesize
 Many of you know that geodesics are in some sense shortest curves.
 We will give here an  exact meaning to this statement 
The proof is using variational principe.
(See the proof, discussions and applications of this statement
in appendices.)

Let $M$ be a Riemannian manifold.

{\bf Theorem}
{\it Let $\x_1$ and $\x_2$ be two points on $M$.
The shortest curve which joins these points is an arc of geodesic.

Let $C$ be a geodesic on $M$ and  $\x_1\in C$. Then for an arbitrary point  $\x_2\in C$ which is
close to the point $\x_1$ the arc of geodesic joining the points $\x_1,\x_2$ is a shortest curve between
these points\footnote{More precisely: for every point $\x_1\in C$ there exists a ball $B_{\delta}(\x_1)$
such that for an arbitrary point $\x_2\in C\cap B_{\delta}(\x_1)$
the arc of geodesic joining the points $\x_1,\x_2$ is a shortest curve between
these points.}
.}

}
\m

  This Theorem makes a bridge between two different 
approach to geodesic: the shortest distance and
  parallel transport of velocity vector.


%\end{document} % 1 March 2019


\section {Surfaces in $\E^3$. Parallel transport of vectors and 
{\it Theorema Egregium}}


%\end{document}  %22 March 2019

 Now equipped by the knowledge of Riemannian geometry we 
consider surfaces in $\E^3$, and formulate the important theorem
about parallel transport of vector over closed curve on the surface.
As an important corollary of this Theorem we will formulate
and prove Gau\ss  {\it Theorema Egregim}






\subsection {Parallel transport of the vector and
     Gaussian curvature of surface.}


{\sl We formulate here very important theorem
about  parallel transport of vectors
over closed curve and deduce 
Theorema Egregium from this theorem.  





\subsubsection  {Theorem of parallel transport over closed curve. Preliminary
formulation}\label{firstformulation}

  We give preliminary formulation of this Theorem. Later
when we will learn Gaussian curvature we will
return again to this theorem (see \ref{secondformulation} above.)



Let $M$ be a surface in  Euclidean space $\E^3$.
  Consider a closed curve $C$ on $M$,
  $M\colon \r=\r(u,v)$, $C\colon \r=\r(u(t,v(t)), 0\leq t\leq t_1, \, \x(0)=\x(t_1)$.
  ($u(t), v(t)$ are internal coordinates of the curve $C$.)

Consider the parallel transport of an 
arbitrary tangent $\X$ vector along the closed curve $C$:
           \begin{equation*}\label{paraltransportalongclosedcurve2}
\X(t)\colon\,\, {\nabla \X(t)\over dt}=0,\,\, 0\leq t\leq t_1\,,
          \end{equation*}
i.e.
                \begin{equation}\label{paraltransportalongclosedcurve3}
{d X^\a(t)\over dt}+
X^\beta(t)\Gamma^\a_{\beta\gamma}(u(t))
{du^\gamma(t)\over dt}=0,\,\, 0\leq t\leq t_1\,,\qquad \a,\beta,\gamma=1,2\,,
          \end{equation}

where $\nabla$ is the connection induced on the surface $M$ by
the Levi-Civita connection \eqref{levi-civitaformula}
of the induced Riemannian metric on the surface $M$,
(this is the same as to say that $\nabla$ is the connection induced on the surface $M$ by canonical flat connection
(see \eqref{inducedconnection1}) ),
and $\Gamma^\a_{\beta\gamma}$ its Christoffel symbols
(see in more detail in subsection \ref{secondformulation} above).

\m

{\bf Theorem}
{\it  Let $M$ be a surface in  Euclidean space $\E^3$.
  Let $C$ be a closed curve $C$ on $M$ such that $C$ 
is a boundary of a compact oriented domain $D\subset M$.
Consider the parallel transport of an arbitrary tangent vector 
along the closed curve $C$.
As a result of parallel transport along this closed curve 
any  tangent vector rotates through the angle

\begin{equation}\label{theoremofrotationonangle}
\angle\phi=\angle\left({\X, P_C\X}\right)=\int_D K d\sigma\,,
             \end{equation}
where $K$ is the Gaussian curvature and 
$d\sigma$
is the area element of induced 
Riemannian metric on the surface $M$, 
i.e.  $d\sigma=\sqrt {\det g}dudv$, where
$g_{\a\beta}=(\r_\a,\r_\beta)$.



}

\m


{\bf Example} Consider the closed curve, "latitude" $C_{\theta_0}\colon\,\theta=\theta_0$ on the sphere of the radius $R$.
Calculations show that
               \begin{equation}\label{rotationforlatitude}
                \angle\phi(C_{\theta_0})=2\pi(1-\cos\theta_0)
               \end{equation}

(see also the Homework 8). On the other hand the latitude $C_{\theta_0}$ 
is the boundary of the segment  $D$
 with area $2\pi RH$ where $H=R(1-\cos \theta_0)$. Hence
           $$
   \angle\left({\X, \R_C\X}\right)={2\pi RH\over R^2}={1\over R^2}\cdot \hbox{area of the segment}=\int_D  Kd\sigma
           $$
since Gaussian curvature is equal to $1\over R^2$


In the statement of this Theorem we use the Gaussian curvature. We will explain
it in next subsections, then will return again to 
this Theorem.



\subsubsection {Weingarten (shape) operator on surfaces and Gaussian curvature}

%\end{document} % 18 April it remians four last lectures.


\def\xip {{\boldsymbol \xi}}
Let $M\colon \r=\r(u,v)$ be a surface 
and $\n(u,v)$ be a unit normal
   vector field at the points of the surface  $M$.
  
We define at every point $\pt=\r(u,v)$ the 
Weingarten (shape) operator $S$ 
acting on the vector space $T_\pt M$ of 
vectors tangent to the surface $M$.


\bigskip
{\bf Definition-Proposition} Let $\n(u,v)$ be a unit normal vector field to the surface  $M$.
Then operator
\begin{equation}\label{defofshapeoperator}
    S\colon\,\,S(\X)=\p_\X (-\n)=
-X_u{\p { \n}(u,v)\over \p u}-X_v{\p {\n}(u,v)\over \p v}
\end{equation}


 maps tangent vectors to the tangent vectors:
\begin{equation}\label{propertyofshapeoperator1}
  S\colon T_\pt M\to T_\pt M\,\, \hbox{for every}\,\, 
 \X=X_u\r_u+X_v\r_v\in T_\pt M,
\qquad S(\X)\in T_\pt M
\end{equation}


This operator is called  {\it Weingarten (shape) operator}.






\m
{\footnotesize
{\bf Remark} The sign $"-"$ seems to be senseless: if $\n$ is unit normal vector field then $-\n$
is normal vector field too. Later we will see why it is convenient 
(see the Example-Motivation and Proposition below).
}

   Show that  property  \eqref{propertyofshapeoperator1} is 
  indeed obeyed, i.e.
   vector $\X'=S(\X)$ is tangent to surface.  Consider derivative of scalar product $(\n,\n)$
with respect to the vector field $\X$. We have that $(\n,\n)=1$. Hence
                 $$
   \p_\X(\n,\n)=0=\p_\X (\n,\n)=(\p_\X \n,\n)+(\n,\p_\X \n)=2(\p_\X \n,\n)\,.
              $$
   Hence $(\p_\X \n,\n)=-(S(\X),\n)=-(\X',\n)=0$, i.e. vector
$\p_\X \n=-\X'$ is orthogonal to the vector $\n$. This means that vector $\X'$ is
tangent to the surface.

Write down the action of shape operator on coordinate basis $\r_u=\p_u$, $\p_v=\r_v$ at
the given point $\pt$:
\begin{equation*}
S(\r_u)=-\p_{\r_u}\n(u,v)=-{\p \n(u,v)\over \p u},\quad
S(\r_v)=-\p_{\r_v}\n(u,v)=-{\p \n(u,v)\over \p v}
\end{equation*}


Since the shape operator transforms tangent vectors to tangent vectors, then
\begin{equation*}\label{propertyofshapeoperator-1}
\begin {matrix}
S(\r_u)=-{\p \n(u,v)\over \p u}=a\,\r_u+c \r_v \cr
S(\r_v)=-{\p \n(u,v)\over \p v}=b\r_u+d\r_v\cr
\end{matrix},\quad
  \end{equation*}
  i.e.
  \begin{equation}\label{propertyofshapeoperator2}
                       S
   =\begin{pmatrix}
   a &b\cr c& d\cr
   \end{pmatrix}
      \hbox {in the coordinate basis $\r_u,\r_v$}
\end{equation}
(This matrix is the matrix of Weingarten operator in the basis
   $(\r_u,\r_v)$.)
%Examples of shape operator see
%in the subsection above (Shape operator, Gaussian and mean 
%curvature for sphere and cylinder)
%and in the Homework 9.

{\bf Remark}. Shape operator as well as normal unit vector is defined up to a sign:
                $$
   \n(u,v)\to -\n(u,v), \quad {\rm then}\quad  S\to -S\,.
                $$

%\end{document} 24 April 2015

 {\bf Example-Motivation}   Consider just for  curve $C$ in $\E^2$.
   analog of shape operator.  Let $\r=\r(t)$ be parameterised curve,
and  $\n=\n(t)$ be unit normal vector field
on the curve, then  for arbitrary tangent vector $\x=c\v$
(where $\v$ is a velocity vector)
            $$
  T_\pt C \ni
 \x=c\v\to S(\x)=-\p_\x\n=-c{d x^i\over dt}{\p \n(\r(t))\over \p x^i}=-
          -{d\n(\r(t))\over dt}=k(\r(t))\v
               $$
since vector field ${d\n(\r(t))\over dt}$
is tangent to the curve.  
Here $k$ is so called curvature of the curve
(Frenet curvature)(Usually curvature is defined
as modulus of this magnitude). 

Explain how curvature is related with normal acceleration
(centripetal acceleration) $\ac_{\rm normal}$. We have:
               $$
\ac_{\rm normal}=(\ac,\n)\n=
\left({d\v\over dt},\n\right)\n=
-\left(\v, {d\n\over dt}\right)\n=
\left(\v, S(\v)\right)\n\
             $$ 
 (you see this equation explains why the sign $``-''$ appears)




Note that it follows from this equation that
if $\v(t)$ is tangent unit vector field, then
        $$
      {d\v(t)\over dt}=k(t)\n
        $$
it is nothing but centripetal acceleration.

Curvature of curves is not intrincis object.




{\footnotesize
\bigskip We show now that normal acceleration of a curve on the surface and
normal curvature are expressed in terms
of shape operator.

Let $C\colon \r(t)$ be a curve on the surface $M$,
$\r(t)=\r(u(t),v(t))$. Let $\v=\v(t)={d\r(t)\over dt}$, 
$\ac=\ac(t)={d^2r(t)\over dt^2}$
  be velocity and acceleration vectors respectively. Recall that
 \begin{equation}\label{internandexterncomponentsofvelocityvector2}
    \v(t)={d\r(t)\over dt}=\dot x\e_x+\dot y\e_y+\dot z\e_z=
    {d\r(u(t),v(t))\over dt}=\dot u\r_u+\dot v\r_v
 \end{equation}
  be velocity vector; $\dot u,\dot v$ are internal components of the velocity vector with respect
  to the basis $\{\r_u=\p_u,\r_v=\p_v\}$ and
  $\dot x, \dot y,\dot z$,
  are external components velocity  vectors with respect to 
the basis $\{\e_x=\p_x,\e_y=\p_y,\e_z=\p_z\}$ .
As always we denote by $\n$ normal unit vector.

\bigskip

  {\bf Proposition} {\it    The normal acceleration
    at an arbitrary point $\pt=\r(u(t_0),v(t_0))$ of the curve $C$ on the surface $M$
    is defined by the scalar product of the velocity vector $\v$ of the curve at the point $\pt$
    on the value of the shape operator on the velocity vector:
\begin{equation}\label{propertyofshapeoperator2}
        \ac_{n}=a_{n}\n=\left(\v, S\v\right)\n\
         \end{equation}
and normal curvature 
%\eqref{normalcurvaturedef} 
is equal to

\begin{equation}\label{propertyofshapeoperator3}
    \kappa_{n}=
{(\n,\ac)\over (\v,\v)}=
{\left(\v, S\v\right)\over (\v,\v)}
\end{equation}
}

\bigskip

{\it Proof of the Proposition}.  According to 
%\eqref{normalacceler2} 
we have
                  $$
  \ac_{n}=(\n,\ac)\n=\n\left(\n,{d\over dt}\v(t)\right)\n=
     \n{d\over dt}\left(\n,\v(t)\right)-
     \n\left({d\over dt}\n(u(t),v(t)),\v(t)\right)
                    $$
                    $$
                    =0+\left(-\p_\v\n,\v\right)\n=(S\v,\v)\n
                  $$
This proves Proposition.

Later we will use equation \eqref{propertyofshapeoperator3}
to find eigenvectors of the operator.
}






\subsubsection {The Weingarten operator; principal curvatures and Gaussian curvature}

Now we introduce on surfaces, principal curvatures and
Gaussian curvature in terms of the Weingarten (shape) operator

\def \bl {{\bf l}}
  Let $\pt$ be an arbitrary point of the surface $M$ and $S$ be 
the Weingarten  operator at this point.
  $S$ is symmetric operator: $(S\ac, {\bf b})=({\bf b}, S\ac)$.
  Consider
 eigenvalues $\lambda_1,\lambda_2$ and  eigenvectors $\bl_1, \bl_2$  
of the shape operator  $S$
 \begin{equation}\label{eigenvaluesofshapeoperator}
\bl_1,\bl_2\in T_\pt M,\qquad    S\,\bl_1=\kappa_1 l_1,\qquad S\,\bl_2=\kappa_2 l_2,
   \end{equation}

\m

\def\k {{\kappa}}

{\bf Definition} Eigenvalues of shape operator $\lambda_1,\lambda_2$ are called {\it principal curvatures}:
             $$
   \lambda_1=\k_1,\,\,\,\lambda_2=\k_2
              $$
 Eigenvectors $\l_1,\l_2$ define the two directions such that  curves directed along
  these vectors have normal curvature equal to the principal curvatures $\k_+, \k_-$.

These directions are called principal directions

\m



{\bf Remark} As it was noted above normal unit vector as well as a shape operator are defined up to a sign.
Hence principal curvatures, i.e. eigenvalues of shape operator are defined up to a sign too:
                \begin{equation}\label{princcurvaturearedefinedup to asign}
\n\to -\n, {\rm then}\,\, S\to -S, \,\,\,{\rm then}\,\, (\k_1,\k_2)\to (-\k_1,-\k_2)
                \end{equation}

{\footnotesize
{\bf Remark}. Principal directions are well-defined in the case if principal curvatures (eigenvalues of shape operator)
are different: $\lambda_1=\kappa_1\not=\kappa_2=\lambda_2$.
 In the case if eigenvalues $\lambda_1=\lambda_2=\lambda$ then $S=\lambda E$ is proportional to unity operator.
 In this case  all vectors
are eigenvectors, i.e. all directions are principal directions.
(This happens for the shape operator of the sphere: see the Homework 9.)



{\bf Remark}
Does shape operator have always two eigenvectors? 
Yes, this follows from the fact that shape operator is 
{\it symmetrical operator}. One can prove that
                $$
\langle S\ac, \b\rangle=\langle S\b, \ac\rangle\,,
              $$ for arbitrary two tangent vectors $\ac,\b$,

  This implies that principal directions are orthogonal to each other.
 Indeed one can see that
             $\lambda_2(\l_2,\l_1)=(S\l_2,\l_1)=(\l_2,S\l_1)=\lambda_1(\l_2,\l_1)$.
It follows from this relation that eigenvectors are orthogonal ($(\l_-,\l_+)=0$) if $\lambda_-\not=\lambda_+$
If $\lambda_-=\lambda_+$ then all vectors are eigenvectors. 
One can choose in this case
$\l_-$, $\l_+$ to be orthogonal.

It has to be mentioned that equation \eqref{propertyofshapeoperator3}
may be used to prove and to find eigenvectors. Indeed following equation
\eqref{propertyofshapeoperator3} consider on unit cirle $|\v|=1$ a function
           $$
   f(\v)=(S(\v),\v)
               $$
One can see that minimu and maximum values of this function
define two eigenvalues of oeprator $S$, and the points
where these extrema atteint define eigenvectors.
}

  \m

{\bf Definition}

      {\it Gaussian curvature} $K$ of the surface $M$ at a point $\pt$ is equal to
                  the product of principal curvatures.
                  \begin{equation}\label{gaussinacurvaturedef0}
                    K=\k_1\k_2
                    \end{equation}




  Recall that the product of eigenvalues of a linear operator is determinant of this operator,
  Thus we immediately come to the useful formulae for calculating the Gaussian 
curvature
                        

 {\bf Proposition}  Let $S$ be a shape operator at 
the point $\pt$ on the surface $M$.  Then

Gaussian curvature $K$ of the  surface $M$ at the point $\pt$ is equal to the determinant of the shape operator:
\begin{equation}\label{gauscurvasdeterm}
    K=\k_1\k_2=\det S
\end{equation}



 E.g. if  in a given coordinate basis  a shape operator is given by the matrix
 $\begin{pmatrix}a &b\cr c &d\end{pmatrix}$ (see e.g. equations \eqref{propertyofshapeoperator1}
 and\eqref{propertyofshapeoperator2} ),
then
                  \begin{equation}\label{exampleofcalcugaussian}
    K=\det S=\det \begin{pmatrix}a &b\cr c &d\cr\end{pmatrix}=ad-bc,\,\,\,
    H={\rm Tr\,} S={\rm Tr\,}\begin{pmatrix}a &b\cr c &d\cr\end{pmatrix}=a+d
          \end{equation}

{\footnotesize   One can define also so called {mean curvature}  $H$
of the surface. {\it Mean curvature} $K$ of the surface $M$ 
at every point is equal to
                  the sum of the principal curvatures:
 $H=\k_1+\k_2$.
Mean curvature  $H$ of the  surface $M$ at an arbitrary point $\pt$ 
is equal to the trace of the shape operator
$S$ at this point: $H=\k_1+\k_2={\rm Tr\,} S$.
}

\m


 \subsubsection {Examples of calculation of Weingarten operator,
curvatures for cylinder sphere and for saddle.}

\m

{\it Cylinder}

\m

  We already calculated induced Riemannian metric on the cylinder (see \eqref {formula forfirstformcyl}).

 Cylinder is given by the equation $x^2+y^2=a^2$. 
One can consider the following
parameterisation
 of this surface:
\begin{equation}\label{surface11}
  \r(h,\varphi)\colon\quad
  \begin{cases}
  x=a\cos\varphi\\
  y=a\sin\varphi\\
  z=h\\
  \end{cases}\,,\quad   \r_h=
  \begin{pmatrix}
        0\\
        0\\
        1\\
   \end{pmatrix}\,,
\quad
  \r_\varphi=\begin{pmatrix}
        -R\sin\varphi\\
        R\cos\varphi\\
          0\\
   \end{pmatrix}\,,
\end{equation}
   Normal unit vector $\n=\pm \begin{pmatrix}
        \cos \varphi\cr
        \sin\varphi\cr
        0\cr
   \end{pmatrix}$. Choose $\n=\begin{pmatrix}
        \cos \varphi\cr
        \sin\varphi\cr
        0\cr
   \end{pmatrix}$. Weingarten operator
     \begin{equation*}\label{weingrartenforcylinder}
        S\p_h=-\p_{\r_h} \n=-\p_h\begin{pmatrix}
        \cos \varphi\cr
        \sin\varphi\cr
        0\cr
   \end{pmatrix}=0\,,
     \end{equation*}
\begin{equation*}\label{weingrartenforcylinder}
        S\p_\varphi=-\p_{\r_\varphi} \n=-{\p_\varphi}
        \begin{pmatrix}
        \cos \varphi\cr
        \sin\varphi\cr
        0\cr
   \end{pmatrix}=\begin{pmatrix}
        \sin\varphi \cr
        -\cos\varphi\cr
        0\cr
   \end{pmatrix}=-{\p_\varphi\over a}.
    \end{equation*}
\begin{equation}\label{weingrartenforcylinder2}
           S
   \begin{pmatrix}
        \p_h \cr
        \p_\varphi\cr
   \end{pmatrix}=
   \begin{pmatrix}
         0\cr
        {-\p_\varphi\over R}\cr
   \end{pmatrix},\quad S=\begin{pmatrix}0&0\cr 0 &{-1\over R}\end{pmatrix}\,.
    \end{equation}
Principal curvaures are $\k_1=0$ and $\k_2=-{1\over R}$.
For the Gaussian curvature we have
     \begin{equation}\label{Gaussianforcylinder}
        K=\det S={\det A\over \det G}=\det
                              \begin{pmatrix}0&0\cr
                                0&-{1\over R}\cr
                                   \end{pmatrix}=0\,.
     \end{equation}
   
If we change $\n\to-\n$ Gaussian curvature
will not change.





{\it Sphere}


\medskip


  Sphere is given by the equation $x^2+y^2+z^2=R^2$. 

To calculate Gaussain curvature for arbitrary surface we may choose
arbitrary parameterisation, since curvature does
dpend on the choice of parameterisation.  Usually we choose
the parameterisation  which is convenient for caluclations.
In the case of sphere  we have freedom to use  
an arbitrary parameterisation.
We will do calculations for arbitrary parametersiation. 
(At the end we will do calculations in spherical coordinates
just to double check that everything is alright).


Let $\r=\r(u,v)$ belong to the sphere $x^2+y^2+z^2=R^2$ of radius $R$;
and let 
$(u,v)$ are arbitrary local coordinates on this sphere

   Consider at the points of the sphere, a unit vector field
                \begin{equation*}
           \n(u,v)={\r(u,v)\over R}
                 \end{equation*} 
since the surface is sphere hence this unit vector field
is orthogonal to the surface of the sphere.
       Then calculate the shape operator:
                       \begin{equation*}
           S(\r_u)=
           -\p_{\r_u}\n(u,v)=
           -{\p \over \p u}
         \left(\r(u,v\over R)\right)=-{\r_u\over R}\,,
                       \end{equation*}
and the same is for the vector $\r_v$:
        \begin{equation*}
           S(\r_v)=
           -\p_{\r_v}\n(u,v)=
           -{\p \over \p v}\left(\r(u,v\over R)\right)=-{\r_v\over R}\,,
                       \end{equation*}

We see that both vectors are eigenvetors with the same eigenvalue
   $-1\over R$, i.e. {\rm all} tangent vectors are eigenvectors
with teh same eigenvalue.  The matrix of shape operator
is $\begin{pmatrix}-{1\over R}  &\cr 0 & -{1\over R}\end{pmatrix}$,
the Gaussian curvater is equal to $K={1\over R^2}$.



   \bigskip

We can repeat calculations in specific coordinates,
e.g. in spherical coordinates.

Consider the  parameterisation
 of sphere in spherical coordinates
\begin{equation}\label{surfacesphere11}
  \r(\theta,\varphi)\colon\quad
  \begin{cases}
  x=R\sin\theta\cos\varphi\\
  y=R\sin\theta\sin\varphi\\
  z=R\cos\theta\\
  \end{cases}
\end{equation}

For the sphere $\r(\theta,\varphi)$ is orthogonal to the surface.
       Hence
   normal unit vector
   $   \n(\theta,\varphi)=\pm {\r(\theta,\varphi)\over R}=\pm
   \begin{pmatrix}
    \sin\theta\cos\varphi\cr
     \sin\theta\sin\varphi\cr
      \cos\theta\cr
   \end{pmatrix}$.
           Choose $\n= {\r\over R}=
   \begin{pmatrix}
    \sin\theta\cos\varphi\cr
     \sin\theta\sin\varphi\cr
      \cos\theta\cr
   \end{pmatrix}.
     $
     Weingarten operator
     \begin{equation*}\label{weingrartenforcylinder}
        S\p_\theta=-\nabla^{\rm can.flat}_{\r_\theta}\n=-\p_\theta \n=
          -\p_\theta\left({\r\over R}\right)=-{\r_\theta\over R}\,,
       \end{equation*}
       \begin{equation*}\label{weingrartenforcylinder}
       S\p_\varphi=-\nabla^{\rm can.flat}_{\r_\varphi}\n=-\p_\varphi \n=
          -\p_\varphi\left({\r\over R}\right)=-{\r_\varphi\over R}\,.
    \end{equation*}
\begin{equation}\label{weingrartenforcylinder2}
           S
   \begin{pmatrix}
        \p_\theta \cr
        \p_\varphi\cr
   \end{pmatrix}=
   \begin{pmatrix}
         -{\p_\theta\over R}\cr
        -{\p_\varphi\over R}\cr
   \end{pmatrix},\quad S=
        -
   \begin{pmatrix}{1\over R}&0\cr 0 &{1\over R}\end{pmatrix}\,.
    \end{equation}
For the Gaussian we have
     \begin{equation}\label{Gaussianforcylinder}
        K=\det S={\det A\over \det G}=\det
                              \begin{pmatrix}-{1\over R}&0\cr
                                0&-{1\over R}\cr
                                   \end{pmatrix}={1\over R^2}\,,
     \end{equation}
Gaussian curvature will not change if we change $\n\to -\n$.


We see that for the sphere  Gaussian curvature is not equal to 
zero, whilst for cylinder and cone Gaussian curvature
equals to zero.

\m


{\it Saddle} Consider saddle: $z=kxy$:

  \begin{equation}\label{saddle2}
     \begin{cases}
      x=u\cr
       y=v\cr
        z=kxy\cr
      \end{cases}
       \end{equation}     

 It is ruled surface.
     For  every point of saddle 
$\pt\colon \begin{cases}
      x=u_0\cr
       y=v_0\cr
        z=ku_0v_0\cr
      \end{cases} $
one can consider two stragit  lines which pass
 thrOugh this point and 
belong to the saddle:  
   \begin{equation*}\label{saddleruledsurface}
     l_1\colon 
      \begin{cases}
      x=u_0\cr
       y=v_0+t\cr
        z=ku_0(v_0+t)\cr
      \end{cases}\,,
         -\infty<t<\infty\quad
     l_2\colon 
      \begin{cases}
      x=u\cr
       y=v_0\cr
        z=k(u_o+t)v_0\cr
      \end{cases}\,,  -\infty<t<\infty
       \end{equation*}     
Calculate the shape operator and Gaussian curvature of
the saddle at the stationary point 
 $x=y=z=0$, i.e. $u=v=0$.
One can see that for the saddle \eqref{saddle2}
      $$
\r_u=\begin{pmatrix}
     1\cr 0\cr kv\cr
    \end{pmatrix}\,,\quad
\r_v=\begin{pmatrix}
     0\cr 1\cr ku\cr
    \end{pmatrix}\,,
         $$
and one can choose normal unit vector field $\n=\n(u,v)$
as
                 $$
      \n(u,v)=
       {1\over \sqrt {1+ku^2+kv^2}}\begin{pmatrix}
     kv\cr ku\cr -1\cr
    \end{pmatrix}\,,
      $$
   Indeed it is easy to see that
   $(\n.n)=1$ and $(\n,\r_u)=(\n,\r_v)=0$,
i.e. $\n(u,v)$ is unit vector field whic is orthogonal
to tangent vectors $\r_u$ and $\r_v$.

 
Now notice that for the point $u=v=0$
calculations for 
${\p\n(u,v)\over \p u}\big\vert_{u=v=0}$ and
${\p\n(u,v)\over \p v}\big\vert_{u=v=0}$ which lead to
calculation of shape operator are
simple:
      $$
{\p\n(u,v)\over \p u}\big\vert_{u=v=0}=
{\p  \left(  
   {1\over \sqrt {1+ku^2+kv^2}}
       \begin{pmatrix}
     kv\cr ku\cr -1\cr
    \end{pmatrix}
         \right)
     \over \p u}\big\vert_{u=v=0}= 
      \begin{pmatrix}
      0\cr k\cr 0\cr
       \end{pmatrix}\,,
         $$
and   $$
{\p\n(u,v)\over \p v}\big\vert_{u=v=0}=
{\p  \left(  
   {1\over \sqrt {1+ku^2+kv^2}}
       \begin{pmatrix}
     kv\cr ku\cr 1\cr
    \end{pmatrix}
         \right)
     \over \p v}\big\vert_{u=v=0}= 
      \begin{pmatrix}
       k\cr 0\cr 0\cr
       \end{pmatrix}\,,
         $$
since
     ${1\over \sqrt {1+k^2u^2+k^2v^2}}\big\vert_{u=v=0}=1$
and
          $$
{\p\over \p u}\left(
{1\over \sqrt {1+k^2u^2+k^2v^2}}
  \right)\big\vert_{u=v=0}=
{k^2u\over \sqrt {1+k^2u+k^2v}}\big\vert_{u=v=0}
=0\,,
     $$
and        $$
{\p\over \p v}\left(
{1\over \sqrt {1+k^2u^2+k^2v^2}}
  \right)\big\vert_{u=v=0}=
{k^2v\over \sqrt {1+k^2u+k^2v}}\big\vert_{u=v=0}
=0\,.
     $$
Hence for the shape operator at the point $u=v=0$ we have
    $$
S(\r_u)\big\vert_{u=v=0}=
-{\p\n(u,v)\over \p u}\big\vert_{u=v=0}=
 -\begin{pmatrix} 0\cr k\cr 0\cr \end{pmatrix}=
-k\r_v\big\vert_{u=v=0}
  $$
and  $$
S(\r_v)\big\vert_{u=v=0}=
-{\p\n(u,v)\over \p v}\big\vert_{u=v=0}=
 -\begin{pmatrix} k\cr 0\cr 0\cr \end{pmatrix}=
-k\r_u\big\vert_{u=v=0}\,.
  $$
Thus at the origin shape operator is
 $S=\begin{pmatrix} &-k\cr -k & 0\end{pmatrix}$
and Gaussian curvature at the otigin is equal to
     $$
K=\det S=-k^2\,.
    $$    
 Saddle has negative curvature at the origin.      


%\end{document}  % 29 March 2019
     $$ $$
{\tt 

   \centerline {Dear Riemannian Geometry students}

Due to the exceptional circumstances the material of this 
subsection  was not  delivered on week 9.
We will recall this subsecion on the 1-st May lecture and will
begin the last section: Curvature.
}

\subsubsection {Theorem of parallel transport over closed curve (detailed formulation)}\label{secondformulation}



{\sl We formulated in subsection \ref{firstformulation} 
very important theorem
about  parallel transport of vectors
over closed curve.

Now after learning the Gaussian curvature
we formulate it again in more detail,
focusing attention on the fact that  using this Theorem we
obtain information about Gaussian curvature using only
induced Riemannian metric. 

We will  deduce 
Theorema Egregium from this theorem.  









We recall here formulation of this Theorem in subsection\ref{firstformulation}
and will formulate it again in more details.

Let $M$ be a surface in  Euclidean space $\E^3$.
  Consider a closed curve $C$ on $M$,
  $M\colon \r=\r(u,v)$, $C\colon \r=\r(u(t,v(t)), 0\leq t\leq t_1, \, \x(0)=\x(t_1)$.
  ($u(t), v(t)$ are internal coordinates of the curve $C$.)

Consider the parallel transport of an 
arbitrary tangent $\X$ vector along the closed curve $C$:
  Recall that we did it in the section 3.1.2 where we already 
considered parallel transport
over curve for arbitrary connection and for Levi-Civita connection
for arbitrary Riemannian manifold
(see equations \eqref{paralleltransportalongthecurve}
and \eqref{paralleltransportorthogonaloperator}).
 Now we will repeat these considerations for this special case.)
          \begin{equation*}\label{paraltransportalongclosedcurve1}
\X(t)= \underbrace {X^\a(t){\p\over \p u_\a}\big\vert_{u^\a(t)}}_{\hbox{Internal observer}} =
 \underbrace{X^\a(t)\r_\a\big\vert_{\r(u(t),v(t))}}_{\hbox{External observer}}\,\,,\left(
 \r_\a={\p x^i\over \p u^\a}{\p\over \p x^i}\right)\,.
          \end{equation*}
          \begin{equation*}\label{paraltransportalongclosedcurve2}
\X(t)\colon\,\, {\nabla \X(t)\over dt}=0,\,\, 0\leq t\leq t_1\,,
          \end{equation*}
i.e.
                \begin{equation}\label{paraltransportalongclosedcurve3}
{d X^\a(t)\over dt}+
X^\beta(t)\Gamma^\a_{\beta\gamma}(u(t))
{du^\gamma(t)\over dt}=0,\,\, 0\leq t\leq t_1\,,\qquad \a,\beta,\gamma=1,2\,,
          \end{equation}

where $\nabla$ is the connection induced on the surface $M$ by canonical flat connection
(see \eqref{inducedconnection1}), or (it is the same)
the Levi-Civita connection \eqref{levi-civitaformula}
of the induced Riemannian metric on the surface $M$
 and $\Gamma^\a_{\beta\gamma}$ its Christoffel symbols:
\begin{equation}\label{connectioninduced12}
          \Gamma^\gamma_{\a\beta}={1\over 2}g^{\gamma\pi}
      \left({\p g_{\pi \a}\over \p u^\beta}+{\p g_{\pi \beta}\over \p u^\a}-
      {\p g_{\a\beta}\over \p u^\pi}\right)\,,{\rm where}\,
      g_{\a\beta}=\langle\r_\a,\r_\beta\rangle={\p x^i\over \p u^\a} {\p x^i\over \p u^\beta}
     \end{equation}
 are components of induced Riemannian metric
 $G_M=g_{\a\beta}du^\a du^\beta$.


Let $\r(0)=\pt$ be a starting (and ending) point of the curve $C$: $\r(0)=\r(t_1)=\pt$.
The differential equation \eqref{paraltransportalongclosedcurve3} defines the linear operator
            \begin{equation}\label{linearoperatoroverclosedcurve}
            P_C\colon T_{\pt}M\longrightarrow T_{\pt}M
            \end{equation}
For any vector $\X\in T_{\pt}M$, its image the vector $R_C\X$ as the solution of the differential equation
\eqref{paraltransportalongclosedcurve3} with 
initial condition $\X(t)\big\vert_{t=0}=\X$.
(See also section 3.1.2, equations \eqref{paralleltransportalongthecurve}
and \eqref{paralleltransportorthogonaloperator}).
 )

On the other hand we know that parallel transport is orthogonal
operator, it does not change the scalar product
of two vectors, and it does not chnage lengths of vectors (see
\eqref{paralleltransportorthogonaloperator} in the subsection 3.1.2):
       \begin{equation}\label{linearoperatoroverclosedcurvescalarproduct}
            \langle\X,\X\rangle=\langle P_C\X,P_C\X\rangle
            \end{equation}
We see that $P_C$ is an orthogonal operator in the $2$-dimensional vector space $T_{\pt}M$.
We know that orthogonal operator preserving
orientation is the operator of rotation on some angle $\phi$.

One can see that $P_C$ preserves orientation
\footnote{We will consider mainly the case 
if the closed curve $C$ is a boundary of
a compact oriented domain $D\subset M$. 
In this case one can see by continuity arguments
that operator $R_C$ preserves an orientation.}
 then the action of operator $P_C$ on vectors is rotation on the angle,
i.e. the result of parallel transport along closed curve is 
rotation on the $\angle\phi$.
 This angle depends depends on the curve.
The very beautiful question arises:  
How to calculate this angle $\Delta\Phi(C)$


\m

{\bf Theorem}
{\it  Let $M$ be a surface in  Euclidean space $\E^3$.
  Let $C$ be a closed curve $C$ on $M$ such that $C$ 
is a boundary of a compact oriented domain $D\subset M$.
Consider the parallel transport of an arbitrary tangent vector 
along the closed curve $C$.
As a result of parallel transport along this closed curve 
any  tangent vector rotates through the angle

\begin{equation}\label{theoremofrotationonangle}
\angle\phi=\angle\left({\X, P_C\X}\right)=\int_D K d\sigma\,,
             \end{equation}
where $K$ is the Gaussian curvature and $d\sigma=\sqrt {\det g}dudv$ is the area element induced by the
Riemannian metric on the surface $M$, i.e.  $d\sigma=\sqrt {\det g}dudv$.



}

\m
{\bf Remark} One can show that the angle of rotation does not depend on initial point of the curve.


\m

{\footnotesize
{\bf Remark} 

%Here we assume that a curve $C$  is smooth curve.
  If $C$ is a {\it smooth closed geodesics} 
then it follows from this 
Theorem and properties of geodesics that rotation angle is
equal to $2\pi n$ (where $n$ is integer). 
Theorem is valid also for piecewise smooth curve. 
In general if $C$
is piecewise smooth curve, then one can see  that
  rotation angle is equal to $\sum \a_i- \pi (n-1)$
where $n$ is number of smooth arcs, and $\a_i$
angles between them (see example above in next subsection). 


}

%To proof this Theorem we  develop the technique of derivatin formulae;
%we apply this technique to caclulate the Gaussain and mena curvature for
%surfaces, and then in the section \ref{proofofrotation}
%we will prove this Theorem.

The proof of this Theorem see in Appendices.


\subsubsection {Gau\ss\,\, {\it Theorema Egregium}}


  Here we wlll formulate and prove very important
corollary of the Theorem on parallel transport
over closed curve. This is Gau\ss\, Theorema   Egregium.

We defined Gaussian curvature in terms of Weingarten (shape) 
operator as a product of principal curvatures.
This definition was in terms of External Observer. 

 The Theorem about transport over closed curve implies
the remarkable Corollary:

{\bf Corollary} {\it Gau\ss \,\, Egregium Theorema}

Gaussian curvature of the surface can be expressed in 
terms of induced Riemannian metric. It
is invariant of isometries.


\bigskip

Indeed
let  $D$ be a small domain around  a given point $\pt$, 
let $C$ its boundary and
$\angle \phi(D)$ be an angle of rotation.   
Denote by $S(D)$ an area of this domain.
Applying the Theorem for the case when area of the domain $D$ 
tends to zero we  we come to the statement that
            $$
      \hbox{if $S(D)\to 0$ then \,} \angle\phi(D)=\int_D Kd\sigma\to K(\pt)S(D),\,{\rm i.e.}
            $$

\begin{equation}\label{egregium1}
K(\pt)=\lim_{S(D)\to 0}{\angle\phi(D)\over S(D)}\,.
\end{equation}

Now notice that  right hand side od this equation defining 
Gaussian curvature $K(\pt)$ depends only on Riemannian
metric on the surface $C$. Indeed numerator of RHS is 
defined by the solution of differential equation
$\eqref{paraltransportalongclosedcurve3}$ which depends on 
Levi-Civita connection depending on
the induced Riemannian metric, and denominator is an area depending 
on Riemannian metric too.
Thus we come to Gau\ss\,, Theorema Egregium.

{\footnotesize Note that   
mean curvature ($H=\k_++\k_-={\rm Tr\,}S$) 
is not the invariant of isometries.  It cannot be calculated
in terms of induced Riemannian metric. E.g. 
mean curvature of cylindre is equal to $H=0+{1\over R}={1\over R}$.
On the other hand cylindre is locally Euclidean,
hence mean curvature {\it cannot be expressed in terms of induceed
Riemannian metric.}}



\bigskip

{\footnotesize

{\bf Example}    

Consider  ``triangle''   $ABC$ on the surface $M$ in $\E^3$,
such that 
  the sides, edges  of triangle are arcs of great circles,
the shortest curves.
Here we consider the case if $M$ is a sphere of radius $R$,
but all our considerations and the final formula
\eqref{defectoftriangle} is valid for arbitrary embedded surface.




  Let $\a,\beta,\gamma$ be angles between edges
of the triangle $ABC$.
 
     We will apply Theorem on closed curve to this triangle.
 
  One can easy to caclulate the parallel transport of
arbitrary vector along geodesic:
during parallell transport along geodesics vector remains tangent,
if it is tangent at the initial point, and in the case if it is not
tangent, the angle between this vector and the tangent vector is 
preserved.   This means that we can calculate for the triangle
$ABC$ the left hand side of the formula 
\eqref{theoremofrotationonangle}.

Denote by $\A,\B,\C$ a (non-zero) vectors such that
they are attached to vertices of this triangle:
   $A\in T_A M$ 
   $B\in T_B M$ 
   $\C\in T_C M$ 
and they are tangent 
to the corresponding sides: 
vector $\A$ is tangent to the side $AB$,
vector $\B$ is tangent to the side $BC$,
and
vector $\C$ is tangent to the side $CA$.
(we suppose that they are oriented anti-clock wise)
      
Suppose also that all these vectors have the same length.       


Let $\X(t)$ be a parallel transport of the vector
$\X$ along the edges of triangle $ABC$:  initial condition is
that vector $\X$ is attached at the vertex $A$,
and coincides with vector $\A$: $\X(t)\vert_{t_A}=\A$.


One can see that due to parallel transport vector $\X=\A$
will rotate on the anlge $\a+\beta+\gamma-\pi$

{\footnotesize  
Show it.
 
Due to the porperty of parallel
trasport along geodesic for parallel transport
$\X(t_B)$ over edges of triangle we see that

 \begin{itemize}
\item For all $t\colon\quad t_A\leq t< t_B$ 
the vector $\X(t)$ is tangent to
the curve $\r(t)$, its length is the same and at the point 
at the final point of the trip it
will have the angle $\pi-\beta$.
with the vector $\B$.

\item For all $t\colon\quad t_B\leq t< t_C$ 
the vector $\X(t)$ will  have the angle $\pi-\beta$
with tangent vector to the edge $BC$.
its length is the same and 
at the final point of the trip it
will have the angle $\pi-\gamma+\pi-\beta$.
with the vector $\C$.

\item For all $t\colon\quad t_B\leq t< t_C$ 
the vector $\X(t)$ will  have the angle $2\pi-\beta-\gamma$
with tangent vector to the edge $CA$.
Its length is the same and 
at the final point of the trip it
will have the angle $\pi-\a+\pi-\gamma+\pi-\beta$.
with the vector $\C$.


\end{itemize}
The angle between the final vector and initial will be 
$3\pi-\a-\beta-\gamma$, this meant 
that  final vector $\A$ will rotate on the angle
      $-(3\pi-\a-\beta-\gamma)=\a+\b+\gamma-\pi+2\pi$.

}

Now apply formula \eqref{theoremofrotationonangle}:
We have that rotation of vector during parallel transport along
the boundary of triangle is equal to
 
    \begin{equation}\label{defectoftriangle}
\a+\beta+\gamma-\pi=\int_{\triangle ABC} Kd\sigma
       \end{equation}

n particular for sphere of radius $R$ we come to 
     \begin{equation}\label{defectoftrianglesphere}
\a+\beta+\gamma-\pi=\int_{\triangle ABC} Kd\sigma=
  {\hbox{the area of $\triangle ABC$}\over R^2}\,.
       \end{equation}

In the case of zero curvatre we come to the formula
which everybody knows: sum of angles of triangle is equal to $\pi$.


 

In Appendices
we develop the technique which  itself is very interesting.
One of the applications of this
technique is the proof of the Theorem \eqref{theoremofrotationonangle}.
%and Theorem \eqref{gaussiancurvatureintermsofisothermal0}.



{\footnotesize  There are different other proofs of Theorema Egregium.
See Appendices.

%E.g. it immediately follows from the formula
%\eqref{gaussiancurvatureintermsofisothermal0}
%in the Theorem in subsection \ref{isothermal1}, where
%the right hand side of the formula 
%\eqref{gaussiancurvatureintermsofisothermal0}
%depends on metric on the surface, i.e. Gaussian curvature 
%maybe independently calculated by
%Internal Observer. 

%Later in the fifth section we will give also 
%another proof of the Theorema  Egregium, calculating
%straightforwardly Gaussian curvature in terms of Riemannian 
%curvature tensor.
}
% 16 April

%\end {document}   % 24 April 2018

}


\section {Curvature tensor}


\subsection {Curvature tensor for connection}




\end{document}  %12 April 2019
%\end{document}  %28 April

{\bf Definition-Proposition}
Let manifold $M$ be equipped with connection $\nabla$.
Consider the following operation which assigns to arbitrary
vector fields $\X,\Y$ and $\Z$ on $M$ the new vector field:
\begin{equation}\label{operationdefinig the tensor}
   {\cal R}(\X,\Y)\Z=
    \left(
    \nabla_\X\nabla_\Y-\nabla_\Y\nabla_\X-
   \nabla_{[\X,\Y]}
    \right)\Z
\end{equation}
This operation is obviously linear over the scalar coefficients.


 One can show that this operation is  $C^{\infty}(M)$-linear
with respect to vector fields $\X,\Y\,\Z$, i.e. for an arbitrary functions
$f,g,h$
        \begin{equation}\label{propertiesoflienarity}
            {\cal R}(f\X,g\Y)(h\Z)=fgh{\cal R}(\X,\Y)\Z\,.
        \end{equation}
This means that the operation defines
the tensor field of the type $\begin{pmatrix}1\cr 3\end{pmatrix}$:
If $\X=X^i\p_i, \X=X^i\p_i,\X=X^i\p_i$
then according to \eqref{propertiesoflienarity}
           $$
        {\cal R}(\X,\Y)\Z={\cal R}(X^m\p_m,Y^n\p_n)(Z^r\p_r)=Z^rR^i_{rmn}X^mY^n
           $$
where we denote by $R^i_{rmn}$ the components of the tensor $\cal R$ in the coordinate basis ${\p_i}$
\begin{equation}\label{componentsofcurvaturetensor}
    R^i_{\,\,rmn}\p_i={\cal R}(\p_m,\p_n)\p_r
\end{equation}
This tensor is called {\it curvature tensor of the connection $\nabla$}.


    

Express components of the curvature tensor in terms of 
Christoffel symbols of the connection.
If $\nabla_m\p_n=\Gamma_{mn}^r\p_r$ then according to the \eqref{operationdefinig the tensor} we have:
                $$
        R^i_{\,\,rmn}\p_i={\cal R}(\p_m,\p_n)\p_r=\nabla_{\p_m}\nabla_{\p_n}\p_r-\nabla_{\p_n}\nabla_{\p_m}\p_r,
                    $$
                    $$
                    R^i_{\,\,rmn}=
              \nabla_{\p_m}\left(\Gamma_{nr}^p\p_p\right)-\nabla_{\p_n}\left(\Gamma_{mr}^p\p_p\right)=
                $$
                \begin{equation}\label{curvatureincomponents}
                \p_m\Gamma_{nr}^i+\Gamma_{mp}^i\Gamma_{nr}^p-\p_n\Gamma_{mr}^i-\Gamma_{np}^i\Gamma_{mr}^p\,.
                \end{equation}


{\footnotesize 

 The proof of the property \eqref{propertiesoflienarity} can be 
given just  by straightforward calculations:
     Consider e.g. the case $f=g=1$, then

        $$
         {\cal R}(\X,\Y)(h\Z)=
    \nabla_{\X}\nabla_\Y(h\Z)-\nabla_{\Y}\nabla_\X(h\Z)-
   \nabla_{[\X,\Y]}(h\Z)=
        $$
        $$
     \nabla_\X\left(\p_\Y h\Z+h\nabla_\Y\Z\right)-\nabla_\Y\left(\p_\X h\Z+h\nabla_\X\Z\right)-
     \p_{[\X,\Y]}h \Z-h\nabla_{[\X,\Y]}\Z=
        $$
        $$
    \p_\X\p_\Y h\Z+\p_\Y h\nabla_\X\Z+
    \p_\X h\nabla_\Y\Z+
    h\nabla_\X\nabla_\Y \Z-
        $$
        $$
        \p_\Y\p_\X h\Z-\p_\X h\nabla_\Y\Z-
    \p_\Y h\nabla_\X\Z+
    h\nabla_\Y\nabla_\X \Z-
        $$
        $$
      \p_{[\X,\Y]}h \Z-h\nabla_{[\X,\Y]}\Z=
        $$
        $$
    h\left[\nabla_{\X}\nabla_\Y\Z-\nabla_{\Y}\nabla_\X\Z)-
   \nabla_{[\X,\Y]}\Z\right]+\left[\p_\X\p_\Y h-\p_\Y\p_\X h-\p_{[\X,\Y]h}\right]\Z=
        $$
        $$
h\nabla_{\X}\nabla_\Y\Z-\nabla_{\Y}\nabla_\X\Z)-
   \nabla_{[\X,\Y]}\Z=h{\cal R}(\X,\Y)\Z\,,
        $$
   since $\p_\X\p_\Y h-\p_\Y\p_\X h-\p_{[\X,\Y]}h=0$.

%\end{document} %24 April
}



\subsubsection {Properties of curvature tensor}

  Tensor $R^i_{\,\,kmn}$ is expressed through derivatives of Christoffel symbols.
  In spite this fact it is is much more "pleasant" object than Christoffel symbols, since the latter is not the tensor.

 It follows from the definition that the tensor $R^i_{kmn}$ is antisymmetrical with respect
to indices $m,n$:
    \begin{equation}\label{antisymmetricitycurvature1}
      R^i_{kmn}=-R^i_{knm}\,.
    \end{equation}



 {\footnotesize One can prove that for symmetric connection this tensor obeys the following identities:
              \begin{equation}\label{ciclicity}
                R^i_{\,\,kmn}+R^i_{mnk}+R^i_{nkm}=0\,,
                    \end{equation}

}


 The curvature tensor corresponding to Levi-Civita connection obeys also another identities too
 (see the next subsection.)

   We know well that
   If Christoffel symbols vanish in a vicinity of a given point 
$\pt$ in some chosen  coordinate system
   then in general Christoffel symbols do not vanish in arbitrary coordinate systems.
   E.g. Christoffel symbols of canonical flat connection in $\E^2$
   vanish in Cartesian coordinates but do not vanish in polar coordinates. This unpleasant property
   of Christoffel symbols is due to the fact that Christoffel symbols do not form a tensor.




  In particular if a tensor vanishes in some coordinate system, then it vanishes in arbitrary coordinate system too.
    This implies
         very simple but important

       {\bf Proposition}
     {\it If curvature tensor $R^i_{\,\,kmn}$ vanishes in some coordinate  system, then it
      vanishes in arbitrary coordinate systems.}

\m
   We see that if Christoffel symbols vanish in a vicinity of a given point
   $\pt$ in some chosen  coordinate system then 
its Riemannian curvature tensor vanishes
   in a vicinity of the  point $\pt$ (see the formula \eqref{curvatureincomponents}) and hence it vanishes
   locally  (in a vicinity of point $\pt$) in arbitrary coordinate system.

  In fact one can prove 

 {\bf Theorem} If a connection is symmetric then
 curvature tensor vanishes in a vicinity of a point if and only 
  there exist local Cartesian coordiantes,n a vicinity of this point
  i.e. coordinates in which Christoffel symbol of connection vanish.

 

%\end{document} % 26 April

   \subsection {Riemann curvature tensor of Riemannian manifolds.}



   Let $M$ be Riemannian manifold equipped  with Riemannian metric $G$

    In this section we will consider curvature tensor of Levi-Civita connection $\nabla$ of Riemannian metric $G$.

    The curvature tensor for Levi-Civita connection will be called later Riemann curvature tensor, or Riemann tensor.

  Using Riemannian metric one can consider Riemann tensor with all low indices
     \begin{equation}\label{lowindices}
    R_{ikmn}=g_{ij}R^j_{\,\,kmn}
\end{equation}



In the subsection above we formulated very important
Theorem that vanishing of curvature tensor means that connection
is locally flat. For Riemann tensor one can formulate the
analogous  Theorem. If Riemannina manifold is
locally Euclidean, i.e. there exist coordinates
 $(x^1,\dots,x^n)$  such that
$G=(dx^1)^2+\dots+(dx^n)^2$ then it is evident that 
Christoffel symbols of Levi-Civita connection
vanish in these coordinates, hence
curvature tensor vanishes also. 
  The converse implication is true also:



  {\bf Theorem}  Riemann curvature tensor vanishes 
if and only if Riemannian manifold is locally 
Euclidean, i.e. if $R^i_{kmn}\equiv 0$ in a vicinity of
the point $\pt$ of Riemannian manifold, then
in a vicinity of this point 
there exist local coordinates
  $(x^1,\dots,x^n)$  such that
Riemannian metric $G=(dx^1)^2+\dots+(dx^n)^2$.

  
 


  For Riemann tensor one can consider Ricci tensor,
      \begin{equation}\label{Riccitensor}
      R_{mn}=R^i_{min}
\end{equation}
which is symmetrical tensor:  $R_{mn}=R_{nm}$.

One can consider scalar curvature:
                        \begin{equation}\label{scalarcurvature}
      R=R^i_{\,kin}g^{kn}=g^{kn}R_{kn}
\end{equation}
where $g^{kn}$ is Riemannian metric with indices above 
(the matrix $||g^{ik}||$ is inverse to the matrix
 $||g_{il}||$).

Ricci tensor and scalar curvature also play essential role for 
formulation of famous Einstein gravity equations.
In particular the space without matter the Einstein equations 
have the following form:
                \begin{equation}\label{gravityequation}
                  R_{ik}-{1\over 2}Rg_{ik}=0\,.
                \end{equation}

\m

{\footnotesize


     Due to identities \eqref{antisymmetricitycurvature1} 
and \eqref{ciclicity}
 for curvature tensor  Riemann tensor obeys  the following identities:
\begin{equation}\label{identityforriemantensor1}
  R_{ikmn}=-R_{iknm}\,\,\,,\quad R_{ikmn}+R_{imnk}+R_{inkm}=0
\end{equation}

   Riemann curvature tensor which is curvature tensor for Levi-Civita connection obeys also the following identities:
            \begin{equation}\label{identityforriemantensor2}
R_{ikmn}=-R_{kimn}\,\,\,,\quad R_{ikmn}=R_{mnki}\,.
\end{equation}

These condition lead to the fact that for $2$-dimensional Riemannian manifold the Riemann curvature tensor
of Levi-Civita connection has essentially only one non-vanishing component: all components
vanish or equal to component $R_{1212}$up to a sign.



Indeed consider for $2$-dimensional Riemannian manifold Riemann tensor $R_{ikmn}$, where
$i,k,m,n=1,2$.  Since antisymmetricity with respect to third and fourth indices
($R_{ikmn}=-R_{iknm}$), $R_{ik11}=R_{ik22}=0$ and $R_{ik12}=-R_{ik21}$.
The same for first and second indices:
 since antisymmetricity with respect to the the first  and second indices
($R_{12mn}=-R_{21mn}$), $R_{11mn}=R_{22mn}=0$ and $R_{12mn}=-R_{ik21}$.
If we denote $R_{1212}=a$ then
                  \begin{equation}\label{2dimriemann}
R_{1212}=R_{2121}=a, R_{1221}=R_{2112}=-a
                  \end{equation}
                  and all other components vanish.

}

   \subsubsection {
Relation between Gaussian curvature and Riemann curvature tensor
   and {\sl Theorema Egregium}}
\label{anotherproofofEgregium}

  For surfaces in $\E^3$ Gaussian curvature is equal to
half of scalar curvature:
              \begin{equation}\label{halfofscalarcurvature}
                K={R\over 2}\,,
              \end{equation}
where $R=R^i_{kin}g^{kn}$ is scalar curvature of Riemann curvature tensor.

  
 This equation is the fundamental relation which claims that the
 Gaussian curvature  (the magnitude defined in terms of
 External observer) equals to the scalar curvature (up to a coefdficient),
 the magnitude defined in terms of Internal Observer.
   This gives us another proof of Theorema Egregium.
(see the proof of equation \eqref{halfofscalarcurvature} in
  subsection ``Theorema Egregium again'')

  \m

    Consider this little bit more in details.

\m

Let $M$ be a surface in $\E^3$ and $R^i_{\,\,kmp}$ be Riemann tensor, Riemann
curvature tensor of Levi-Civita connection. Recall that this means that
$R^i_{\,\,kmp}$ is curvature tensor of the connection $\nabla$, which is
Levi-Civita connection of the Riemannian metric $g_{\a\beta}$ induced on the surface
$M$ by standard Euclidean metric $dx^2+dy^2+dz^2$.
Recall that
  Riemann curvature tensor is expressed via Christoffel symbols of 
connection by the formula
  \begin{equation}\label{curvaturetensorintermsofconnection4}
    R^i_{\,\,kmn}=\p_m\Gamma_{nk}^i+\Gamma_{mp}^i\Gamma^p_{nk}-\p_n\Gamma_{mk}^i-\Gamma_{np}^i\Gamma^p_{mk}
\end{equation}
(see he formula \eqref{curvatureincomponents})
where Christoffel symbols of Levi-Civita connection are defined by the formula
\begin{equation}\label{Levi-Civitaagain}
                     \Gamma^i_{mk}={1\over 2}g^{ij}\left({\p g_{jm}\over\p x^k}+
    {\p g_{jk}\over\p x^m}-{\p g_{mk}\over \p x^j}\right)
\end{equation}
(see Levi-Civita Theorem)

Recall that scalar curvature $R$ of Riemann tensor equals to
$R=R^i_{kim}g^{km}$, where $g^{km}$ is Riemannian metric tensor with 
upper indices
(matrix $||g^{ik}||$ is inverse to the matrix $||g_{ik}||$). 

Now consider $2$-dimensional case. One can show that in this case
scalar curvature $R$ can be expressed via the component $R_{1212}=a$ by 
the formula
\begin{equation}\label{scalarcurvatureviaonecomponent}
    R={2R_{1212}\over \det g}
\end{equation}
where $\det g=\det g_{ik}=g_{11}g_{22}-g_{12}^2$.

{\footnotesize

To see it note
that as it was mentioned in the subsection above
the formula for scalar curvature becomes very simple in two-dimensional 
case (see formulae
\eqref{identityforriemantensor1}
and \eqref{2dimriemann} above) and in this case
it is very easy
to calculate Ricci tensor
and scalar curvature $R$. Indeed let $R_{1212}=a$. 
For $2$-dimensional Riemannian surface  all 
other components of Riemann tensor equal
to zero or equal to $\pm a$ (see
\eqref{identityforriemantensor1}
and \eqref{2dimriemann}).
{\footnotesize
 Show it. Using identities \eqref{identityforriemantensor1}
and \eqref{identityforriemantensor1} we see that
 \begin{equation}\label{ricchitensor1}
    R_{11}=R^i_{\,\,1i1}=R^2_{\,\,121}=g^{22}R_{2121}+g^{21}R_{1121}=g^{22}R_{1212}=g^{22}a
\end{equation}
\begin{equation}\label{ricchitensor1}
    R_{22}=R^i_{\,\,2i2}=R^1_{\,\,212}=g^{11}R_{1212}+g^{12}R_{2221}=g^{11}R_{1212}=g^{11}a
\end{equation}
\begin{equation}\label{ricchitensor1}
    R_{12}=R_{21}=R^i_{\,\,1i2}=R^1_{\,\,112}=g^{12}R_{2112}=-g^{12}R_{1212}=-g^{12}a
\end{equation}
Thus using the formula for inverse $2\times 2$ matrix we come to the relation
         \begin{equation*}
     R_{ik}=
   \begin{pmatrix} R_{11}&  R_{12}\cr R_{21}& R_{22}\end{pmatrix}=
   \begin{pmatrix}g^{22} a& -g^{12}a\cr -g^{21}a & g^{11} a\end{pmatrix}=
   {1\over \det g}
  \begin{pmatrix}g_{11} a& g_{12}a\cr g_{21}a & g_{11} a\end{pmatrix}\,,
         \end{equation*}
  i.e. for $2$-dimensional Riemannian manifold
           \begin{equation}\label{identity4}
               R_{ik}={1\over \det g}R_{1212}g_{ik}\,,
              \end{equation}
Hence for scalar curvatre of $2$-dimension Riemannian manifold 
\begin{equation}\label{scalarcurvatureintermsofonecomponent}
    R=R^i_{\,\,kim}g^{km}=R_{km}g^{km}=
    {1\over \det g}R_{1212}g_{ik}g^{ik}={2\over \det g}R_{1212}\,.
\end{equation}

}

{\footnotesize Note that the relations \eqref{identity4} 
and \eqref{scalarcurvatureintermsofonecomponent} imply thatt
    \begin{equation}\label{2-gravity}
    R_{ik}={1\over 2}Rg_{ik}\,.
\end{equation}
One can say that gravity equation for $n=2$ are trivial. 
The mathematical meaning of this formula is
the following: equation \eqref{2-gravity} means that 
variation of functional $S=\int R\sqrt {\det g}d\sigma$
vanishes and this is one of corollaries of Gauss-Bonnet 
Theorem (see later).


}


}
\m

Formula \eqref{scalarcurvatureviaonecomponent} expresses 
scalar curvature for surface in terms of
non-trivial component $R_{1212}$. 
On the other hand  Gaussian curvature
of $2$-dimensional surface is equal to the half of
the scalar curvature (see equation \eqref{halfofscalarcurvature}).
  Hence we come to 


{\bf Proposition} {\it For an arbitrary point of the surface $M$
 \begin{equation}\label{egregiumagain}
    K={R\over 2}={R_{1212}\over \det g}\,.
\end{equation}
where $R=R^i_{kim}g^{km}$ is scalar curvature and $K$ is Gaussian curvature.}


We know also that for surface $M$ the scalar curvature $R$ is expressed via Riemann curvature tensor
by the formula \eqref{scalarcurvatureviaonecomponent}. 
Hence if we know the Gaussian curvature
then we know all components of Riemann curvature tensor 
(since all components vanish or equal to $\pm a$.).
This is nothing but Theorema Egregium!
Theorema Egregium  (see beginning of the section 4) immediately 
follows from this Proposition which states
that Gaussian curvature is equal (up to a coefficient)
to scalar curvature whcih is expressed in terms of Riemannian metric.

 
One can check equation \eqref{egregiumagain}
just by brute force calculating  Riemannian metric
 and Riemanian curvature tensor (see Appendices)

Here we consider just   a simple example.


\m

{\bf Example}  Let $M=S^2$ be sphere of radius $R$ in $\E^3$.
 Show that one cannot find local coordinates $u,v$ on the sphere such that induced
Riemannian metric equals to $du^2+dv^2$ in these coordinates.

This immediately follows from the Proposition. Indeed suppose there exist
local coordinates $u,v$ on the sphere such that induced
Riemannian metric equals to $du^2+dv^2$, i.e. Riemannian metric is given by unity matrix.
Then according to the formulae for Levi-Civita connection, the Christoffel symbols
equal to zero in these coordinates. Hence Riemann curvature tensor equals to zero,
and scalar curvature too. Due to Proposition this is in contradiction with the fact
that  Gaussian curvature of the sphere equals to $1\over R^2$.

(The straightforward proof see in the next paragraph)



\m




  \subsection {Gauss Bonnet Theorem}



  Consider the integral of curvature over whole closed surface $M$. According to the Theorem
  above the answer has to be equal to $0$ (modulo $2\pi$), i.e. $2\pi N$ where $N$ is an integer,
  because this integral is a limit when we consider very small curve. We come to the formula:
         $$
         \int_D Kd\sigma=2\pi N
         $$
(Compare this formula with formula \eqref{theoremofrotationonangle}).


  What is the value of integer $N$?


We present now one remarkable Theorem which answers this question and prove this Theorem using the
formula \eqref{theoremofrotationonangle}.


Let $M$ be a closed orientable surface.\footnote{Closed means compact surface without boundaries.
Intuitively orientability means that one can define out and inner side of the surface.
  In terms of normal vectors
orientability means that
one can define the continuous field of normal vectors at all the points of $M$.
The direction of normal vectors at any point defines outward direction.
Orientable surface is called oriented if the direction of normal vector is chosen.}
All these surfaces can be classified up to a diffeomorphism.
Namely arbitrary  closed oriented surface $M$
   is diffeomorphic either to  sphere (zero holes),
   or torus (one hole), or pretzel (two holes),...
"Number k" of holes is intuitively evident characteristic of the surface.
It is related with very important characteristic---Euler characteristic
$\chi(M)$ by the following formula:
\begin{equation}\label{defofeuler00}
  \chi(M)=2(1-g(M)),\quad \hbox {where $g$ is number of holes}
\end{equation}


{\bf Remark} What we have called here "holes" in a surface is often referred
to as "handles" attached o the sphere, so that the sphere itself does not have any handles,
the torus has one handle, the pretzel has two handles and so on. The number of handles is also called genus.

\smallskip

Euler characteristic appears in many different way. The simplest appearance is the following:

Consider on the surface $M$  an arbitrary  set of points (vertices) connected
with edges (graph on the surface) such that surface is divided on
 polygons with (curvilinear sides)---plaquets. ("Map of world")

Denote by $P$ number of plaquets (countries of the map)

Denote by $E$ number of edges (boundaries between countries)

Denote by $V$ number of vertices.

Then it turns out that
\begin{equation}\label{defofeulerchar100}
  P-E+V=\chi(M)
\end{equation}
It does not depend on the graph, it depends only on how much holes has surface.

E.g. for every graph on $M$, $P-E+V=2$ if $M$ is diffeomorphic to sphere.
For every graph on $M$ $P-E+V=0$ if $M$ is diffeomorphic to torus.


\bigskip


Now we formulate Gau\ss\,-Bonnet Theorem.


Let $M$ be closed oriented surface in $\E^3$.

Let $K(p)$ be Gaussian curvature at any point $p$ of this surface.



\medskip

{\bf  Theorem} (Gau\ss\,\,-Bonnet)
The integral of Gaussian curvature over the
closed compact oriented surface  $M$ is equal to $2\pi$ multiplied
by the Euler characteristic of the surface $M$
\begin{equation}\label{gaussbonnet}
  {1\over 2\pi}\int_M Kd\sigma=\chi(M)=2(1-\hbox{number of holes})
\end{equation}

In particular for the surface $M$ diffeomorphic to the sphere   $\kappa(M)=2$,
for the surface diffeomorphic to the torus it is equal to $0$.

\smallskip

The value of the integral does not change under continuous deformations of
surface! It is integer number (up to the factor $\pi$) which characterises
topology of the surface.

E.g. consider surface $M$ which is diffeomorphic to the sphere.
If it is sphere of the radius $R$ then curvature is equal to $1\over R^2$,
 area of the sphere is equal to $4\pi R^2$ and left hand side
 is equal to ${4\pi \over 2\pi}=2$.

If surface $M$ is an arbitrary surface diffeomorphic to $M$
then  metrics and curvature depend from point to the point,
Gau\ss\,-Bonnet states that integral nevertheless remains unchanged.

\smallskip

Very simple but impressive corollary:

{\it Let $M$ be surface diffeomorphic to sphere in $\E^3$. Then
there exists at least one point where Gaussian curvature is positive.}

Proof: Suppose it is not right. Then $\int_M K\sqrt {\det g}dudv\leq 0$.
On the other hand according to the Theorem it is equal to $4\pi$. Contradiction.






\bigskip
{\footnotesize
 {\it Proof of Gau\ss-Bonet Theorem}

 Consider triangulation of the surface $M$. Suppose $M$ is covered by $N$ triangles. Then number
 of edges will be $3N/over 2$. If $V$ number of vertices then according to Euler Theorem
                $$
        N-{3N\over 2}+V=V-{N\over 2}=\chi(M).
                $$
Calculate the sum of the angles of all triangles.
On the one hand it is equal to $2\pi V$. On the other hand
according the formula \eqref{theoremofrotationonangle} it is equal to
           $$
   \sum_{i=1}^N\left(\pi+\int_{\triangle_i}K d\sigma\right)=
   \pi N+\sum_{i=1}^N\left(\int_{\triangle_i}K d\sigma\right)=
   N\pi+\int_M Kd\sigma
           $$
We see that $2\pi V=N\pi+\int_M Kd\sigma$, i.e.
            $$
         \int_M Kd\sigma=\pi \left(2V-{N\over 2}\right)=2\pi \chi(M)\hbox{\finish}
            $$
}


%\end{document}

% this has to be erased  15 April 2018

\end{document}


We have
     $$
x\to  {{x\over x^2+y^2}+\vare\over
  \left({x\over x^2+y^2}+\vare\right)^2+
     \left({y\over x^2+y^2}\right)^2 }=
      {x+\vare(x^2+y^2)\over 
      1+2\vare x}=x+\vare (y^2-x^2)\,,\, (\vare^2=0)\,
     $$
and
      $$
y\to {{y\over x^2+y^2}\over
  \left({x\over x^2+y^2}+\vare\right)^2+
     \left({y\over x^2+y^2}\right)^2 }= 
      {y\over 
      1+2\vare x}=y-2\vare xy\,,\, (\vare^2=0)\,
      $$
The corresponding vector field is $D_3=(y^2-x^2)\p_x-2xy\p_y$.

(these fiellds form complex representation of $so(3)$.)

  Every Killing vector field $K^i\p_i$ induces integral of motion
$I_k=K^i{\p L\over\p \dot x^i}$. Hence we have:
      $$
\begin{matrix}
D_1=\p_x\mapsto I_1={\p L\over\p \dot x}={\dot x\over y^2}\cr
D_2=x\p_x+y\p_y\mapsto I_2=x{\p L\over\p \dot x}+y{\p L\over\p \dot y}=
 {x\dot x+y\dot y\over y^2}\cr
D_3=(y^2-x^2)\p_x-2xy\p_y\mapsto I_3=
(y^2-x^2){\p L\over\p \dot x}-2xy{\p L\over \p \dot y}=
     {(y^2-x^2)\dot x-2xy\dot y\over y^2}\cr
\end{matrix}
      $$
Geodesics is trajectory of free particle. It is well-defined by 
three conditions $I_1=C_1, I_2=C_2, I_3=C_3$. We have linear equations
 on $\dot x,\dot y$           $$
                         \begin{cases}
                    \dot x =       y^2 C_1\cr
                   x \dot x+y \dot y=y^2 C_2\cr
                 (y^2-x^2)\dot x-2xy \dot y=y^2 C_3\cr
                               $$
Compatibility of these three linear equations 
%on two variables $\dot x $, $\dot y$ 
 implies that
              $$
\det\begin{pmatrix} 
         1 &0 & y^2 C_1\cr
         x &y & y^2 C_2\cr
        (y^2-x^2) &-2xy & y^2 C_3  
            \end{pmatrix}=0
              $$

Calculating the determinant we come to
          $$
    y^3(C_3-2xC_2+C_1(x^2+y^2))=0\,.
         $$
The curves which are given by these equations are non-prameterised geodesics
%($y\not=0$). 
We see that these curves are half-circles with centre 
%on the absolute ($y=0$), or vertical rays (if $C_1=0$).
   Parameter of geodesic is proportional 
to the natural (length) since connection is Levi-Civita connection.

%\end{document}

