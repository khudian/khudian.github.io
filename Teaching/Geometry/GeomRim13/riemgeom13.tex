% It is a new course. I began this document on 31-st January 2010.
% This is version of 2013


\def\vare {\varepsilon}
\def\A {{\bf A}}
\def\t {\tilde}
\def\a {\alpha}
\def\K {{\bf K}}
\def\N {{\bf N}}
\def\V {{\cal V}}
\def\s {{\sigma}}
\def\S {{\bf S}}
\def\s {{\sigma}}
\def\p{\partial}
\def\vare{{\varepsilon}}
\def\Q {{\bf Q}}
\def\D {{\cal D}}
\def\G {{\Gamma}}
\def\C {{\bf C}}
\def\M {{\cal M}}
\def\Z {{\bf Z}}
\def\U  {{\cal U}}
\def\H {{\cal H}}
\def\R  {{\bf R}}
\def\E  {{\bf E}}
\def\l {\lambda}
\def\degree {{\bf {\rm degree}\,\,}}
\def \finish {${\,\,\vrule height1mm depth2mm width 8pt}$}
\def \m {\medskip}
\def\p {\partial}
\def\r {{\bf r}}
\def\v {{\bf v}}
\def\n {{\bf n}}
\def\t {{\bf t}}
\def\b {{\bf b}}
\def\e{{\bf e}}
\def\f{{\bf f}}
\def\ac {{\bf a}}
\def \X   {{\bf X}}
\def \Y   {{\bf Y}}
\def\diag {\rm diag\,\,}
\def\pt {{\bf p}}
\def\w {\omega}
\def\la{\langle}
\def\ra{\rangle}
\def\x{{\bf x}}

\documentclass[12pt]{article}
\usepackage{amsmath,amsthm}


\usepackage{amsmath,amssymb,amsfonts,amsthm}


\theoremstyle{theorem}
\newtheorem{thm}{Khimera}

\numberwithin{equation}{section}


\title{Riemannian Geometry}
\date{}
\begin{document}
\maketitle

  \centerline {it is a draft of Lecture Notes of H.M. Khudaverdian.}

  \centerline { Manchester, 28 February  2013}





\tableofcontents


\section {Riemannian manifolds}


\subsection { Manifolds. Tensors. (Recalling)}


I recall briefly basics of manifolds and tensor fields on manifolds.

An $n$-dimensional manifold is a space such that in a vicinity of any point
one can consider local coordinates $\{x^1,\dots,x^n\}$ (charts). One can consider different local coordinates.
If both coordinates $\{x^1,\dots,x^n\}$, $\{y^1,\dots,y^n\}$ are defined in a vicinity of the given point
then they are related by  bijective transition functions (functions defined on domains in $\R^n$ and taking values in $\R^n$).
              $$
             \begin{cases}
             x^{1'}=x^{1'}(x^1,\dots,x^n)\cr
             x^{2'}=x^{2'}(x^1,\dots,x^n)\cr
                \dots\cr
            x^{{n-1}'}=x^{{n-1}'}(x^1,\dots,x^n)\cr
              x^{{n}'}=x^{{n}'}(x^1,\dots,x^n)\cr

             \end{cases}
              $$
 We say that manifold is {\it differentiable} or {\it smooth} if transition functions are diffeomorphisms,
i.e. they are smooth and rank of Jacobian is equal to $k$, i.e.
              \begin{equation}\label{transfunctions---diffeomorphisms}
      \det
          \begin{pmatrix}
          {\p x^{1'}\over \p x^1} &{\p x^{1'}\over \p x^2}\dots &{\p x^{1'}\over \p x^n}\cr
         {\p x^{2'}\over \p x^1} &{\p x^{2'}\over \p x^2} \dots &{\p x^{2'}\over \p x^n}\cr
                           \dots\cr
      {\p x^{n'}\over \p x^1} &{\p x^{n'}\over \p x^2}\dots &{\p x^{n'}\over \p x^n}\cr
          \end{pmatrix}\not=0\,.
              \end{equation}




    A good example of manifold is an open domain  $D$ in  $n$-dimensional vector space $\R^n$.
    Cartesian coordinates on $\R^n$ define global coordinates on $D$.
    On the other hand one can consider an arbitrary local coordinates in different domains in $\R^n$. E.g. one can consider polar coordinates $\{r,\varphi\}$ in a domain $D=\{x,y\colon y>0\}$ of $\R^2$
    defined by standard formulae:
              \begin{equation}\label{exampleofpolarcoordinates}
                \begin{cases}
                x=r\cos\varphi\cr y=r\sin\varphi\cr
                \end{cases}\,,
              \end{equation}
            \begin{equation}\label{jacobianofpolartransform}
                \det
          \begin{pmatrix}
          {\p x\over \p r} &{\p x\over \p \varphi}\cr
          {\p y\over \p r} &{\p y\over \p \varphi}\cr
          \end{pmatrix}=
                \det
          \begin{pmatrix}
          \cos\varphi &-r\sin\varphi\cr
          \sin\varphi &r\cos\varphi\cr
          \end{pmatrix}=r
            \end{equation}
             or one can consider spherical coordinates  $\{r,\theta,\varphi\}$ in a domain
  $D=\{x,y,z\colon x>0,y>0, z>0\}$ of $\R^3$ (or in other domain of $\R^3$)
  defined by standard formulae
  \begin{equation*}
                \begin{cases}
                x=r\sin\theta\cos\varphi\cr y=r\sin\theta\sin\varphi\cr z=r\cos\theta
                \end{cases}\,,
              \end{equation*}\,,
              \begin{equation}\label{jacobianofspherictransform}
                \det
          \begin{pmatrix}
          {\p x\over \p r} &{\p x\over \p \theta} &{\p x\over \p \varphi}\cr
          {\p y\over \p r} &{\p y\over \p \theta} &{\p y\over \p \varphi}\cr
           {\p z\over \p r} &{\p z\over \p \theta} &{\p z\over \p \varphi}\cr
          \end{pmatrix}=
                \det
          \begin{pmatrix}
          \sin\theta\cos\varphi &r\cos\theta\cos\varphi &-r\sin\theta\sin\varphi\cr
          \sin\theta\sin\varphi &r\cos\theta\sin\varphi &r\sin\theta\cos\varphi\cr
           \cos\theta &-r\sin\theta &0\cr
                 \end{pmatrix}=r^2\sin\theta
            \end{equation}

 Choosing domain where polar (spherical) coordinates are well-defined
we have to be award
that coordinates have to be well-defined and
transition functions \eqref{transfunctions---diffeomorphisms}
  have to be diffeomorphisms. (E.g. for domain $D$
in example \eqref{exampleofpolarcoordinates} Jacobian
\eqref{jacobianofpolartransformation} does not vanish: $r>0$ in $D$.)
\m


Examples of manifolds: $\R^n$, Circle $S^1$,  Sphere $S^2$, in general sphere $S^n$, torus $S^1\times S^1$,
  cylinder, cone, \dots.

{\bf Example} Consider in detail circle $S^1$.
  Suppose it is given as $x^2+y^2=1$.
One can consider two local polar coordinates:
1) $\varphi$, $0<\varphi<2\pi$  which covers all points except the point
$(1,0)$ and $\varphi'\colon -\pi<\varphi'<\pi$ which covers all points
except point $(-1,0)$. In a vicinity of
any point $M$ on the circle (except
these two exceptional points) one can consider both coordinates
$\varphi$ and $\varphi'$


We come to another
 very useful coordinates on a circle using {\it stereographic projection}.
Take north pole of the circle: the point $N=(0,1)$.
Assign to every point  $M=(x,y)$ on the circle
the point $(t,0)$ on the $x$-axis such that the point $(t,0)$,
the point $M$ and the north pole $N$ are on the one line.
This can be done for every point
of circle except the north pole $(0,1)$ itself.
We come to stereographic projection of circle $S^1$ without North pole
on the line $\R$. In the same way we can define
stereographic projection of circle
without south pole (the point $(0,-1)$)
on the $x$-axis. We come to coordinate $t'$.
One can see that these coordinates are related
by the following simple formula:
             $$
          t'={1\over t}\,,
             $$
{\small $^\dagger$ One very important property of stereographic projetion
which we do not use in this course but it is too beautiful not to mention it:
under stereographic projection  all points
on the circle $x^2+y^2=1$
with rational coordinates $x$ and $y$ and only these points
transform to rational points on line. Thus we come to
Pythagorean triples $a^2+b^2=c^2$.}






        \centerline {\it Tensors on Manifold}

Recall briefly what are tensors on manifold.
For every point $\pt$ on manifold $M$
one can consider tangent vector space $T_\pt M$---
the space of vectors tangent to the manifold at the point $M$.


 Tangent vector $\A=A^i{\p\over \p x^i}$.
Under changing of coordinates it transforms as follows:
             $$
         \A=A^i{\p\over \p x^i}=
A^i{\p x^{m'}\over \p x^i}{\p\over \p x^{m'}}=A^{m'}{\p\over \p x^{m'}}\,.
             $$
  Hence

            \begin{equation}\label{transformofvector}
                   A^{i'}={\p x^{i'}\over \p x^i}A^i\,.
            \end{equation}
              Consider also cotangent space $T^*_\pt M$
(for every point $\pt$ on manifold $M$)---
              space of linear functions on tangent vectors, i.e. space of $1$-forms which sometimes are called
              {\it covectors.}:

         One-form (covector) $\w=\w_idx^i$ transforms as follows
            $$
      \w=\w_m dx^m=\w_m {\p x^m\over \p x^{m'}}dx^{m'}=\w_{m'}dx^{m'}\,.
        $$
      Hence
         \begin{equation}\label{transformofoneform}
            \w_{m'}={\p x^{m}\over \p x^{m'}}\w_m\,.
         \end{equation}
Differential form sometimes is called {\it covector}.

          {\it Tensors}:
\smallskip

One can consider contravariant tensors  of the rank $p$
                   $$
             T=  T^{i_1i_2\dots i_p}{\p\over \p x^{i_1}}\otimes
             {\p\over \p x^{i_2}}\otimes \dots \otimes {\p\over \p x^{i_k}}
                   $$
with components $\{T^{i_1i_2\dots i_k}\}$.


One can consider covariant tensors  of the rank $q$
                   $$
             S=  S_{j_1j_2\dots j_q}dx^{j_1}\otimes
              dx^{j_2}\otimes \dots dx^{j_q}
                   $$
with components $\{S_{j_1j_2\dots j_q}\}$.

One can also consider mixed tensors:
          $$
        Q=Q^{i_1i_2\dots i_p}_{j_1j_2\dots j_q}{\p\over \p x^{i_1}}\otimes
             {\p\over \p x^{i_2}}\otimes \dots \otimes {\p\over \p x^{i_k}}\otimes
             dx^{j_1}\otimes
              dx^{j_2}\otimes \dots dx^{j_q}
          $$
with components $\{Q^{i_1i_2\dots i_p}_{j_1j_2\dots j_q}\}$.
We call these tensors
{\it tensors of the type  $\begin{pmatrix} p\cr q\cr\end{pmatrix}$}.
Tensors of the type $\begin{pmatrix} p\cr 0\cr\end{pmatrix}$ are called
{\it contravariant tensors of the rank p}.
They have $p$ upper indices.

Tensors of the type $\begin{pmatrix} 0\cr q\cr\end{pmatrix}$ are called
{\it covariant tensors of the rank q}.
They have $q$ lower indices.

Having in mind \eqref{transformofvector} and
\eqref{transformofoneform} we come  to
the rule of transformation for tensors
which have $p$ upper and $q$ lover indices, tensors
of type  $\begin{pmatrix} p\cr q\cr\end{pmatrix}$:
\begin{equation}\label{ruleoftransformationofarbitrarytensors}
    Q^{i'_1i'_2\dots i'_p}_{j'_1j'_2\dots j'_q}=
    {\p x^{i'_1}\over \p x^{i_1}}
    {\p x^{i'_2}\over \p x^{i_2}}
    \dots
    {\p x^{i'_p}\over \p x^{i_p}}
    {\p x^{j_1}\over \p x^{j'_1}}
    {\p x^{j_2}\over \p x^{j'_2}}
    \dots
    {\p x^{j_q}\over \p x^{j'_q}}
    Q^{i_1i_2\dots i_p}_{j_1j_2\dots j_q}
\end{equation}



E.g. if $S_{ik}$ is a covariant tensor
of rank 2 then
            \begin{equation*}\label{ruleoftransformfor 2-tensor}
                S_{i'k'}=
   {\p x^i\over \p x^{i'}}
   {\p x^k\over \p x^{k'}}
             S_{ik}\,.
            \end{equation*}
 If $A^i_k$ is a tensor of rank $\begin{pmatrix} 1\cr 1\cr\end{pmatrix}$ (linear operator on $T_\pt M$)
 then

             $$
    A^{i'}_{k'}=        {\p x^{i'}\over \p x^{i}}{\p x^k\over \p x^{k'}}A^i_k
             $$

If $S^m_{ik}$ is a
tensor of the type $\begin{pmatrix} 1\cr 2\cr\end{pmatrix}$) then
            \begin{equation}\label{ruleoftransformfor 2-tensor2}
                S^{m'}_{i'k'}=
   {\p x^{m'}\over \p x^m}
   {\p x^i\over \p x^{i'}}
   {\p x^k\over \p x^{k'}}
             S^{m}_{ik}\,.
            \end{equation}





{\bf Remark} Transformations formulae
\eqref{transformofvector}---\eqref{ruleoftransformfor 2-tensor2}  define
vectors, covectors and in generally any tensor fields in components. E.g. covariant tensor (covariant tensor field)
of the rank 2 can be defined as matrix $S_{ik}$ (matrix valued function $S_{ik}(x)$) such that
under changing of coordinates $\{x^1,x^2,\dots,x^n\}\mapsto \{x^{1'},x^{2'},\dots,x^{n'}\}$
 $S_{ik}$ change by the rule \eqref{ruleoftransformfor 2-tensor}.


{\bf Remark} {\it Einstein summation rules}


 In our lectures we always use so called {\it Einstein summation convention}.
 it  implies that when an index occurs twice
in the same expression in upper and in lower postitions, then
 the expression is implicitly summed over all possible values
for that index.
  Sometimes it is called dummy indices summation rule.

%31 January

%\end{document}


    \subsection {Riemannian manifold---
manifold equipped with Riemannian metric}

{\bf Definition} The Riemannian manifold is a
manifold equipped with a Riemannian metric.




  The Riemannian metric on the manifold $M$ defines the
  length of the tangent vectors and the length of the curves.

{\bf Definition}
  Riemannian metric $G$ on n-dimensional manifold $M^n$
  defines for every point $\pt\in M$ the scalar product
  of tangent vectors in the tangent space $T_\pt M$
  smoothly depending on the point $\pt $.

  It means that in every coordinate system $(x^1,\dots,x^n)$
  a metric $G=g_{ik}dx^idx^k$ is defined by a matrix valued smooth function $g_{ik}(x)$ ($i=1,\dots,n;k=1,\dots n$)
  such that for any two vectors
       $$
  {\bf A}=A^i(x){\p\over \p x^i},\,\, {\bf B}=B^i(x){\p\over \p x^i},
      $$
tangent to the manifold $M$ at the point $\pt$ with coordinates $x=(x^1,x^2,\dots,x^n)$ ($\A,{\bf B}\in T_{\pt}M$)
the scalar product is equal to:
              $$
              \langle\A,{\bf B}\rangle_G\big\vert_\pt= G({\bf A},{\bf B})\big\vert_\pt=
A^i(x)g_{ik}(x)B^k(x)=
            $$
            \begin{equation}\label{scalarproduct}
  \begin{pmatrix}
   A^1 \dots A^n\\
   \end{pmatrix}
  \begin{pmatrix}
     g_{11}(x)&\dots &g_{1n}(x)\\
      \dots &\dots & \dots \\
         g_{n1}(x)&\dots &g_{nn}(x)\\  \\
   \end{pmatrix}
\begin{pmatrix}
   B^1\\
     \cdot \\
   \cdot\\
   \cdot\\
   B^n\\
   \end{pmatrix}
\end{equation}

where
\begin{itemize}

  \item  $G({\bf A},{\bf B})=G({\bf B},{\bf A})$, i.e.  $g_{ik}(x)=g_{ki}(x)$ (symmetricity condition)

    \item
       $G({\bf A},{\bf A})>0$ if $\bf A\not=0$, i.e.

    $g_{ik}(x)u^iu^k\geq 0$, $g_{ik}(x)u^iu^k=0$ iff $u^1=\dots=u^n=0$  (positive-definiteness)

   \item  $G({\bf A},{\bf B})\big\vert_{\pt=x}$, i.e. $g_{ik}(x)$ are smooth functions.


\end{itemize}


{\it One can say that Riemannian metric is defined by  symmetric covariant smooth tensor field $G$ of the rank 2
which defines scalar product in the tangent spaces $T_{\pt}M$ smoothly depending on the point $\pt$.
 Components of tensor field  $G$ in coordinate system are matrix valued functions $g_{ik}(x)$}:
\begin{equation}\label{symb}
    G=g_{ik}(x)dx^i\otimes dx^k\,.
\end{equation}

  \m

 The matrix $||g_{ik}||$ of components of the metric $G$ we also sometimes denote by $G$.

    {\it Rule of transformation for entries of matrix $g_{ik}(x)$}

  $g_{ik}(x)$-entries of the matrix $||g_{ik}||$  are components of tensor field $G$ in a given coordinate system.

    How do these components transform under transformation of coordinates $\{x^{i}\}\mapsto \{x^{i'}\}$?
    \begin{equation*}
        G=g_{ik}dx^i\otimes dx^k=g_{ik}
        \left({\p x^i\over \p x^{i'}}dx^{i'}\right)
        \otimes
        \left({\p x^k\over \p x^{k'}}dx^{k'}\right)=
    \end{equation*}
    \begin{equation*}\label{ruleoftransformation}
         {\p x^i\over \p x^{i'}}g_{ik}{\p x^k\over \p x^{k'}}
        dx^{i'}
        \otimes dx^{k'}=
        g_{i'k'}
        dx^{i'}
        \otimes dx^{k'}
            \end{equation*}
    Hence
    \begin{equation}\label{transformationlaw}
     g_{i'k'}={\p x^i\over \p x^{i'}}g_{ik}{\p x^k\over \p x^{k'}}\,.
       \end{equation}
     One can derive transformations formulae also
     using general formulae \eqref{ruleoftransformfor 2-tensor} for tensors.



Important remark

     \begin{equation}\label{imporatantremark2}
 g_{ik}=\left\langle{{\p \over \p x^i},{\p\over \p x^k}}\right\rangle\,.
     \end{equation}
Later by some abuse of notations we sometimes omit the sign of tensor product
and write a metric just as
\begin{equation*}\label{symb1}
    G=g_{ik}(x)dx^idx^k\,.
\end{equation*}


{\bf Examples}

   \begin{itemize}
\item


   $\R^n$ with canonical coordinates $\{x^i\}$ and with metric
               $$
           G=(dx^1)^2+(dx^2)^2+\dots+(dx^n)^2
               $$
           $G=||g_{ik}||=\diag [1,1,\dots,1]$

Recall that this is a basis example of $n$-dimensional Euclidean space, where scalar product
is defined by the formula:
              $$
     G(\X,\Y)=\la\X,\Y\ra=g_{ik}X^iY^k=X^1Y^1+X^2Y^2+\dots+X^nY^n\,.
              $$
In the general case if $G=||g_{ik}||$ is an
arbitrary symmetric positive-definite metric then
                $$
            G(\X,\Y)=\la\X,\Y\ra=g_{ik}X^iY^k\,.
                $$
One can show that there exists a new basis $\{\e_i\}$ such that in this basis
              $$
              G(\e_i,\e_k)=\delta_{ik}\,.
               $$
This basis is called orthonormal basis. (See the Lecture notes in Geometry)



Scalar product in vector space defines the {\it same}
scalar product at all the points. In general case
for Riemannian manifold scalar product depends on a point.

In Riemannian manifold we consider arbitrary transformations from
local coordinates to new local coordinates.


\item  $\R^2$ with polar coordinates in the domain $y>0$
($x=r\cos\varphi, y=r\sin\varphi$):

   $dx=\cos\varphi dr-r\sin\varphi d\varphi, dy=\sin\varphi dr+r\cos\varphi d\varphi$.
  In new coordinates the Riemannian metric $G=dx^2+dy^2$ will
have the following appearance:
             $$
        G=(dx)^2+(dy)^2=(\cos\varphi dr-r\sin\varphi d\varphi)^2+(\sin\varphi dr+r\cos\varphi d\varphi)^2=
         dr^2+r^2(d\varphi)^2
            $$
   We see that for matrix    $G=||g_{ik}||$
             \begin{equation*}
                 \underbrace{G=
           \begin{pmatrix}        g_{xx} &g_{xy}\cr  g_{yx} &g_{yy} \end{pmatrix}=
      \begin{pmatrix}    1 &0\cr  0 &1\cr
\end{pmatrix}}_{\hbox{in Cartesian coordinates}},\qquad
                 \underbrace
                 {G=
      \begin{pmatrix}g_{rr} &g_{r\varphi}\cr g_{\varphi r}
&g_{\varphi\varphi}\end{pmatrix}=
      \begin{pmatrix}1 &0\cr 0 &r\end{pmatrix}}_ {\hbox{in polar coordinates}}
             \end{equation*}



    \item   Circle

     Interval $[0,2\pi)$ in the line $0\leq x< 2\pi$ with Riemannian  metric
           \begin{equation}\label{circle1}
          G=  a^2dx^2
           \end{equation}
Renaming $x\mapsto \varphi $ we come to habitual formula for
metric
for circle of the radius $a$: $x^2+y^2=a^2$ embedded in
the Euclidean space $\E^2$:
           \begin{equation}\label{circle2}
          G= a^2d\varphi^2\qquad
          \begin{cases}
          x=a\cos\varphi\cr
          y=a\sin\varphi
          \end {cases},
          0\leq \varphi <2\pi,
           \end{equation}




    \item Cylinder surface

     Domain in $\R^2$ $D=\{(x,y)\colon,\,\, 0\leq x< 2\pi$ with Riemannian  metric
           \begin{equation}\label{domainascylinder1}
          G=  a^2dx^2+dy^2
           \end{equation}
We see that renaming variables $x\mapsto \varphi $, $y\mapsto h$ we come to habitual, familiar formulae for
metric in standard polar coordinates
for cylinder surface of the radius $a$ embedded in the Euclidean space $\E^3$:
           \begin{equation}\label{domainascylinder2}
          G= a^2d\varphi^2+dh^2\qquad
          \begin{cases}
          x=a\cos\varphi\cr
          y=a\sin\varphi\cr
          z=h\cr
          \end {cases},
          0\leq \varphi <2\pi, -\infty<h<\infty
           \end{equation}


 \item  Sphere

   Domain in $\R^2$,  $0<x<2\pi$, $0<y<\pi$ with metric $G=dy^2+\sin^2 y dx^2$
We see that renaming variables $x\mapsto \varphi $, $y\mapsto h$ we come to habitual, familiar formulae for
metric in standard spherical coordinates
for sphere $x^2+y^2+z^2=a^2$ of the radius $a$ embedded in the Euclidean space $\E^3$:
           \begin{equation}\label{domainascylinder2}
          G= a^2d\theta^2+a^2\sin^2\theta
          d\varphi^2\qquad
          \begin{cases}
          x=a\sin\theta\cos\varphi\cr
          y=a\sin\theta\sin\varphi
          z=a\cos\theta
          \end {cases},
          0\leq \varphi <2\pi, -\infty<h<\infty
           \end{equation}

\end{itemize}
(See examples also in the Homeworks.)



\subsubsection {$^*$ Pseudoriemannian manifold} If we omit the
condition of positive-definiteness for Riemannian metric we come to so
called  Pseudoriemannian metric.
Manifold equipped with pseudoriemannian metric is called
pseudoriemannian manifold.  Pseudoriemannian manifolds
appear in applications in the special and general
relativity theory.

In pseudoriemanninan space scalar product $(\X,\X)$ may take an arbitrary real
values: it can be positive, negative, it can be equal to zero. Vectors
$\X$ such that $(\X,\X)=0$ are called null-vectors.
(See the problem 6 in Homework 1).

{\bf Example}  Consider $n+1$-dimensional
linear space $\R^{n+1}$ with pseudometric
\footnote{In the case $n=3$ it is so called Minkovski space.
The coordinate $x^0$ plays a
role of the time: $x^0=ct$, where $c$ is
the value of the speed of the light. Vectors $\X$ such that $(\X,\X)>0$
are called time-like vectors and they called space-like vectors
if $(\X,\X)<0$}
            $$
    G=(dx^0)^2-(dx^1)^2-(dx^2)^2-\dots-(dx^n)^2\,.
            $$
 For an arbitrary vector $\X=(a^0,a^1,a^2,\dots, a^n)$
scalar product $(\X,\X)$ is positive if
$(a^0)^2>(a_1)^2+(a_2)^2+\dots+(a_n)^2$, it is negative if
$(a^0)^2<(a_1)^2+(a_2)^2+\dots+(a_n)^2$, and
$\X$ is null-vector if  $(a^0)^2=(a_1)^2+(a_2)^2+\dots+(a_n)^2$.








\subsection {Scalar product, length of tangent vectors
and angle between them. Length of the curve}

The Riemannian metric defines scalar product of tangent vectors attached at the given point.
Hence it defines the length of tangent vectors and angle between them.
  If $\X=X^m{\p \over \p x^m}, \Y=Y^m{\p \over \p x^m}$ are two tangent vectors at the given point
  $\pt$ of Riemannian manifold with coordinates $x^1,\dots,x^n$, then we have that lengths of
  these vectors equal to
   \begin{equation}\label{lengthandangle1}
    |\X|=\sqrt {\langle \X,\X\rangle}=\sqrt {g_{ik}(x)X^iX^k},\quad
    |\Y|=\sqrt {\langle \Y,\Y\rangle}=\sqrt {g_{ik}(x)Y^iY^k},\quad
\end{equation}
and an angle $\theta$ between these vectors is defined by the relation
         \begin{equation}\label{lengthandangle1}
    \cos\theta={{\langle \X,\Y\rangle}\over |\X|\cdot|\Y|}={g_{ik}X^iY^k\over \sqrt {g_{ik}(x)X^iX^k}
    \sqrt {g_{ik}(x)Y^iY^k}}
\end{equation}

\m

{\bf Example} Let $M$ be $3$-dimensional Riemannian manifold.
 Consider the vectors $\X=2\p_x+2\p_y-\p_z$ and $\Y=\p_x-2\p_y-2\p_z$
attached at the point $\pt$ of $M$ with
local coordinates $(x,y,z)$, where $x=y=1,z=0$.
Find the lengths of these vectors and angle between
them if the expression of Riemannian
metric in these coordinates is $G=dx^2+dy^2+dz^2\over (1+x^2+y^2)^2$.

We see that matrix of Riemannian $g_{ik}=\sigma(x,y,z)\delta_{ik}$, where
$\sigma(x,y,z)={1\over (1+x^2+y^2+z^2)^2}$ is a scalar function,
i.e. matrix $G=||g_{ik}||$ is proportional to unity matrix.
According to formulae above
    $$
    |\X|=\sqrt {\langle \X,\X\rangle}=\sqrt {g_{ik}(x)X^iX^k}=\sqrt {\sigma(x,y,z)}\sqrt {X^iX^i}=
    3\sqrt {\sigma(x,y,z)}=1\,.
    $$
The same answer for $|\Y|$. The scalar product between vectors
$\X,\Y$ equal to zero:
           $$
         \langle\X,\Y\rangle=\sigma(x,y,z)\delta_{ik}X^iY^k=0
           $$
Hence these vectors are unit vectors which are
orthogonal to each other.

\noindent ($\delta_{ik}$ is Kronecker symbol: $\delta_{ik}=1$ if $i=k$
and it vanishes otherwise.)

\subsubsection {Length of  curves}

Let $\gamma\colon\,\, x^i=x^i(t), (i=1,\dots,n))$
 $(a\leq t\leq b)$ be a curve on the Riemannian manifold $(M,G)$.

  At the every point of the curve the velocity vector (tangent vector)
  is defined:
\begin{equation*}\label{velvector}
  \v(t)=\begin{pmatrix}
       \dot x^1 (t)\\
             \cdot\\
             \cdot\\
             \cdot\\
             \dot x^n(t)
         \end{pmatrix}
\end{equation*}

The length of velocity vector $\v\in T_xM$
(vector $\v$ is tangent to the manifold $M$ at the point $x$)
equals to
     \begin{equation*}\label{speedforaunt}
       |\v|_x=\sqrt {\la \v,\v\ra_G\big\vert_{x}}=
       \sqrt{g_{ik}v^iv^k}\big\vert_{x}=
       \sqrt{g_{ik}{dx^i(t)\over dt}{dx^k(t)\over dt}_x}\big\vert_{x}\,.
     \end{equation*}
For an arbitrary curve its length is equal
 to the integral of the length of velocity vector:
\begin{equation}\label{lengthofthecurve}
  L_\gamma=\int_a^b \sqrt {\langle\v,\v\rangle_G\big\vert_{x(t)}}dt=
  \int_a^b \sqrt {g_{ik}(x(t))\dot x^i(t) \dot x^k(t)}dt\,.
\end{equation}

Bearing in mind that metric \eqref{symb} defines the length
we often write metric in the following form
\begin{equation*}\label{metric}
  ds^2=g_{ik}dx^idx^k
\end{equation*}
For example consider $2$-dimensional Riemannian manifold with metric
                    $$
                 ||g_{ik}(u,v)||=
                     \begin{pmatrix}
                     g_{11}(u,v) &g_{12}(u,v)\cr
                     g_{21}(u,v) &g_{22}(u,v)\cr
                    \end{pmatrix}\,.
                    $$
        Then
                           $$
 G=ds^2=g_{ik}du^idv^k=g_{11}(u,v)du^2+2g_{12}(u,v)dudv+g_{22}(u,v)dv^2\,.
                         $$
     The length of the curve
$\gamma\colon u=u(t),v=v(t)$, where $t_0\leq t\leq t_1$
                according to \eqref{lengthofthecurve} is equal to
$L_\gamma=\int_{t_0}^{t_1} \sqrt {\langle\v,\v\rangle}=
  \int_{t_0}^{t_1} \sqrt {g_{ik}(x)\dot x^i \dot x^k}=$
  \begin{equation}
\int_{t_0}^{t_1}
\sqrt {{g_{11}}\left(u\left(t\right),v\left(t\right)\right)u_t^2
+2{g_{12}}\left(u\left(t\right),v\left(t\right)\right)u_tv_t+
{g_{22}}\left(u\left(t\right),v\left(t\right)\right)v_t^2}dt\,.
\end{equation}

The length of  curves defined by the
formula\eqref{lengthofthecurve} obeys the following natural conditions

\begin{itemize}

\item It coincides with the usual length in the Euclidean space $\E^n$
($\R^n$ with standard metric  $G=(dx^1)^2+\dots+(dx^n)^2$
in Cartesian coordinates). E.g. for $3$-dimensional Euclidean space
\begin{equation*}\label{coincidewithlength}
      L_\gamma=
  \int_a^b \sqrt {g_{ik}(x(t))\dot x^i(t) \dot x^k(t)}dt=
  \int_a^b \sqrt{(\dot x^1(t))^2+(\dot x^2(t))^2+(\dot x^3(t))^2}dt
\end{equation*}


\item It does not depend on parameterisation of the curve

\begin{equation*}\label{independenceon parameterisation}
    L_\gamma=  \int_a^b \sqrt {g_{ik}(x(t))\dot x^i(t) \dot x^k(t)}dt=
        \int_{a'}^{b'} \sqrt {g_{ik}(x(\tau))\dot x^i(\tau) \dot x^k(\tau)}
                 d\tau\,,
\end{equation*}
($x^i(\tau)=x^i(t(\tau))$,
$a'\leq \tau\leq b'$ while $a\leq t\leq b$) since under
changing of parameterisation
  $$
  \dot x^i(\tau)={dx(t(\tau))\over d\tau}=
  {dx(t(\tau))\over d t}{dt\over d\tau}=
  \dot x^i(t){dt\over d\tau}\,.
  $$


\item It {\it does not depend on coordinates on Riemannian manifold $M$}
 \begin{equation*}\label{independence on coordinates}
    L_\gamma=  \int_a^b \sqrt {g_{ik}(x(t))\dot x^i(t) \dot x^k(t)}dt=
    \int_a^b \sqrt {g_{i'k'}(x'(t))\dot x^{i'}(t) \dot x^{k'}(t)}dt\,.
 \end{equation*}
This immediately follows from transformation rule
\eqref{transformationlaw}  for Riemannian metric:
       $$
g_{i'k'}\dot x^{i'}(t) \dot x^{k'}(t)=
   g_{ik}
 \left({\p x^i\over \p x^{i'}(t)}\dot x^{i'}(t)\right)
 \left({\p x^k\over \p x^{k'}}\dot x^{k'}(t)\right)
    g_{ik}\,\dot x^{i}(t) \dot x^{k}(t)\,.
        $$
\item  It is additive: length of the sum of two curves is equal to the sum of
their lengths.
If a curve  $\gamma=\gamma_1+\gamma$, i.e.
$\gamma\colon x^i(t), a\leq t\leq b$,
$\gamma_1\colon x^i(t), a\leq t\leq c$ and
$\gamma_2\colon x^i(t), c\leq t\leq b$ where
a point $c$ belongs to the interval $(a,b)$ then
$L_{\gamma}=L_{\gamma^1}+L_{\gamma^2}$.

\end{itemize}

{\footnotesize One can show that formula \eqref{lengthofthecurve}
for length is defined
uniquely by these conditions.
More precisely one can show under some technical conditions
one may show that any local additive functional on curves which does not
depend on coordinates and parameterisation, and depends on
derivatives of curves of order $\leq 1$  is equal
to \eqref{lengthofthecurve} up to a constant multiplier. To feel the taste of this statement
you may do the following exercise:

 {\bf Exercise} Let
$A=A\left(x(t), y(t),
 {dx(t)\over dt},{dy(t)\over dt}\right)$ be a function such that
an integral $L=\int A\left(x(t), y(t),
  {dx(t)\over dt},{dy(t)\over dt}\right)dt$ over an arbitrary curve
$\gamma$
in $\E^2$ does not change
under reparameterisation of this curve and under an arbitrary isometry,
i.e. translation and rotation of the curve.
Then one can easy show (show it!) that
        $$
   A\left(x(t), y(t),
 {dx(t)\over dt},{dy(t)\over dt}\right)=
  c\sqrt {\left({dx(t)\over dt}\right)^2+
   \left({dy(t)\over dt}\right)^2}\,,
    $$
where $c$ is a constant, i.e. it is a usual length up to a multiplier
}


% 8 February 2013

%\end{document}

\subsection{Riemannian structure on the surfaces embedded in Euclidean space}

Let $M$ be a surface embedded in Euclidean space. Let $G$ be Riemannian structure on the manifold $M$.

  Let $\X, \Y$ be two vectors tangent to the surface
$M$ at a point $\pt\in M$. An External Observer calculate this scalar product viewing
these two vectors as vectors in $\E^3$ attached at the point $\pt\in \E^3$
using scalar product in   $\E^3$.  An Internal Observer will calculate the scalar product
viewing these two vectors as vectors  tangent to the surface $M$
using the Riemannian
metric $G$ (see the formula \eqref{scalarproduct}).  Respectively


If $L$ is a curve in $M$ then an External Observer consider this curve as a curve in $\E^3$,
calculate the modulus of velocity  vector (speed) and the length of the curve using Euclidean scalar
product of ambient space. An Internal Observer ("an ant") will define the modulus of the velocity vector and
the length of the curve using Riemannian metric.


 {\bf Definition}  Let $M$ be a surface embedded in the Euclidean space. We say that metric $G_M$ on the surface is
 induced by the Euclidean metric
if the scalar product of arbitrary two vectors ${\bf A,B}\in T_\pt M$ calculated in terms of the metric $G$
equals to Euclidean scalar product of these two vectors:
          \begin{equation}\label{induced metric 1}
      \langle{\bf A,B}\rangle_{G_M}=\langle{\bf A,B}\rangle_{G_{\rm Euclidean}}
          \end{equation}
In other words  we say that Riemannian metric on the embedded surface is induced by the Euclidean structure of the ambient space
    if External and Internal Observers come to the same results calculating scalar product of vectors tangent to the surface.

    In this case modulus of velocity vector (speed) and the length of the curve is the same for
    External and Internal Observer.


Before going in details of this definition recall the conception of Internal and External Observers
when dealing with surfaces in Euclidean space:




\subsubsection {Internal and external coordinates of tangent vector}


\centerline {\it Tangent plane}

Here we recall basic notions from the course of Geometry which we will need here.

    Let $\r=\r(u,v)$ be parameterisation of the surface $M$ embedded in the Euclidean space:
                           \begin{equation*}
                            \r(u,v)=\begin{pmatrix}x(u,v)\cr
                              y(u,v)\cr
                              z(u,v)\cr
                              \end{pmatrix}
                           \end{equation*}
Here as always $x,y,z$ are Cartesian coordinates in $\E^3$.



Let $\pt$ be an arbitrary point on the surface $M$.
Consider the plane formed by the vectors which are adjusted to the point $\pt$
and tangent to the surface $M$. We call this plane
{\it plane tangent to $M$ at the point $\pt$ } and denote it by $T_\pt M$.

For a point  $\pt\in M$ one can consider a basis
in the tangent plane $T_pM$ adjusted to the parameters $u,v$.
       Tangent basis vectors at any point $(u,v)$
   are
              \begin{equation*}
                \r_u={\p \r(u,v)\over \p u}=\begin{pmatrix}
                              {\p x(u,v)\over \p u}\cr
                              {\p y(u,v)\over \p u}\cr
                              {\p z(u,v)\over \p u}\cr
                              \end{pmatrix}=
                              {\p x(u,v)\over \p u}{\p\over \p x}+
                              {\p y(u,v)\over \p u}{\p\over \p y}+
                              {\p z(u,v)\over \p u}{\p\over \p z}
              \end{equation*}




 Every vector $\X\in T_pM$
can be expanded over this basis:
\begin{equation*}\label{expansion}
  \X=X_u\r_u+X_v\r_v,
\end{equation*}
where $X_u, X_v$ are coefficients, components of the vector $\X$.

Internal Observer views  the basis vector $\r_u\in T_pM$, as a  velocity vector
for the curve $u=u_0+t,v=v_0$, where $(u_0,v_0)$ are coordinates of the point $p$.
Respectively the basis vector $\r_v\in T_pM$ for an Internal Observer, is velocity vector
for the curve $u=u_0,v=v_0+t$, where $(u_0,v_0)$ are coordinates of the point $p$.



  \subsubsection {Explicit formulae for induced Riemannian metric (First Quadratic form)}

Now we are ready to write down the explicit formulae for the Riemannian metric on the surface
induced by metric (scalar product) in ambient Euclidean space (see the Definition \eqref{induced metric 1}).

     Let $M\colon \r=\r(u,v)$ be a surface embedded in $\E^3$.

The formula \eqref{induced metric 1} means that scalar products of basic vectors  $\r_u=\p_u, \r_v=\p_v$
has to be  the same calculated in the ambient space and on the surface:
For example scalar product  $\langle\p_u,\p_v\rangle_M=g_{uv}$ calculated by the Internal Observer
is the same as a scalar product  $\langle\r_u,\r_v\rangle_{\E^3}$ calculated by the External Observer,
scalar product  $\langle\p_v,\p_v\rangle_M=g_{uv}$ calculated by the Internal Observer
is the same as a scalar product  $\langle\r_v,\r_v\rangle_{\E^3}$ calculated by the External Observer and so on:
               \begin{equation}\label{inducedmetric 3}
   G=\begin{pmatrix}g_{uu} &g_{uv}\cr g_{vu} &g_{vv}\cr\end{pmatrix}=
    \begin{pmatrix}\langle\p_u,\p_u\rangle &\langle\p_u,\p_v\rangle\cr
     \langle\p_v,\p_u\rangle &\langle\p_v,\p_v\rangle\cr\end{pmatrix}=
     \begin{pmatrix}\langle\r_u,\r_u\rangle_{\E^3} &\langle\r_u,\r_v\rangle_{\E^3}\cr
     \langle\r_v,\r_u\rangle_{\E^3} &\langle\r_v,\r_v\rangle_{\E^3}\cr\end{pmatrix}
     \end{equation}
where as usual we denote by $\langle\,\,,\,\,\rangle_{\E^3}$ the scalar product in the ambient Euclidean space.
(Here see also the important remark \eqref{imporatantremark2})

  {\bf Remark}   It is convenient sometimes to denote
    parameters $(u,v)$ as $(u^1,u^2)$ or $u^\a$ ($\a=1,2$)
   and to write $\r=\r(u^1,u^2)$ or $\r=\r(u^\a)$ ($\a=1,2$)
   instead $\r=\r(u,v)$

In these notations:
\begin{equation*}
    G_M=
\begin{pmatrix}
   g_{11} & g_{12} \\
   g_{12}& g_{22} \\
   \end{pmatrix}=
   \begin{pmatrix}
   \la\r_u,\r_u\ra_{\E^3} & \la\r_u,\r_v\ra_{\E^3} \\
   \la\r_u,\r_v\ra_{\E^3} & \la\r_v,\r_v\ra_{\E^3} \\
   \end{pmatrix},\quad g_{\a\beta}=\la\r_\a,\r_\beta\ra\,,
\end{equation*}
             \begin{equation}\label{firstquadraticform}
          G_M=g_{\a\beta}du^\a du^\beta=g_{11}du^2+2g_{12}dudv+g_{22}dv^2
             \end{equation}

where $(\,,\,)$ is a scalar product in Euclidean space.


The formula \eqref{firstquadraticform} is the formula for induced Riemannian
metric on the surface $\r=\r(u,v)$. It is First Quadratic Form of this surface.

  If ${\X,\Y}$ are two tangent vectors in the tangent plane $T_pC$ then $G(\X,\Y)$
  at the point $p$ is equal to scalar product of vectors $\X,\Y$:
\begin{equation}\label{scalarproduct}
(\X,\Y)=(X^1\r_1+X^2\r_2, Y^1\r_1+Y^2\r_2)=
\end{equation}
                $$
X^1 (\r_1,\r_1)Y^1+X^1 (\r_1,\r_2)Y^2+X^2 (\r_2,\r_1)Y^1+X^2 (\r_2,\r_2)Y^2=
$$
$$X^\a (\r_\a,\r_\beta)Y^\beta=
X^\a g_{\a\beta}Y^\beta=G(\X,\Y)
$$





We can come to this formula just transforming differentials.
 In Cartesian coordinates $\la\X,\Y\ra=X^1Y^1+X^2Y^2+X^3Y^3$,
i.e. the  Riemannian structure of Euclidean space  in Cartesian coordinates is given by
           \begin{equation}\label{RiemEuclid}
            G_{\E^3}=(dx)^2+(dy)^2+(dz)^2\,.
           \end{equation}
The condition that Riemannian metric \eqref{firstquadraticform} is induced by Euclidean scalar product means
that

       \begin{equation}\label{intermsofdifferentials1}
        G_{\E^3}\big\vert_{\r=\r(u,v)}=\left((dx)^2+(dy)^2+(dz)^2\right)\big\vert_{\r=\r(u,v)}=
        G_M=g_{\a\beta}du^\a du^\beta
       \end{equation}
i.e.  $\left((dx)^2+(dy)^2+(dz)^2\right)\big\vert_{\r=\r(u,v)}=$
          $$
\left({\p x(u,v)\over \p u}du+{\p x(u,v)\over \p u}du\right)^2+
\left({\p x(u,v)\over \p u}du+{\p x(u,v)\over \p u}du\right)^2+
\left({\p x(u,v)\over \p u}du+{\p x(u,v)\over \p u}du\right)^2=
          $$
          $$
(x_u^2+y_u^2+z_u^2)du^2+2(x_ux_v+y_uy_v+z_uz_v)dudv+(x_v^2+y_v^2+z_v^2)dv^2
          $$
          We see that
\begin{equation}\label{firstquadraticform1}
G_M=g_{\a\beta}du^\a du^\beta=g_{11}du^2+2g_{12}dudv+g_{22}dv^2,
\end{equation}
where $g_{11}=g_{uu}=(x_u^2+y_u^2+z_u^2)=\langle\r_u,\r_u\rangle_{\E^3}$,
$g_{12}=g_{21}=g_{uv}=g_{vu}=(x_ux_v+y_uy_v+z_uz_v)=\langle\r_u,\r_v\rangle_{\E^3}$,
$g_{22}=g_{vv}=(x_v^2+y_v^2+z_v^2)=\langle\r_v,\r_v\rangle_{\E^3}$.
We come to same formula \eqref{firstquadraticform}.

(See the examples of calculations in the next subsection.)



{\bf Remark}  Sometimes it is convenient to denote Cartesian coordinates of Euclidean space
by $x^i$, ($i=1,2,3$).   Let surface $M$ be given in local  parameterisation $x^i=x^i(u^\a)$.
Riemannian metric of Euclidean space \eqref{RiemEuclid} has appearance
                  \begin{equation}\label{RiemanianmetricofEuclideanspace}
                   G_\E=dx^i\delta_{ik}dx^k\,.
                    \end{equation}
and induced metric \eqref{intermsofdifferentials1} has appearance
                  \begin{equation}\label{RiemanianmetricofEuclideanspace2}
                 G_M=dx^i\delta_{ik}dx^k\big\vert_{x^i=x^i(u^\a)}=
                 {\p x^i(u)\over \p u^\a}
                 \delta_{ik}
                 {\p x^k(u)\over \p u^\beta}
                 du^\a du^\beta=g_{\a\beta}(u)du^\a du^\beta\,
                    \end{equation}
{\footnotesize
Why this representation is useful? it is easy to see that formulae \eqref{RiemanianmetricofEuclideanspace}
,\eqref{RiemanianmetricofEuclideanspace2} work for arbitrary dimensions, i.e. if we have
$m$-dimensional manifold embedded in $n$-dimensional Euclidean space.
We just have to suppose that in this case $i=1,\dots, n$ and $\a=1,\dots,m$;
manifold is given by parameterisation $x^i=x^i(u^\a)$ ($\a=1,\dots,m$).
Moreover in the face if manifold is embedded not in Euclidean space but in an arbitrary
Riemannian space then one can see comparing formulae
\eqref{induced metric 1}  and \eqref{RiemanianmetricofEuclideanspace} we come to the induced metric

   \begin{equation*}\label{RiemanianmetricofEuclideanspace3}
                 G_M=dx^i g_{ik}\left(\left(x(u)\right)\right)dx^k\big\vert_{x^i=x^i(u^\a)}=
                 {\p x^i(u)\over \p u^\a}
                 g_{ik}\left(\left(x(u)\right)\right)
                 {\p x^k(u)\over \p u^\beta}
                 du^\a du^\beta=g_{\a\beta}(x(u))du^\a du^\beta\,
                    \end{equation*}



}





Check explicitly again that  length of the tangent vectors and curves on the surface
calculating by External observer (i.e. using Euclidean metric \eqref{RiemEuclid})
{\it is the same} as calculating by Internal Observer, ant
(i.e. using the induced Riemannian metric \eqref{firstquadraticform}, \eqref{firstquadraticform1}).
 Let $\X=X^\a\r_\a=a\r_u+b\r_v$ be a vector tangent to the surface $M$.
  The square of the length $|\X|$ of this vector calculated by External observer
  (he calculates using the scalar product in $\E^3$) equals to
\begin{equation}\label{lengthforexternalobserver}
 |\X|^2=\la\X,\X\ra=\la\r_u+b\r_v,a\r_u+b\r_v\ra=a^2\la\r_u,\r_u\ra+
   2ab\la\r_u,\r_v\ra+b^2\la\r_v,\r_v\ra
\end{equation}
where $\la\,,\,\ra$ is a scalar product in $\E^3$.
The internal observer will calculate the length using Riemannian metric \eqref{firstquadraticform}
\eqref{firstquadraticform1}:
\begin{equation}\label{thevalueoffirstquadraticform}
  G(\X,\X)=
  \begin{pmatrix}
   a, &b
  \end{pmatrix}
   \cdot
   \begin{pmatrix}
   g_{11} & g_{12} \\
   g_{21}& G_{22} \\
   \end{pmatrix}\cdot
  \begin{pmatrix}
   a\\ b\\
  \end{pmatrix}=
  g_{11}a^2+2g_{12}ab+g_{22}b^2
\end{equation}

{\it External observer (person living in ambient space $\E^3$) calculate the length
of the tangent vector  using formula
\eqref{lengthforexternalobserver}. An ant living on the surface
calculate length of this vector in internal coordinates using formula
\eqref{thevalueoffirstquadraticform}. External observer deals
with external coordinates of the vector, ant on the surface with internal coordinates.
 They come to the same answer.}


\smallskip



  Let $\r(t)=\r(u(t),v(t))$ $a\leq t\leq b$ be a curve on the surface.


  Velocity of this curve at the  point $\r(u(t),v(t))$ is equal to
  $$
  \v=\X=\xi \r_u+\eta\r_v
  \hbox
  {where
  $\xi=u_t,\eta=v_t\colon\quad \v={d\r(t)\over dt}=
    u_t\r_u+v_t\r_v$}
  \,.
  $$


The length of the curve is equal to
\begin{equation}\label{lengthofthecurveexternal}
L=\int_a^b |\v(t)|d t= \int_a^b\sqrt
{\la \v(t),\v(t)\ra_{\E^3}}d t = \int_a^b\sqrt
{\la u_t\r_u+v_t\r_v,u_t\r_u+v_t\r_v\ra_{\E^3}}dt=
\end{equation}
           $$
\int_a^b\sqrt {\la\r_u,\r_u\ra_{\E^3}u_t^2+2\la\r_u,\r_v\ra_{\E^3}u_tv_t+\la\r_v,\r_v\ra_{\E^3}v_t^2}d\tau=
           $$
\begin{equation}\label{lengthofthecurveinternal}
  \int_a^b\sqrt {g_{11} u_t^2+2g_{12}u_tv_t+g_{22}v_t^2}dt
\end{equation}

{\it An external observer will calculate the length of the curve using
\eqref{lengthofthecurveexternal}.  An ant living on the surface calculate
length of the curve using
\eqref{lengthofthecurveinternal} using  Riemannian
metric on the surface. They will come to the same answer.}


%\end{document}

\subsubsection{Induced Riemannian metrics. Examples.}

We consider here examples of calculating induced Riemannian metric
on some quadratic surfaces in $\E^3$.
using calculations for tangent vectors (see \eqref{firstquadraticform}) or
explicitly in terms of differentials (see \eqref{intermsofdifferentials1} and \eqref{firstquadraticform1}).




\m

First of all  consider  the general case when
 a surface $M$ is defined by the
 equation $z-F(x,y)=0$. One can consider the following parameterisation
 of this surface:
\begin{equation}\label{surface}
  \r(u,v)\colon\quad
  \begin{cases}
  x=u\\
  y=v\\
  z=F(u,v)
  \end{cases}
\end{equation}

Then

  \begin{equation}\label{c}
  \r_u=\begin{pmatrix}
        1\\
        0\\
        F_u\\
   \end{pmatrix}
\quad
  \r_v=\begin{pmatrix}
        0\\
        1\\
        F_v\\
   \end{pmatrix}
 \end{equation},
            $$
     (\r_u,\r_u)=1+F_u^2,\quad
     (\r_u,\r_v)=F_uF_v,\quad
     (\r_v,\r_v)=1+F_v^2
            $$
and induced Riemannian metric (first quadratic form) \eqref{firstquadraticform} is equal to
\begin{equation}\label{formula forfirstform}
   ||g_{\a\beta}||=
\begin{pmatrix}
   g_{11} & g_{12} \\
   g_{12}& g_{22} \\
   \end{pmatrix}=
   \begin{pmatrix}
   (\r_u,\r_u) & (\r_u,\r_v) \\
   (\r_u,\r_v) & (\r_v,\r_v) \\
   \end{pmatrix}=   \begin{pmatrix}
   1+F_u^2 & F_uF_v \\
   F_uF_v& 1+F_v^2 \\
   \end{pmatrix}
\end{equation}

\begin{equation}\label{formula forfirstformgeneral1}
   G_M=ds^2=(1+F_u^2)du^2+2F_uF_vdudv+(1+F_v^2)dv^2
\end{equation}
  and the length of the curve $\r(t)=\r (u(t), v(t))$ on $C$  $(a\leq t\leq b)$
  can be calculated by the formula:
               \begin{equation*}
             L=\int
             \int_a^b\sqrt{(1+F_u^2)u_t^2+2F_uF_vu_tv_t+(1+F_v)^2v^2_t}dt
               \end{equation*}
One can calculate \eqref{formula forfirstformgeneral1} explicitly using \eqref{intermsofdifferentials1}:
                     $$
                                    G_M=\left(dx^2+dy^2+dz^2\right)\big\vert_{x=u,y=v,z=F(u,v)}=
               (du)^2+(dv)^2+(F_udu+F_vdv)^2=
                     $$
               \begin{equation}\label{formula forfirstformgeneral2}
=(1+F_u^2)du^2+2F_uF_vdudv+(1+F_v^2)dv^2\,.
               \end{equation}



\medskip

         \medskip


       \centerline  {\it Cylinder}


  Cylinder is given by the equation $x^2+y^2=a^2$. One can consider the following
parameterisation
 of this surface:
\begin{equation}\label{surface1}
  \r(h,\varphi)\colon\quad
  \begin{cases}
  x=a\cos\varphi\\
  y=a\sin\varphi\\
  z=h\\
  \end{cases}
\end{equation}

\medskip

We have   $G_{cylinder}=\left(dx^2+dy^2+dz^2\right)
       \big\vert_{x=a\cos\varphi,y=a\sin\varphi,z=h}=$
        \begin{equation}\label{firstquadraticformcylinder}
               =(-a\sin\varphi d\varphi)^2+(a\cos\varphi
              d\varphi)^2+dh^2=a^2d\varphi^2+dh^2
        \end{equation}

The same formula in terms of scalar product of tangent vectors:



  \begin{equation}\label{cyl1}
  \r_h=\begin{pmatrix}
        0\\
        0\\
        1\\
   \end{pmatrix}
\quad
  \r_\varphi=\begin{pmatrix}
        -a\sin\varphi\\
        a\cos\varphi\\
          0\\
   \end{pmatrix}
 \end{equation},
            $$
     (\r_h,\r_h)=1,\quad
     (\r_h,\r_\varphi)=0,\quad
     (\r_\varphi,\r_\varphi)=a^2
            $$
and
\begin{equation*}\label{formula forfirstformcyl}
||g_{\a\beta}||=
   \begin{pmatrix}
   (\r_u,\r_u) & (\r_u,\r_v) \\
   (\r_u,\r_v) & (\r_v,\r_v) \\
   \end{pmatrix}=   \begin{pmatrix}
   1 & 0 \\
   0& a^2 \\
   \end{pmatrix}\,,
\end{equation*}
\begin{equation}\label{formula forfirstformcyl}
   G=dh^2+a^2d\varphi^2
\end{equation}
  and the length of the curve $\r(t)=\r (h(t), \varphi(t))$ on the cylinder
    $(a\leq t\leq b)$
  can be calculated by the formula:
               \begin{equation}
             L=
             \int_a^b\sqrt{h_t^2+a^2\varphi_t}dt
               \end{equation}



\medskip

  \centerline {\it Cone}
 Cone is given by the equation $x^2+y^2-k^2z^2=0$.
One can consider the following
parameterisation
 of this surface:
\begin{equation}\label{surfacecone}
  \r(h,\varphi)\colon\quad
  \begin{cases}
  x=kh\cos\varphi\\
  y=kh\sin\varphi\\
  z=h\\
  \end{cases}
\end{equation}

\medskip

   Calculate induced Riemannian metric:

We have               $$
             G_{conus}=\left(dx^2+dy^2+dz^2\right)
\big\vert_{x=kh\cos\varphi,y=kh\sin\varphi,z=h}=
                      $$
                      $$
                      (k\cos\varphi dh-kh\sin\varphi d\varphi)^2+
             (k\sin\varphi dh+kh\cos\varphi d\varphi)^2+dh^2
                      $$
        \begin{equation}\label{firstquadraticformforconus}
            G_{conus} =k^2h^2d\varphi^2+(1+k^2)dh^2,\,\,
                        ||g_{\a\beta}||=
   \begin{pmatrix}
   1+k^2 & 0 \\
   0&  k^2h^2 \\
   \end{pmatrix}
                       \end{equation}
The length of the curve $\r(t)=\r (h(t), \varphi(t))$ on the
  cone
    $(a\leq t\leq b)$
  can be calculated by the formula:
               \begin{equation}
             L=\int_a^b
             \sqrt{(1+k^2)h_t^2+k^2h^2\varphi_t^2}dt
               \end{equation}

\medskip
 \centerline {\it Circle}
  Circle of radius $R$ is given by the equation $x^2+y^2=R^2$. Consider standard parameterisation
  $\varphi$
 of this surface:
\begin{equation*}\label{surfacecircle}
  \r(\varphi)\colon\quad
  \begin{cases}
  x=R\cos\varphi\\
  y=R\sin\varphi\\
  \end{cases}
\end{equation*}
   Calculate induced Riemannian metric (first quadratic form)
              $$
              G_{S^2}=\left(dx^2+dy^2\right)\big\vert_{x=R\cos\varphi,y=R\sin\varphi}=
                      $$
                      $$
                      (-R\sin\varphi d\varphi)^2+
                      (a\cos\varphi d\varphi)^2=
          (R^2\cos^2\varphi+R^2\sin^2\varphi)d\varphi^2=
             R^2 d\varphi^2\,.
             $$

One can consider stereographic coordinates on the circle (see Example in the subsection 1.1)
A point $x,y\colon x^2+y^2=R^2$ has stereographic coordinate $t$ if points $(0,1)$ (north pole),
the point $(x,y)$ and the point $(t,0)$ belong to the same line, i.e.
     ${x\over t}={R-y\over R}$, i.e.
              $$
     t={Rx\over R-y}, \qquad \begin{cases}
      x={2tR^2\over R^2+t^2}\cr
      y={t^2-R^2\over t^2+R^2}R\cr
      \end{cases}\,. \quad{\rm since\,\,} x^2+y^2=R^2\,.
                   $$
 Induced metric in coordinate $t$ is
               $$
                       G=(dx^2+dy^2)\big\vert_{x=x(t),y=y(t)}=
\left(d\left(2tR^2\over R^2+t^2\right)\right)^2+
         \left(d\left({t^2-R^2\over R^2+t^2}R\right)\right)^2=
         $$
         $$
       \left({2R^2dt\over R^2+t^2}-{4t^2R^2dt\over (R^2+t^2)^2}\right)^2+
         \left(-{4R^2 tdt\over (t^2+R^2)^2}\right)^2=
         {4R^4dt^2\over (R^2+t^2)^2}\,.
               $$
(See for detail Homework 2)

{\bf Remark} Stereographic coordinates very often are preferable since
 they define birational equivalence between circle and line.


\m

 \centerline {\it Sphere}
  Sphere of radius $R$ is given by the
equation $x^2+y^2+z^2=R^2$. Consider the following
(standard ) parameterisation
 of this surface:
\begin{equation}\label{surfacesphere}
  \r(\theta,\varphi)\colon\quad
  \begin{cases}
  x=R\sin\theta\cos\varphi\\
  y=R\sin\theta\sin\varphi\\
  z=R\cos\theta\\
  \end{cases}
\end{equation}

\medskip

   Calculate induced Riemannian metric (first quadratic form)
              $$
              G_{S^2}=\left(dx^2+dy^2+dz^2\right)
              \big\vert_{x=R\sin\theta\cos\varphi,y=R\sin\theta\sin\varphi,
              z=R\cos\theta}=
                      $$
                      $$
                      (R\cos\theta\cos\varphi d\theta-R\sin\theta\sin\varphi d\varphi)^2+
                      (R\cos\theta\sin\varphi d\theta+R\sin\theta\cos\varphi d\varphi)^2+
                         (-R\sin\theta d\theta)^2=
                      $$
                      $$
          R^2\cos^2\theta d\theta^2+R^2\sin^2\theta d\varphi^2+R^2\sin^2\theta d\theta^2=
                      $$
        \begin{equation}\label{firtsquadraticformforsphere(diff)}\,,\qquad
             =R^2d\theta^2+R^2\sin^2\theta d\varphi^2\,,\qquad
                        ||g_{\a\beta}||=
   \begin{pmatrix}
   R^2 & 0 \\
   0&  R^2\sin^2\theta \\
   \end{pmatrix}
                       \end{equation}

One comes to the same answer calculating scalar product of tangent vectors:
  \begin{equation*}\label{c}
  \r_\theta=\begin{pmatrix}
        R\cos\theta\cos\varphi\\
        R\cos\theta\sin\varphi\\
        -R\sin\theta\\
   \end{pmatrix}
\quad
  \r_\varphi=\begin{pmatrix}
        -R\sin\theta\sin\varphi\\
        R\sin\theta\cos\varphi\\
          0\\
   \end{pmatrix}
 \end{equation*},
            $$
     (\r_\theta,\r_\theta)=R^2,\quad
     (\r_h,\r_\varphi)=0,\quad
     (\r_\varphi,\r_\varphi)=R^2\sin^2\theta
            $$
and
\begin{equation*}\label{formula forfirstform3*}
   ||g||=
   \begin{pmatrix}
   (\r_u,\r_u) & (\r_u,\r_v) \\
   (\r_u,\r_v) & (\r_v,\r_v) \\
   \end{pmatrix}=
\end{equation*}
\begin{equation*}\label{formula forfirstformsphere}
   \begin{pmatrix}
   R^2 & 0 \\
   0&  R^2\sin^2\theta \\
   \end{pmatrix}, \quad
   G_{S^2}=ds^2=R^2d\theta^2+R^2\sin^2\theta d\varphi^2
\end{equation*}
  The length of the curve $\r(t)=\r (\theta(t), \varphi(t))$ on the
  sphere of the radius $a$
    $(a\leq t\leq b)$
  can be calculated by the formula:
               \begin{equation}
             L=\int_a^b
             R\sqrt{\theta_t^2+\sin^2\theta\cdot \varphi_t^2}dt
               \end{equation}

One can consider on sphere as well as on a circle stereographic coordinates:
                 \begin{equation}
                 \label{stereographiccoordinatesonsphere}
                           \begin{cases}
                    u={Rx\over R-z}
                 \cr
                 v={Ry\over R-z}\cr
                 \end{cases},
                 \qquad \begin{cases}
                 x={2uR^2\over R^2+u^2+v^2}\cr
                 y={2vR^2\over R^2+u^2+v^2}\cr
                 z={u^2+v^2-R^2\over u^2+v^2+R^2}R
                    \end{cases}
                    \end{equation}
In these coordinates Riemannian metric is
                     $$
G=(dx^2+dy^2+dz^2)\big\vert_{x=x(u,v),y=y(u,v),z=z(u,v)}=
                        $$
                        $$
  \left(d\left(2uR^2\over R^2+u^2+v^2\right)\right)^2+
                     \left(d\left(2vR^2\over R^2+u^2+v^2\right)\right)^2+
         \left(d\left(1-{2R^2\over R^2+u^2+v^2}\right)R\right)^2=
                        $$
                        $$
                        =
         {4R^4(du^2+dv^2)\over (R^2+u^2+v^2)^2}\,.
            $$



(see in detail Homework 2 and Solutions)

\m


\centerline {\it Saddle (paraboloid)\footnote{{\it This example was not
considered on lectures. It could be useful for learning purposes.}
}}

Consider paraboloid $z=x^2-y^2$.
It can be rewritten as $z=axy$ and it is called sometimes ``saddle''
(rotation on the angle $\varphi=\pi/4$ transforms $z=x^2-y^2$ onto $z=2xy$.)
{\footnotesize Paraboloid and saddle they are  ruled surfaces which are formed by lines.}
We considered this surface in the course of Geometry.

  Consider the following
(standard ) parameterisation
 of this surface:
\begin{equation}\label{surface3}
  \r(u,v)\colon\quad
  \begin{cases}
  x=u\\
  y=v\\
  z=uv\\
  \end{cases}
\end{equation}



Calculate induced metric:
                       $$
 G_{saddle}=\left(dx^2+dy^2+dz^2\right)\big\vert_{x=u\cos\varphi,y=v\sin\varphi,z=uv}=
                      du^2+dv^2+(udv+vdu)^2=
                      $$
        \begin{equation*}\label{firtsquadraticformforsaddle(diff)}
            G_{saddle} =(1+v^2)du^2+2uvdudv+(1+u^2)dv^2\,.
            \end{equation*}


    \centerline {\it One-sheeted and two-sheeted hyperboloids.}
{\it These
    examples were mostly considered on tutorials.}

    Consider surface given by the equation
                $$
              x^2+y^2-z^2=c
                $$
    If $c=0$ it is a cone. We considered it already above.

     If $c>0$ it is  one-sheeted hyperboloid---connected surface in $\E^3$.

     If $c<0$ it is two-sheeted hyperboloid--- a surface with two sheets:
      upper sheet $z>0 $ and another sheet: $z<0$.



     Consider these cases separately.

     \m

    1) {\sl One-sheeted hyperboloid}: $x^2+y^2-z^2=a^2$. It is ruled surface.

    {\bf Exercise}$^\dagger$ Find the lines on two-sheeted hyperboloid



One-sheeted hyperboloid is given by the equation $x^2+y^2-z^2=a^2$.  It is convenient to
choose parameterisation:
\begin{equation}\label{surfaceonesheeteed}
  \r(\theta,\varphi)\colon\quad
  \begin{cases}
  x=a\cosh\theta\cos\varphi\\
  y=a\cosh\theta\sin\varphi\\
  z=a\sinh\theta\\
  \end{cases}
\end{equation}
   $$
 x^2+y^2-z^2=a^2\cosh^2\theta-a^2\sinh^2\theta=a^2\,.
   $$
(Compare the calculations with calculations for sphere! We changed functions
$\cos, \sin$ on $\cosh,\sinh$.)

Induced Riemannian metric (first quadratic form) is equal to
              $$
              G_{Hyperbol I}=\left(dx^2+dy^2+dz^2\right)\big\vert_{x=a\cosh\theta\cos\varphi,y=a\cosh\theta\sin\varphi,
              z=a\sinh\theta}=
                      $$
                      $$
                      (a\sinh\theta\cos\varphi d\theta-a\cosh\theta\sin\varphi d\varphi)^2+
                      (a\sinh\theta\sin\varphi d\theta+a\cosh\theta\cos\varphi d\varphi)^2+
                         (a\cosh\theta d\theta)^2=
                      $$
                      $$
          a^2\sinh^2\theta d\theta^2+a^2\cosh^2\theta d\varphi^2+a^2\cosh^2\theta d\theta^2=
                      $$
        \begin{equation*}\label{firtsquadraticformforsphere(diff)}\,,\qquad
             =a^2(1+2\sinh^2\theta) d\theta^2+a^2\cosh^2\theta d\varphi^2\,,\qquad
                        ||g_{\a\beta}||=
   \begin{pmatrix}
   a^2(1+2\sinh^2\theta) & 0 \\
   0&  a^2\cosh^2\theta \\
   \end{pmatrix}
                       \end{equation*}
 \m

2) {\sl Two-sheeted hyperboloid}: $z^2-x^2-y^2=a^2$. It is not ruled surface!

 For two-sheeted hyperboloid calculations will be very similar.


 In the same way as for one-sheeted hyperboloid (see equation \eqref{surfaceonesheeteed})
 it is convenient to
choose parameterisation:
\begin{equation}\label{surfacehyperbol2}
  \r(\theta,\varphi)\colon\quad
  \begin{cases}
  x=a\sinh\theta\cos\varphi\\
  y=a\sinh\theta\sin\varphi\\
  z=a\cosh\theta\\
  \end{cases}
\end{equation}
   $$
 z^2-x^2-y^2=a^2\cosh^2 \theta-a^2\sinh^2\theta=a^2
   $$
(Compare the calculations with calculations for sphere and one-sheeted hyperboloid.
\medskip

Induced Riemannian metric (first quadratic form) is equal to
              $$
              G_{Hyperbol I}=\left(dx^2+dy^2+dz^2\right)\big\vert_{x=a\sinh\theta\cos\varphi,y=a\sinh\theta\sin\varphi,
              z=a\cosh\theta}=
                      $$
                      $$
                      (a\cosh\theta\cos\varphi d\theta-a\sinh\theta\sin\varphi d\varphi)^2+
                      (a\cosh\theta\sin\varphi d\theta+a\sinh\theta\cos\varphi d\varphi)^2+
                         (a\sinh\theta d\theta)^2=
                      $$
                      $$
          a^2\cosh^2\theta d\theta^2+a^2\sinh^2\theta d\varphi^2+a^2\sinh^2\theta d\theta^2=
                      $$
        \begin{equation}\label{firtsquadraticformforsphere(diff)}\,,\qquad
             =a^2(1+2\sinh^2\theta)d\theta^2+a^2\sinh^2\theta d\varphi^2\,,\qquad
                        ||g_{\a\beta}||=
   \begin{pmatrix}
   a^2(1+2\sinh^2\theta) & 0 \\
   0&  a^2\sinh^2\theta \\
   \end{pmatrix}
\end{equation}


\m
{\footnotesize
  We calculated examples of induced Riemannian structure embedded in Euclidean space almost for all quadratic surfaces.

Quadratic surface is a surface defined by the equation
              $$
             Ax^2+By^2+Cz^2+2Dxy+2Exz+2Fyz+ex+fy+dz+c=0
              $$
One can see that any quadratic surface by affine transformation can be transformed to one of these surfaces

  \begin{itemize}
  \item  cylinder (elliptic cylinder) $x^2+y^2=1$

  \item  hyperbolic cylinder: $x^2-y^2=1$)

  \item  parabolic cylinder  $z=x^2$

  \item  paraboloid   $x^2+y^2=z$

  \item      hyperbolic paraboloid $x^2-y^2=z$

  \item  cone $x^2+y^2-z^2=0$

  \item  sphere $x^2+y^2+z^2=1$

  \item  one-sheeted hyperboloid  $x^2+y^2-z^2=1$

  \item   two-sheeted hyperboloid   $z^2-x^2-y^2=1$

  \end{itemize}
(We exclude degenerate cases such as "point" $x^2+y^2+z^2=0$, planes, e.t.c.)
}


  \subsubsection {$^*$Induced metric on two-sheeted hyperboloid embedded in pseudo-Euclidean space.}

{\small
   Consider the same two-sheeted hyperboloid $z^2-x^2-y^2=1$
   embedded $\R^3$ (See equation \eqref{surfacehyperbol2}.
   For simplicity we assume now that $a=1$.) Now we consider the ambient space
   $\R^3$ not as Euclidean space
   but as {\it pseudo-Euclidean space},
   i.e. in $\R^3$ instead standard scalar product
      $$
      \la\X,\Y\ra= X^1Y^1+X^2Y^2+X^3Y^3
      $$
   we consider pseudo-scalar product defined by
   bilinear form
   \begin{equation*}\label{pseudoscalar produc}
    \la\X,\Y\ra_{pseud}= X^1Y^1+X^2Y^2-X^3Y^3
   \end{equation*}
 The "pseudoscalar" product is bilinear, symmetric. It is defined by non-degenerate matrix.
  But it is not positive-definite. E.g.  The "pseudo-length"
  of vectors $\X=(a\cos\varphi,a\sin\varphi,\pm a)$ is equals to zero (such vectors are called null vectors):
       \begin{equation*}\label{pseudolength}
     \X=(a\cos\varphi,a\sin\varphi,\pm a)\Rightarrow   \la\X,\X\ra_{pseudo}=0,
       \end{equation*}
The corresponding pseudo-Riemannian metric is:
              \begin{equation}\label{pseudoriemannianmetric}
                G_{pseudo}=dx^2+dy^2-dz^2
              \end{equation}



It turns out that the  following remarkable fact occurs:

 {\bf Proposition} {\it The pseudo-Riemannian metric \eqref{pseudoriemannianmetric} in the ambient $3$-dimensional
 pseudo-Euclidean space induces Riemannian metric on two-sheeted hyperboloid $x^2+y^2-z^2=1$.}



{\bf Remark} This is not the fact for one-sheeted hyperboloid (see problem 7 in Homework 2)

 Show it. (See also problems 5 and 6 in Homework 2. )
 Repeat the calculations above for two-sheeted hyperboloid changing in the ambient space
 Riemannian metric $G=dx^2+dy^2+dz^2$ on pseudo-Riemannian $dx^2+dy^2-dz^2$:

\m


Using \eqref{surfacehyperbol2} and  \eqref{pseudoriemannianmetric}
we come now to


              $$
              G=
              \left(dx^2+dy^2-dz^2\right)\big\vert_{x=a\sinh\theta\cos\varphi,y=a\sinh\theta\sin\varphi,
              z=a\cosh\theta}=
                      $$
                      $$
                      (a\cosh\theta\cos\varphi d\theta-a\sinh\theta\sin\varphi d\varphi)^2+
                      (a\cosh\theta\sin\varphi d\theta+a\sinh\theta\cos\varphi d\varphi)^2-
                         (a\sinh\theta d\theta)^2=
                      $$
                      $$
          a^2\cosh^2\theta d\theta^2+a^2\sinh^2\theta d\varphi^2-a^2\sinh^2\theta d\theta^2
                      $$
        \begin{equation}\label{firtsquadraticformforsphere(diff)}\,,\qquad
           G_L  =a^2d\theta^2+a^2\sinh^2\theta d\varphi^2\,,\qquad
                        ||g_{\a\beta}||=
   \begin{pmatrix}
   1 & 0 \\
   0&  \sinh^2\theta \\
   \end{pmatrix}
\end{equation}


The two-sheeted hyperboloid equipped with this metric is called hyperbolic or Lobachevsky plane.

 \m

Now express Riemannian metric in stereographic coordinates.
(We did it in detail in homework 2)

Calculations are very similar to the case of stereographic coordinates of $2$-sphere
 $x^2+y^2+z^2=1$. (See homework 1). Centre of projection $(0,0,-1)$:
 For stereographic coordinates $u,v$ we have ${u\over x}={y\over v}={1\over 1+z}$.  We come to
                   $$
                    \begin{cases}
             u={x\over 1+z}\cr
             v={y\over 1+z}\cr
                   \end{cases},
                    \qquad
                  \begin{cases}
                 x={2u\over 1-u^2-v^2}\cr
                 y={2v\over 1-u^2-v^2}\cr
                 z={u^2+v^2+1\over 1-u^2-v^2}
                    \end{cases}
                 \eqno (4)
                     $$

The image of upper-sheet is an open disc $u^2+v^2=1$ since
$u^2+v^2={x^2+y^2\over (1+z)^2}={z^2-1\over (1+z)^2}={z-1\over z+1}$.
Since for upper sheet $z>1$ then $0\leq {z-1\over z+1}<1$.
\m


    $$
                         G=(dx^2+dy^2-dz^2)\big\vert_{x=x(u,v),y=y(u,v),z=z(u,v)}=
                     \left(d\left(2u\over 1-u^2-v^2\right)\right)^2+
                     $$
                     $$
                     \left(d\left(2v\over 1-u^2-v^2\right)\right)^2-
         \left(d\left(u^2+v^2+1\over 1-u^2-v^2\right)\right)^2=
            {4(du)^2+4(dv)^2\over (1+u^2+v^2)^2}\,.
    $$
These coordinates are very illuminating. One can show that we come to so called
hyperbolic plane (see in detail Homework 2)






\subsection {Isometries of Riemannian manifolds.}


  Let $(M_1,G_{(1)})$, $(M_2,G_{(2)})$ be two Riemannian manifolds---
 manifolds equipped
with Riemannian metric $G_{(1)}$ and $G_{(2)}$ respectively.

Loosely  speaking isometry is the
diffeomorphism of Riemannian manifolds which preserves the distance.



{\bf Definition}

We say that Riemannian manifolds
 $(M_1,G_{(1)})$, $(M_2,G_{(2)})$ are {\it isometric} if
there exists a diffeomorphism $F$
(one-one smooth map with smooth inverse)
which preserves the metrics, i.e.
$G_{(1)}$ is pull-back of $G_{(2)}$:
              \begin{equation}\label{pullback1}
            F^*G_{(2)}=G_{(1)}\,.
               \end{equation}
  In local coordinates this means the following:
  Let $\pt_1$ be an arbitrary point on manifold $M_1$
and $\pt_2\in M_2$ be its image:$F(\pt_1)=\pt_2$.
      Let $\{x^i\}$ be arbitrary
oordinates in a vicinity of a point $\pt_1\in M_1$
      and $\{y^a\}$ be arbitrary
 coordinates in a vicinity of a point $\pt_2\in M_2$.
      Let  Riemannian  metrics  $G_{(1)}$ on $M_1$
  has local expression
      $G_{(1)}=g_{(1)ik}(x)dx^idx^k$ in coordinates $\{x^i\}$
 and respectively
      Riemannian  metrics  $G_{(2)}$ has local expression
      $G_{(2)}=g_{(2)ab}(y)dy^ady^b$ in coordinates $\{y^a\}$ on $M_2$.
                 Then the formula \eqref{pullback1} has the following
appearance in these local coordinates:
                 $$
      F^*\left(g_{_{(2)}ab}(y)dy^a dy^b\right)=
       g_{_{(2)}ab}(y)dy^a dy^b\big\vert_{y=y(x)}=
                 $$
                \begin{equation}\label{transformationlaw}
               g_{_{(2)}ab}(y(x))
   {\p y^a(x)\over \p x^i}dx^i{\p y^b(x)\over \p x^k}dx^k
   =g_{_{(1)}ik}(x)dx^idx^k\,,
               \end{equation}
    i.e.  \begin{equation}\label{transformationlaw2}
   g_{_{(1)}ik}(x)={\p y^a(x)\over \p x^i}g_{_{(2)}ab}(y(x))
        {\p y^b(x)\over \p x^k}\,,
                 \end{equation}
 where $y^a=y^a(x)$ is local expression for diffeomorphism  $F$.
We say that diffeomorphism $F$ is {\it isometry}
 of Riemanian manifolds
$(M_1, G_{(1)})$ and $M_2,G_{(2)}$.

Diffeomorphism $F$ establishes one-one correspondence between local
coordinates on manifolds $M_1$ and $M_2$.
The left hand side of equation \eqref{transformationlaw} can be
considered as a local expression of metric $G_{(2)}$ in coordinates
$x^i$ on $M_2$ and the right hand side of this equation is local
expression of metric $G_{(1)}$ in coordinates $x^i$ on $M_1$.
Diffeomorphism $F$ identifies manifolds $M_1$ and $M_2$
and it can be considered as changing of coordinates.



\m

% 15 February

%\end{document}




{\bf Example} Consider surface of cylinder
$C$, $x^2+y^2=a^2$ in $\E^3$
with induced Riemannian metric
 $G_{C}=a^2 d\varphi^2+dh^2$ (see equations
\eqref{surface1} and \eqref{firstquadraticformcylinder}).
If we remove the line $l\colon\,\,x=a,y=0$
from the cylinder surface $C$ we come to
surface $C'=C\backslash l$.
Consider a map $F$ of this surface
in Euclidean space $E^2$ with Cartesian coordinates $u,v$
(with standard Euclidean metric $G_{Eucl}=du^2+dv^2$):
         \begin{equation}\label{diffofcylinderondomain}
        F\colon\qquad
        \begin{cases}
           u=a\varphi\cr
           v=h\cr
          \end{cases}\,\quad 0<\varphi < 2\pi\,.
           \end{equation}
One can see that $F$ is the diffeomorphism  of $C'$ on the domain
$0<u<2\pi a$ in $\E^2$. This diffeomorphism  transforms the metric
$G_{Eucl}$ on Euclidean space  in metric $G_{C}$ on cylinder, i.e.
pull-back condition \eqref{pullback1} is obeyed:
       $$
    F^*G_{Eucl}=F^*\left(du^2+dv^2\right)=
    \left(du^2+dv^2\right)\big\vert_{u=a\varphi,v=h}=
     a^2d\varphi^2+dh^2=G_1\,.
       $$
We see that cylinder surface with removed line is isometric to domain
in $\E^2$.



{\footnotesize One can consider local version of this notion.

{\bf Definition} Let  a map $F\colon M_1\to M_2$
be a local diffeomorphism, i.e. every point $\pt_1\in M_1$
 has a neighboorhood $V_1$ such that its image $V_2=F(V_1)$,
is an open neighboorhood
of the point $\pt_2=F(\pt_1)$, and the restriction
$F\big\vert_{V_1}$ is diffeomorphism of $V_1$ on $V_2$.
We say that local diffeomorphism $F$ is local isometry
if pull-back condition \eqref{pullback1} is obeyed.




E.g. for cylinder surface considered above, which
is not diffeomorphic to Euclidean plane
and to any domain in it
we can consider
local diffeomorphism $P_{local}$
(covering) of $\E^2$ on cylinder surface
         \begin{equation}\label{localdiffofcylinderondomain}
        P_{local}\colon\qquad
        \begin{cases}
          \varphi={u\over a}\cr
           h=v\cr
          \end{cases}\,\quad 0<\varphi < 2\pi\,.
           \end{equation}
   The pre-image of an arbitrary point $(\varphi,h)$ on cylindre
is the set of points $(\varphi+{2\pi k\over a})$ in $\E^2$.
it is inverse to the diffeomorphism \eqref{diffofcylinderondomain}
$P\circ F={\bf id}$ on $C'$.

 The map $P_{local}$
is local isometry of Euclidean plane on cylinder surface. and
cylinder surface without
line is isometric (globally) to domain in Eulcideand plane.
Formulae \eqref{diffofcylinderondomain} and
\eqref{localdiffofcylinderondomain} establish changing of coordinates
under which metric on cylinder transforms to the metric on
plane and vice versa.
}




\subsubsection{Locally Euclidean Riemannian
 manifolds }
It is useful to formulate the  local isometry condition between
Riemannian manifold and Euclidean space.
A neighbourhood of every point of $n$-dimensional
 manifold is diffeomorphic to $\R^n$.
Let as usual $\E^n$ be $n$-dimensional
Euclidean space, i.e. $\R^n$  with standard Riemannian metric
$G=dx^i\delta_{ik}dx^k=(dx^1)^2+\dots+(dx^n)^2$
in Cartesian coordinates $(x^1,\dots,x^n)$.

{\bf Definition} We say that $n$-dimensional Riemannian manifold
$(M,G)$ is locally isometric to Euclidean space $\E^n$,
i.e.  it is locally Euclidean Riemannian manifold,
if for every point $\pt\in M$ there exists an  open
neighboorhood $D$ (domain) containing this point, $\pt\in D$
such that $D$ is isometric to a domain in Euclidean plane. In other words
in a vicinity of every point $\pt$ there exist local coordinates
$u^1,\dots,u^n$ such that Riemannian metric $G$ in
these coordinates has an appearance
            \begin{equation}\label{defofisometrytoEuclid}
           G=du^i\delta_{ik}du^k=(du^1)^2+\dots+(du^n)^2\,.
             \end{equation}

\m






   Consider examples.

{\bf Example} Consider again cylinder surface..



 We know that cylinder is not diffeomorphic to plane
(there are plenty reasons for this).
 In the previous subsection we cutted the line from cylindre.
Thus we came to surface diffeomorphic to plane. We established that
this surface is isometric to Euclidean plane. (See equation
 \eqref{diffofcylinderondomain} and considerations above.)
Local isometry of cylinder to the Euclidean plane, i.e. the fact
that it is locally Euclidean Riemannian surface
 immediately follows from the fact
that under changing of local coordinates $u=a\varphi, v=h$
  in equation \eqref{diffofcylinderondomain},
the standard Euclidean metric
$du^2+dv^2$ transforms to the metric
$G_{cylinder}=a^2d\varphi^2+dh^2$ on cylinder.



  \m

{\bf Example}
Now show that cone is locally Euclidean Riemannian surface, i.e, it is
locally isometric to the Euclidean  plane.
This means that we have to find local coordinates $u,v$ on the cone such that in these coordinates
  induced metric $G\vert_c$ on cone would have the appearance $G\vert_c=du^2+dv^2$.  Recall calculations of the metric on cone in
coordinates $h,\varphi$ where
             $$
          \r(h,\varphi)\colon
          \begin{cases}
          x=kh\cos\varphi\cr
          y=kh\sin\varphi\cr
          z=h\cr
          \end{cases},
             $$
$x^2+y^2-k^2z^2=k^2h^2\cos^2\varphi+k^2h^2\sin^2\varphi-
k^2h^2=k^2h^2-k^2h^2=0$.
 We have that metric $G_c$ on the cone in coordinates
$h,\varphi$  induced with
the Euclidean metric $G=dx^2+dy^2+dz^2$ is equal to
                $$
            G_c=\left(dx^2+dy^2+dz^2\right)
\big\vert_{x=kh\cos\varphi, y=kh\sin\varphi, z=h}=
            (k\cos\varphi dh-kh\sin\varphi d\varphi)^2+
            $$
            $$
            (k\sin\varphi dh+kh\cos\varphi d\varphi)^2+dh^2=
            (k^2+1)dh^2+k^2h^2d\varphi^2\,.
                $$
In analogy with polar coordinates try to find new local coordinates $u,v$
such that $\begin{cases} u=\a h\cos \beta\varphi\cr v=\a h\sin \beta\varphi\end{cases}$,
where $\alpha, \beta$ are parameters. We come to  $du^2+dv^2=$
             $$
    \left(\a\cos\beta\varphi dh-\a\beta h\sin\beta\varphi d\varphi\right)^2+
  \left(\a\sin\beta\varphi dh+\a\beta h\cos\beta\varphi d\varphi\right)^2=
  \alpha^2 dh^2+\a^2\beta^2 h^2d\varphi^2.
             $$
Comparing with the metric on the cone $G_c=(1+k^2)dh^2+k^2h^2d\varphi^2$  we see that if we put $\alpha=k$ and
$\beta={k\over \sqrt{1+k^2}}$
then $du^2+dv^2=\alpha^2 dh^2+\a^2\beta^2 h^2d\varphi^2=(1+k^2)dh^2+k^2h^2d\varphi^2$.

Thus in new local coordinates
                  $$
                  \begin{cases}
             u= \sqrt {k^2+1}h\cos {k\over \sqrt {k^2+1}}\varphi\cr
             v=  \sqrt {k^2+1} h\sin {k\over \sqrt {k^2+1}}\varphi\cr
                  \end{cases}
                  $$
induced metric on the cone becomes
$G\vert_c= du^2+dv^2$, i.e. cone locally is isometric to the Euclidean plane \finish

Of course these coordinates are local.---  Cone and plane are not homeomorphic, thus they are not globally isometric.

\m

{\bf Example and couterexample}

  Consider domain $D$ in Euclidean plane with two metrics:
              \begin{equation}\label{twoexamples}
    G_{(1)}=du^2+\sin^2v dv^2\,,\quad {\rm and}\quad
    G_{(2)}=du^2+\sin^2u dv^2
              \end{equation}
Thus we have two different Rimeannian manifolds $(D,G_{(1)})$
and $(D, G_{(2)})$.
Metrics in \eqref{twoexamples} look similar. But....
It is easy to see that the first one is locally
isometric to Euclidean plane, i.e. it is locally Euclidean Riemannian manifold
since $\sin^2 vdv^2=d(-\cos v)^2$: in new coordinates
$u'=u,v'=\cos v$ Riemannian metric $G_{(1)}$ has appearance of standard
Euclidean metric:
            $$
(du')^2+(dv')^2=(du)^2+(d(\cos v))^2=du^2+\sin^2 vdv^2=G_{(1)}\,.
            $$
This is not the case for second metric $G_{(2)}$. If we change notations
$u\mapsto \theta$, $v\mapsto \varphi$ then
 $G_{(2)}=d\theta^2+sin\theta^2 d\varphi^2$.
This is local
expression for Riemannian metric induced on the sphere of radius $R=1$.
Suppose that there exist coordinates $u'=u'(\theta,\varphi)$
 $v'=v'(\theta,\varphi)$ such that in these coordinates
metric has Eucldean appearance. This means that locally geometry of
sphere is as  a geometry of Euclidean plane.
On the other hand we know from the course of Geometry
that this is not the case:
sum of angles of triangels on the sphere is not equal to $\pi$,
sphere cannot be bended without shrinking. Later in this course we will
return to this question....

     \m

   There are plenty other examples:

      2) Plane with metric $4R^2(dx^2+dy^2)\over (1+x^2+y^2)^2$
is isometric to the sphere with radius $R$.

    3) Disc with metric $du^2+dv^2\over (1-u^2-v^2)^2$ is isometric to half plane with metric $dx^2+dy^2\over 4y^2$.

(see also exercises in Homeworks and Coursework.)

% 20 February
%\end{document}




    \subsection {Volume element in Riemannian manifold}

   The volume element
in $n$-dimensional Riemannian manifold with metric $G=g_{ik}dx^i
    dx^k$ is defined by the formula
     \begin{equation}\label{volumelement}
  \sqrt {\det g_{ik}}\,dx^1dx^2\dots dx^n\,.
\end{equation}
    If $D$ is a domain in the $n$-dimensional Riemannian manifold with metric
    $G=g_{ik}dx^i$
    then its volume is equal to to the integral of volume element over this domain.
          \begin{equation}\label{volumeofriemannianmanifold}
  V(D)=\int_D \sqrt {\det g_{ik}}\,dx^1dx^2\dots dx^n\,.
\end{equation}

{\bf Remark} Students who know the concept of exterior forms
 can read the volume element as $n$-form
     \begin{equation*}\label{volumelement2}
  \sqrt {\det g_{ik}}\,dx^1\wedge dx^2\wedge \dots \wedge dx^n\,.
\end{equation*}

\bigskip

     Note that in the case of $n=1$ volume is just the length, in the case
     if $n=2$ it is area.

\subsubsection {Volume of parallelepiped}

  Use formulae \eqref{volumelement},\eqref{volumeofriemannianmanifold}
to calculate volume of $n$-dimensional parallelepiped.
   Let $\E^n$ be Euclidean
  vector space with orthonormal basis $\{\e_i\}$. Let
  ${\ac_i}$ be an arbitrary basis in
this vector space (vectors $\v_i$ in general have not unit length and
  are not orthogonal to each other).
Consider $n$-parallelepiped spanned by vectors $\{\v_i\}$:
                 $$
     \Pi_{\v_i}\colon  \r=t^i\v_i, 0\leq t^i\leq 1.
                 $$
  We know that the volume of this parallelepiped is equal to
             \begin{equation}\label{parallelepipinn-dimspace1}
                Vol(\Pi_{\v_i})=\det ||a_i^m||\,,
             \end{equation}
where $A=||a_i^m||$ is  transition matrix, $\v_i=\e_m a_i^m$.
(We know this formula t least for $n=1$--length of interval, $n=2$---
  area of parallelogram and $n=3$ volume of paralleliped
and vector product (see below))

On the other hand
             $$
\r=x^i\e_i=t^m\v_m, \,
\hbox{hence $x^i=a^i_m t^m$, where $\v_m=\e_ia^i_m$.}
             $$
Let $G=(dx^1)^2+\dots+(dx^n)^2=g_{ik}dt^idt^k$ be usual
Euclidean metric in new coordinates $t^i$. Then
          $$
     G=(dx^1)^2+\dots+(dx^n)^2=dx^i\delta_{ik}dx^k=
     dt^{i}{\p x^{i'}\over \p t^{i}}\delta_{i'k'}{\p x^{k'}\over \p t^{k}}dt^{k}.
          $$
Since   ${\p x^{i'}\over \p t^{i}}=a^{i'}_i$ then
                          $$
g_{ik}=\sum_{i'}a^{i'}_i a^{i'}_k\Rightarrow \det g=(\det A)^2, \det g=\sqrt {\det A}.
  $$
and according to the formula \eqref{volumeofriemannianmanifold}
          $$
      Vol(\Pi_{\v_i})=\int_{0\leq t^i\leq 1} \sqrt{\det g} dt^1dt^2\dots dt^n=\det A\,.
          $$
We come to \eqref{parallelepipinn-dimspace1}.
\m

Perform these calculations in detail for $3$-dimensional case.

\m


Let $\E^3$ be $3$-dimensional Euclidean space.
Consider  parallelepiped spanned by vectors
$\{\ac_1\ac_2,\ac_3\}$:
                 $$
                 \Pi_{\ac_1,\ac_2,\ac_3}=t^1\ac_1+t^2\ac_2+t^3\ac_3\,,
                    \quad 0\leq t^1,t^2,t^3\leq 1.
                 $$
We know form standard course of Geometry (calculus, e.t.c.) that
volume of parallelepiped equals to
\begin{equation}\label{volparallelepindim3}
Vol \Pi_{\ac_1,\ac_2,\ac_3}=\left(\ac_1,\ac_2\times \ac_3\right)=
\begin{pmatrix}
a_1^1&a_1^2 &a_1^3\cr
a_2^1&a_2^2&a_2^3\cr
a_3^1&a_3^2&a_3^3\cr
\end{pmatrix}=\det {||a_i^k||}\,,
\end{equation}
(rows of matirx are components of vectors $\ac_i$).
 Vectors $\ac_i=\e_k a^k_{i}$, where $\e_k$ is standard basis in $\E^3$:
 $\Pi_{\ac_1,\ac_2,\ac_3} \ni\r=t^i\ac_i=t^i a^k_i\e_k=x^k\e_k$.
Hence we come from  Cartesian coordinates $x^i$
to new coordinates $t^i$ by changing of coordinates
         $$
         x^k=a^k_it^i\,.
       $$
 The Riemannian metric in new coordinates $(t^1,t^2,t^3)$ is equal to
               $$
   (dx^1)^2+(dx^2)^2+(dx^3)^2=
     dx^i\delta_{ik}\delta^k=a^i_mdt^m\delta_{ik}a^k_n dt^n,
    $$
i.e. in coordinates $t^i$ $g_{mn}=a^i_ma^i_n$ and
            $$
 \sqrt {\det g}=\sqrt {\det(a^T \cdot a)}=\det ||a^i_k||,
     $$
Volume of parallelpiped is equal to
       $$
   \int_{0\leq t^i\leq 1} \left(\sqrt{\det g}\right)dt^1 dt^2 dt^3=
    \sqrt{\det g}\int_0^1dt^1 \int_0^1 dt^2 \int_0^1 dt^3=
     \sqrt{\det g}=\det ||a^k_i||\,.
      $$



\subsubsection{
Invariance of volume element under changing of coordinates }
\medskip

    Prove that volume element is invariant under coordinate transformations,
    i.e. if $y^1,\dots,y^n$ are new coordinates:
    $x^1=x^1(y^1,\dots,y^n), x^2=x^2(y^1,\dots,y^n)...$,
                      $$
          x^i=x^i(y^p), i=1,\dots,n\quad, p=1,\dots,n
                      $$
and   $\tilde g_{pq}(y)$ matrix of the metric in new coordinates:
                       \begin{equation}\label{changingofcoord2}
    \tilde g_{pq}(y)={\p x^i\over \p y^{p}}g_{ik}(x(y))
             {\p x^k\over \p y^{q}}\,.
\end{equation}
Then
                    \begin{equation}\label{invarianceofvolumeelement}
 \sqrt {\det g_{ik}(x)}\,dx^1 dx^2 \dots dx^n=
 \sqrt {\det \tilde g_{pq}(y)}\,dy^1 dy^2 \dots dy^n
\end{equation}
This follows from \eqref{changingofcoord2}. Namely
                             $$
\sqrt {\det g_{ik}(y)}\,dy^1 dy^2 \dots dy^n=
\sqrt {\det \left({\p x^i\over \p y^{p}}g_{ik}(x(y))
             {\p x^k\over \p y^{q}}\right)}\,dy^1 dy^2 \dots dy^n
                             $$
Using the fact that $\det (ABC)=\det A\cdot \det B\cdot \det C$
and  $\det \left({\p x^i\over \p y^{p}}\right)=
\det \left({\p x^k\over \p y^{q}}\right)$\footnote{determinant of matrix does not change
if we change the matrix on the adjoint, i.e. change columns on rows.} we see that
from the formula above follows:
           $$
           \sqrt {\det g_{ik}(y)}\,dy^1 dy^2 \dots dy^n=
\sqrt {\det \left({\p x^i\over \p y^{p}}g_{ik}(x(y))
             {\p x^k\over \p y^{q}}\right)}
          dy^1 dy^2 \dots dy^n=
           $$
           $$
\sqrt {\left(\det
        \left(
    {\p x^i\over \p y^{p}}
    \right)\right)^2}
     \sqrt {\det g_{ik}(x(y))}
     dy^1 dy^2 \dots dy^n=
           $$
\begin{equation}\label{transformofvolumeform4}
\sqrt {\det g_{ik}(x(y))}
         \det
        \left(
    {\p x^i\over \p y^{p}}
    \right)
    dy^1 dy^2 \dots dy^n=
\end{equation}


Now note that   $$
\det \left({\p x^i\over \p y^{p}}\right)dy^1 dy^2 \dots dy^n=dx^1\dots dx^n
                $$
according to the formula for changing coordinates in $n$-dimensional integral
\footnote{Determinant of the matrix $\left({\p x^i\over \p y^{p}}\right)$ of changing of coordinates
is called sometimes Jacobian. Here we consider the case if Jacobian is positive.
If Jacobian is negative then formulae above remain valid just the symbol of modulus appears.}.
Hence
\begin{equation}\label{transformofvolumeform5}
\sqrt {\det g_{ik}(x(y))}\det
      \left(
 {\p x^i\over \p y^{p}}
 \right) dy^1 dy^2 \dots dy^n=
\sqrt {\det g_{ik}(x(y))}dx^1 dx^2 \dots dx^n
\end{equation}
Thus we come to \eqref{invarianceofvolumeelement}.}

% 24 February

%\end{document}

\subsubsection { Examples of calculating volume element}


Consider first very simple example: Volume element of plane in Cartesian coordinates,
metric $g=dx^2+dy^2$. Volume element  is equal to
                      $$
         \sqrt {\det g}dxdy=
         \sqrt
           {
         \det
         \left(
    \begin{array}{cc}
  1 & 0 \\
  0&  1 \\
\end{array}
\right)}dxdy=dxdy
                      $$
  Volume of the domain $D$ is equal to
              $$
          V(D)=\int_D\sqrt {\det g}dxdy=\int_D dxdy
              $$

If we go to polar coordinates:
 \begin{equation}\label{polarcoord1}
    x=r\cos\varphi, y=r\sin\varphi
\end{equation}
Then we have for metric:
              $$
G=dr^2+r^2d\varphi^2
              $$
because
               \begin{equation}\label{polarknow}
       dx^2+dy^2=(dr\cos\varphi-r\sin\varphi d\varphi)^2+
    (dr\sin\varphi+r\cos\varphi d\varphi)^2
       =dr^2+r^2 d\varphi^2
              \end{equation}
Volume element in polar coordinates is equal to
  $$
           \sqrt {\det g}drd\varphi=
         \sqrt
           {
         \det
         \left(
    \begin{array}{cc}
  1 & 0 \\
  0&  r^2 \\
\end{array}
\right)}drd\varphi=drd\varphi\,.
  $$

\m

{\it Lobachesvky plane.}

 In coordinates $x,y$ ($y>0$) metric $G={dx^2+dy^2\over y^2}$,
 the corresponding matrix $G=\begin{pmatrix}
      {1/y^2} & _0 \\
   _0 & {1/y^2} \
 \end{pmatrix}$. Volume element is equal to
 $\sqrt {\det g} dxdy={dxdy\over y^2}$\,.

\medskip


 {  \it Sphere in stereographic coordinates}
Consider the two dimensional plane with Riemannian metrics
                     \begin{equation}\label{stereographic}
    G={4R^2(du^2+dv^2)\over (1+u^2+v^2)^2}
\end{equation}

(It is isometric to the sphere of the radius $R$ without North pole in stereographic coordinates (see the Homeworks.))

   Calculate its volume element and volume.
            It is easy to see that:
                    \begin{equation}\label{}
    G=\left(
    \begin{array}{cc}
  {4R^2\over (1+u^2+v^2)^2} & 0 \\
  0&  {4R^2\over (1+u^2+v^2)^2} \\
\end{array}
\right)
\qquad
  \det g= {16R^4\over (1+u^2+v^2)^4}
\end{equation}
and volume element is equal to  $\sqrt {\det g}dudv= {4R^2dudv\over (1+u^2+v^2)^2}$

One can calculate volume in coordinates $u,v$ but it is better to
consider volume form in polar coordinates
$u=r\cos\varphi, v=r\sin\varphi$.
Then it is easy to see that according to \eqref{polarknow}
 we have for the metric $G={R^2(du^2+dv^2)\over (1+u^2+v^2)^2}={R^2(dr^2+r^2d\varphi^2)\over (1+r^2)^2}$
and volume form is equal to
$\sqrt {\det g}drd\varphi={4R^2rdr d \varphi\over (1+r^2)^2}$

Now calculation of integral becomes easy:
                  $$
    V=\int   {4R^2rdr d \varphi\over (1+r^2)^2}=8\pi R^2 \int_0^\infty {rdr\over (1+r^2)^2}=
                  4\pi  R^2 \int_0^\infty {du\over (1+u)^2}=4\pi R^2\,.
                  $$

                  \medskip
  {\it Segment of the sphere.}

    Consider sphere of the radius $a$ in Euclidean space with standard Riemannian metric
                       $$
            a^2 d\theta^2+a^2\sin^2\theta d\varphi^2
                       $$
  This metric is nothing but first quadratic form   on the sphere (see \eqref{formula forfirstformsphere}).
   The volume element is
                             $$
           \sqrt {\det g}d\theta  d\varphi=
         \sqrt
           {
         \det
         \left(
    \begin{array}{cc}
  a^2 & 0 \\
  0&  a^2\sin \theta \\
\end{array}
\right)}d\theta d\varphi=a^2\sin \theta d\theta d\varphi
                             $$
 Now calculate the volume of the segment of the sphere between two parallel planes,
 i.e. domain restricted  by parallels $\theta_1\leq \theta \leq \theta_0$:
   Denote by  $h$ be the height of this segment. One can see that
                 $$
               h=a\cos\theta_0-a\cos\theta_1=a(\cos\theta_0-a\cos \theta_1)
                $$
   There is remarkable formula which express the area of segment via the height $h$:
       $$
   V= \int_{\theta_1\leq \theta \leq \theta_0}\left (a^2\sin \theta \right)d\theta d\varphi=
         \int_{\theta^0}^{\theta^1}\left(\int_0^{2\pi}\left(a^2\sin \theta\right) d\varphi\right)d\theta=
                $$
\begin{equation}\label{areaoftheseqment}
       \int_{\theta^1}^{\theta^0} 2\pi a^2\sin\theta d\theta=2\pi a^2 (\cos \theta_0-\cos\theta_1)=
            2\pi a (a\cos \theta_0-a\\cos \theta_1)=2\pi a h
\end{equation}
q E.g. for all the sphere $h=2a$. We come to $S=4\pi a^2$.
It is remarkable formula: area of the segment is a polynomial function of radius of the sphere
 and height (Compare with formula for length of the arc of the circle)

24 February

%\end{document}


\section {Covariant differentiaion. Connection. Levi Civita  Connection on Riemannian manifold}

\subsection {Differentiation of vector field along the vector field.---Affine connection}

How to differentiate vector fields on a (smooth )manifold  $M$?


Recall  the differentiation  of functions on a (smooth )manifold  $M$.







Let $\bf X=X^i(x){\bf e}_i(x)={\p\over \p x^i}$ be a vector field on $M$.
Recall that vector field
\footnote{here like always we suppose by default the summation over repeated indices.
E.g.$\X=X^i{\bf e}_i$ is nothing but
$\X=\sum_{i=1}^nX^i{\bf e}_i$}
 $\bf X=X^i{\bf e}_i$ defines at the
every point $x_0$ an infinitesimal curve: $x^i(t)=x^i_0+tX^i$
(More exactly the equivalence class $[\gamma(t)]_\X $
of curves $x^i(t)=x^i_0+tX^i+\dots$).


Let $f$ be an arbitrary (smooth) function on $M$ and $\X=X^i{\p\over \p x^i}$.
 Then derivative
of function $f$ along vector field $\X=X^i{\p\over \p x^i}$ is equal to
                              $$
             \p_{\bf X}f= \nabla_{\bf X}f
                  =X^i{\p f\over \p x^i}
                                $$
The geometrical meaning of this definition is following:
If $\X$ is a velocity vector of the curve $x^i(t)$ at the point $x^i_0=x^i(t)$ at the "time"
$t=0$ then the value of the derivative $\nabla_{\bf X}f$ at the point $x^i_0=x^i(0)$
is equal just to the derivative by $t$ of the function $f(x^i(t))$ at the "time" $t=0$:
\begin{equation}\label{meaningofderivative}
{\rm if}\quad
    X^i(x)\big\vert_{x_0=x(0)}={dx^i(t)\over dt}\big\vert_{t=0},\quad
    {\rm then}\quad
\nabla_\X f\big\vert_{x^i=x^i(0)}=
{d\over dt}f\left(x^i\left(t\right)\right)\big\vert_{t=0}
\end{equation}


{\bf Remark} In the course of Geometry and Differentiable Manifolds the operator
 of taking derivation of function along the vector field
was denoted by "$\p_{\bf X}f$". In this course we prefer to denote it by "$\nabla_{\bf X}f$"
to have the uniform notation for both operators of taking derivation of functions and vector fields
along the vector field.


One can see that the operation $\nabla_X$ on the space $C^\infty(M)$ (space of smooth functions on the manifold)
 satisfies the following conditions:

\begin{itemize}
\item
   $\nabla_{\X}\left(\lambda f+\mu g\right)=
   a\nabla_{\X}f+b\nabla_{\X}g$
 where $\lambda,\mu\in \R$ (linearity over numbers )

\item
 $\nabla_{h\X+g\bf Y}(f)=h\nabla_{\X}(f)+g\nabla_{\bf Y}(f)$
                   (linearity over the space of functions)


\item

  $\nabla_\X(\lambda fg)=f\nabla_\X(\lambda g)+g\nabla_\X(\lambda f)$                       (Leibnitz rule)

\begin{equation}\label{conditionsforderivative}
\end{equation}

\end{itemize}

{\bf Remark}
One can prove that these properties characterize  vector fields:operator on smooth functions
obeying the conditions above is a vector field.
(You will have a detailed analysis of this statement in the course of Differentiable Manifolds.)

\m

How to define differentiation of vector fields along vector fields.

The formula \eqref{meaningofderivative} cannot be generalised straightforwardly because
vectors at the point $x_0$ and $x_0+t X$ are vectors from different vector spaces.
(We cannot substract the vector from one vector space from the
vector from the another vector space, because {\it apriori}
we cannot compare vectors from different vector space.
One have to define an operation of transport of vectors from the space
$T_{x_0}M$ to the point $T_{x_0+tX}M$ defining the transport
from the point $T_{x_0}M$ to the point $T_{x_0+tX}M$).

Try to define the operation $\nabla$ on vector fields such
that conditions \eqref{conditionsforderivative} above  be satisfied.



\subsubsection{Definition of connection. Christoffel symbols of connection}

{\bf Definition} Affine connection on $M$ is the {\it operation} $\nabla$
which assigns to every vector field $\bf X$ a linear map, (but not necessarily $C(M)$-linear map!)
(i.e. a map which is linear over numbers not necessarily over functions)
 $\nabla_{\bf X}$
 on the space ${\cal O}(M)$ of
vector fields:
\begin{equation}\label{defofconnection}
  \nabla_{\bf X}\left(\lambda{\bf Y}+\mu{\bf Z}\right)=
   \lambda\nabla_{\bf X}{\bf Y}+\mu\nabla_{\bf X}{\bf Z},\qquad
   \hbox{for every $\lambda,\mu\in \R$}
\end{equation}
(Compare the first condition in \eqref{conditionsforderivative}).

\noindent which satisfies the following conditions:


\begin{itemize}

\item

for arbitrary (smooth) functions $f,g$ on $M$
\begin{equation}\label{linearityonfunctions}
  \nabla_{f\bf X+g\bf Y}\left({\bf Z}\right)=
   f\nabla_{\bf X}\left({\bf Z}\right)+
   g\nabla_{\bf Y}\left({\bf Z}\right)\qquad
   \hbox {($C(M)$-linearity)}
\end{equation}


(compare with second condition in \eqref{conditionsforderivative})

\item
for arbitrary function $f$


\begin{equation}\label{leibnitzrule}
  \nabla_{\bf X} \left( f\bf Y \right)=
   \left(\nabla_{\bf X}f\right){\bf Y}+
   f\nabla_{\bf X}\left({\bf Y}\right)\qquad
   \hbox {(Leibnitz rule)}
\end{equation}
Recall that $\nabla_\X f$ is just usual derivative of a function $f$ along vector field:
$\nabla_\X f=\p_\X f$.

(Compare with Leibnitz rule in \eqref{conditionsforderivative}).



{\it The operation $\nabla_\X\Y$ is called  covariant derivative
of vector field $\Y$ along the vector field $\X$.}


\end{itemize}



  Write down explicit formulae in a given local coordinates $\{x^i\}$ ($i=1,2,\dots,n$) on manifold $M$.

Let
         $$
      \X=X^i\e_i=X^i{\p\over \p x^i}\,\quad \Y=Y^i\e_i=Y^i{\p\over \p x^i}\,\quad
         $$
The basis vector fields ${\p\over x^i}$ we denote sometimes by $\p_i$ sometimes by $\e_i$


  Using properties above one can see that
\begin{equation}\label{explicitexpression}
  \nabla_{\bf X}{\bf Y}=\nabla_{X^i\p_i}{Y^k \p_k}
  =X^i\left(\nabla_i\left(Y^k\p_k\right)\right),\qquad
  \hbox{where $\nabla_i=\nabla_{\p_i}$}
\end{equation}
Then  according to \eqref{linearityonfunctions}
                 $$
           \nabla_i
            \left(
             Y^k \p_k
            \right)=
              \nabla_i
              \left(Y^k\right)\p_k+
            Y^k \nabla_i \p_k
                   $$

 Decompose the vector field  $\nabla_i \p_k$ over the basis $\p_i$:
             \begin{equation}\label{cristoffelinlocalcoordiantes1}
                \nabla_i \p_k=\Gamma_{ik}^m\p_m
              \end{equation}
             and
\begin{equation}\label{covariantderivative}
    \nabla_i\left({Y^k \p_k}\right)=
    {\p Y^k(x)\over \p x^i}{\p}_k+Y^k\Gamma_{ik}^m\p_m,
\end{equation}
   \begin{equation}\label{covariantderivative2}
    \nabla_{\bf X} {\bf Y}=
    X^i{\p Y^m(x)\over \p x^i}\p_m+X^iY^k\Gamma_{ik}^m\p_m,\quad
    \end{equation}

    In components
           \begin{equation}\label{covderivincomponents}
             \left( \nabla_{\bf X} {\bf Y}\right)^m=
    X^i\left({\p Y^m(x)\over \p x^i}+Y^k\Gamma_{ik}^m\right)
           \end{equation}
    Coefficients $\{\Gamma_{ik}^m\}$ are called {\it Christoffel symbols} in coordinates $\{x^i\}$.
These coefficients define covariant derivative---{\bf connection}.


If operation of taking covariant derivative is given we say that the connection is given on the manifold.
Later it will be explained why we us the word "connection"


We see from the formula above that to define covariant derivative of vector fields, connection,
we have to define Christoffel symbols in local coordinates.

\subsubsection {Transformation of Christoffel symbols for an arbitrary connection}

 Let $\nabla$ be a connection on manifold $M$.
  Let $\{\Gamma^i_{km}\}$ be Christoffel symbols of this connection in  given local coordinates $\{x^i\}$.
  Then according \eqref{cristoffelinlocalcoordiantes1} and \eqref{covariantderivative} we have
                 $$
              \nabla_\X\Y=X^m{\p Y^i\over \p x^m}{\p\over \p x^i}+X^m\Gamma^i_{mk}Y^k{\p\over \p x^i},
                 $$
and in particularly
               $$
               \Gamma^i_{mk}\p_i=\nabla_{\p_m}\p_k
               $$
  Use this relation to calculate Christoffel symbols in new coordinates $x^{i'}$
               $$
            \Gamma^{i'}_{m'k'}\p_{i'}=
            \nabla_{\p_m'}\p_{k'}
               $$
  We have that $\p_{m'}={\p\over \p x^{m'}}={\p x^m\over \p x^{m'}}{\p\over \p x^{m}}= {\p x^m\over \p x^{m'}}\p_m $.
  Hence due to properties \eqref{linearityonfunctions}, \eqref{leibnitzrule} we have
    $$
          \Gamma^{i'}_{m'k'}\p_{i'}=
            \nabla_{\p_{m'}}\p_{k'}=\nabla_{\p_m'}\left({\p x^k\over \p x^{k'}}\p_{k}\right)=
            \left({\p x^k\over \p x^{k'}}\right)\nabla_{\p_m'}\p_{k}+
            {\p\over \p x^{m'}}\left({\p x^k\over \p x^{k'}}\right)\p_{k}=
    $$
              $$
    \left({\p x^k\over \p x^{k'}}\right)\nabla_{{\p x^m\over \p x^{m'}}\p_m}\p_{k}+
            {\p^2  x^k\over \p x^{m'}\p x^{k'}}\p_{k}=
                      {\p x^k\over \p x^{k'}}
          {\p x^m\over \p x^{m'}}
          \nabla_{\p_m}\p_{k}+
           {\p^2  x^k\over \p x^{m'}\p x^{k'}}\p_{k}
          $$
          $$
     {\p x^k\over \p x^{k'}}
          {\p x^m\over \p x^{m'}}
          \Gamma_{mk}^i\p_i+{\p^2  x^k\over \p x^{m'}\p x^{k'}}\p_{k}=
          {\p x^k\over \p x^{k'}}
          {\p x^m\over \p x^{m'}}\Gamma_{mk}^i{\p x^{i'}\over \p x^i}\p_{i'}+
          {\p^2  x^k\over \p x^{m'}\p x^{k'}}{\p x^{i'}\over \p x^k}\p_{i'}
          $$
Comparing the first and the last term in this formula we come to the transformation law:

\m

If $\{\Gamma^i_{km}\}$ are Christoffel symbols of the connection $\nabla$ in  local coordinates $\{x^i\}$
and $\{\Gamma^{i'}_{k'm'}\}$ are Christoffel symbols of this connection  in  new local coordinates $\{x^{i'}\}$
then
 \begin{equation}\label{formualfortransformationofconnection}
    \Gamma^{i'}_{k'm'}=
          {\p x^k\over \p x^{k'}}
          {\p x^m\over \p x^{m'}}
          {\p x^{i'}\over \p x^i}
          \Gamma_{km}^i+
    {\p^2  x^r\over \p x^{k'}\p x^{m'}}{\p x^{i'}\over \p x^r}
 \end{equation}

{\bf Remark}  Christoffel symbols do not transform as tensor.
If the second term is equal to zero, i.e. transformation of coordinates are linear
(see the Proposition on flat connections)  then the transformation rule above is the the same as a
transformation rule for tensors of the type $\begin{pmatrix} 1\cr 2\cr\end{pmatrix}$
(see the formula \eqref{ruleoftransformationofarbitrarytensors}).
 In general case this is not true. Christoffel symbols do not transform as tensor
 under arbitrary non-linear coordinate transformation: see the second term in the formula above.

%28 February

 \end{document}



\subsubsection {Canonical flat affine connection }

  It follows from the properties of connection that it is suffice
   to define connection at vector fields which form basis at the every point
   using \eqref{cristoffelinlocalcoordiantes1}, i.e. to define Christoffel symbols of this connection.

   {\bf Example} Consider $n$-dimensional Euclidean space $\E^n$ with Cartesian coordinates
   $\{x^1,\dots,x^n\}$.

   Define connection such that all Christoffel symbols
    are equal to zero in these Cartesian coordinates $\{x^i\}$.
\begin{equation}\label{flatconnection1}
  \nabla_{\e_i}\e_k=\Gamma_{ik}^m\e_m=0,\quad   \Gamma_{ik}^m=0
\end{equation}
Does  this mean that Christoffel symbols are equal to zero in
an arbitrary Cartesian coordinates if they equal to zero in given Cartesian coordinates?

 Does this mean that  Christoffel symbols of this connection equal to zero in arbitrary coordinates system?



 To answer these questions note that the relations \eqref{flatconnection1} mean that
 \begin{equation}\label{flatconnection2}
    \nabla_\X \Y=X^m{\p Y^i\over \p x^m}
    {\p\over \p x^i}
 \end{equation}
in coordinates $\{x^i\}$

Consider  an arbitrary new coordinates $x^{i'}=x^{i'}(x^1,\dots,x^n)$.
Recall the transformation rule  for an arbitrary vector field (see subsection 1.1)
            $$
   {\bf R}=R^{m}{\p \over \p x^{m}}=R^{m}{\p x^{m'}\over \p x^{m}}{\p \over \p x^{m'}}
   \,,\quad {\rm i.e.} R^{m'}={\p x^{m'}\over \p x^{m}}R^m \,,{\rm and}\,,
              R^{m}={\p x^{m}\over \p x^{m'}}R^{m'}\,.
            $$
Hence we have from \eqref{flatconnection2} that
            $$
 \nabla_\X \Y=X^{m}{\p Y^{i}\over \p x^{m}}{\p\over \p x^i}=
 X^{m}{\p \over \p x^{m}}\left(Y^{i}\right){\p\over \p x^i}=
 X^{m}{\p x^{m'}\over \p x^{m}}{\p \over \p x^{m'}}\left({\p x^{i}\over \p x^{i'}}Y^{i'}\right){\p\over \p x^i}=
                     $$
                     $$
                     X^{m'}{\p \over \p x^{m'}}\left({\p x^{i}\over \p x^{i'}}Y^{i'}\right)
                     {\p\over \p x^i}=
         X^{m'}{\p \over \p x^{m'}}\left(Y^{i'}\right)
         {\p x^{i}\over \p x^{i'}}{\p\over \p x^i}+
          X^{m'}{\p^2 x^{i}\over \p x^{m'}\p x^{i'}}
          \left(Y^{i'}\right)
          {\p\over \p x^i}=
                        $$
                        \begin{equation}\label{flatconnecctiontransformation}
         X^{m'}{\p Y^{i'}\over \p x^{m'}}{\p\over \p x^{i'}}+
         \underbrace {X^{m'}{\p^2 x^{i}\over \p x^{m'}\p x^{i'}}Y^{i'}
          {\p\over \p x^i}}_{\hbox {an additional term}}=
                        \end{equation}
We see that an additional term  equals to zero for arbitrary vector fields $\X,\Y$ if and only if
the relations between new and old coordinates are linear:
                    \begin{equation}\label{additionalterm}
                    {\p^2 x^{i}\over \p x^{m'}\p x^{i'}}=0, \,\,{\rm i.e.}\,\,
                    x^i=b^i+a^i_kx^k
                    \end{equation}
 Comparing formulae \eqref{additionalterm} and \eqref{flatconnection2}
 we come to simple but very important

\m

 {\bf Proposition} {\it
   Let all Christoffel symbols of a given connection be equal to zero
 in a given coordinate system $\{x^i\}$.
  Then   all Christoffel symbols of this connection are equal to zero in an arbitrary
 coordinate system  $\{x^{i'}\}$ such that the relations between new and old coordinates are linear:
               \begin{equation}\label{additionalterm1}
                   x^{i'}=b^i+a^i_kx^k
                    \end{equation}
 If transformation to new coordinate system is not linear, i.e.  ${\p^2 x^{i}\over \p x^{m'}\p x^{i'}}\not=0$
 then Christoffel symbols of this connection in general are not equal to zero in new
 coordinate system  $\{x^{i'}\}$.}

\m

{\bf Definition} We call connection $\nabla$ flat if there exists coordinate system such that
all Christoffel symbols of this connection are equal to zero in a given coordinate system.

\m

In particular connection \eqref{flatconnection1} has zero Christoffel symbols in arbitrary Cartesian coordinates.



{\bf Corollary} Connection has zero Christoffel symbols in arbitrary Cartesian coordinates
if it has zero Christoffel symbols in a given Cartesian coordinates.


Hence the following definition is correct:

\m


{\bf Definition}  A connection on $\E^n$ which Christoffel symbols vanish in Cartesian coordinates
is called {\it canonical flat connection.}


{\bf Remark} {\small   Canonical flat connection in Euclidean space is uniquely defined,
since Cartesian coordinates
are defined globally.  On the other hand
on arbitrary manifold one can define flat connection locally just choosing any arbitrary {\it local}
coordinates and define {\it locally flat connection} by condition that Christoffel symbols
vanish in these local coordinates. This does not mean that one can define flat connection {\it globally.}
 We will study this question after learning transformation law for Christoffel symbols.}


\m
{\bf Remark}   One can see that flat connection is symmetric connection.


\m


  {\bf Example} Consider a connection \eqref{flatconnection1} in $\E^2$. It is a flat connection.
            Calculate Christoffel symbols of this connection in polar coordinates
                       \begin{equation}\label{polarcoordinates}
                        \begin{cases}
                        x=r\cos\varphi\cr
                        y=y\sin\varphi\cr
                        \end{cases}
                        \qquad
                        \begin{cases}
                        r=\sqrt {x^2+y^2}\cr
                        \varphi={\rm arctan\,} {y\over x}\cr
                           \end{cases}
                                \end{equation}
Write down Jacobians of transformations---matrices of partial derivatives:
                      \begin{equation}\label{polarcoordinates2}
                      \begin{pmatrix}
                      x_r &y_r\cr
                      x_\varphi &y_\varphi\cr
                      \end{pmatrix}=
                      \begin{pmatrix}
                      \cos\varphi &\sin\varphi\cr
                      -r\sin\varphi & r\cos\varphi\cr
                      \end{pmatrix},\qquad
                      \begin{pmatrix}
                      r_x &\varphi_x\cr
                      r_y &\varphi_y\cr
                      \end{pmatrix}=
                      \begin{pmatrix}
                      {x\over \sqrt{x^2+y^2}} &-{y\over {x^2+y^2}}\cr
                      {y\over \sqrt{x^2+y^2}} & {x\over {x^2+y^2}}\cr
                      \end{pmatrix}
                      \end{equation}
According \eqref{formualfortransformationofconnection} and since Chrsitoffel symbols are equal to zero in Cartesian coordinates
$(x,y)$ we have
             \begin{equation}\label{christoffelsymbolsinpolarcoordinates}
\Gamma^{i'}_{k'm'}=
    {\p^2  x^r\over \p x^{k'}\p x^{m'}}{\p x^{i'}\over \p x^r},
                             \end{equation}
           where $(x^{1},x^{2})=(x,y)$ and $(x^{1'},x^{2'})=(r,\varphi)$. Now using \eqref{polarcoordinates2}
           we have
            $$
         \Gamma^r_{rr}={\p^2 x\over \p r\p r}{\p r\over \p x}+{\p^2 y\over \p r\p r}{\p r\over \p y}=0
            $$
              $$
        \Gamma^r_{r \varphi}=\Gamma^r_{\varphi r}={\p^2 x\over \p r\p \varphi}{\p r\over \p x}+
        {\p^2 y\over \p r\p \varphi}{\p r\over \p y}=-\sin\varphi\cos\varphi+\sin\varphi\cos\varphi=0\,.
              $$
               $$
           \Gamma^r_{\varphi \varphi}={\p^2 x\over \p \varphi\p \varphi}{\p r\over \p x}+
        {\p^2 y\over \p r\p \varphi}{\p r\over \p y}=-x{x\over r}-y{y\over r}=-r\,.
               $$
                $$
                 \Gamma^\varphi_{rr}=
        {\p^2 x\over \p r\p r}{\p \varphi\over \p x}+{\p^2 y\over \p r\p r}{\p \varphi\over \p y}=0\,.
                $$
                $$
     \Gamma^\varphi_{\varphi r}=
     \Gamma^\varphi_{r\varphi}=
        {\p^2 x\over \p r\p \varphi}{\p \varphi\over \p x}+{\p^2 y\over \p r\p \varphi}{\p \varphi\over \p y}=
        -\sin\varphi{-y\over r^2}+\cos\varphi{x\over r^2}={1\over r}
                $$
             \begin{equation}\label{clhristoffelsymbolsinpolarcoordinates2}
                \Gamma^\varphi_{\varphi\varphi}=
        {\p^2 x\over \p \varphi\p \varphi}{\p \varphi\over \p x}+
        {\p^2 y\over \p \varphi\p \varphi}{\p \varphi\over \p y}=
        -x{-x\over r^2}-y{y\over r^2}=0\,.
                     \end{equation}
       Hence  we have that the covariant derivative \eqref{flatconnection2}
        in polar coordinates has the following appearance
                  \begin{equation*}
                  \nabla_r \p_r=
                  \Gamma_{rr}^r\p_r+
                  \Gamma_{rr}^\varphi\p_\varphi=0\,,
                  , \qquad
                  \nabla_r\p_\varphi=\Gamma_{r\varphi}^r\p_r+
                  \Gamma_{r\varphi}^\varphi\p_\varphi={\p_\varphi\over r}
                  \end{equation*}
                  \begin{equation}\label{covariantderivativeinpolarcoordinates}
                  \nabla_\varphi \p_r=
                  \Gamma_{\varphi r}^r\p_r+
                  \Gamma_{\varphi r}^\varphi\p_\varphi={\p_\varphi\over r}, \qquad
                  \nabla_\varphi\p_\varphi=\Gamma_{\varphi\varphi}^r\p_r+
                  \Gamma_{\varphi\varphi}^\varphi\p_\varphi=-r\p_r
                  \end{equation}
{\bf Remark}  Later when we study geodesics we will learn a very quick method to calculate
Christoffel symbols.



\subsubsection {$^*$ Global aspects of existence of connection}

{\small We defined connection as an operation on vector fields obeying the special axioms (see the subsubsection 2.1.1).
Then we showed that in a given coordinates
connection is defined by Christoffel symbols. On the other hand we know that in general
coordinates on manifold are not defined  globally. (We had not this trouble in Euclidean space where there are globally
defined Cartesian coordinates.)

\begin{itemize}

\item How to define connection globally using local coordinates?

\item Does there exist at least one globally defined connection?

\item   Does there exist globally defined flat connection?

  \end{itemize}

  These questions are not  naive questions.   Answer on first and second questions is "Yes".
   It sounds bizzare but answer on the first question is not "Yes" \footnote{
    Topology of the manifold can be an obstruction to existence
of global flat connection.  E.g. it does not exist on sphere $S^n$ if $n>1$.}


     \centerline{Global definition of connection}

  The formula  \eqref{formualfortransformationofconnection} defines the transformation for
  Christoffer symbols if we go from one coordinates to another.


  Let $\{(x^i_\a), U_\a\}$ be an atlas of charts on the manifold $M$.

  If connection $\nabla$ is defined on the manifold $M$ then
  it defines in any chart (local coordinates) $(x^i_\a)$ Christoffer symbols
  which we denote by $_{(\a)}\Gamma^{i}_{km}$.
  If $(x^i_\a)$, $(x^{i'}_{(\beta)})$ are different local coordinates in a vicinity of a given point then
  according to \eqref{formualfortransformationofconnection}
     \begin{equation}\label{symbolsinlocalchart}
                        _{(\beta)}\Gamma^{i'}_{k'm'}=
          {\p x^k_{(\a)}\over \p x^{k'}_{(\beta)}}
          {\p x^m_{(\a)}\over \p x^{m'}_{(\beta)}}
          {\p x^{i'}_{(\beta)}\over \p x^i_{(\a)}}
          _{(\beta)}\Gamma_{mk}^{(\alpha)i}+
      {\p^2  x^{k}_{(\a)}\over \p x^{m'}_{(\beta)}\p x^{k'}_{(\beta)}}
                  {\p x^{i'}_{(\beta)}\over \p x^{k}_{(\a)}}
                 \end{equation}


  {\bf Definition}  Let $\{(x^i_\a), U_\a\}$ be an atlas of charts on the manifold $M$

  We say that the collection of Christoffel symbols  $\{\Gamma^{(\a)i}_{km}\}$
  defines globally a connection on the manifold $M$ in this atlas if for every two local coordinates
    $(x^i_{(\a)})$, $(x^i_{(\beta)})$ from this atlas the transformation rules  \eqref{symbolsinlocalchart}
    are obeyed.



  \m

  Using partition of unity one can prove the existence of global connection constructing it in explicit way.
  Let $\{(x^i_\a), U_\a\}$ $(\a=1,2,\dots,N)$ be a finite  atlas on the manifold $M$
  and let $\{\rho_\a\}$ be a partition of unity adjusted to this atlas.
    Denote by  $^{(\a)}\Gamma^i_{km}$ local connection defined in domain $U_\a$
    such that its components in these coordinates are equal to zero.
  Denote by $_{(\beta)}^{(\a)}\Gamma^i_{km}$ Christoffel symbols of this local connection
  in coordinates $(x^i_{(\beta)})$ ($_{(\beta)}^{(\a)}\Gamma^i_{km}=0$).
  Now one can define globally the connection by the formula:
                      \begin{equation}\label{formulaforglobalconnection}
                      {}_{(\beta)}\Gamma^i_{km}({\bf x})=
                                   \sum_\a \rho_\a(\x)\,\,{}^{(\a)}_{(\beta)}\Gamma^{i}_{km}(\x)=
           \sum_\a \rho_\a(\x) {\p x^{i}_{(\beta)}\over \p x^{i'}_{(\a)}}
              {\p^2 x^{i'}_{(\a)}(\x)\over \p x^{k}_{(\beta)}\p x^{m}_{(\beta)}}\,.
                      \end{equation}

 This connection in general is not flat connection\footnote{
 See for detail the text: "Global affine connection on manifold"
 " in my homepage: "www.maths.mancheser.ac.uk/khudian" in subdirectory Etudes/Geometry}

}


\subsection {Connection induced on the surfaces}

   Let $M$ be a manifold (surface) embedded in Euclidean space\footnote{We know that every $n$-dimensional manifodl can be embedded
    in $2n+1$-dimensional Euclidean space}.
    Canonical flat connection on $\E^N$  induces the connection on surface in the following way.

    Let $\X,\Y$ be tangent vector fields to the surface $M$ and $\nabla^{\rm can.flat}$ a canonical flat
    connection in $\E^N$.
    In general
            \begin{equation}\label{nottangent!}
                \Z=    \nabla^{\rm can.flat}_\X\Y\quad \hbox{is not tangent to manifold $M$}
                \end{equation}
                Consider its decomposition on two vector fields:
                          \begin{equation}\label{nottangent!2}
             \Z=\Z_{tangent}+\Z_\bot,       \nabla^{\rm can.flat}_\X,\Y=
                    \left(\nabla^{\rm can.flat}_\X\Y\right)_{tangent}+
                    \left(\nabla^{\rm can.flat}_\X\Y\right)_{\bot}\,,
                \end{equation}
where $\Z_{\bot}$ is a component of vector which is orthogonal to the surface $M$ and $\Z_{||}$ is a component which
is tangent to the surface. Define an induced connection $\nabla^{M}$ on the surface $M$ by the following formula
\begin{equation}\label{inducedconnection1}
\nabla^{M}\colon\qquad \nabla^{M}_\X\Y \colon =\left(\nabla^{\rm can.flat}_\X\Y\right)_{tangent}
\end{equation}

{\bf Remark} One can imply this construction for an arbitrary connection in $\E^N$.

\subsubsection {Calculation of induced connection on surfaces in $\E^3$.}

Let $\r=\r(u,v)$ be a surface in $\E^3$. Let $\nabla^{\rm can.flat}$ be a flat connection in $\E^3$.
   Then
\begin{equation}\label{inducedconnection1}
\nabla^{M}\colon\qquad \nabla^{M}_\X\Y \colon =\left(\nabla^{\rm can.flat}_\X\Y\right)_{||}=
\nabla^{\rm can.flat}_\X\Y-
\n(\nabla^{\rm can.flat}_\X\Y,\n),
\end{equation}
where $\n$ is normal unit vector field to $M$. Consider a special example

  {\bf Example} (Induced connection on sphere)
   Consider a sphere of the radius $R$ in $\E^3$:
              $$
           \r(\theta,\varphi)\colon \begin{cases}
           x=R\sin\theta \cos\varphi\cr
           y=R\sin\theta \sin\varphi\cr
           z=R\cos\theta\cr
           \end{cases}
              $$
then
           $$
      \r_{\theta}=\begin{pmatrix}
            R\cos\theta\cos\varphi\cr
            R\cos\theta\sin\varphi\cr
            -R\sin\theta\cr
                  \end{pmatrix},
       \r_{\varphi}=\begin{pmatrix}
            -R\sin\theta\sin\varphi\cr
            R\sin\theta\cos\varphi\cr
            0\cr
                  \end{pmatrix},
                  \n=\begin{pmatrix}
            sin\theta\cos\varphi\cr
            sin\theta\sin\varphi\cr
            \cos\theta\cr
                  \end{pmatrix},
           $$
where $\r_\theta={\p \r(\theta,\varphi)\over \p \theta}$,
$\r_\varphi={\p \r(\theta,\varphi)\over \p \varphi}$ are basic tangent vectors and $\n$ is normal unit vector.

Calculate an induced connection  $\nabla$ on the sphere.

First calculate $\nabla_{\p_\theta}\p_\theta$.
\begin{equation*}
    \nabla_{\p_\theta}\p_\theta=\left({\p \r_\theta\over \p\theta}\right)_{tangent}=
    \left(\r_{\theta\theta}\right)_{tangent}.
\end{equation*}
On the other hand one can see that $\r_{\theta\theta}=\begin{pmatrix}
            -Rsin\theta\cos\varphi\cr
            -Rsin\theta\sin\varphi\cr
            -R\cos\theta\cr
                  \end{pmatrix}=-R\n$ is proportional to normal vector, i.e. $\left(\r_{\theta\theta}\right)_{tangent}=0$.
                  We come to
\begin{equation}\label{inducedonsphere4}
    \nabla_{\p_\theta}\p_\theta=\left(\r_{\theta\theta}\right)_{tangent}=0 \Rightarrow
    \Gamma^\theta_{\theta\theta}=\Gamma^\varphi_{\theta\theta}=0\,.
\end{equation}

Now calculate $\nabla_{\p_\theta}\p_\varphi$ and $\nabla_{\p_\varphi}\p_\theta$.
\begin{equation*}
    \nabla_{\p_\theta}\p_\varphi=\left({\p \r_\varphi\over \p\theta}\right)_{tangent}=
    \left(\r_{\theta\varphi}\right)_{tangent},\,
    \nabla_{\p_\varphi}\p_\theta=\left({\p \r_\theta\over \p\varphi}\right)_{tangent}=
    \left(\r_{\varphi\theta}\right)_{tangent}
\end{equation*}
 We have
           $$
           \nabla_{\p_\theta}\p_\varphi=\nabla_{\p_\varphi}\p_\theta=\left(\r_{\varphi\theta}\right)_{tangent}=
           \begin{pmatrix}
            -R\cos\theta\sin\varphi\cr
            R\cos\theta\cos\varphi\cr
            0\cr
                  \end{pmatrix}_{tangent}.
           $$
We see that the vector $\r_{\varphi\theta}$ is orthogonal to $\n$:
\begin{equation*}
     \langle\r_{\varphi\theta},\n\rangle=-R\cos\theta\sin\varphi \sin\theta\cos\varphi+
     R\cos\theta\cos\varphi\sin\theta\sin\varphi=0.
\end{equation*}
Hence
\begin{equation*}
 \nabla_{\p_\theta}\p_\varphi=\nabla_{\p_\varphi}\p_\theta=
 \left(\r_{\varphi\theta}\right)_{tangent}=\r_{\varphi\theta}=
 \begin{pmatrix}
            -R\cos\theta\sin\varphi\cr
            R\cos\theta\cos\varphi\cr
            0\cr
                  \end{pmatrix}=
                  {\rm cotan\,}\theta  \r_\varphi\,.
\end{equation*}
We come to

\begin{equation}\label{inducedonsphere8}
    \nabla_{\p_\theta}\p_\varphi=\nabla_{\p_\varphi}\p_\theta={\rm cotan\,}\theta  \p_\varphi \Rightarrow
    \Gamma^\theta_{\theta\varphi}=\Gamma^\theta_{\varphi\theta}=0,\,\,
    \Gamma^\varphi_{\theta\varphi}=\Gamma^\varphi_{\varphi\theta}={\rm cotan\,}\theta
\end{equation}

Finally calculate $\nabla_{\p_\varphi} \p_\varphi$
                $$
            \nabla_{\p_\varphi} \p_\varphi=\left(\r_{\varphi\varphi}\right)_{tangent}=
            \left(
            \begin{pmatrix}
            -R\sin\theta\cos\varphi\cr
            -R\sin\theta\sin\varphi\cr
            0\cr
                  \end{pmatrix}
                  \right)_{tangent}
                $$
Projecting on the tangent vectors to the sphere (see \eqref{inducedconnection1}) we have
     $$
\nabla_{\p_\varphi} \p_\varphi=\left(\r_{\varphi\varphi}\right)_{tangent}=
\r_{\varphi\varphi}-\n\langle\n,\r_{\varphi\varphi}\rangle=
         $$
         $$
     \begin{pmatrix}
            -R\sin\theta\cos\varphi\cr
            -R\sin\theta\sin\varphi\cr
            0\cr
                  \end{pmatrix}-
                  \begin{pmatrix}
            \sin\theta\cos\varphi\cr
            \sin\theta\sin\varphi\cr
            \cos\theta\cr
                  \end{pmatrix}\left(-R\sin\theta\cos\varphi\sin\theta\cos\varphi-
                  R\sin\theta\sin\varphi\sin\theta\sin\varphi\right)=
     $$
         $$
      -\sin\theta\cos\theta
      \begin{pmatrix}
            R\cos\theta\cos\varphi\cr
            R\cos\theta\sin\varphi\cr
            -R\sin\theta\cr
                  \end{pmatrix}=-\sin\theta\cos\theta\r_\theta,
         $$
i.e.
\begin{equation}\label{connectiononsphere12}
   \nabla_{\p_\varphi}\p_\varphi=-\sin\theta\cos\theta\r_\theta \Rightarrow
    \Gamma^\theta_{\varphi\varphi}=-\sin\theta\cos\theta,\,\,
    \Gamma^\varphi_{\varphi\varphi}=\Gamma^\varphi_{\varphi\varphi}=0\,.
\end{equation}

\end{document}

\subsection {Levi-Civita connection}


  \subsubsection {Symmetric connection}

 {\bf Definition}. We say that connection is symmetric if its Christoffel symbols $\Gamma^i_{km}$ are symmetric with respect to
  lower indices
  \begin{equation}\label{symmetricconnection}
    \Gamma^i_{km}=\Gamma^i_{mk}
  \end{equation}
  The canonical flat connection and induced connections considered above are symmetric connections.

  \m

  {\small

   \centerline {\it Invariant definition of symmetric connection}

  A connection $\nabla$ is symmetric if for an arbitrary vector fields $\X,\Y$

  \begin{equation}\label{symmetricconnectioninvariant}
    \nabla_\X \Y-\nabla_\Y \X-[\X,\Y]=0
  \end{equation}
  If we apply this definition to basic fields $\p_k,\p_m$ which commute: $[\p_k,\p_m]=0$
  we come to the
  condition
          $$
     \nabla_{\p_k}\p_m-\nabla_{\p_m}\p_k=\Gamma_{mk}^i\p_i-\Gamma_{km}^i\p_i=0
          $$
 and this is the condition  \eqref{symmetricconnection}.
  }



  \subsubsection {Levi-Civita connection. Theorem and Explicit formulae}
Let $(M, G)$ be a Riemannian manifold.

{\bf Definition. Theorem}

{\it A symmetric connection $\nabla$ is called Levi-Civita connection if it is compatible with metric, i.e.
if it preserves the scalar product:
        \begin{equation}\label{connpreservingmetric}
  \p_\X\langle\Y,\Z\rangle=\langle\nabla_\X\Y,\Z\rangle+\langle\Y,\nabla_\X\Z\rangle
        \end{equation}
        for arbitrary vector fields $\X,\Y,\Z$.

  There exists unique levi-Civita connection on the Riemannian manifold.

  In local coordinates Christoffel symbols of Levi-Civita connection are given by the following formulae:
 \begin{equation}\label{levi-civitaformula}
    \Gamma^i_{mk}={1\over 2}g^{ij}\left({\p g_{jm}\over\p x^k}+
    {\p g_{jk}\over\p x^m}-{\p g_{mk}\over \p x^j}\right)\,.
 \end{equation}
 where $G=g_{ik}dx^idx^k$ is Riemannian metric in local coordinates and
 $||g^{ik}||$ is the matrix inverse to the matrix $||g_{ik}||$.

 }

 {\sl Proof}

{\small Suppose that this connection exists and $\Gamma^i_{mk}$ are its Christoffel symbols.
 Consider vector fields $\X=\p_m,\Y=\p_i$ and $\Z=\p_k$ in \eqref{connpreservingmetric}.
  We have that
            \begin{equation}\label{conditionforallindices}
         \p_mg_{ik}=\langle\Gamma_{mi}^r\p_r,\p_k\rangle+\langle\p_i,\Gamma^r_{mk}\p_r\rangle=
              \Gamma^r_{mi}g_{rk}+g_{ir}\Gamma^r_{mk}\,.
            \end{equation}
   for arbitrary indices $m,i,k$.

    Denote by $\Gamma_{mik}=\Gamma^r_{mi}g_{rk}$ we come to
                $$
          \p_mg_{ik}=\Gamma_{mik}+\Gamma_{mki}, \,\,{\rm i.e.}
                $$
    Now using the symmetricity $\Gamma_{mik}=\Gamma_{imk}$ since $\Gamma_{mi}^k=\Gamma_{im}^k$ we have
                $$
    \Gamma_{mik}=\p_mg_{ik}-\Gamma_{mki}=\p_mg_{ik}-\Gamma_{kmi}=\p_mg_{ik}-\left(\p_kg_{mi}-\Gamma_{kim}\right)=
                $$
                $$
\p_mg_{ik}-\p_kg_{mi}+\Gamma_{kim}=\p_mg_{ik}-\p_kg_{mi}+\Gamma_{ikm}=
\p_mg_{ik}-\p_kg_{mi}+\left(\p_ig_{km}-\Gamma_{imk}\right)=
                $$
                $$
       \p_mg_{ik}-\p_kg_{mi}+\p_ig_{km}-\Gamma_{mik}\,.
                $$
 Hence
                \begin{equation}\label{levi-vivitaforlowerindices}
   \Gamma_{mik}={1\over 2}(\p_mg_{ik}+\p_ig_{mk}-\p_kg_{mi})\Rightarrow \Gamma^k_{im}=
   {1\over 2}g^{kr}\left(\p_mg_{ir}+\p_ig_{mr}-\p_rg_{mi}\right)
                \end{equation}
  We see that if this connection exists then it is given by the formula\eqref{levi-civitaformula}.

  On the other hand one can see that \eqref{levi-civitaformula} obeys the condition
  \eqref{conditionforallindices}. We prove the uniqueness and existence.

since $\nabla_{\p_i}\p_k=\Gamma_{ik}^m\p_m$.}

\m

Consider examples.

\m

\subsubsection {Levi-Civita connection on $2$-dimensional Riemannian
manifold with metric $G=adu^2+bdv^2$.}

{\bf Example} Consider $2$-dimensional manifold with Riemannian metrics
                  $$
                  G=a(u,v)du^2+b(u,v)dv^2, \qquad
                  G=\begin{pmatrix}
                       g_{11} &g_{12}\\
                       g_{21}  &g_{22}\\
                     \end{pmatrix}=
         \begin{pmatrix}
                       a(u,v) &0\\
                       0  &b(u,v)\\
                     \end{pmatrix}
                  $$
Calculate Christoffel symbols of Levi Civita connection.

 Using  \eqref{levi-vivitaforlowerindices} we see that:
                       \begin{equation}\label{explicitcalculationofchristoffel}
                       \begin{matrix}
        \Gamma_{111}&={1\over 2}\left(\p_{1} g_{11}+\p_{1} g_{11}-\p_{1} g_{11}\right)=
        {1\over 2}\p_{1} g_{11}=&{1\over 2}a_u\\
             &\\
        \Gamma_{211}=\Gamma_{121}&={1\over 2}\left(\p_{1} g_{12}+
        \p_{2} g_{11}-\p_{1} g_{12}\right)=
        {1\over 2}\p_{2} g_{11}=&{1\over 2}a_v\\
           &\\
\Gamma_{221}&={1\over 2}\left(\p_{2} g_{12}+\p_{2} g_{12}-\p_{1} g_{22}\right)=
        -{1\over 2}\p_{1} g_{22}=-&{1\over 2}b_u\\ &\\
        \Gamma_{112}&={1\over 2}\left(\p_{1} g_{12}+\p_{1} g_{12}-\p_{2} g_{11}\right)=
        -{1\over 2}\p_{2} g_{11}=-&{1\over 2}a_v\\&\\
        \Gamma_{122}=\Gamma_{212}&=
        {1\over 2}\left(\p_{2} g_{21}+\p_{1} g_{22}-\p_{2} g_{21}\right)=
        {1\over 2}\p_{1} g_{22}=&{1\over 2}b_u\\&\\
        \Gamma_{222}&={1\over 2}\left(\p_{2} g_{22}+\p_{2} g_{22}-\p_{2} g_{22}\right)=
        {1\over 2}\p_{2} g_{22}=&{1\over 2}b_v\\
                           \end {matrix}
                       \end{equation}
To calculate $\G^i_{km}=g^{ir}\Gamma_{kmr}$ note that for the metric
$a(u,v)du^2+b(u,v)dv^2$
        $$
          G^{-1}=\begin{pmatrix}
                       g^{11} &g^{12}\\
                       g^{21}  &g^{22}\\
                     \end{pmatrix}=
         \begin{pmatrix}
                       {1\over a(u,v)} &0\\
                       0  &{1\over b(u,v)}\\
                     \end{pmatrix}
              $$
Hence
\begin{equation}\label{christophelfortwodimensionalmetric}
      \begin{matrix}
  \Gamma^{1}_{11}=&g^{11}\Gamma_{111}=&{a_u\over 2a},\qquad
  \Gamma^1_{21}=\Gamma^{1}_{12}=&g^{11}\Gamma_{121}=&{a_v\over 2a},\qquad
  \Gamma^{1}_{22}=&g^{11}\Gamma_{221}=&{-b_u\over 2a}\\&&&&&&\\
  \Gamma^{2}_{11}=&g^{22}\Gamma_{112}=&{-a_v\over 2b},\qquad
  \Gamma^2_{21}=\Gamma^{2}_{12}=&g^{22}\Gamma_{122}=&{b_u\over 2b},\qquad
  \Gamma^{2}_{22}=&g^{22}\Gamma_{222}=&{b_v\over 2b}\\
  \end{matrix}
\end{equation}




\subsubsection {Example of the sphere again}

 {Calculate Levi-Civita connection on the sphere.

 On the sphere first quadratic form (Riemannian metric)
 $G=R^2d\theta^2+R^2\sin^2\theta d\varphi^2$.  Hence we
 use calculations from the previous example
 with $a(\theta,\varphi)=R^2, b(\theta,\varphi)=R^2\sin^2 \theta$
  ($u=\theta, v=\varphi$).
 Note that $a_\theta=a_\varphi=b_\varphi=0$.
  Hence only non-trivial components
 of $\Gamma$ will be:
\begin{equation}\label{Christoffelforsphere:}
  \Gamma^{\theta}_{\varphi\varphi}={-b_\theta\over 2a}={-\sin 2\theta\over 2},
  \qquad
  \left(\Gamma_{\varphi\varphi\theta}={-R^2\sin 2\theta\over 2}\right),
       \end{equation}
  \begin{equation}\label{connectionforsphere}
  \Gamma^{\varphi}_{\theta\varphi}=\Gamma^{\varphi}_{\varphi\theta}=
  {b_\theta\over 2b}={\cos\theta \over \sin\theta}\,
  \qquad
  \left(\Gamma_{\theta\varphi\varphi}={R^2\sin 2\theta\over 2}\right)
\end{equation}
All other components are equal to zero:
                 $$
  \Gamma^{\theta}_{\theta\theta}=\Gamma^{\theta}_{\theta\varphi}=\Gamma^{\theta}_{\varphi\theta}=
  \Gamma^{\varphi}_{\theta\theta}=\Gamma^{\varphi}_{\varphi\varphi}=0
             $$

{\bf Remark} Note that Christoffel symbols of Levi-Civita connection on the sphere coincide
with  Christoffel symbols of induced connection calculated
in the subsection "Connection induced on surfaces".
later we will understand the geometrical meaning of this fact.


\subsection {Levi-Civita connection = induced connection on surfaces in $\E^3$}






 Christoffel symbols of canonical flat connection vanish in Cartesian coordinates.
On the other hand standard Riemannian metric on Euclidean space has the appearance: $G=dx^i\delta_{ik}dx^k$.
The matrix $g_{ik}$ has constant entries. Hence according to Levi-Civita formula \eqref{levi-civitaformula}
the Christoffel symbols of Levi-Civita connection vanish also. Hence we proved very important fact:

{\it Canonical flat connection of Euclidean space is the Levi-Civita connection
of the standard metric on Euclidean space.}


Now we show that Levi-Civita connection on surfaces in Euclidean space coincides with
 the connection induced on the surfaces by canonical flat connection.


We perform our analysis for surfaces in $\E^3$.

Let $M\colon \r=\r(u,v)$ be a surface in $\E^3$. Let  $G$ be induced Riemannian metric on $M$
and $\nabla$ Levi--Civita connection of this metric.

We know that the induced connection $\nabla^{(M)}$ is defined
    in the following way: for arbitrary vector fields  $\X,\Y$ tangent to the surface $M$,
    $\nabla^{M}_\X\Y$ equals to the projection on the tangent space of the vector field $\nabla^{{\rm can.flat}}_\X\Y$:
                   $$
                   \nabla^{M}_\X\Y =
                   \left(\nabla^{\rm can.flat}_\X\Y\right)_{\rm tangent}\,,
                   $$
where  $\nabla^{\rm can.flat}$ is canonical flat connection
   in $\E^3$ (its Christoffel symbols
   vanish in Cartesian coordinates).
   We denote by $\A_{\rm tangent}$ a projection of the vector $A$ attached at the point of the surface on the
tangent space:
             $
          \A_{\perp}=\A-\n(\A,\n)\,,
             $
($\n$ is normal unit vector field to the surface.)

\m


{\bf Theorem}    {\it Induced connection on the surface $\r=\r(u,v)$ in $\E^3$ coincides with
  Levi-Civita connection of Riemannian metric induced by the canonical metric on Euclidean space $\E^3$.}

\m

{\sl Proof}

Let $\nabla^{M}$ be induced connection on a surface $M$  in $\E^3$ given by equations
$\r=\r(u,v)$. Considering this connection on
the basic vectors $\r_h,\r_v$ we see that it is symmetric connection. Indeed
          $$
      \nabla^{M}_{\p_u}\p_v=\left(\r_{uv}\right)_{\rm tangent}=\left(\r_{vu}\right)_{\rm tangent}=
          \nabla^{M}_{\p_v}\p_u\,. \Rightarrow \Gamma^{u}_{uv}=\Gamma^{u}_{vu},
          \Gamma^{v}_{uv}=\Gamma^{v}_{vu}\,.
          $$
Prove that this connection preserves scalar product on $M$.
For arbitrary tangent vector fields  $\X,\Y,\Z$ we have
         $$
         \p_\X\langle \Y,\Z\rangle_{\E^3}=\langle \nabla_\X ^{\rm can.\,flat}\Y,\Z\rangle_{\E^3}+
    \langle \Y,\nabla_\X ^{\rm can.\,flat}\Z\rangle_{\E^3}\,.
         $$
since canonical flat connection in $\E^3$
preserves Euclidean metric in $\E^3$ (it is evident in Cartesian coordinates).
Now project the equation above on the surface $M$.
If $\A$ is an arbitrary vector attached to the surface and $\A_{\rm tangent}$ is its
 projection on the tangent space to the surface, then
 for every tangent vector $\bf B$ scalar product $\langle\A,{\bf B}\rangle_{\E^3}$ equals to
 the scalar product $\langle\A_{\rm tangent},{\bf B}\rangle_{\E^3}=
 \langle\A_{\rm tangent},{\bf B}\rangle_{M}$
  since vector $\A-\A_{\rm tangent}$
 is orthogonal to the surface.  Hence we deduce from  (2) that  $ \p_\X\langle \Y,\Z\rangle_{M}=$
                    $$
   \langle \left(\nabla_\X ^{\rm can.\,flat}\Y\right)_{\rm tangent},\Z\rangle_{\E^3}+
    \langle \Y,\left(\nabla_\X ^{\rm can.\,flat}\Z\right)_{\rm tangent}\rangle_{\E^3}=
                    \langle \nabla_\X ^{M}\Y,\Z\rangle_{M}+
    \langle \Y,\nabla_\X ^{M}\Z\rangle_{M}\,.
                    $$
 We see that induced connection is symmetric connection which preserves the induced metric.
 Hence due to Levi-Civita Theorem it is unique and is expressed as in the formula \eqref{levi-civitaformula}.

\m




{\bf Remark} {\footnotesize One can easy to reformulate and prove more general statement:
Let $M$ be a submanifold in Riemannian manifold $(E,G)$. Then Levi-Civita connection
of the metric induced on this submanifold coincides with the connection induced on the manifold
by Levi-Civita connection of the metric $G$.}


   \section {Parallel transport and geodesics}

   \subsection{Parallel transport}

\subsubsection {Definition}

   Let $M$ be a manifold equipped with affine connection $\nabla$.

   {\bf Definition} Let $C\colon \x(t)$ with coordinates
   $x^i=x^i(t),\,\, t_0\leq t\leq t_1$ be a curve on the manifold $M$.
   Let $\X=\X(t_0)$ be an arbitrary tangent  vector attached at the initial point
  $\x_0$ (with coordinates $x^i(t_0)$) of the curve $C$, i.e.
  $\X(t_0)\in T_{\x_0}M$ is a vector tangent to the manifold $M$
  at the point $\x_0$ with coordinates $x^i(t_0)$.
  (The vector $\X$ is not necessarily tangent to the curve $C$)

  We say that $\X(t)$, $t_0\leq t\leq t_1$ is a parallel transport of the
  vector $\X(t_0)\in T_{\x_0}M$ along the curve $C\colon x^i=x^i(t), t_0\leq t\leq t_1$ if

\begin {itemize}

\item For an arbitrary $t$, $t_0\leq t\leq t$, vector $\X=\X(t)$, ($\X(t)\vert_{t=t_0}=\X(t_0)$)
 is a vector attached at the point
  $\x(t)$ of the curve $C$,
 i.e. $\X(t)$ is a vector tangent to the manifold $M$
  at the point $\x(t)$  of the curve $C$.

  \item The covariant derivative of $\X(t)$ along the curve $C$ equals to zero:

  \begin{equation}\label{conditofparalleltransport}
    {\nabla \X\over dt}=\nabla_\v \X=0\,.
  \end{equation}
 In components: if $X^m(t)$ are components of the vector field $\X(t)$
 and   $v^m(t)$ are components of the velocity vector $\v$ of the curve  $C$ ,
                $$
                \X(t)=X^m(t){\p \over \p x^m}\vert_{\x(t)}\,,\quad
                \v={d\x(t)\over dt}={dx^i\over dt}{\p \over \p x^m}\vert_{\x(t)}
                $$
  then the condition \eqref{conditofparalleltransport} can be rewritten as
  \begin{equation}\label{conditofparalleltransportcomponents}
    {dX^i(t)\over dt}+v^k(t)\Gamma^i_{km}(x^i(t))X^m(t)\equiv 0\,.
  \end{equation}


\end{itemize}

\m

{\bf Remark}    We say sometimes that $\X(t)$ is {\it covariantly constant along the curve $C$}
   If $\X(t)$ is parallel transport of the vector $\X$ along the curve $C$.
  If we consider Euclidean space with canonical flat connection then in Cartesian coordinates
  Christoffel symbols vanish and parallel transport is nothing but
   ${d \X\over dt}=\nabla_\v \X=0$, $\X(t)$ is constant vector.



  \m


{\bf Remark} Compare this definition of parallel transport with the definition which we consider
in the course of "Introduction to Geometry" where we consider parallel transport of the vector along the
curve on the surface embedded in $\E^3$ and define parallel transport by the condition, that only orthogonal component
of the vector changes during parallel transport, i.e.
${d\X(t)\over dt}$ is a vector orthogonal to the surface  (see the Exercise in the Homework 7).

\subsubsection {$^*$Parallel transport is a linear map }

Consider two different points $\x_0,\x_1$ on the manifold  $M$ with connection $\nabla$.
Let $C$ be a curve $\x(t)$ joining these points.
The parallel transport \eqref{conditofparalleltransport}  $\X(t)$ defines the map between  tangent vectors
at the point $\x_0$ and tangent vectors at the point $\x_1$. This map depends on the curve $C$. Parallel
transport along different curves joining the same points is in general different (if we are not in Euclidean space).

On the other hand parallel transport is a linear map of tangent spaces which {\it does not depend on the
parameterisation of the curve} joining these points.

\m

{\bf Proposition}  {\it Let $\X(t), t_0\leq t\leq t_1$ be a parallel transport of the vector $\X(t_0)\in T_{\x_0}M$
along the curve $C\colon \x=\x(t), t_0\leq t\leq t_1$, joining the points $\x_0=\x(t_0)$ and $\x_1=\x(t_1)$.
 Then the map
      \begin{equation}\label{paralleltransportalongthecurve}
 \tau_{C}\colon \quad   T_{\x_0}M \ni  \X(t_0)\longrightarrow  \X(t_1)\in T_{\x_1}M
               \end{equation}
 is a linear map from the vector space  $T_{\x_0}M$ to the vector space $T_{\x_1}M$
   which does not depend on the parametersiation of the curve $C$.}

The fact that the map \eqref{paralleltransportalongthecurve} does not depend on the parameterisation
follows from the differential equation \eqref{conditofparalleltransportcomponents} also.}

Indeed let $t=t(\tau)$, $\tau_0\leq \tau\leq \tau_1$, $t(\tau_0)=t_0, t(\tau_1)=\tau_1$ be another parameterisation
of the curve $C$. Then multiplying the equation \eqref{conditofparalleltransportcomponents} on ${dt\over d\tau}$
and using the fact that velocity $\v'(\tau)=t_\tau\v(t)$
we come to differential equation:
  \begin{equation}\label{conditofparalleltransportcomponentsnewparameter}
    {dX^i(t(\tau))\over d\tau}+v'^k(t(\tau))\Gamma^i_{km}(x^i(t(\tau)))X^m(t(\tau))\equiv 0\,.
  \end{equation}

The functions $X^(t(\tau))$ with the same initial conditions are the solutions of this equation.

The fact that it is a linear map follows immediately from the fact that
differential equations \eqref{conditofparalleltransportcomponents} are linear.
E.g. let vector fields $\X(t),\Y(t)$ be covariantly constant along the curve $C$.
Then since linearity  $\X(t)+\Y(t)$ is a solution too.




  \subsubsection {Parallel transport with respect to Levi-Civita connection}


We mostly consider parallel transport on Riemannian manifold.  If $(M,G)$ is Riemannian manifold
we mostly consider parallel transport with respect to connection $\nabla$ which is Levi-Civita connection
of the  Riemannian metric $G$


{\bf Proposition} {The length of the vector preserves during parallel transport with
respect to Levi-Civita connection.}

Proof follows immediately from the definition \eqref{conditofparalleltransport} of aprallel traspotr
and the definition  \eqref{connpreservingmetric} of Levi-Civita connection
              \begin{equation}\label{lengthdoesnotchangeduringparalleltransport}
{d\over dt}\langle\X(t),\X(t)\rangle=\p_\v\langle\X(t),\X(t)\rangle=
2\left\langle\nabla_\v \X(t), \X(t)\right\rangle=2\left\langle 0, \X(t)\right\rangle=0\,.
     \end{equation}
Hence $|\X(t)|$ is constant.

Exercise Show that scalar product preserves during parallel transport.



\subsection {Geodesics}

\subsubsection {Definition.  Geodesic on Riemannian manifold.}

Let $M$ be manifold equipped with connection $\nabla$.

{\bf Definition} A parameterised curve  $C\colon\quad x^i=x^i(t)$ is called geodesic if velocity vector
$\v(t)\colon v^i(t)={dx^i(t)\over dt}$ is covariantly constant along this curve, i.e.
it remains parallel along the curve:
\begin{equation}\label{geodesdef1}
    \nabla_\v \v={\nabla\v\over dt}={dv^i(t)\over dt}+v^k(t)\Gamma^i_{km}(x(t))v^m(t)=0,\,\,\,{i.e.}
\end{equation}
\begin{equation*}
    {d^2x^i(t)\over dt^2}+{d x^k(t)\over dt}\Gamma^i_{km}(x(t)){d x^m(t)\over dt}=0\,.
\end{equation*}
  These are linear second order differential equations. One can prove that
  this equations have solution and it is unique\footnote{this is true under additional technical conditions which we do not
  discuss here} for an arbitrary initial data
  ($x^i(t_0)=x^i_0, \dot x^i(t_0)=\dot x^i_0$. )



In other words the curve $C\colon x(t)$ is a geodesic if parallel transport of velocity vector to
along the curve is  a velocity vector at any point of the curve.
\m

Geodesics defined with Levi-Civita connection on the Riemannian manifold is called geodesic on Riemannian manifold.
We mostly consider geodesics on Riemannian manifolds.

Since velocity vector of the geodesics on Riemannian manifold  at any point is a parallel transport
with the Levi-Civita connection, hence due to Proposition \eqref{lengthdoesnotchangeduringparalleltransport}
the length of the velocity vector remains constant:

{\bf Proposition}   {\it If $C \colon\,\, \x(t)$ is a geodesics on Riemannian manifold then the length of
velocity vector is preserved along the geodesic.}


{\sl Proof}   Since the connection is Levi-Civita connection then it preserves scalar product of tangent vectors,
(see \eqref{connpreservingmetric}) in particularly the length of the velocity vector $\v$:




\subsubsection {Un-parameterised geodesic}

We defined a geodesic as a parameterised curve such that the velocity vector
is covariantly constant along the curve.

\m

What happens if we change the parameterisation of the curve?


\m

  Another  question:  Suppose a tangent vector to the curve remains tangent to the curve during parallel transport.
  Is it true that this curve (in a suitable parameterisation) becomes geodesic?

\m

 {\bf Definition}  We call un-parameterised curve geodesic if under suitable parameterisation it obeys
   the equation \eqref{geodesdef1} for geodesics.

\m


   Let $C$--be un-parameterised geodesic.
   Then the following statement is valid.

\m

   {\bf Proposition} {\it A curve $C$ (un-parameterised) is geodesic
   if an only if a non-zero vector tangent to the curve remains tangent to the curve
   during parallel transport.}



   \m
   {\sl Proof}.  Let $\A$ be tangent vector at the point $\pt\in C$ of the curve.
  Parallel transport does not depend on parametersiation of the curve
  (see 3.1.2).  Choose a suitable parameterisation $x^i=x^i(t)$  such that
 $x^i(t)$ obeys the equations \eqref{geodesdef1} for geodesics,
 i.e. the velocity vector $\v(t)$ is covariantly constant along the curve:
 $\nabla_\v\v=0$. If $\A(t_0)=c\v(t_0)$
 at the given point $\pt$ ($c$ is a scalar coefficient) then due to linearity $\A(t)=c\v(t)$ is a parallel
 transport of the vector $\A$. The vector $\A(t)$ is tangent to the curve since it is proportional to velocity vector.
  We proved that any tangent vector remains tangent during parallel transport.

Now prove the converse: Let $\A(t)$ be a parallel transport of non-zero vector and it is proportional to velocity.
  If in a given parameterisation $\A(t)=c(t)\v(t)$ choose a reparameterisation $t=t(\tau)$ such that
  ${dt(\tau)\over d\tau}=c(t)$. In the new parameterisation the velocity vector
  $\v'(\tau)={dt(\tau)\over d\tau}\v(t(\tau))=
  c(t)\v(t)=\A(t(\tau))$. We come to parameterisation such that velocity vector remains covariantly constant.
  Thus we come to parameterised geodesic.
  Hence $C$ is a geodesic.

\m

{\bf Remark} In particularly it follows from the Proposition above the following important observation:

Let  $C$ is un-parameterised geodesic, $x^i(t)$ be its  arbitrary parameterisation
and $\v(t)$ be velocity vector in this parameterisation. Then the velocity vector remains parallel to the curve
since it is a tangent vector.

In spite of the fact that velocity vector is not covariantly constant along the curve,
i.e. it will not remain velocity vector during parallel transport, since
it will be remain tangent to the curve during parallel transport.




{\bf Remark} One can see
that if $x^i=x^i(t)$ is geodesic in an arbitrary parameterisation and
$s=s(t)$ is a natural parameter (which defines the length of the curve) then
$x^i(t(s))$ is parameterised geodesic.


\subsubsection {Geodesics on surfaces in $\E^3$}

 Let $M\colon \r=\r(u,v)$ be a surface in $\E^3$.  Let $G_M$ be induced Riemannian metric
 and $\nabla$ a Levi-Civita connection on $M$. We consider on $M$ Levi-Civita connection of the metric $G_M$.

  Let $C$ be an arbitrary  geodesic and $\v(t)={d\r(t)\over dt}$ the velocity vector.
  According to the definition of geodesic
        $\nabla_\v\v=0$. On the other hand we know that Levi-Civita connection
coincides with the connection induced on the surface by canonical flat connection in $\E^3$
 (see the Theorem in subsection 2.4). Hence
                 \begin{equation}\label{proofofcoicidence ofconnections}
    \nabla_\v\v=0=\nabla^M_\v\v=\left(\nabla^{\rm can.flat}_\v\v\right)_{\rm tangent}
\end{equation}
In Cartesian coordinates  $\nabla^{\rm can.flat}_\v\v=\p_\v \v={d\over dt}\v(u(t),v(t))={d^2\r(t)\over dt^2}=\ac$.

Hence according to \eqref{proofofcoicidence ofconnections} the tangent component of acceleration equals to zero.

Converse if for the curve $\r(t)=\r(u(t),v(t))$ the acceleration vector $\ac(t)$ is orthogonal to the surface
then due to  \eqref{proofofcoicidence ofconnections} $\nabla_\v\v=0$.

We come to very beautiful observation:

{\bf Proposition} {\it The acceleration vector of an curve  $\r=\r(u(t),v(t))$ on $M$
 is orthogonal to the surface $M$ if and only if this curve is geodesic.

\m

In other words due to Newton second law particle moves along along geodesic on the surface if and only if
the force is
orthogonal to the surface.}


\m


One can very easy using this Proposition to calculate geodesics of cylinder and sphere.


{\it Geodesic on the cylinder}

Let $\r(h(t),\varphi(t))$ be a geodesic on the cylinder $\begin{cases}x=a\cos\varphi\cr y=a\sin\varphi \cr z=h\cr
\end{cases}$.
    We have  $\v={d\r\over dt}=\begin{pmatrix}-a\dot \varphi\sin\varphi\cr a\dot
    \varphi\cos\varphi\cr \dot h\end{pmatrix}$
and for acceleration:
                       \begin{equation*}\label{}
    \ac={d\v\over dt}= \underbrace
              {
    \begin{pmatrix}-a{\buildrel \cdot\cdot\over \varphi\,}\sin\varphi\cr
    a{\buildrel \cdot\cdot\over \varphi\,}\cos\varphi\cr
    {\buildrel \cdot\cdot\over h\,}\cr
    \end{pmatrix}
         }_{\hbox {tangent acceleration}}+
         \underbrace
            {
    \begin{pmatrix}-a\dot \varphi^2\cos\varphi\cr
     -a\dot\varphi^2\sin\varphi\cr
         0\cr\end{pmatrix}
    }_{\hbox {normal acceleration}}
\end{equation*}
Since tangential acceleration equals to zero hence ${d^2h\over dt^2}=0$ and $h(t)=h_0+ct$
Normal acceleration is centripetal acceleration of the rotation over circle with constant speed
(projection on the plane $OXY$). The geodesic is helix.

\m

   {\it Geodesics on sphere}


\noindent Let $\r=\r(\theta(t),\varphi(t))$ be a geodesic on the sphere  of the radius $a$:
$\r(\theta,\varphi)\colon \begin{cases}x=a\sin\theta\cos\varphi\cr y=a\sin\theta\sin\varphi \cr z=a\cos\theta\cr
\end{cases}$


Consider the vector product of the vectors $\r(t)$ and velocity vector $\v(t)$
${\bf M}(t)=\r(t)\times \v(t)$.
Acceleration vector $\ac(t)$ is proportional to the $\r(t)$
since due to Proposition it is orthogonal to the surface of the sphere. This implies that
$\M(t)$ is constant vector:

               \begin{equation}\label{momentofmotion}
{d\over dt}{\bf M}(t)={d\over dt}\left(\r(t)\times \v(t)\right)=
\left(\v(t)\times \v(t)\right)+\left(\r(t)\times \ac(t)\right)=0
              \end{equation}
         We have ${\bf M}(t)={\bf M}_0$.  $\r(t)$ is orthogonal to ${\bf M}=\r(t)\times \v(t)$.
         We see that $\r(t)$ belongs to the sphere and to the plane orthogonal to the vector ${\bf M}_0=\r(t)\times \v(t)$.
         The intersection of this plane with sphere is a great circle.
         We proved that if $\r(t)$ is geodesic hence it belongs to great circle (as un-parameterised curve).

         The converse is evident since if particle moves along the great circle with constant velocity
         then obviously acceleration vector is orthogonal to the surface.

{\bf Remark} The vector $\M=\r(t)\times \v(t)$ is the torque. The torque is integral of motion
 in isotropic space.---This is the core of the considerations for geodesics on the sphere.




   \subsection {Geodesics and Lagrangians of "free" particle on Riemannian manifold.}


     \subsubsection {Lagrangian and Euler-Lagrange equations}

   A function $L=L(x,\dot x)$ on points and velocity vectors on manifold $M$
   is a {\it Lagrangian} on manifold $M$.


   We assign  to  Lagrangian $L=L(x,\dot x)$ the following second order differential equations
\begin{equation}\label{ELequations1}
    {d\over dt}\left({\p L\over \p \dot x^i}\right)={\p L\over \p  x^i}
\end{equation}
In detail
\begin{equation}\label{ELequationsindetail}
    {d\over dt}\left({\p L\over \p \dot x^i}\right)={\p^2 L\over \p x^m \p \dot x^i}\dot x^m+
    {\p^2 L\over \p \dot x^m \p \dot x^i}{\buildrel \cdot\cdot\over x^m}
    =
    {\p L\over \p  x^i}\,.
\end{equation}

  These equations are called {\it Euler-Lagrange equations} of the Lagrangian $L$.
  We will explain later the variational origin of these equations
  \footnote{To every mechanical system one can put in correspondence  a Lagrangian on configuration space.
      The dynamics of the system is described by Euler-Lagrange equations.
      The advantage of Lagrangian approach is that it works in an arbitrary coordinate system:
      Euler-Lagrange equations are invariant
      with respect to changing of coordinates since they arise from variational principe.}.

    \subsubsection {Lagrangian of "free" particle}


     Let $(M,G)$, $G=g_{ik}dx^idx^k$ be a Riemannian manifold.


     {\bf Definition} We say that {\it Lagrangian} $L=L(x,\dot x)$ is the Lagrangian of a "free" particle on the
     Riemannian manifold $M$ if
      \begin{equation}\label{frepaticlelagrangian}
        L={g_{ik}\dot x^i\dot x^k\over 2}
      \end{equation}

{\bf Example} "Free" particle in Euclidean space.
    Consider $\E^3$ with standard metric $G=dx^2+dy^2+dz^2$
      \begin{equation}\label{freeparticleineuclideanspace}
        L={g_{ik}\dot x^i\dot x^k\over 2}={\dot x^2+\dot y^2+\dot z^2\over 2}
      \end{equation}

Note that this is the Lagrangian that describes the dynamics od free particle.

\m

{\bf Example} "Free" particle on sphere.

The metric on the sphere of radius $R$ is
$G=R^2d\theta^2+R^2\sin^2\theta d\varphi^2$.
Respectively for the Lagrnagian of "free" particle we have
\begin{equation}\label{freeparticleonthesphere}
        L={g_{ik}\dot x^i\dot x^k\over 2}={R^2\dot\theta^2+R^2\sin^2\theta\dot \varphi^2\over 2}
      \end{equation}


\m
\subsubsection {Equations of geodesics and Euler-Lagrange equations}

{\bf Theorem}. {\it Euler-Lagrange equations of Lagrangian of free particle are equivalent
to the second order differential equations
for geodesics.}

This Theorem makes very easy calculations for Christoffel indices.

\m

 This Theorem can be proved by direct calculations.

Calculate Euler-Lagrange equations \eqref{ELequations1} for the Lagrangian \eqref{frepaticlelagrangian}:
               $$
    {d\over dt}\left({\p L\over \p \dot x^i}\right)=
    {d\over dt}\left({\p \left({g_{mk}\dot x^m\dot x^k\over 2}\right)\over \p \dot x^i}\right)=
     {d\over dt}\left(g_{ik}\dot x^k\right)=
    g_{ik}{\buildrel \cdot\cdot\over x{^k}}+{\p g_{ik}\over \p x^m}\dot x^m\dot x^k
               $$
and
            $$
            {\p L\over \p  x^i}={\p \left({g_{mk}\dot x^m\dot x^k\over 2}\right)\over \p  x^i}=
            {1\over 2}{\p g_{mk}\over \p x^i}\dot x^m\dot x^k.
            $$
Hence we have
           $$
 {d\over dt}\left({\p L\over \p \dot x^i}\right)=
 g_{ik}{\buildrel \cdot\cdot\over x{^k}}+{\p g_{ik}\over \p x^m}\dot x^m\dot x^k=
    {\p L\over \p  x^i}={1\over 2}{\p g_{mk}\over \p x^i}\dot x^m\dot x^k\,,
           $$
  i.e.
       $$
 g_{ik}{\buildrel \cdot\cdot\over x{^k}}+\p_mg_{ik}\dot x^m\dot x^k=
 {1\over 2}\p_i g_{mk}\dot x^m\dot x^k\,.
       $$
 Note that    $\p_mg_{ik}\dot x^m\dot x^k={1\over 2}\left(\p_mg_{ik}\dot x^m\dot x^k+\p_kg_{im}\dot x^m\dot x^k\right)$.
Hence we come to equation:
               $$
               g_{ik}{d^2 x^k\over dt^2}+
               {1\over 2}\left(
 \p_mg_{ik}+\p_kg_{im}-\p_ig_{mk}
                  \right)\dot x^m\dot x^k
               $$
 Multiplying on the inverse matrix $g^{ik}$ we come
\begin{equation}\label{ELequationforgeodesic1}
    {d^2 x^i\over dt^2}+{1\over 2}g^{ij}\left({\p g_{jm}\over\p x^k}+
    {\p g_{jk}\over\p x^m}-{\p g_{mk}\over \p x^j}\right){dx^m\over dt}{dx^k\over dt}=0\,.
\end{equation}
We recognize here Christoffel symbols of Levi-Civita connection (see \eqref{levi-civitaformula}) and we rewrite
this equation as
   \begin{equation}\label{ELequationforgeodesic2}
    {d^2 x^i\over dt^2}+{dx^m\over dt}\Gamma^i_{mk}{dx^k\over dt}=0\,.
\end{equation}
This is nothing but the equation \eqref{geodesdef1}.

Applications of this Theorem: calculation of Christoffel symbols of Levi-Civita connection.

\subsubsection{Examples of calculations of Christoffel symbols and geodesics using Lagrangians.}

Consider two examples: We calculate Levi-Civita connection on sphere in $\E^3$ and on Lobachevsky plane
using Lagrangians and find geodesics.

1) {\it Sphere of the radius $R$ in $\E^3$}:


   Lagrangian of "free" particle on the sphere is given by \eqref{freeparticleonthesphere}:
    $$
    L={R^2\dot \theta^2+R^2\sin^2\theta\dot \varphi^2\over 2}
    $$
Euler-Lagrange equations defining geodesics are
          \begin{equation}\label{EL eqforsphere}
  {d\over dt}\left({\p L\over \p  \dot\theta}\right)-
  {\p L\over \p \theta}=
  {d\over dt} \left(R^2{\buildrel \cdot\over \theta}\right)-
  R^2\sin\theta\cos\theta\dot \varphi^2\Rightarrow  {\buildrel \cdot\cdot\over \theta}-
  \sin\theta\cos\theta\dot \varphi^2=0\,,
          \end{equation}
              $$
{d\over dt}\left({\p L\over \p  \dot\varphi}\right)-
  {\p L\over \p \varphi}=
  {d\over dt} \left(R^2{\sin^2\theta\dot\varphi}\right)=0\Rightarrow
  {\buildrel \cdot\cdot\over \varphi}+
  {\rm cotan\,}\theta\dot\theta\dot\varphi=0\,.
                            $$
Comparing Euler-Lagrange equations with equations for geodesic in terms of Christoffel symbols:
         $$
   {\buildrel \cdot\cdot\over \theta}+\Gamma^\theta_{\theta\theta}\dot\theta^2+
   2\Gamma^\theta_{\theta\varphi}\dot\theta\dot\varphi+\Gamma^\theta_{\varphi\varphi}\dot\varphi^2=0,
         $$
    $$
  {\buildrel \cdot\cdot\over \varphi}+\Gamma^\varphi_{\theta\theta}\dot\theta^2+
   2\Gamma^\varphi_{\theta\varphi}\dot\theta\dot\varphi+\Gamma^\varphi_{\varphi\varphi}\dot\varphi^2=0
    $$
    we come to
 \begin{equation}\label{christsymbolsthroughlagrangiansforsphere1}
    \Gamma^\theta_{\theta\theta}=
    \Gamma^\theta_{\theta\varphi}=\Gamma^\theta_{\varphi\theta}=0\,,
    \Gamma^\theta_{\varphi\varphi}=-\sin\theta\cos\theta\,,
 \end{equation}
 \begin{equation}\label{christsymbolsthroughlagrangiansforsphere2}
    \Gamma^\varphi_{\theta\theta}=\Gamma^\varphi_{\varphi\varphi}=0,\,
    \Gamma^\varphi_{\theta\varphi}=\Gamma^\varphi_{\varphi\theta}={\rm cotan\, }\theta\,.
 \end{equation}
 (Compare with previous calculations for connection in subsections 2.2.1 and 2.3.4)



{\footnotesize  We proved that geodesics on the sphere are great circles.
(see subsection 3.2.3 above).
 Consider another more straightforward proof of this fact.
To find geodesics one have to solve second order differential equations \eqref{EL eqforsphere}:

One can see that the great circles: $\varphi=\varphi_0$, $\theta=\theta_0+t$ are solutions of
second order differential equations \eqref{EL eqforsphere} with initial conditions
\begin{equation}\label{initialconditions}
  \theta(t)\big\vert_{t=0}=\theta_0, \dot\theta(t)\big\vert_{t=0}=1,\quad
  \varphi(t)\big\vert_{t=0}=\varphi_0, \dot\theta(t)\big\vert_{t=0}=0\,.
\end{equation}
The rotation of the sphere is isometry, which does not change Levi-Civta connection.
Hence an arbitrary great circle is geodesic.

Prove that an arbitrary geodesic is an arc of great circle.
Let the curve $\theta=\theta(t),\varphi=\varphi(t)$,
$0\leq t\leq t_1$ be geodesic. Rotating the sphere
we can come to the curve $\theta=\theta'(t),\varphi=\varphi'(t)$, $0\leq t\leq t_1$
such that  velocity vector at the initial time is direccted along meridian, i.e.
initial conditions are
\begin{equation}\label{initialconditionsnew}
  \theta'(t)\big\vert_{t=0}=\theta_0, \dot\theta'(t)\big\vert_{t=0}=a,\quad
  \varphi'(t)\big\vert_{t=0}=\varphi_0, \dot\varphi'(t)\big\vert_{t=0}=0\,.
\end{equation}
(Compare with initial conditions \eqref{initialconditions})
Second order differential equations with boundary conditions for coordinates and velocities at $t=0$ have unique
solution. The solutions of second order differential equations \eqref{EL eqforsphere} with initial conditions
\eqref{initialconditionsnew} is a curve
                $\theta'(t)=\theta_0+at$, $\varphi'(t)=\varphi_0$. It is great circle.
                Hence initial curve the geodesic $\theta=\theta(t),\varphi=\varphi(t)$,
$0\leq t\leq t_1$ is an arc of great circle too.

This is another proof that geodesics are great circles.}


\m


2) {\it Lobachevsky plane.}


   Lagrangian of "free" particle on the Lobachevsky plane with metric $G={dx^2+dy^2\over y^2}$ is
      $$
   L={1\over 2}{\dot x^2+\dot y^2\over y^2}.
      $$
Euler-Lagrange equations are

  $$
  {\p L\over \p x}=0={d\over dt}{\p L\over \p \dot x}={d\over dt}\left({\dot x\over y^2}\right)=
            {{\buildrel \cdot\cdot\over x}\over y^2}-{2\dot x\dot y\over y^3}, {\rm i.e.}\quad
            {\buildrel \cdot\cdot\over x}-{2\dot x\dot y\over y}=0\,,
             $$
             $$
              {\p L\over \p y}=-{\dot x^2+\dot y^2\over y^3}=
              {d\over dt}{\p L\over \p \dot y}={d\over dt}\left({\dot y\over y^2}\right)=
            {{\buildrel \cdot\cdot\over y}\over y^2}-{2\dot y^2\over y^3}, {\rm i.e.}\quad
            {\buildrel \cdot\cdot\over y}+{\dot x^2\over y}-{\dot y^2\over y}=0\,.
            $$
Comparing these equations with equations for geodesics:
${\buildrel \cdot\cdot\over x^i}-\dot x^k\Gamma^i_{km}\dot x^m=0$
($i=1,2$, $x=x^1,y=x^2$) we come to
                $$
\Gamma^{x}_{xx}=0,
\Gamma^{x}_{xy}=\Gamma^{x}_{yx}=-{1\over y},\,
\Gamma^{x}_{yy}=0,\,\Gamma^{y}_{xx}={1\over y}, \Gamma^{y}_{xy}=\Gamma^{y}_{yx}=0, \Gamma^{y}_{yy}=-{1\over y}\,.
              \hbox{\finish}  $$
{\footnotesize In  a similar way as for a sphere one can find geodesics on Lobachevsky plane.
First we note that vertical rays are geodesics.  Then using the inversions with centre on the absolute
one can see that arcs of the circles with centre at the absolute ($y=0$) are geodesics too.
}




\subsubsection {Variational principe and Euler-Lagrange equations }

   Here very briefly we will explain how Euler-Lagrange equations follow from variational principe.

\m
\def\confspace {\M^{^{\x_2,t_2}}_{_{\x_1,t_1}}}

Let $M$ be a manifold (not necessarily Riemannian) and $L=L(x^i,\dot x^i)$ be a Lagrangian on it.

   Denote my $\M^{^{\x_2,t_2}}_{_{\x_1,t_1}}$ the space of curves (paths)
such that they start at the point $\x_1$ at the "time" $t=t_1$ and end at the
point  $\x_2$ at the "time" $t=t_2$:
\begin{equation}\label{spacceofpaths}
\M^{^{\x_2,t_2}}_{_{\x_1,t_1}}=\{C\colon\,\, \x(t), t_1\leq t\leq t_2,\,\,\x(t_1)=\x_1, \x(t_2)=\x_2\}\,.
\end{equation}

  Consider the following functional $S$ on the space $\confspace$:
                \begin{equation}\label{actionfunctional1}
  S\left[\x(t)\right]=\int_{t_1}^{t_2}L\left(x^i(t), \dot x^i(t)\right)dt\,.
                \end{equation}
    for every curve $\x(t)\in \confspace$.

This functional is called {\it action} functional.

{\bf Theorem}  {\it Let functional $S$ attaints the minimal value on the path $\x_0(t)\in\confspace$, i.e.
                             \begin{equation}\label{minimalvalueofthefunctional}
                    \forall \x(t)\in \confspace\quad            S[\x_0(t)]\leq  S[\x(t)]\,.
                             \end{equation}
Then the path $\x_0(t)$ is a solution of Euler-Lagrange equations of the Lagrangian $L$:
            \begin{equation}\label{minimalfunctionsolution1}
{d\over dt}\left({\p L\over \p \dot x^i}\right)={\p L\over \p  x^i}\,\,
            {\rm if}\,\, \x(t)=\x_0(t)\,.
            \end{equation}
 }
 \m


 {\bf Remark} The path $\x(t)$ sometimes is called {\it extremal} of the action functional
\eqref{actionfunctional1}.

   We will use this Theorem to show that the geodesics are in some sense shortest curves
   \footnote{The statement of this Theorem is enough for our purposes.
  In fact in classical mechanics another more useful statement
  is used:
  the path $\x_0(t)$ is a solution of Euler-Lagrange equations of the Lagrangian $L$
  if and only if it is the stationary "point" of the action functional  \eqref{actionfunctional1}, i.e.
           \begin{equation}\label{stationarpath}
            S[\x_0(t)+\delta \x(t)]-S[\x_0(t)+\delta \x(t)]=0(\delta \x(t))
           \end{equation}
 for an arbitrary infinitesimal variation of the path $\x_0(t)$:
   $\delta \x_(t_1)=\delta \x_(t_2)=0$.}.



\subsection {Geodesics and shortest distance.}

 Many of you know that geodesics are in some sense shortest curves.
 We will give an exact meaning to this statement and prove it using variational principe:

Let $M$ be a Riemannian manifold.

{\bf Theorem}
{\it Let $\x_1$ and $\x_2$ be two points on $M$.
The shortest curve which joins these points is an arc of geodesic.

Let $C$ be a geodesic on $M$ and  $\x_1\in C$. Then for an arbitrary point  $\x_2\in C$ which is
close to the point $\x_1$ the arc of geodesic joining the points $\x_1,\x_2$ is a shortest curve between
these points\footnote{More precisely: for every point $\x_1\in C$ there exists a ball $B_{\delta}(\x_1)$
such that for an arbitrary point $\x_2\in C\cap B_{\delta}(\x_1)$
the arc of geodesic joining the points $\x_1,\x_2$ is a shortest curve between
these points.}
.}

\m

  This Theorem makes a bridge between two different approach to geodesic: the shortest disntance and
  parallel transport of velocity vector.

\m

Sketch a proof:


Consider the following two Lagrangians: Lagrangian of a "free "
particle $L_{\rm free}={g_{ik}(x)\dot x^i\dot x^k\over 2}$
and the length Lagrangian
         $$
      L_{\rm length}(x,\dot x)=\sqrt{g_{ik}(x)\dot x^i\dot \x^k}=\sqrt {2L_{\rm free}}\,.
         $$
   If $C\colon \,\,x^i(t), t_1\leq t\leq t_2$ is a curve on $M$ then
                  $$
             \hbox{Length of the curve $C$}=
                  $$
    \begin{equation}\label{lengthlagrangian}
    \int_{t_1}^{t_2}L_{\rm length}(x^i(t),\dot x^i(t))dt=
    \int_{t_1}^{t_2}\sqrt{g_{ik}(x(t))\dot x^i(t)\dot x^k(t)}dt\,.
    \end{equation}


    The proof of the Theorem follows from the following observation:

 {\it Observation}  Euler-Lagrange equations for the length  functional \eqref{lengthlagrangian}
 are equivalent to the  Euler-Lagrange equations for action functional \eqref{actionfunctional1}.
        This means that extremals of the length functional and action functionals coincide.

        \m

    Indeed it follows from this observation and  the variational principe that
     the shortest curves obey
     the Euler-Lagrange equations for the action functional.
     We showed before that Euler-Lagrange equations for action functional \eqref{actionfunctional1}
    define geodesics.   Hence the shortest curves are geodesics.

    \m

    One can  check the observation by direct calculation:  Calculate Euler-Lagrange equations for the Lagrangian
    $L_{\rm length}=\sqrt{g_{ik}(x)\dot x^i\dot x^k}=\sqrt {2L_{\rm free}}$:
                    $$
     {d\over dt}\left({\p L_{\rm length}\over \p \dot x^i}\right)-{\p L_{\rm length}\over \p  x^i}=
     {d\over dt}\left( {1\over \sqrt{g_{ik}\dot x^i\dot x^k}}g_{ik}\dot x^k\right)-
     {1\over 2\sqrt{g_{ik}\dot x^i\dot x^k}}{\p g_{km}\dot x^k\dot x^m\over  \p x^i}
                    $$
                    \begin{equation}\label{geodesiclagrequation4}
=    {d\over dt}\left( {1\over L_{\rm length}}{\p L_{\rm free}\over  \dot \p x^i}\right)-
     {1\over L_{\rm length}}{\p L_{\rm free}\over  \p x^i}=0\,.
                                         \end{equation}


To facilitate calculations note that the length functional \eqref{lengthlagrangian}
    is reparameterisation invariant.  Choose the natural parameter $s(t)$ or a parameter proportional
    to the natural parameter on the curve $x^i(t)$.  We come to $L_{\rm length}=const$ and
    it follows from  \eqref{geodesiclagrequation4} that
        $$
   {d\over dt}\left({\p L_{\rm length}\over \p \dot x^i}\right)-{\p L_{\rm length}\over \p  x^i}=
    {1\over L_{\rm length}}\left( {d\over dt}\left( {\p L_{\rm free}\over  \dot \p x^i}\right)-
     {\p L_{\rm free}\over  \p x^i}\right)=0\,.
            $$
 We prove that Euler-Lagrange equations for length and action Lagrangians coincide.\finish


\m

  In the Euclidean space straight lines are the shortest distances between two points. On the other hand
  their velocity vectors are constant.
We realise now that in general Riemannian manifold the role of geodesic is twofold also:
  they are locally shortest and have covariantly constant velocity vectors.

\m

\subsubsection {Again geodesics for sphere and Lobachevsky plane}

The fact that geodesics are shortest gives us another tool to calculate geodesics.

Consider again examples of sphere and Lobachevsky plane and find geodesics using the fact that they are shortest.
The fact that geodesics are locally the shortest curves

\m


Consider  again sphere in $\E^3$ with the radius $R$:
  Coordinates $\theta,\varphi$, induced Riemannian metrics (first quadratic form):
\begin{equation}\label{metricsonsphere5}
  G=R^2(d\theta^2+\sin^2\theta d\varphi^2)\,.
\end{equation}
Consider two arbitrary points $A$ and $B$ on  the sphere.
Let $(\theta_0,\varphi_0)$ be coordinates
  of the point $A$ and $(\theta_1,\varphi_1)$ be coordinates
  of the point $B$


 Let $C_{AB}$ be a curve which connects these points:
   $C_{AB}\colon \theta(t),\varphi(t)$ such that
    $\theta(t_0)=\theta_0, \theta(t_1)=\theta_1$,
    $\varphi(t_0)=\theta_0, \theta(t_1)=\theta_1$
  then:
\begin{equation}\label{lengthonsphere5}
  L_{C_{AB}}=\int R\sqrt {\theta_t^2+\sin^2\theta(t)\varphi_t^2}dt
\end{equation}

  Suppose that points $A$ and $B$
  have the same latitude, i.e.
  if $(\theta_0,\varphi_0)$ are coordinates
  of the point $A$ and $(\theta_1,\varphi_1)$ are coordinates
  of the point $B$ then $\varphi_0=\varphi_1$
  (if it is not the fact then we can come to this condition rotating the sphere)

Now it is easy to see that an arc of meridian, the  curve $\varphi=\varphi_0$ is geodesics:
Indeed consider an arbitrary curve $\theta (t),\varphi(t)$
which connects the points $A,B$:
$\theta(t_0)=\theta(t_1)=\theta_0$,
$\varphi(t_0)=\varphi(t_1)=\varphi_0$.
Compare its length with the length
of the meridian which connects the points $A$, $B$:
\begin{equation}\label{geodforsphereap}
\int_{t_0}^{t_1} R\sqrt {\theta_t^2+\sin^2\theta\varphi_t^2}dt\geq
R\int_{t_0}^{t_1} \sqrt {\theta_t^2}dt=R\int_{t_0}^{t_1} {\theta_t}dt=
R(\theta_1-\theta_0)
\end{equation}
Thus we see that  the great circle joining points $A,B$ is the shortest.
{\it The great circles on sphere are geodesics.}
 It corresponds to geometrical intuition:
     The geodesics on the sphere are the circles of intersection of the sphere
     with the plane which crosses the centre.


\bigskip

     {\it Geodesics on Lobachevsky plane}

     Riemannian metric on Lobachevsky plane:
            \begin{equation}\label{lobachmetric5}
               G={dx^2+dy^2\over y^2}
            \end{equation}



The length of the curve $\gamma\colon x=x(t, y=y(t))$ is equal to
                     $$
   L=\int\sqrt {{x_t^2+y_t^2\over y^2(t)}}dt
                        $$

In particularly the length of the vertical interval $[1,\vare]$ tends
to infinity if $\vare\to 0$:
                    $$
   L=\int\sqrt {{x_t^2+y_t^2\over y^2(t)}}dt=
   \int_\vare^1\sqrt {{1\over t^2}}dt=\log {1\over \vare}
                     $$
One can see that the distance from  every point to the line $y=0$ is
equal to infinity. This motivates the fact that the line $y=0$
is called {\it absolute}.

Consider two points $A=(x_0,y_0)$, $B=(x_1,y_1)$ on Lobachevsky plane.


It is easy to see that vertical lines are geodesics of Lobachevsky plane.

Namely let points $A, B$ are on the ray $x=x_0$.
Let  $C_{AB}$ be an arc of the ray $x=x_0$ which joins these points:
    $C_{AB}\colon x=x_0, y=y_0+t$
Then it is easy to see that the length of the curve $C_{AB}$ is less or equal than the length of the arbitrary curve
  $x=x(t),y=y(t)$ which joins these points: $x(t)\big\vert_{t=0}=x_0, y(t)\big\vert_{t=0}=y_0$,
$x(t)\big\vert_{t=t_1}=x_0, y(t)\big\vert_{t=t_1}=y_1$:
          $$
 \int_{0}^t\sqrt {x_t^2+y_t^2\over y^2(t)}dt\geq
 \int_{0}^t\sqrt {y_t^2\over y^2(t)}dt=\int_{y_0}^{y_1} {dt\over t}dt=\log {y_1\over y_0}=\hbox{length of $C_{AB}$}
          $$
Hence $C_{AB}$ is shortest. We prove that vertical rays are geodesics.

Consider now the case if $x_0\not=x_1$.
Find  geodesics which connects two points $A,B$ which are not on the same vertical ray.
Consider semicircle which passes
these two points such that its centre is on the absolute.
We prove that it is a geodesic.

\medskip
{\footnotesize
{\it Proof} Let coordinates of the centre of the circle
are $(a,0)$. Then consider polar coordinates $(r,\varphi)$:
\begin{equation}\label{semicircle}
  x=a+r\cos\varphi, y=r\sin\varphi
\end{equation}
In these polar coordinates $r$-coordinate of the semicircle is constant.

Find Lobachevsky metric  in these coordinates:
 $dx=-r\sin\varphi d\varphi+\cos\varphi dr$, $dy=r\cos\varphi d\varphi+\sin\varphi dr$,
 $dx^2+dy^2=dr^2+r^2d\varphi^2$. Hence:
\begin{equation}\label{lob2}
  G={dx^2+dy^2\over y^2}={dr^2+r^2d\varphi^2\over r^2\sin^2\varphi}=
={d\varphi^2\over \sin^2\varphi}+{dr^2\over r^2\sin^2\varphi}
\end{equation}
We see
that the length of the arbitrary curve which connects points $A,B$
is greater or equal to the length of the arc of the circle:
\begin{equation}\label{lengthmin1}
 L_{AB}=
 \int_{t_0}^{t_1}\sqrt
 {{\varphi_t^2\over \sin^2 \varphi}+{r_t^2\over r^2\sin^2\varphi}}dt\geq
\int_{t_0}^{t_1}\sqrt
 {{\varphi_t^2\over \sin^2 \varphi}}dt=
 \end{equation}
 $$
\int_{t_0}^{t_1}
 {{\varphi_t\over \sin \varphi}}dt=
\int_{\varphi_0}^{\varphi_1}
 {{d\varphi\over \sin \varphi}}=
 \log {\tan \varphi_1\over \tan \varphi_1}
$$

The proof is finished.}





\section {Surfaces in $\E^3$}

 Now equipped by the knowledge of Riemannian geometry we consider surfaces in $\E^3$. We
  reconsider again conceptions of Shape (Weingarten) operator, Gaussian and mean curvatures,
  focusing attention on the the fact what properties are internal and what properties are external.






\subsection {Formulation of the main result. Theorem of parallel transport over closed curve
and {\it Theorema Egregium}}

Let $M$ be a surface in  Euclidean space $\E^3$.
  Consider a closed curve $C$ on $M$,
  $M\colon \r=\r(u,v)$, $C\colon \r=\r(u(t,v(t)), 0\leq t\leq t_1, \, \x(0)=\x(t_1)$.
  ($u(t), v(t)$ are internal coordinates of the curve $C$.)

Consider the parallel transport of an arbitrary tangent $\X$ vector along the closed curve $C$:
          \begin{equation*}\label{paraltransportalongclosedcurve1}
\X(t)= \underbrace {X^\a(t){\p\over \p u_\a}\big\vert_{u^\a(t)}}_{\hbox{Internal observer}} =
 \underbrace{X^\a(t)\r_\a\big\vert_{\r(u(t),v(t))}}_{\hbox{External observer}}\,\,,\left(
 \r_\a={\p x^i\over \p u^\a}{\p\over \p x^i}\right)\,.
          \end{equation*}
          \begin{equation*}\label{paraltransportalongclosedcurve2}
\X(t)\colon\,\, {\nabla \X(t)\over dt}=0,\,\, 0\leq t\leq t_1\,,
          \end{equation*}
i.e.
                \begin{equation}\label{paraltransportalongclosedcurve3}
{d X^\a(t)\over dt}+
X^\beta(t)\Gamma^\a_{\beta\gamma}(u(t))
{du^\gamma(t)\over dt}=0,\,\, 0\leq t\leq t_1\,,
          \end{equation}
where $\nabla$ is the connection induced on the surface $M$ by canonical flat connection
(see \eqref{christoffelsymbols1forinducedconnection}), i.e.
the Levi-Civita connection of the induced Riemannian metric on the surface $M$
 and $\Gamma^\a_{\beta\gamma}$ its Christoffel symbols:
\begin{equation}\label{connectioninduced12}
          \Gamma^\gamma_{\a\beta}={1\over 2}g^{\gamma\pi}
      \left({\p g_{\pi \a}\over \p u^\beta}+{\p g_{\pi \beta}\over \p u^\a}-
      {\p g_{\a\beta}\over \p u^\pi}\right)\,,{\rm where}\,
      g_{\a\beta}=\langle\r_\a,\r_\beta\rangle={\p x^i\over \p u^\a} {\p x^i\over \p u^\beta}
     \end{equation}
 are components of induced Riemannian metric
 $G_M=g_{\a\beta}du^\a du^\beta$/


Let $\r(0)=\pt$ be a starting (and ending) point of the curve $C$: $\r(0)=\r(t_1)=\pt$.
The differential equation \eqref{paraltransportalongclosedcurve3} defines the linear operator
            \begin{equation}\label{linearoperatoroverclosedcurve}
            R_C\colon T_{\pt}M\longrightarrow T_{\pt}M
            \end{equation}
For any vector $\X\in T_{\pt}M$, its image the vector $R_C\X$ as the solution of the differential equation
\eqref{paraltransportalongclosedcurve3} with initial condition $\X(t)\big\vert_{t=0}=\X$.

On the other hand we know that parallel transport of the vector does not change its length (see
\eqref{lengthdoesnotchangeduringparalleltransport} in the subsection 3.2.1):
       \begin{equation}\label{linearoperatoroverclosedcurvescalarproduct}
            \langle\X,\X\rangle=\langle R_C\X,R_C\X\rangle
            \end{equation}
We see that $\R_C$ is an orthogonal operator in the $2$-dimensional vector space $T_{\pt}M$.
We know that orthogonal operator preserving orientation is the operator of rotation on some angle $\Phi$.

We see that if $\R_C$ preserves orientation\footnote{We consider the case if the operator $R_C$ preserves orientation.
In our considerations we consider only the case if the closed curve $C$ is a boundary of
a compact oriented domain $D\subset M$. In this case one can see that operator $R_C$ preserves an orientation.}
 then the action of operator $\R_C$ on vectors is rotation on the angle,
i.e. the result of parallel transport along closed curve is rotation on the angle
                               \begin{equation}\label{rotationontheangle}
                               \Delta\Phi=\Delta\Phi(C)
                               \end{equation}
 which depends on the curve.


The very beautiful question arises:  How to calculate this angle $\Delta\Phi(C)$


\m

{\bf Theorem}
{\it  Let $M$ be a surface in  Euclidean space $\E^3$.
  Let $C$ be a closed curve $C$ on $M$ such that $C$ is a boundary of a compact oriented domain $D\subset M$.
Consider the parallel transport of an arbitrary tangent vector along the closed curve $C$.
As a result of parallel transport along this closed curve any  tangent vector rotates through the angle

\begin{equation}\label{theoremofrotationonangle}
\Delta\Phi=\angle\left({\X, \R_C\X}\right)=\int_D K d\sigma\,,
             \end{equation}
where $K$ is the Gaussian curvature and $d\sigma=\sqrt {\det g}dudv$ is the area element induced by the
Riemannian metric on the surface $M$, i.e.  $d\sigma=\sqrt {\det g}dudv$.



}

\m
{\bf Remark} One can show that the angle of rotation does not depend on initial point of the curve.


\m


{\bf Example} Consider the closed curve, "latitude" $C_{\theta_0}\colon\,\theta=\theta_0$ on the sphere of the radius $R$.
Calculations show that
               \begin{equation}\label{rotationforlatitude}
                \Delta\Phi(C_{\theta_0})=2\pi(1-\cos\theta_0)
               \end{equation}

(see the Homework 6). On the other hand the latitude $C_{\theta_0}$ is the boundary of the segment  $D$
 with area $2\pi RH$ where $H=R(1-\cos \theta_0)$. Hence
           $$
   \angle\left({\X, \R_C\X}\right)={2\pi RH\over R^2}={1\over R^2}\cdot \hbox{area of the segment}=\int_D  Kd\sigma
           $$
since Gaussian curvature is equal to $1\over R^2$

The  proof of this Theorem is the content of the subsections above.

\subsubsection {Gau\ss {\it Theorema Egregium}}

  Show  that this Theorem implies the remarkable corollary.

Let  $D$ be a small domain around  a given point $\pt$, let $C$ its boundary and
$\Delta \Phi(D)$ the angle of rotation.   Denote by $S(D)$ an area of this domain.
Applying the Theorem for the case when area of the domain $D$ tends to zero we  we come to the statement that
            $$
      \hbox{if $S(D)\to 0$ then \,} \Delta\Phi(D)=\int_D Kd\sigma\to K(\pt)S(D),\,{\rm i.e.}
            $$

\begin{equation}\label{egregium1}
K(\pt)=\lim_{S(D)\to 0}{\Delta\Phi(D)\over S(D)},\,{\rm i.e.}
\end{equation}

Now notice that  left hand side od this equation defining Gaussian curvature $K(\pt)$ depends only on Riemannian
metric on the surface $C$. Indeed numerator of LHS is defined by the solution of differential equation
$\eqref{paraltransportalongclosedcurve3}$ which depends on Levi-Civita connection depending on
the induced Riemannian metric, and denominator is an area depending on Riemannian metric too.

Thus we come to very important

{\bf Corollary} {\it Gau\ss \,\, Egregium Theorema}

Gaussian curvature of the surface is invariant of isometries.


\bigskip


In next subsections we develop the technique which  itself is very interesting.
One of the applications of this
technique is the proof of the Theorem \eqref{theoremofrotationonangle}.
Thus we will prove Theorema Egregium too.

Later in the fifth section we will give another proof of the Theorema  Egregium.

\subsection{ Derivation formulae}

Let $M$ be a surface embedded in  Euclidean space $\E^3$,
$M\colon \r=\r(u,v)$.

   Let ${\e,\f,\n}$ be three vector fields defined on the points of this surface
   such that they form an orthonormal basis at any point, so that the vectors $\e,\f$ are tangent to the surface
   and  the vector $\n$ is orthogonal to the surface\footnote{
   One can say that $\{\e,\f,\n\}$ is an orthonormal basis in $T_{\pt}\E^3$ at every point of surface  $\pt\in M$
such that $\{\e,\f\}$ is an orthonormal basis in $T_{\pt}\E^3$ at every point of surface  $\pt\in M$.}.
   Vector fields $\e,\f,\n$ are functions on the surface $M$:
                   $$
          \e=\e(u,v),\,\, \f=\f(u,v)\,, \n=\n(u,v)\,.
                   $$
 Consider $1$-forms $d\e,d\f,d\n$:
              $$
       d\e={\p \e\over \p u}du+{\p \e\over \p v}dv,\,
       d\f={\p \f\over \p u}du+{\p \f\over \p v}dv\,,
       d\n={\p \n\over \p u}du+{\p \n\over \p v}dv
              $$
 These $1$-forms take values in the vectors in $\E^3$, i.e. they are {\it vector valued} $1$-forms.
 Any vector in $\E^3$ attached at an arbitrary point of the surface
 can be expanded over the basis  $\{\e,\f,\n\}$. Thus vector valued $1$-forms $d\e,d\f,d\n$ can be expanded
 in a sum of $1$-forms with values in basic vectors  $\e,\f,\n$. E.g.
 for $d\e={\p \e\over \p u}du+{\p \e\over \p v}dv$ expanding   vectors
 ${\p \e\over \p u}$ and ${\p \e\over \p v}$ over basis vectors we come to
            $$
      {\p \e\over \p u}=A_1(u,v)\e+B_1(u,v)\f+C_1(u,v)\n,\,\,
      {\p \e\over \p v}=A_2(u,v)\e+B_2(u,v)\f+C_2(u,v)\n
            $$
 thus
          $$
   d\e={\p \e\over \p u}du+{\p \e\over \p v}dv=\left(A_1\e+B_1\f+C_1\n\right)du+\left(A_2\e+B_2\f+C_2\n\right)dv=
          $$
           \begin{equation}\label{one-forms4}
 =\underbrace{(A_1du+A_2dv)}_{M_{11}}\e+
 \underbrace{(B_1du+B_2dv)}_{M_{12}}\f+\underbrace{(C_1du+C_2dv)}_{M_{11}}\e,
           \end{equation}
 i.e.
            $$
       d\e=M_{11}\e+M_{12}\f+M_{13}\n,
            $$
where $M_{11}, M_{12}$ and $M_{13}$ are $1$-forms on the surface $M$ defined by the relation \eqref{one-forms4}.

In the same way we do the expansions of vector-valued $1$-forms $d\f$ and $d\n$ we come to
                 \begin{equation*}
                \begin{matrix}
                d\e=M_{11}\e+M_{12}\f+M_{13}\n\cr
                d\f=M_{21}\e+M_{22}\f+M_{23}\n\cr
                d\n=M_{31}\e+M_{32}\f+M_{33}\n\cr
                \end{matrix}
                    \end{equation*}
   This equation can be rewritten in the following way:
                  \begin{equation}\label{matrixequation1}
                       d
                  \begin{pmatrix}
                \e\cr
                \f\cr
                \n\cr
                \end{pmatrix}
                    =
                \begin{pmatrix}
                M_{11} &M_{12} &M_{13}\cr
                M_{21} &M_{22} &M_{23}\cr
                M_{31} &M_{32} &M_{33}\cr
                \end{pmatrix}
                      \begin{pmatrix}
                \e\cr
                \f\cr
                \n\cr
                \end{pmatrix}
                    \end{equation}

       \m

  {\it Proposition} {\it The matrix $M$ in the equation \eqref{matrixequation1} is antisymmetrical matrix, i.e.
                     \begin{equation}
                       \begin{matrix}
                    M_{11}=M_{22}=M_{33}=0\cr
                       M_{12}=-M_{21}=a\cr
                       M_{13}=-M_{31}= b\cr
                       M_{23}=-M_{32}=-b\cr
                       \end{matrix}
                       \end{equation}
i.e.



                      \begin{equation}\label{derivationformulae1}
                    d\begin{pmatrix}
                    \e\cr\f\cr\n\cr
                    \end{pmatrix}=
                    \begin{pmatrix}
                    0&a&b\cr -a&0&c\cr -b&-c&0\cr
                    \end{pmatrix}
                 \begin{pmatrix}
                    \e\cr\f\cr\n\cr
                    \end{pmatrix}\,,
                       \end{equation}
 where $a,b,c$ are $1$-forms on the surface $M$.}


 \m
Formulae \eqref{derivationformulae1} are called {\it derivation formulae}.


 Prove this Proposition. Recall that $\{\e,\f,\n\}$ is orthonormal basis, i.e. at every point of the surface
                  $$
  \langle\e,\e\rangle=\langle\f,\f\rangle=\langle\n,\n\rangle=1, \,{\rm and}\,\,
   \langle\e,\f\rangle=\langle\e,\n\rangle=\langle\f,\n\rangle=0
         $$
 Now using \eqref{matrixequation1} we have
         $$
  \langle\e,\e\rangle=1\Rightarrow d\langle\e,\e\rangle=0=2\langle\e,d\e\rangle
  =\langle\e,M_{11}\e+M_{12}\f+M_{13}\n\rangle=
               $$
               $$
               M_{11}\langle\e,\e\rangle+M_{12}\langle\e,\f\rangle+
  M_{13}\langle\e,\n\rangle=M_{11}\Rightarrow M_{11}=0\,.
              $$
 Analogously
           $$
     \langle\f,\f\rangle=1\Rightarrow d\langle\f,\f\rangle=0=2\langle\f,d\f\rangle
  =\langle\f,M_{21}\e+M_{22}\f+M_{23}\n\rangle=M_{22}\Rightarrow M_{22}=0\,,
           $$
            $$
             \langle\n,\n\rangle=1\Rightarrow d\langle\n,\n\rangle=0=2\langle\n,d\n\rangle
  =\langle\n,M_{31}\e+M_{32}\f+M_{33}\n\rangle=M_{33}. \Rightarrow M_{33}=0\,.
            $$
 We proved already that $M_{11}=M_{22}=M_{33}=0$. Now prove that
 $M_{12}=-M_{21}$, $M_{13}=-M_{31}$ and $M_{13}=-M_{31}$.
                       $$
  \langle\e,\f\rangle=0\Rightarrow d\langle\e,\f\rangle=0=\langle\e,d\f\rangle+\langle d\e,\f\rangle=
             $$
             $$
  \langle\e,M_{21}\e+M_{22}\f+M_{23}\n\rangle+\langle M_{11}\e+M_{12}\f+M_{13}\n,\f\rangle=
  M_{21}+M_{12}=0.
          $$
 Analogously
       $$
       \langle\e,\n\rangle=0\Rightarrow d\langle\e,\n\rangle=0=\langle\e,d\n\rangle+\langle d\e,\n\rangle=
             $$
             $$
  \langle\e,M_{31}\e+M_{32}\f+M_{33}\n\rangle+\langle M_{11}\e+M_{12}\f+M_{13}\n,\n\rangle=
  M_{31}+M_{13}=0
       $$
 and
          $$
          \langle\f,\n\rangle=0\Rightarrow d\langle\f,\n\rangle=0=\langle\f,d\n\rangle+\langle d\f,\n\rangle=
             $$
             $$
  \langle\f,M_{31}\e+M_{32}\f+M_{33}\n\rangle+\langle M_{21}\e+M_{22}\f+M_{23}\n,\n\rangle=
  M_{32}+M_{23}=0.
          $$

       {\bf Remark} This proof could be performed much more shortly in condensed notations. Derivation formulae
       \eqref{derivationformulae1} in condensed notations are
          \begin{equation}\label{derivationcondensed}
          d\e_i=M_{ik}\e_k
          \end{equation}
     Orthonormality condition means that $\langle\e_i,\e_k\rangle=\delta_{ik}$. Hence
        \begin{equation}
   d\langle\e_i,\e_k\rangle=0=\langle d\e_i,\e_k\rangle+\langle\e_i,d\e_k\rangle=
   \langle  M_{im}\e_m,\e_k\rangle+\langle\e_i,M_{kn}\e_n\rangle=M_{ik}+M_{ki}=0\hbox{\finish}
        \end{equation}
    Much shorter, is not it?


 \subsubsection  {$^*$Gauss condition (structure equations)}
    Derive the relations between $1$-forms $a,b$ and $c$ in derivation formulae.

    Recall that $a,b,c$ are $1$-forms, $\e,f,\n$ are vector valued functions ($0$-forms)
    and  $d\e,d\f,d\n$ are vector valued $1$-forms.
    (We use the simple identity that $ddf=0$ and the fact that for $1$-form $\w\wedge \w=0$.)
     We have from derivation formulae \eqref{derivationformulae1} that
         $$
     d^2\e=0= d(a\f+b\n)=da\f-a\wedge d\f+db\n-b\wedge d\n=
        $$
        $$
        da\, \f-a\wedge (-a \e+c\n)+db\,\n-b\wedge(-b\e-c\f)=
                       $$
                       $$
          (da+b\wedge c)\f+(a\wedge a+b\wedge b)\e+(db-a\wedge c)\n=
          (da+b\wedge c)\f+(db+c\wedge a)\n=0\,.
                      $$
We see that
                \begin{equation}\label{gausscondition1}
                    (da+b\wedge c)\f+(db+c\wedge a)\n=0
                \end{equation}
                Hence components of the left hand side equal to zero:
\begin{equation}\label{gausscondition2}
                    (da+b\wedge c)=0\,\,(db+c\wedge a)=0\,.
                \end{equation}
Analogously
              $$
                       d^2\f=0= d(-a\e+c\n)=-da\e+a\wedge d\e+dc\n-c\wedge d\n=
        $$
        $$
        -da\e+a\wedge (a \f+b\n)+dc\,\n-c\wedge(-b\e-c\f)=
                       $$
                       $$
          (-da+c\wedge b)\e+(dc+a\wedge b)\n=0\,.
              $$


 Hence we come to structure equations:

               \begin{equation}\label{firststructureformula}
               \begin{matrix}
                da+b\wedge c=0\cr
                 db+c\wedge a=0\cr
                 dc+a\wedge a=0\cr
                 \end{matrix}
               \end{equation}


          \subsection {Geometrical meaning of derivation formulae.
         Weingarten operator and second quadratic form in terms of derivation formulae. }




   Let $M$ be a surface in $\E^3$.

          Let ${\e,\f,\n}$ be three vector fields defined on the points of this surface
   such that they form an orthonormal basis at any point, so that the vectors $\e,\f$ are tangent to the surface
   and  the vector $\n$ is orthogonal to the surface.
   Note that in generally these vectors are not coordinate vectors.

     Describe  Riemannian geometry on the surface
     $M$ in terms of this basis and derivation formulae \eqref{derivationformulae1}.




    \m



    {\it Induced Riemannian metric}

    If $G$ is the Riemannian metric induced on the surface $M$ then since $\e,\f$ is orthonormal basis
    at every tangent space $T_\pt M$ then
        \begin{equation}\label{inducriemmetricinnonholonombasis}
        G(\e,\e)=G(\f,\f)=1,\,\,\,G(\e,\f)=G(\f,\e)=0
        \end{equation}
       The matrix of the Riemannian metric in the basis $\{\e,\f\}$ is
     \begin{equation}\label{inducedriemmetricinnonholonombasis2}
        G=
        \begin{pmatrix}
        1& 0\cr
        0& 1\cr
        \end{pmatrix}
     \end{equation}


      \m


    {\it Induced connection}
    Let $\nabla$ be the connection induced by the canonical flat connection on the surface $M$.

    Then according equations \eqref{inducedconnection4} and derivation formulae \eqref{derivationformulae1}
    for every tangent vector $\X$
        \begin{equation}\label{inducedconnectionintermsofnonholonombasis1}
        \nabla_\X \e=\left(\p_\X\e\right)_{\rm tangent}=\left(d\e(\X)\right)_{\rm tangent}=
        \left(a(\X)\f+b(\X)\n\right)_{\rm tangent}=a(\X)\f\,.
        \end{equation}
and
    \begin{equation}\label{inducedconnectionintermsofnonholonombasis2}
        \nabla_\X \f=\left(\p_\X\f\right)_{\rm tangent}=\left(d\f(\X)\right)_{\rm tangent}=
        \left(-a(\X)\e+c(\X)\n\right)_{\rm tangent}=-a(\X)\e\,.
        \end{equation}
 In particular
              \begin{equation}\label{inducedconnectionintermsofnonholonombasis3}
              \begin{matrix}
        \nabla_\e \e=a(\e)\f  &   \nabla_\f \e=a(\f)\f\cr
        \nabla_\e \f=-a(\e)\e  &   \nabla_\f \f=-a(\f)\e\cr
        \end{matrix}
        \end{equation}
We know that the connection $\nabla$ is Levi-Civita connection of the induced Riemannian metric
\eqref{inducedconnectionintermsofnonholonombasis1} (see the subsection 4.2.1)\footnote{In
particular this implies that this is symmetric connection, i.e.
\begin{equation}\label{commutatorofvectorfields}
    \nabla\f\e-\nabla_\e\f-[\f,\e]=a(\f)\f+a(\e)\e-[\f,\e]=0\,.
\end{equation}
}.


{\footnotesize
\m
{\it Second Quadratic form}
   Second quadratic form is  a bilinear symmetric function $A(\X,\Y)$ on tangent vectors
   which is well-defined by the condition  $A(\X,\Y)\n=(\p_\X\Y)_{\rm orthogonal}$
   (see e.g. subsection 6.4 in Appendices.)

  Let $A(\X,\Y)$ be second quadratic form. Then according to
derivation formulae \eqref{derivationformulae1} we have
               $$
  A(\e,\e)=\langle\p_\e\e,\n\rangle=\langle d\e(\e),\n\rangle=\langle a(\e)\f+b(\e)\n,\n\rangle=b(\e)\,,
               $$
         $$
   A(\f,\e)=\langle\p_\f\e,\n\rangle=\langle d\e(\f),\n\rangle=\langle a(\f)\f+b(\f)\n,\n\rangle=b(\f)\,,
         $$
         $$
     A(\e,\f)=\langle\p_\e\f,\n\rangle=\langle d\f(\e),\n\rangle=\langle -a(\e)\f+c(\e)\n,\n\rangle=c(\e)\,,
           $$
           $$
           A(\f,\f)=\langle\p_\f\f,\n\rangle=\langle d\f(\f),\n\rangle=\langle -a(\f)\f+c(\f)\n,\n\rangle=c(\f)\,,
           $$
     The matrix of the second quadratic form in the basis $\{\e,\f\}$ is
     \begin{equation}\label{inducedquadrforminnonholonombasis}
        A=
        \begin{pmatrix}
        A(\e,\e)& A(\f,\e)\cr
        A(\e,\f)& A(\f,\f)\cr
        \end{pmatrix}
        =
        \begin{pmatrix}
        b(\e)&  b(\f)\cr
        c(\e)& c(\f)\cr
        \end{pmatrix}
     \end{equation}
 This is symmetrical matrix (see the subsection 4.3.2):
            \begin{equation}\label{symmericinnonholonom}
 A(\f,\e)=b(\f)=A(\e,\f)=c(\e)\,.
            \end{equation}

\m
}
{\it Weingarten operator}

Let $S$ be Weingarten operator: $S\X=-\p_\X\n$ (see the subsection 6.4 in Appendix, or last year Geometry letures). Then
it follows from derivation formulae that
                  $$
     S\X=-\p_\X\n=-d\n(\X)=-\left(-b(X)\e-c(\X)\f\right)=b(\X)\e+c(\X)f
                  $$
In particular
 \begin{equation*}\label{inducedweingartenoperator}
    S(\e)=b(\e)\e+c(\e)\f\,, S(\f)=b(\f)\e+c(\f)\f
\end{equation*}
and the matrix of the Weingarten operator in the basis $\{\e,\f\}$ is
\begin{equation}\label{inducedweingartenoperator}
S=
\begin{pmatrix}
b(\e)  &c(\e)\cr
b(\f)  &c(\f)\cr
\end{pmatrix}
\end{equation}

{\footnotesize {\bf Remark} According to the condition \eqref{symmericinnonholonom}  the matrix $S$ is symmetrical.
The relations $A=GS, S=G^{-1}A$ for Weingarten operator, Riemannian metric and second quadratic form
are evidently obeyed for matrices of these operators in the basis ${\e,\f}$ where $G=1$, $A=S$.}

\subsubsection{Gaussian and mean curvature in terms of derivation formulae}
Now we are equipped to express Gaussian and mean curvatures in terms of derivation formulae.
  Using \eqref{inducedweingartenoperator} we have for Gaussian curvature
           \begin{equation}\label{gaussiancurvintermsofderformulae}
            K=\det S=b(\e)c(\f)-c(\e)b(\f)=(b\wedge c)(\e,\f)
           \end{equation}
 and for mean curvature
   \begin{equation}\label{meancurvintermsofderformulae}
            H={\rm Tr\,} S=b(\e)+c(\f)
           \end{equation}
 What next? We will  study in more detail formula \eqref{gaussiancurvintermsofderformulae} later.

 Now consider some examples of calculation derivation formulae, Weingarten operator, e.t..c. for
 some examples using derivation formulae.

 \subsection {Examples of calculations of derivation formulae and curvatures for cylinder, cone and sphere }

 Last year  we calculated Weingarten oeprator, second quadratic form and curvatures
 for cylinder, cone and sphere (see also the subsection 6.4 in Appendices.).
 Now we do the same but in terms of derivation formulae.
\m

 {\it Cylinder}

\m



We have to define three vector fields
${\e,\f,\n}$ on the points of the cylinder $\r(h,\varphi)\colon\quad
  \begin{cases}
  x=R\cos\varphi\\
  y=R\sin\varphi\\
  z=h\\
  \end{cases}$
such that they form an orthonormal basis at any point, so that the vectors $\e,\f$ are tangent to the surface
   and  the vector $\n$ is orthogonal to the surface.
  We calculated many times coordinate vector fields $\r_h,\r_\varphi$ and normal unit vector field:
 \begin{equation}\label{surface115}
    \r_h=
  \begin{pmatrix}
        0\\
        0\\
        1\\
   \end{pmatrix}\,,
\quad
  \r_\varphi=\begin{pmatrix}
        -R\sin\varphi\\
        R\cos\varphi\\
          0\\
   \end{pmatrix}\,,\quad
      \n=\begin{pmatrix}
        \cos \varphi\cr
        \sin\varphi\cr
        0\cr
   \end{pmatrix}\,.
\end{equation}

Vectors $\r_h,\r_\varphi$ and $\n$ are orthogonal to each other but not all of them have unit length. One can choose
           \begin{equation}\label{nonholonombasisfor cylinder}
        \e=\r_h=  \begin{pmatrix}
        0\cr
        0\cr
        1\cr
   \end{pmatrix},
   \f={\r_\varphi\over R}=\begin{pmatrix}
        -\sin\varphi\cr
        \cos\varphi\cr
          0\cr
   \end{pmatrix},\,\n=\begin{pmatrix}
        \cos \varphi\cr
        \sin\varphi\cr
        0\cr
   \end{pmatrix}
           \end{equation}
   These vectors form an orthonormal basis and $\e,\f$  form an orthonormal basis in tangent space.

   Derive for this basis derivation formulae \eqref{derivationformulae1}. For vector fields $\e,\f,\n$
   in \eqref{nonholonombasisfor cylinder} we have
                   $$
             d\e=0, d\f=d\begin{pmatrix}
        -\sin\varphi\cr
        \cos\varphi\cr
          0\cr
   \end{pmatrix}=\begin{pmatrix}
        -\cos\varphi\cr
        -\sin\varphi\cr
          0\cr
   \end{pmatrix}d\varphi=-\n d\varphi,
          $$
          $$
   d\n=d\begin{pmatrix}
        \cos \varphi\cr
        \sin\varphi\cr
        0\cr
   \end{pmatrix}=
   \begin{pmatrix}
        -\sin\varphi \cr
        \cos\varphi\cr
        0\cr
   \end{pmatrix}=\f d\varphi,
                   $$
   i.e.
   \begin{equation}\label{derivationformulaefor cylinder}
                    d\begin{pmatrix}
                    \e\cr\f\cr\n\cr
                    \end{pmatrix}=
                    \begin{pmatrix}
                    0&a&b\cr -a&0&c\cr -b&-c&0\cr
                    \end{pmatrix}
                 \begin{pmatrix}
                    \e\cr\f\cr\n\cr
                    \end{pmatrix}=
                     \begin{pmatrix}
                    0&0&0\cr 0&0&-d\varphi\cr 0&d\varphi&0\cr
                    \end{pmatrix}
                 \begin{pmatrix}
                    \e\cr\f\cr\n\cr
                    \end{pmatrix}\,,
                   \end{equation}
  i.e. in derivation formulae $a=b=0$, $c=-d\varphi$.

  The matrix of Weingarten operator in the basis  $\{\e,\f\}$ is
             $$
     S=
\begin{pmatrix}
b(\e)  &c(\e)\cr
b(\f)  &c(\f)\cr
\end{pmatrix}=
\begin{pmatrix}
0  &-d\varphi(\e)\cr
0  &-d\varphi(\f)\cr
\end{pmatrix}=
\begin{pmatrix}
0  &0\cr
0  &-{1\over R}\cr
\end{pmatrix}
       $$
    According to  \eqref{gaussiancurvintermsofderformulae}  and\eqref{meancurvintermsofderformulae}
                Gaussian curvature $K=b(\e)c(\f)-b(\e)c(\f)=0$ and
                mean curvature
                 $$
    H=b(\e)+c(\f)=-d\varphi(\f)=-d\varphi\left({\r_\varphi\over R}\right)=-{1\over R}
                 $$
  (Compare with calculations in the subsection 4.3.4)


\bigskip

{\it Cone}


For  cone:
          $$
  \r(h,\varphi)\colon\quad
  \begin{cases}
  x=kh\cos\varphi\\
  y=kh\sin\varphi\\
  z=h\\
  \end{cases}\,   \,,
   $$
        $$
        \r_h=
  \begin{pmatrix}
        k\cos\varphi\cr
        k\sin\varphi\cr
        1\cr
   \end{pmatrix},\quad
  \r_\varphi=\begin{pmatrix}
        -kh\sin\varphi\cr
        kh\cos\varphi\cr
          0\cr
   \end{pmatrix}\,,\quad
   \n=
   {1\over {\sqrt {1+k^2}}}
      \begin{pmatrix}
        \cos \varphi\cr
        \sin\varphi\cr
        -k\cr
   \end{pmatrix}
            $$
Tangent vectors  $\r_h,\r_\varphi$ are orthogonal to each other.
The length of the vector $\r_h$ equals to $\sqrt {1+k^2}$ and  the length of the vector $\r_\varphi$
equals to $kh$.
Hence we can choose orthonormal basis $\{\e,\f,\n\}$ such that vectors $\e,\f$ are unit vectors in the
directions of the vectors  $\r_h,\r_\varphi$:
               $$
               \e= {\r_h\over \sqrt {1+k^2}}=
               {1\over \sqrt {1+k^2}}
  \begin{pmatrix}
        k\cos\varphi\cr
        k\sin\varphi\cr
        1\cr
   \end{pmatrix},\,
  \f={\r_\varphi\over hk}=\begin{pmatrix}
        -\sin\varphi\cr
        \cos\varphi\cr
          0\cr
   \end{pmatrix},\,
   \n=
   {1\over {\sqrt {1+k^2}}}
      \begin{pmatrix}
        \cos \varphi\cr
        \sin\varphi\cr
        -k\cr
   \end{pmatrix}
               $$
Calculate $d\e,d\f$ and $d\n$:
           $$
      d\e=d\begin{pmatrix}
        k\cos\varphi\cr
        k\sin\varphi\cr
        1\cr
   \end{pmatrix}={kd\varphi\over \sqrt{1+k^2}}\begin{pmatrix}
        -\sin\varphi\cr
        \cos\varphi\cr
        0\cr
   \end{pmatrix}={ kd\varphi\over \sqrt{1+k^2}}\f,
           $$
           $$
   d\f=d\begin{pmatrix}
        -\sin\varphi\cr
        \cos\varphi\cr
          0\cr
   \end{pmatrix}=
     \begin{pmatrix}
        -\cos\varphi\cr
        -\sin\varphi\cr
          0\cr
   \end{pmatrix}d\varphi=
        $$
        $$
   {-k\over 1+k^2}\begin{pmatrix}
        k\cos\varphi\cr
        k\sin\varphi\cr
        1\cr
   \end{pmatrix}d\varphi-{d\varphi\over 1+k^2}
             \begin{pmatrix}
        \cos \varphi\cr
        \sin\varphi\cr
        -k\cr
   \end{pmatrix}={-kd\varphi\over \sqrt{1+k^2}}\e-{d\varphi\over \sqrt {1+k^2}}\n\,,
     $$
and
       $$
   d\n={1\over \sqrt {1+k^2}}d\begin{pmatrix}
        \cos \varphi\cr
        \sin\varphi\cr
        -k\cr
   \end{pmatrix}=
   {d\varphi\over \sqrt {1+k^2}}
   \begin{pmatrix}
        -\sin \varphi\cr
        \cos\varphi\cr
        0\cr
   \end{pmatrix}\,.
       $$
We come to
   \begin{equation}\label{derivationformulaefor cone}
                    d\begin{pmatrix}
                    \e\cr\f\cr\n\cr
                    \end{pmatrix}=
                    \begin{pmatrix}
                    0&a&b\cr -a&0&c\cr -b&-c&0\cr
                    \end{pmatrix}
                 \begin{pmatrix}
                    \e\cr\f\cr\n\cr
                    \end{pmatrix}=
                     \begin{pmatrix}
                    0&{kd\varphi\over \sqrt {1+k^2}}&0\cr
                     -{kd\varphi\over \sqrt {1+k^2}}&0&{-d\varphi\over \sqrt {1+k^2}}\cr
                      0&{d\varphi\over \sqrt {1+k^2}}&0\cr
                    \end{pmatrix}
                 \begin{pmatrix}
                    \e\cr\f\cr\n\cr
                    \end{pmatrix}\,,
                   \end{equation}
  i.e. in derivation formulae for $1$-forms $a={kd\varphi\over \sqrt {1+k^2}}$, $b=0$ and
  and $c=-{-d\varphi\over \sqrt {1+k^2}}$.

  The matrix of Weingarten operator in the basis  $\{\e,\f\}$ is
             $$
     S=
\begin{pmatrix}
b(\e)  &c(\e)\cr
b(\f)  &c(\f)\cr
\end{pmatrix}=
S=
\begin{pmatrix}
0  &{-d\varphi (\e)\over \sqrt {1+k^2}}\cr
0  &{-d\varphi (\f)\over \sqrt {1+k^2}}\cr
\end{pmatrix}=
\begin{pmatrix}
0  &0\cr
0  &{-1\over kh \sqrt {1+k^2}}\cr
\end{pmatrix}\,.
       $$
since $d\varphi(\f)=d\varphi\left({\r_\varphi\over kh}\right)={1\over kh}d\varphi(\p_\varphi)={1\over kh}$.


According to \eqref{gaussiancurvintermsofderformulae}, \eqref{meancurvintermsofderformulae}
                Gaussian curvature  $$
                K=b(\e)c(\f)-b(\e)c(\f)=0
                $$
and
                mean curvature
                 $$
    H=b(\e)+c(\f)=-d\varphi(\f)=-d\varphi\left({\r_\varphi\over R}\right)=-{1\over R}
                 $$
  {\bf Remark} Equipped by the properties of derivation formulae we do not need to
  calculate $d\f$. The calculation of $d\e$ and $d\n$ and the property that the matrix
  in derivation formulae is antisymmetric gives us the answer for $d\f$.



\bigskip




{\it Sphere}


\medskip


For sphere
\begin{equation}\label{surfacesphere11}
  \r(\theta,\varphi)\colon\quad
  \begin{cases}
  x=R\sin\theta\cos\varphi\\
  y=R\sin\theta\sin\varphi\\
  z=R\cos\theta\\
  \end{cases}
\end{equation}
    $$
  \r_\theta(\theta,\varphi)={\p \r\over \p \theta}=
  \begin{pmatrix}
        R\cos\theta\cos\varphi\cr
        R\cos\theta\sin\varphi\cr
        -R\sin\theta\cr
   \end{pmatrix}\,,
\quad
  \r_\varphi(\theta,\varphi)={\p \r\over \p \varphi}=
         \begin{pmatrix}
        -R\sin\theta\sin\varphi\cr
        R\sin\theta\cos\varphi\cr
          0\cr
   \end{pmatrix},
     $$
     $$
   \n(\theta,\varphi)={\r\over R}=
   \begin{pmatrix}
    \sin\theta\cos\varphi\cr
     \sin\theta\sin\varphi\cr
      \cos\theta\cr
   \end{pmatrix}.
  $$
Tangent vectors  $\r_\theta,\r_\varphi$ are orthogonal to each other.
The length of the vector $\r_\theta$ equals to $R$ and the
length of the vector $\r_\varphi$ equals to $R\sin\theta$.
Hence we can choose orthonormal basis $\{\e,\f,\n\}$ such that vectors $\e,\f$ are unit vectors in the
directions of the vectors  $\r_\theta,\r_\varphi$:
         $$
    \e(\theta,\varphi)={\r_\theta\over R}=
         \begin{pmatrix}
        \cos\theta\cos\varphi\cr
        \cos\theta\sin\varphi\cr
        -\sin\theta\cr
   \end{pmatrix},\,
   \f(\theta,\varphi)={\r_\varphi\over R\sin\theta}=
        \begin{pmatrix}
        -\sin\varphi\cr
        \cos\varphi\cr
          0\cr
   \end{pmatrix},\,
     \n(\theta,\varphi)={\r\over R}=
   \begin{pmatrix}
    \sin\theta\cos\varphi\cr
     \sin\theta\sin\varphi\cr
      \cos\theta\cr
   \end{pmatrix}.
         $$
Calculate $d\e, d\f$ and $d\n$:
            $$
        d\e=d
        \begin{pmatrix}
        \cos\theta\cos\varphi\cr
        \cos\theta\sin\varphi\cr
        -\sin\theta\cr
   \end{pmatrix}=
           $$
           $$
\begin{pmatrix}
        -\cos\theta\sin\varphi\cr
        \cos\theta\cos\varphi\cr
           0\cr
   \end{pmatrix}d\varphi-\begin{pmatrix}
        \cos\theta\cos\varphi\cr
        \cos\theta\sin\varphi\cr
        -\sin\theta\cr
   \end{pmatrix}=\cos\theta d\varphi \f-d\theta \n,
            $$
             $$
            d\f=d
            \begin{pmatrix}
        -\sin\varphi\cr
        \cos\varphi\cr
          0\cr
   \end{pmatrix}=
              -
   \begin{pmatrix}
        \cos\varphi\cr
        \sin\varphi\cr
          0\cr
   \end{pmatrix}d\varphi=
            $$
            $$
       -\cos\theta d\varphi \begin{pmatrix}
        \cos\theta\cos\varphi\cr
        \cos\theta\sin\varphi\cr
        -\sin\theta\cr
   \end{pmatrix}-\sin\theta d\varphi
   \begin{pmatrix}
    \sin\theta\cos\varphi\cr
     \sin\theta\sin\varphi\cr
      \cos\theta\cr
   \end{pmatrix}=
   -\cos\theta d\varphi \e-\sin\theta d\varphi \n\,,
            $$
             $$
             d\n=
   d\begin{pmatrix}
    \sin\theta\cos\varphi\cr
     \sin\theta\sin\varphi\cr
      \cos\theta\cr
   \end{pmatrix}=
   \begin{pmatrix}
    \cos\theta\cos\varphi\cr
     \cos\theta\sin\varphi\cr
      -\sin\theta\cr
   \end{pmatrix}d\theta+
   \begin{pmatrix}
    -\sin\theta\sin\varphi\cr
     \sin\theta\cos\varphi\cr
        0\cr
   \end{pmatrix}d\varphi
        $$
        $$
   =d\theta \e+\sin\theta d\varphi \f\,.
        $$
   i.e.
   \begin{equation}\label{derivationformulaefor cylinder}
                    d\begin{pmatrix}
                    \e\cr\f\cr\n\cr
                    \end{pmatrix}=
                    \begin{pmatrix}
                    0&a&b\cr -a&0&c\cr -b&-c&0\cr
                    \end{pmatrix}
                 \begin{pmatrix}
                    \e\cr\f\cr\n\cr
                    \end{pmatrix}=
                     \begin{pmatrix}
                    0&\cos\theta d\varphi& -d\theta\cr
                    -\cos\theta d\varphi&0&-\sin\theta d\varphi\cr
                     d\theta&\sin\theta d\varphi&0\cr
                    \end{pmatrix}
                 \begin{pmatrix}
                    \e\cr\f\cr\n\cr
                    \end{pmatrix}\,,
                   \end{equation}
  i.e. in derivation formulae $a=\cos\theta d\varphi$,
  $b=-d\theta$, $c=-\sin\theta d\varphi$.

  The matrix of Weingarten operator in the basis  $\{\e,\f\}$ is
             $$
     S=
\begin{pmatrix}
b(\e)  &c(\e)\cr
b(\f)  &c(\f)\cr
\end{pmatrix}=
\begin{pmatrix}
-d\theta (\e) &-\sin\theta d\varphi (\e)\cr
-d\theta (\f)  &-\sin\theta d\varphi(\f)\cr
\end{pmatrix}=
\begin{pmatrix}
{-1\over R}  &0\cr
0  &-{1\over R}\cr
\end{pmatrix}
       $$
       since $d\theta(\e)=d\theta\left({\p_\theta\over R}\right)={1\over R}d\theta(\p_\theta)={1\over R}$,
       $\,d\varphi(\e)=d\varphi\left({\p_\theta\over R}\right)={1\over R}d\varphi(\p_\theta)=0$.

    According to  \eqref{gaussiancurvintermsofderformulae}  and\eqref{meancurvintermsofderformulae}
                Gaussian curvature $$
                K=b(\e)c(\f)-b(\e)c(\f)={1\over R^2}
                                    $$
                                    and
                mean curvature
                 $$
    H=b(\e)+c(\f)=-{2\over R}
                 $$
    Notice that for calculation of Weingarten operator and curvatures we used only
    $1$-forms $b$ and $c$, i.e. the derivation equation for $d\n$, ($d\n=d\theta \e+\sin\theta d\varphi \f$).
  (Compare with calculations in the subsection 4.3.4)





 Mean curvature is define up to a sign. If we change $\n\to-\n$ mean curvature $H\to {1\over R}$ and Gaussian curvature
will not change.


We see that for the sphere  Gaussian curvature is not equal to zero, whilst for cylinder and cone Gaussian curvature
equals to zero.


  {\bf Remark} The same remark as for cone: equipped by the properties of derivation formulae we do not need to
  calculate $d\f$. The calculation of $d\e$ and $d\n$ and the property that the matrix
  in derivation formulae is antisymmetric gives us the answer for $d\f$.






\subsection { $^*$Proof of the Theorem of parallel transport along closed curve.}

We are ready now to prove the Theorem.  Recall that the Theorem states following:

If $C$ is a closed curve on a surface $M$ such that $C$ is a boundary of a compact oriented domain $D\subset M$,
then during the  parallel transport of an arbitrary tangent vector  along the closed curve $C$
the vector rotates through the angle
\begin{equation}\label{theoremofrotationonangle4}
\Delta\Phi=\angle\left({\X, \R_C\X}\right)=\int_D K d\sigma\,,
             \end{equation}
where $K$ is the Gaussian curvature and $d\sigma=\sqrt {\det g}dudv$ is the area element induced by the
Riemannian metric on the surface $M$, i.e.  $d\sigma=\sqrt {\det g}dudv$.

(see \eqref{theoremofrotationonangle}.


Recall that for derivation formulae
\eqref{derivationformulae1} we obtained structure equations
                  \begin{equation}\label{firststructureformula4}
               \begin{matrix}
                da+b\wedge c=0\cr
                 db+c\wedge a=0\cr
                 dc+a\wedge a=0\cr
                 \end{matrix}
               \end{equation}

We need to use only one of these equations, the equation
             \begin{equation}\label{gausscondition}
                da+b\wedge c=0\,.
             \end{equation}
This condition sometimes is called {\it Gau\ss \,\,condition}.



Let as always  $\{\e,\f,\n\}$ be an orthonormal basis in $T_{\pt}\E^3$ at every point of surface  $\pt\in M$
such that $\{\e,\f\}$ is an orthonormal basis in $T_{\pt} M$ at every point of surface  $\pt\in M$.
Then the  Gau\ss\, condition \eqref{gausscondition} and equation \eqref{gaussiancurvintermsofderformulae} mean
 that  for Gaussian curvature on the surface $M$ can be expressed through the $2$-form $da$ and base vectors $\{\e,\f\}$:
              \begin{equation}\label{gaussiancurvatureintermsof form a}
                K=b\wedge c(\e,\f)=-da(\e,\f)
              \end{equation}
  We use this formula to prove the Theorem.



             Now calculate the parallel transport of an arbitrary tangent vector over the closed curve $C$
             on the surface $M$.



 Let $\r=\r(u,v)=\r(u\a)$ ($\a=1,2$, $(u,v)=(u^1,v^1)$) be an equation of the surface $M$.

Let $u^\a=u^\a(t)$ ($\a=1,2$) be the equation of the curve $C$.
Let $\X(t)$ be the parallel transport of vector field along the closed curve $C$,
i.e. $\X(t)$ is tangent to the surface $M$ at the point $u(t)$ of the curve $C$ and
vector field $\X(t)$ is covariantly constant along the curve:
      $$
    {\nabla \X(t)\over dt}=0
      $$
     To write this equation in components we usually  expanded the vector field
in the coordinate basis $\{\r_u=\p_u,\r_v=\p_v\}$ and used Christoffel symbols of the connection
  $\Gamma^\a_{\beta\gamma}\colon \nabla_\beta\p_\gamma=\Gamma^\a_{\beta\gamma}\p_\a$.

  Now we will do it in different way: {\it instead coordinate basis $\{\r_u=\p_u,\r_v=\p_v\}$ we will use
  the basis $\{\e,\f\}$.}   In the subsection 3.4.4 we obtained that the connection $\nabla$ has the following appearance
  in this basis
  \begin{equation}\label{indconforrotation}
    \nabla_\v\e=a(\v)\f,\,, \nabla_\v\f=-a(\v)\e
  \end{equation}
  (see the equations \eqref{inducedconnectionintermsofnonholonombasis1} and \eqref{inducedconnectionintermsofnonholonombasis2})

Let        $$
\X=\X(u(t))=X^1(t)\e(u(t))+X^2(t)\f(u(t))
           $$
Lbe an expansion of tangent vector field $\X(t)$ over basis $\{\e,\f\}$.
Let $\v$ be velocity vector of the curve $C$.
  Then the equation of parallel transport ${\nabla \X(t)\over dt}=0$
                 will have the following appearance:
                      $$
              {\nabla \X(t)\over dt}=0=\nabla_\v \left(X^1(t)\e(u(t))+X^2(t)\f(u(t))\right)=
                      $$
                      $$
                      {dX^1(t)\over dt}\e(u(t))+X^1(t)\nabla_\v\e(u(t))+
                      {dX^2(t)\over dt}\f(u(t))+X^2(t)\nabla_\v\f(u(t))=
                      $$
                      $$
                      {dX^1(t)\over dt}\e(u(t))+X^1(t)a(\v)\f(u(t))+
                      {dX^2(t)\over dt}\f(u(t))-X^2(t)a(\v)\e(u(t))=
                      $$
              $$
          \left({dX^1(t)\over dt}-X^2(t)a(\v)\right)\e(u(t))+
           \left({dX^2(t)\over dt}+X^1(t)a(\v)\right)\f(u(t))=0.
              $$
Thus we come to equation:
              $$
             \begin{cases}
             \dot X^1(t)-a(\v(t))X^2=0\cr
             \dot X^2(t)+a(\v(t))X^1=0\cr
             \end{cases}
              $$
There are many ways to solve this equation. It is very convenient to consider complex variable
             $$
           Z(t)=X^1(t)+iX^2(t)
             $$
We see that
             $$
         \dot Z(t)=\dot X^1(t)+i\dot X^2(t)=a(\v(t))X^2-ia(\v(t)X^1=-ia(\v)Z(t),
                 $$
i.e.
   \begin{equation}\label{complexrotation}
    {dZ(t)\over dt}=-ia(\v(t))Z(t)
   \end{equation}
The solution of this equation is:
           \begin{equation}\label{complexrotation}
      Z(t)=Z(0)e^{-i\int_0^t a(\v(\tau))d\tau}
   \end{equation}
   Calculate $\int_0^{t_1} a(\v(\tau))d\tau$ for closed curve $u(0)=u(t_1)$. Due to Stokes Theorem:
            $$
            \int_0^{t_1} a(\v(t))dt=\int_C a=\int_D da
            $$
   Hence using Gauss condition \eqref{gausscondition} we see that
      $$
      \int_0^{t_1} a(\v(t))dt=\int_C a=\int_D da=-\int_D b\wedge c
      $$

{\bf Claim}
            \begin{equation}\label{claimoncurvature}
            \int_D b\wedge c=-\int_D da=\int K d\sigma\,.
            \end{equation}

Theorem follows from this claim:
           \begin{equation}\label{complexrotation2}
      Z(t_1)=Z(0)e^{-i\int_C a}=Z(0)e^{i\int_D b\wedge C}
   \end{equation}
Denote the integral ${i\int_D b\wedge C}$ by $\Delta \Phi\colon \Delta \Phi={i\int_D b\wedge C}$. We have
           \begin{equation}\label{complexrotation2}
      Z(t_1)=X^1(t_1)+iX^2(t_1)=\left(X^1(0)+iX^2(0)\right)e^{i\Delta\Phi}=
   \end{equation}

\medskip

 It remains to prove the claim.  The induced volume form $d\sigma$ is $2$-form. Its value
 on two orthogonal unit vector $\e,\f$ equals to $1$:
              \begin{equation}\label{valueonorthonormalvectors}
                d\sigma (\e,\f)=1
              \end{equation}
(In coordinates $u,v$ volume form $d\sigma=\sqrt{\det g}du\wedge dv$).

The value of the form $b\wedge c$ on vectors $\{\e,\f\}$ equals to Gaussian curvature according to \eqref{gaussiancurvatureintermsof form a}.
We see that
             $$
        b\wedge c(\e,\f)=-da(\e,\f)=K d\sigma (\e,\f)
             $$
Hence $2$-forms $b\wedge c$, $-da$ and volume form $d\sigma$ coincide. Thus we prove \eqref{claimoncurvature}.


\section {Curvtature tensor}

\subsection {Definition}

Let $\X,\Y,\Z$ be an arbitrary vector fields on the manifold equipped with affine connection  $\nabla$.

Consider the following operation which assings to the vector fields $\X,\Y$ and $\Z$ the new vector field.
\begin{equation}\label{operationdefinig the tensor}
   {\cal R}(\X,\Y)\Z=
    \left(
    \nabla_\X\nabla_\Y-\nabla_\Y\nabla_\X-
   \nabla_{[\X,\Y]}
    \right)\Z
\end{equation}
This operation is obviously linear over the scalar coefficients and it is $C^{\infty}(M)$
with respect to vector fields $\X,\Y\,\Z$, i.e. for an arbitrary functions
$f,g,h$
        \begin{equation}\label{propertiesoflienarity}
            {\cal R}(f\X,g\Y)(h\Z)=fgh{\cal R}(\X,\Y)\Z,
        \end{equation}
        i.e. it defines
the tensor field of the type $\begin{pmatrix}1\cr 3\end{pmatrix}$: If $\X=X^i\p_i, \X=X^i\p_i,\X=X^i\p_i$
then according to \eqref{propertiesoflienarity}
           $$
        {\cal R}(\X,\Y)\Z={\cal R}(X^m\p_m,Y^n\p_n)(Z^r\p_r)=Z^rR^i_{rmn}X^mY^n
           $$
where we denote by $R^i_{rmn}$ the components of the tensor $\cal R$ in the coordinate basis ${\p_i}$
\begin{equation}\label{componentsofcurvaturetensor}
    R^i_{\,\,rmn}\p_i={\cal R}(\p_m,\p_n)\p_r
\end{equation}
Express components of the curvature tensor in terms of Christoffel symbols of the connection.
If $\nabla_m\p_n=\Gamma_{mn}^r\p_r$ then according to the \eqref{operationdefinig the tensor} we have:
                $$
        R^i_{\,\,rmn}\p_i={\cal R}(\p_m,\p_n)\p_r=\nabla_{\p_m}\nabla_{\p_n}\p_r-\nabla_{\p_n}\nabla_{\p_m}\p_r,
                    $$
                    $$
                    R^i_{\,\,rmn}=
              \nabla_{\p_m}\left(\Gamma_{nr}^p\p_p\right)-\nabla_{\p_n}\left(\Gamma_{mr}^p\p_p\right)=
                $$
                \begin{equation}\label{curvatureincomponents}
                \p_m\Gamma_{nr}^i+\Gamma_{mp}^i\Gamma_{nr}^p-\left(m\leftrightarrow n\right)
                \end{equation}


 The proof of the property \eqref{propertiesoflienarity} can be given just  by straightforward calculations:
     Consider e.g. the case $f=g=1$, then

        $$
         {\cal R}(\X,\Y)(h\Z)=
    \nabla_{\X}\nabla_\Y(h\Z)-\nabla_{\Y}\nabla_\X(h\Z)-
   \nabla_{[\X,\Y]}(h\Z)=
        $$
        $$
     \nabla_\X\left(\p_\Y h\Z+h\nabla_\Y\Z\right)-\nabla_\Y\left(\p_\X h\Z+h\nabla_\X\Z\right)-
     \p_{[\X,\Y]}h \Z-h\nabla_{[\X,\Y]}\Z=
        $$
        $$
    \p_\X\p_\Y h\Z+\p_\Y h\nabla_\X\Z+
    \p_\X h\nabla_\Y\Z+
    h\nabla_\X\nabla_\Y \Z-
        $$
        $$
        \p_\Y\p_\X h\Z-\p_\X h\nabla_\Y\Z-
    \p_\Y h\nabla_\X\Z+
    h\nabla_\Y\nabla_\X \Z-
        $$
        $$
      \p_{[\X,\Y]}h \Z-h\nabla_{[\X,\Y]}\Z=
        $$
        $$
    h\left[\nabla_{\X}\nabla_\Y\Z-\nabla_{\Y}\nabla_\X\Z)-
   \nabla_{[\X,\Y]}\Z\right]+\left[\p_\X\p_\Y h-\p_\Y\p_\X h-\p_{[\X,\Y]h}\right]\Z=
        $$
        $$
h\nabla_{\X}\nabla_\Y\Z-\nabla_{\Y}\nabla_\X\Z)-
   \nabla_{[\X,\Y]}\Z=h{\cal R}(\X,\Y)\Z\,,
        $$
   since $\p_\X\p_\Y h-\p_\Y\p_\X h-\p_{[\X,\Y]}h=0$.



\subsubsection {Properties of curvature tensor}

  Tensor $R^i_{\,\,kmn}$ is expressed trough derivatives of Christoffel symbols.
  In spite this fact it is is much more "pleasant" object than Christoffel symbols, since the latter is not the tensor.

 It follows from the definition that the tensor $R^i_{kmn}$ is antisymmetrical with respect
to indices $m,n$:
    \begin{equation}\label{antisymmetricitycurvature1}
      R^i_{kmn}=-R^i_{knm}\,.
    \end{equation}



 {\footnotesize One can prove that for symmetric connection this tensor obeys the following identities:
              \begin{equation}\label{ciclicity}
                R^i_{\,\,kmn}+R^i_{mnk}+R^i_{nkm}=0\,,
                    \end{equation}

}


 The curvature tensor corresponding to Levi-Civita connection obeys also another identities too
 (see the next subsection.)

   We know well that
   If Christoffel symbols vanish in a vicinity of a given point $\pt$ in some chosen  coordinate system
   then in general Christoffel symbols do not vanish in arbitrary coordinate systems.
   E.g. Christoffel symbols of canonical flat connection in $\E^2$
   vanish in Cartesian coordinates but do not vanish in polar coordinates. This unpleasant property
   of Christoffel symbols is due to the fact that Christoffel symbols do not form a tensor.




  In particular if a tensor vanishes in some coordinate system, then it vanishes in arbitrary coordinate system too.
    This implies
         very simple but important

       {\bf Proposition}
     {\it If curvature tensor $R^i_{\,\,kmn}$ vanishes in some coordinate  system, then it
      vanishes in arbitrary coordinate systems.}

\m

   We see that if Christoffel symbols vanish in a vicinity of a given point
   $\pt$ in some chosen  coordinate system then it Riemannian curvature tensor vanishes
   at the point $\pt$ (see the formula \eqref{curvatureincomponents}) and hence it vanishes
   at the point $\pt$ in arbitrary coordinate system.


   \subsection {Riemann curvature tensor of Riemannian manifolds.}



   Let $M$ be Riemannian manifold equipped  with Riemannian metric $G$

    In this section we will consider curvature tensor of Levi-Civita connection $\nabla$ of Riemannian metric $G$.

    The curvature tensor for Levi-Civita connection will be called later Riemann curvature tensor, or Riemann tensor.

  Using Riemannian metric one can consider Riemann tensor with all low indices
     \begin{equation}\label{lowindices}
    R_{ikmn}=g_{ij}R^j_{\,\,kmn}
\end{equation}


 Due to identities \eqref{antisymmetricitycurvature1} and \eqref{ciclicity}
 for curvature tensor  Riemann tensor obeys  the following identities:
\begin{equation}\label{identityforriemantensor1}
  R_{ikmn}=-R_{iknm}\,\,\,,\quad R_{ikmn}+R_{imnk}+R_{inkm}=0
\end{equation}

   Riemann curvature tensor which is curvature tensor for Levi-Civita connection obeys also the following identities:
            \begin{equation}\label{identityforriemantensor2}
R_{ikmn}=-R_{kimn}\,\,\,,\quad R_{ikmn}=R_{mnki}\,.
\end{equation}

These condition lead to the fact that for $2$-dimensional Riemannian manifold the Riemann curvature tensor
of Levi-Civita connection has essentially only one non-vanishing component: all components
vanish or equal to component $R_{1212}$up to a sign.
Indeed consider for $2$-dimensional Riemannian manifold Riemann tensor $R_{ikmn}$, where
$i,k,m,n=1,2$.  Since antisymmetricity with respect to third and fourth indices
($R_{ikmn}=-R_{iknm}$), $R_{ik11}=R_{ik22}=0$ and $R_{ik12}=-R_{ik21}$.
The same for first and second indices:
 since antisymmetricity with respect to the the first  and second indices
($R_{12mn}=-R_{21mn}$), $R_{11mn}=R_{22mn}=0$ and $R_{12mn}=-R_{ik21}$.
If we denote $R_{1212}=a$ then
                  $$
R_{1212}=R_{2121}=a, R_{1221}=R_{2112}=-a
                  $$
                  and all other components vanish.



  For Riemann tensor one can consider Ricci tensor,
      \begin{equation}\label{Riccitensor}
      R_{mn}=R^i_{min}
\end{equation}
which is symmetrical tensor:  $R_{mn}=R_{nm}$.

One can consider scalar curvature:
                        \begin{equation}\label{scalarcurvature}
      R=R^i_{\,kin}g^{kn}=g^{kn}R_{kn}
\end{equation}
where $g^{kn}$ is Riemannian metric with indices above (the matrix $||g^{ik}||$ is inverse to the matrix
 $||g_{il}||$).


Ricci tensor and scalar curvature play essential role for formulation of Einstein gravity equations.
In particular the space without matter the Einsten equations have the following form:
                \begin{equation}
                  R_{ik}-{1\over R}g_{ik}=0\,.
                \end{equation}

{\footnotesize
\subsection {$^\dagger$Curvature of surfaces in $\E^3$.. {\it Theorema Egregium} again}

  Express Riemannian curvature of surfaces in $\E^3$ in terms of derivation formulae \eqref{derivationformulae1}.

   Consider derivation formulae for the orthonormal basis $\{\e,\f,\n,\}$ adjusted to the surface $M$:
                      \begin{equation}\label{derivationformulae1}
                    d\begin{pmatrix}
                    \e\cr\f\cr\n\cr
                    \end{pmatrix}=
                    \begin{pmatrix}
                    0&a&b\cr -a&0&c\cr -b&-c&0\cr
                    \end{pmatrix}
                 \begin{pmatrix}
                    \e\cr\f\cr\n\cr
                    \end{pmatrix}\,,
                       \end{equation}

where as usual $\e,\f,\n$ vector fields of unit length which are  orthogonal
to each other and $\n$ is orthogonal to the surface $M$.  As we know the induced
connection on the surface $M$ is defined by the formulae \eqref{inducedconnectionintermsofnonholonombasis1}
and \eqref{inducedconnectionintermsofnonholonombasis2}:
   \begin{equation}\label{inducedconnectionintermsofnonholonombasis11}
        \nabla_\Y \e=\left(d\e(\Y)\right)_{\rm tangent}=a(\X)\f\,,
        \nabla_\Y \f=\left(d\f(\Y)\right)_{\rm tangent}=-a(\X)\e\,,
        \end{equation}

 According to the definition of curvature calculate
         $$
 R(\e,\f)\e=\nabla_{\e}\nabla_{\f}\e-\nabla_{\f}\nabla_{\e}\e-\nabla_{[\e,\f]}\e\,.
         $$
Note that since the induced connection is symmetrical connection then:
        \begin{equation}\label{commutatorofnonholonombasis}
 \nabla_\e \f-\nabla_\f \e- [\e,\f]=0\,.
\end{equation}
hence due to \eqref{inducedconnectionintermsofnonholonombasis11}

       \begin{equation}\label{commutatorofnonholonombasis}
 [\e,\f]=\nabla_\e \f-\nabla_\f \e=a(\e)\e+a(\f)\f
\end{equation}
Thus we see that  $R(\e,\f)\e=$
       $$
 \nabla_{\e}\nabla_{\f}\e-\nabla_{\f}\nabla_{\e}\e-\nabla_{[\e,\f]}\e=
       \nabla_\e\left(a(\f)\f\right)-\nabla_\f\left(a(\e)\f\right)-\nabla_{a(\e)\e+a(\f)\f}\e=
       $$
       $$
  \p_{\e}a(\f)\f+a(\f)\nabla_\e\f-\p_{\f}a(\e)\f-a(\e)\nabla_\f\f-a(\e)\nabla_\e\e-a(\f)\nabla_\f\e=
       $$
       $$
  \p_{\e}a(\f)\f-a(\f)a(\e)\e-\p_{\f}a(\e)\f+a(\e)a(\f)\e-a(\e)a(\e)\f-a(\f)a(\f)\f=
       $$
       $$
       \left[\p_{\e}a(\f)\f-\p_{\f}a(\e)\f-a\left((\e)\e-a(\f)\f\right)\right]\f=
       da(\e,\f)\f\,.
       $$
  Recall that we established in \ref{gaussiancurvatureintermsof form a} that for Gaussian curvature $K$
          $$
       K=b\wedge c(\e,\f)=-da(\e,\f)
          $$
       Hence we come to the relation:
       \begin{equation}\label{relationbetweengausscurvatureand riemtensor}
        R(\e,\f)\e=da(\e,\f)=-K\f\,.
       \end{equation}
       This means that
       $$
         R^2_{112}=-K
       $$
(in the basis ${\e,\f}$), i.e.
 the scalar curvature
   $$
R=2R_{1212}=2K
   $$
 We come to fundamental relation which claims that the Gaussian curvature  (the magnitude defined in terms of
 External observer) equals to the scalar curvature, the magnitude defined in terms of Internal observer.
   This gives us the straightforward proof of Theorema Egregium.

   Below we will give the proof of Theorema Egregium by straightforward calcultaions.
}

   \subsection {Relation between Gaussian curvature and Riemann curvature tensor.
   Straightforward proof of {\sl Theorema Egregium}}

Let $M$ be a surface in $\E^3$ and $R^i_{\,\,kmp}$ be Riemann tensor, Riemann
curvature tensor of Levi-Civita connection. Recall that this means that
$R^i_{\,\,kmp}$ is curvature tensor of the connection $\nabla$, which is
Levi-Civita connection of the Riemannian metric $g_{\a\beta}$ induced on the surface
$M$ by standard Euclidean metric $dx^2+dy^2+dz^2$.
Recall that
  Riemann curvature tensor is expressed via Christoffel symbols of connection by the formula
  \begin{equation}\label{curvaturetensorintermsofconnection4}
    R^i_{\,\,kmn}=\p_m\Gamma_{nk}^i+\Gamma_{mp}^i\Gamma^p_{nk}-\p_n\Gamma_{mk}^i-\Gamma_{np}^i\Gamma^p_{mk}
\end{equation}
(see he formula \eqref{curvatureincomponents})
where Christoffel symbols of Levi-Civita connection are defined by the formula
\begin{equation}\label{Levi-Civitaagain}
                     \Gamma^i_{mk}={1\over 2}g^{ij}\left({\p g_{jm}\over\p x^k}+
    {\p g_{jk}\over\p x^m}-{\p g_{mk}\over \p x^j}\right)
\end{equation}
(see Levi-Civita Theorem)

Recall that scalar curvature $R$ of Riemann tensor equals to
$R=R^i_{kim}g^{km}$, where $g^{km}$ is Riemannina metric tensor with upper indices
(matrix $||g^{ik}||$ is inverse to the matrix $||g_{ik}||$). Note that as it was mentioned before
the formula for scalar curvature becomes very simple in two-dimensional case (see formulae
\eqref{identityforriemantensor1}
and \eqref{identityforriemantensor1} above). Using these formulae calculate Ricci tensor
and scalar curvature $R$. Let $R_{1212}=a$. We know that all other components of Riemann tensor equal
to zero or equal to $\pm a$ (see
\eqref{identityforriemantensor1}
and \eqref{identityforriemantensor1}).
One can show that scalar curvature $R$ can be expressed via the component $R_{1212}=a$ by the formula
\begin{equation}\label{scalarcurvatureviaonecomponent}
    R={2R_{1212}\over \det g}
\end{equation}
where $\det g=\det g_{ik}=g_{11}g_{22}=g_{12}^2$.

{\footnotesize
 Show it. Using identities \eqref{identityforriemantensor1}
and \eqref{identityforriemantensor1} we see that
 \begin{equation}\label{ricchitensor1}
    R_{11}=R^i_{\,\,1i1}=R^2_{\,\,121}=g^{22}R_{2121}+g^{21}R_{1121}=g^{22}R_{1212}=g^{22}a
\end{equation}
\begin{equation}\label{ricchitensor1}
    R_{22}=R^i_{\,\,2i2}=R^1_{\,\,212}=g^{11}R_{1212}+g^{12}R_{2221}=g^{11}R_{1212}=g^{11}a
\end{equation}
\begin{equation}\label{ricchitensor1}
    R_{12}=R_{21}=R^i_{\,\,1i2}=R^1_{\,\,112}=g^{12}R_{2112}=-g^{12}R_{1212}=-g^{12}a
\end{equation}
Thus
         $$
     R_{ik}=\begin{pmatrix}g^22 a& -g^{12}a\cr -g^{21}a & g^{11} a\end{pmatrix}
         $$
Now for scalar curvature we have
\begin{equation}\label{scalarcurvatureintermsofonecomponent}
    R=R^i_{\,\,kim}g^{km}=R_{km}g^{km}=g^{11}R_{11}+g^{12}R_{12}+g^{21}R_{21}+g^{22}R_{22}=
\end{equation}
         $$
    2R_{1212}(g^{11}g^{22}-g^{12}g^{12})=2a\det g^{-1}={2a\over \det g}
         $$


}

\m


Now we formulate very important


{\bf Proposition} {\it For an arbitrary point of the surface $M$
 \begin{equation}\label{egregiumagain}
    R=2K
\end{equation}
where $R=R^i_{kim}g^{km}$ is scalar curvature and $K$ is Gaussian curvature.}


We know also that for surface $M$ the scalar curvature $R$ is expressed via Riemann curvature tensor
by the formula \eqref{scalarcurvatureviaonecomponent}. Hence if we know the Gaussian curvature
then we know all components of Riemann curvature tensor (since all components vanish or equal to $\pm a$.):
using \eqref{scalarcurvatureviaonecomponent} one can rewrite the formula \eqref{egregiumagain}

         \begin{equation}\label{egregiumagain2}
   {R_{1212}\over \det g}=K
\end{equation}

This is nothing but Theorema Egregium!
Indeed Theorema Egregium  (see beginning of the section 4) immediately follows from this Proposition. Indeed according
to the formulae \eqref{curvaturetensorintermsofconnection4} and \eqref{Levi-Civitaagain}
the left hand side of the relation $R=2K$ depends only on induced Riemannian metric. Hence
  Gaussian curvature $K$ depends only on induced Riemannian metric.



It remains to prove the Proposition.
We do it by direct calculations in the next subsection.



{\bf Example}  Let $M=S^2$ be sphere of radius $R$ in $\E^3$.
 Show that one cannot find local coordinates $u,v$ on the sphere such that induced
Riemannian metric equals to $du^2+dv^2$ in these coordinates.

This immediately follows from the Proposition. Indeed suppose there exist
local coordinates $u,v$ on the sphere such that induced
Riemannian metric equals to $du^2+dv^2$, i.e. Riemannian metric is given by unity matrix.
Then according to the formulae for Levi-Civita connection, the Christoffel symbols
equal to zero in these coordinates. Hence Riemann curvature tensor equals to zero,
and scalar curvature too. Due to Proposition this is in contradiction with the fact
that  Gaussian curvature of the sphere equals to $1\over R^2$.



\m


\subsubsection {$^*$Proof of the Proposition \eqref{egregiumagain}}

We prove this fact by direct calculations.
 The plan of calculations is following:

Let $M$ be a surface in $\E^3$
 For an arbitrary point $\pt$ of the surface $M$ we consider Cartesian coordinates
 $x,y,z$ such that orgin coincides with the point $\pt$ and
 coordinate plane $OXY$ is the palne attached at the surface at the point $\pt$
 and the axis $OZ$ is orthogonal to the surface. In these coordinates calculations become easy.
  The surface $M$ in these Cartesian coordinates can be expressed by the equation
   \begin{equation}\label{surfaceinadjusted4}
   \begin{cases}
   x=u\cr y=v\cr z=F(u,v)
   \end{cases}
   \end{equation}
where $F(u,v)$ has local extremum at the point $u=v=0$\footnote{more in detail this is stationary point. It
is local extremum if quadratic form corresponding to second differential is positive or negative definite, i.e.
$F_{uu}F_{vv}-F_{uv}^2$ keeps the sign. This means that  the Gaussian curvature is not negative}.

Calculate explicitly Gaussian curvature at the point $\pt$.
In a vicinity of the point $\pt$ basis vector $\r_u=\begin{pmatrix}1\cr 0\cr F_u\end{pmatrix}$,
basis vector $\r_v=\begin{pmatrix}0\cr 1\cr F_v\end{pmatrix}$. One can see that
       \begin{equation}\label{noralunitvector4}
    \n(u,v)={1\over 1+F_u^2+F_v^2}\begin{pmatrix}-F_u\cr -F_v\cr 1\end{pmatrix}
\end{equation}
is unit normal vector. In particular in the origin at the point $\pt$,
we have
               \begin{equation}\label{vectorsattheorigin4}
    \r_u=\begin{pmatrix} 1\cr 0\cr 0\end{pmatrix},\,\,
    \r_v=\begin{pmatrix}0\cr 1\cr 0\end{pmatrix},\,\,{\rm and}\,\,
    \n=\begin{pmatrix}0\cr 0\cr 1\end{pmatrix}
\end{equation}
Now calculate shape operator. We need this operator only at the origin, hence during calculations
of all derivatives we have to note that we need the final result only at the point $u=v=0$. This will
essentially simplify
calculations since at the point $\pt$ derivatives  $F_u, F_v$ equal to zero.
             $$
 S\r_u=-\p_u \n(u,v)\vert_\pt=-{\p\over \p u}
 \left({1\over 1+F_u^2+F_v^2}\begin{pmatrix}-F_u\cr -F_v\cr 1\end{pmatrix}\right)\vert_{u=v=0}=
             $$
             $$
    \begin{pmatrix}F_{uu}\cr -F_{vu}\cr 1\end{pmatrix}\vert_{u=v=0}=F_{uu}\r_u+F_{uv}\r_v
             $$
and
       $$
       S\r_v=-\p_v \n(u,v)\vert_\pt=-{\p\over \p v}
 \left({1\over 1+F_u^2+F_v^2}\begin{pmatrix}-F_u\cr -F_v\cr 1\end{pmatrix}\right)\vert_{u=v=0}=
             $$
             $$
    \begin{pmatrix}F_{uv}\cr -F_{vv}\cr 1\end{pmatrix}\vert_{u=v=0}=F_{uv}\r_u+F_{vv}\r_v
      $$
since $F_u=F_v=0$. We see that at the origin the Shape (Weingarten) operator equals to
   \begin{equation}\label{shapeatorigin4}
    S=\begin{pmatrix}F_{uu} &F_{uv}\cr F_{vu} & F_{vv}\end{pmatrix}
\end{equation}
(all the derivatives at the origin).

Gaussian curvature at the point $\pt$ equals to
\begin{equation}\label{gaussiancurvature4}
    K=\det S=F_{uu}F_{vv}-F_{uv}^2\,.
\end{equation}
(all the derivatives at the origin).

Now it is time to calculate the Riemann curvature tensor at the origin.

First of all recall the expression for Riemannian metric for the surface $M$ in a vicinity of origin is
             \begin{equation}\label{metriconsurface12}
       G=
       \begin{pmatrix}\langle\r_u,\r_u\rangle & \langle\r_u,\r_v\rangle \cr  \langle\r_v,\r_u\rangle
       & \langle\r_v,\r_v\rangle
       \end{pmatrix}=
       \begin{pmatrix}1+F_u^2 & F_uF_v \cr  F_vF_u & 1+F_v^2
       \end{pmatrix}\,.
             \end{equation}
This immediately follows from the expression for basic vectors  $\r_u,\r_v$





Note that Riemannian metric $g_{ik}$ at the point $u=v=0$ is defined by unity matrix
$g_{uu}=g_{vv}=1$, $g_{uv}=g_{vu}=0$ since
$\pt$ is extremum point:
$G= \begin{pmatrix}1+F_u^2 & F_uF_v \cr  F_vF_u & 1+F_v^2\end{pmatrix}\big\vert_\pt=
        \begin{pmatrix}1 & 0 \cr  0 & 1\end{pmatrix}$ since $\pt$ is stationary point, extremum ($F_u=F_v=0$).
Hence the components of the tensor $R^i_{\,\,kmn}$ and $R_{ikmn}=g_{ij}R^j_{\,\,kmn}$ at the point $\pt$ are the same.

  Recall that the components of  $R^i_{\,\,kmn}$ are defined by the formula
               $$
R^i_{\,\,kmn}=\p_m\Gamma_{nk}^i+\Gamma_{mp}^i\Gamma^p_{nk}-\p_n\Gamma_{mk}^i-\Gamma_{np}^i\Gamma^p_{mk}\,.
               $$
Notice that at the point $\pt$ not only the matrix of the metric $g_{ik}$ equals to unity matrix,
 but more: Christoffel symbols vanish at this point in coordinates $u,v$
since the derivatives of metric at this point vanish.
(Why they vanish: this immediately follows from Levi-Civita formula applied to the metric
\eqref{metriconsurface12},
see also in detail the file "The solution of the problem 5 in the coursework, revisited").
Hence to calculate $R^i_{kmn}$ at the point $\pt$ one can consider more simple formula:
                   $$
                   R^i_{\,\,kmn}\vert_\pt=
                   \p_m\Gamma_{nk}^i\vert_\pt-
                   \p_n\Gamma_{mk}^i\vert_\pt
                   $$
Try to calculate in a more "economical" way. Due to Levi-Civita formula
             $$
                 \Gamma^i_{mk}={1\over 2}g^{ij}\left({\p g_{jm}\over\p x^k}+
    {\p g_{jk}\over\p x^m}-{\p g_{mk}\over \p x^j}\right)
             $$
Since metric $g_{ik}$ equals to unity matrix
$g=\begin{pmatrix} 1&0\cr 0&1\end{pmatrix}$ at the point $\pt$ hence $g^{ij}$ is unity matrix also:
                          $$
                       g^{ik}\vert_\pt=
                       \begin{pmatrix} 1&0\cr 0&1\end{pmatrix}=\delta^{ik}\,.
                          $$
(We denote $\delta^{ik}$ the unity matrix: all diagonal components equal to $1$, all other components equal to zero.
(It is so called Kronecker symbols))
Moreover we know also that all the first derivatives of the metric vanish at the point $\pt$:
             $$
             {\p g_{ik}\over \p x^m}\vert_\pt=0\,.
             $$
Hence it follows from the formulae above  that for an arbitrary indices $i,j,k,m,n$
          $$
       {\p\over \p x^i}\left(g^{km}{\p g_{pr}\over \p x^j}\right)\big\vert_\pt=
        {\p g^{km}\over \p x^i}\big\vert_\pt {\p g_{pr}\over \p x^j}\big\vert_\pt+
      g^{km}\big\vert_\pt {\p^2 g_{pr}\over \p x^i\p x^j}\big\vert_\pt=
    \delta^{km} {\p^2 g_{pr}\over \p x^i\p x^j}\big\vert_\pt\,.
          $$
Now using the Levi-Civita formula for the Christoffel symbols of connection:
 $$
                  \Gamma^i_{mk}={1\over 2}g^{ij}\left({\p g_{jm}\over\p x^k}+
    {\p g_{jk}\over\p x^m}-{\p g_{mk}\over \p x^j}\right)
 $$
we come to
            $
       \p_n\Gamma_{mk}^i\vert_\pt={\p\over \p x^n}
       \left({1\over 2}g^{ij}\left({\p g_{jm}\over\p x^k}+
    {\p g_{jk}\over \p x^m}-{\p g_{mk}\over \p x^j}\right)\right)\big\vert_\pt=
         $
         \begin{equation}\label{formulaforconnectionatextremumpoint}
{1\over 2}\delta^{ij}\left({\p^2 g_{jm}\over \p x^n\p x^k}+
    {\p^2 g_{jk}\over \p x^n\p x^m}-{\p^2 g_{mk}\over \p x^n\p x^j}\right)\big\vert_\pt\,.
            \end{equation}
Now using this formula we are ready to calculate Riemann curvature tensor $R^i_{\,\,kmn}$.
Remember that it is enough to calculate $R^1_{\,\,212}$ and $R^1_{\,\,212}=R_{1212}$ at the point $\pt$
since $g_{ik}=\delta_{ik}$ at the point $\pt$. We have that at $\pt$
             $
                                R^1_{\,\,212}\vert_\pt=
                   \p_1\Gamma_{22}^1\vert_\pt-
                   \p_2\Gamma_{12}^1\vert_\pt
             $
and according to \eqref{formulaforconnectionatextremumpoint}
             $$
\p_1\Gamma_{22}^1={1\over 2}\delta^{1j}\left({\p^2 g_{j2}\over \p x^1\p x^2}+
    {\p^2 g_{j2}\over \p x^1\p x^2}-{\p^2 g_{22}\over \p x^1\p x^j}\right)\big\vert_\pt=
          $$
          $$
    {1\over 2}\left({\p^2 g_{12}\over \p x^1\p x^2}+
    {\p^2 g_{12}\over \p x^1\p x^2}-{\p^2 g_{22}\over \p x^1\p x^1}\right)\big\vert_\pt=
{\p^2 g_{12}\over \p x^1\p x^2}-{1\over 2}{\p^2 g_{22}\over \p x^1\p x^1}\big\vert_\pt,
             $$
             $$
             \p_2\Gamma_{12}^1={1\over 2}\delta^{1j}\left({\p^2 g_{j1}\over \p x^2\p x^2}+
    {\p^2 g_{j2}\over \p x^2\p x^1}-{\p^2 g_{12}\over \p x^2\p x^j}\right)\big\vert_\pt=
           $$
           $$
    {1\over 2}\left({\p^2 g_{11}\over \p x^2\p x^2}+
    {\p^2 g_{12}\over \p x^2\p x^1}-{\p^2 g_{12}\over \p x^2\p x^1}\right)\big\vert_\pt=
    {1\over 2}{\p^2 g_{11}\over \p x^2\p x^2}\big\vert_\pt
    \,,
             $$
   Hence
          $$
          R_{1212}\vert_\pt=
          R^1_{\,\,212}\vert_\pt=
                   \p_1\Gamma_{22}^1\vert_\pt-
                   \p_2\Gamma_{12}^1\vert_\pt=
\left({\p^2 g_{12}\over \p x^1\p x^2}-{1\over 2}{\p^2 g_{22}\over \p x^1\p x^1}-{1\over 2}{\p^2 g_{11}\over \p x^2\p x^2}
\right)\big\vert_\pt
          $$

Now using expression \eqref{metriconsurface12} for metric calculate  second derivatives
${\p^2 g_{12}\over \p x^1\p x^2},{\p^2 g_{11}\over \p x^2\p x^2}$ and ${\p^2 g_{22}\over \p x^1\p x^1}$:

             $$
             {\p^2 g_{12}\over \p x^1\p x^2}\vert_{\pt}={\p^2 (F_uF_v)\over \p u\p v}\big\vert_\pt=
             {\p \over \p u}\left(F_uF_{vv}+F_{uv}F_v\right)\vert_\pt=
                \left(F_{uu}F_{vv}+F_{uv}^2\right) \vert_\pt\,.
             $$
since $F_u=F_v=0$ at the extremum point $\pt$ ($x^1=u,x^2=v$).
Analogoulsy
         $$
          {\p^2 g_{11}\over \p x^2\p x^2}\vert_{\pt}={\p^2 (1+F_u^2)\over \p v\p v}\big\vert_\pt=
             {\p \over \p v}\left(2F_uF_{uv}\right)\vert_\pt=2F_{uv}^2\vert_\pt\
         $$
  and
  $$
       {\p^2 g_{22}\over \p x^1\p x^1}\vert_{\pt}={\p^2 (1+F_v^2)\over \p u\p u}\big\vert_\pt=
             {\p \over \p u}\left(2F_vF_{uv}\right)\vert_\pt=2F_{uv}^2\vert_\pt\,,
             $$
We have finally that
            $$
      R_{1212}\vert_\pt=
          R^1_{\,\,212}\vert_\pt=
                   \p_1\Gamma_{22}^1\vert_\pt-
                   \p_2\Gamma_{12}^1\vert_\pt=
\left({\p^2 g_{12}\over \p x^1\p x^2}-{1\over 2}{\p^2 g_{22}\over \p x^1\p x^1}-{1\over 2}{\p^2 g_{11}\over \p x^2\p x^2}
\right)\big\vert_\pt=F_{uu}F_{vv}-F_{uv}^2\,.
          $$
     The scalar curvature according to \eqref{scalarcurvatureviaonecomponent}
     equals to
            $R=2R_{1212}=2(F_{uu}F_{vv}-F_{uv}^2)$. Comparing with the expression
            $K=F_{uu}F_{vv}-F_{uv}^2$ in \eqref{gaussiancurvature4} for Gaussian curvature we come to $R=2K$\finish.








  \subsection {Gauss Bonnet Theorem}



  Consider the integral of curvature over whole closed surface $M$. According to the Theorem
  above the answer has to be equal to $0$ (modulo $2\pi$), i.e. $2\pi N$ where $N$ is an integer,
  because this integral is a limit when we consider very small curve. We come to the formula:
         $$
         \int_D Kd\sigma=2\pi N
         $$
(Compare this formula with formula \eqref{theoremofrotationonangle}).


  What is the value of integer $N$?


We present now one remarkable Theorem which answers this question and prove this Theorem using the
formula \eqref{theoremofrotationonangle}.


Let $M$ be a closed orientable surface.\footnote{Closed means compact surface without boundaries.
Intuitively orientability means that one can define out and inner side of the surface.
  In terms of normal vectors
orientability means that
one can define the continuous field of normal vectors at all the points of $M$.
The direction of normal vectors at any point defines outward direction.
Orientable surface is called oriented if the direction of normal vector is chosen.}
All these surfaces can be classified up to a diffeomorphism.
Namely arbitrary  closed oriented surface $M$
   is diffeomorphic either to  sphere (zero holes),
   or torus (one hole), or pretzel (two holes),...
"Number k" of holes is intuitively evident characteristic of the surface.
It is related with very important characteristic---Euler characteristic
$\chi(M)$ by the following formula:
\begin{equation}\label{defofeuler00}
  \chi(M)=2(1-g(M)),\quad \hbox {where $g$ is number of holes}
\end{equation}


{\bf Remark} What we have called here "holes" in a surface is often referred
to as "handles" attached o the sphere, so that the sphere itself does not have any handles,
the torus has one handle, the pretzel has two handles and so on. The number of handles is also called genus.

\smallskip

Euler characteristic appears in many different way. The simplest appearance is the following:

Consider on the surface $M$  an arbitrary  set of points (vertices) connected
with edges (graph on the surface) such that surface is divided on
 polygons with (curvilinear sides)---plaquets. ("Map of world")

Denote by $P$ number of plaquets (countries of the map)

Denote by $E$ number of edges (boundaries between countries)

Denote by $V$ number of vertices.

Then it turns out that
\begin{equation}\label{defofeulerchar100}
  P-E+V=\chi(M)
\end{equation}
It does not depend on the graph, it depends only on how much holes has surface.

E.g. for every graph on $M$, $P-E+V=2$ if $M$ is diffeomorphic to sphere.
For every graph on $M$ $P-E+V=0$ if $M$ is diffeomorphic to torus.


\bigskip


Now we formulate Gau\ss\,-Bonnet Theorem.


Let $M$ be closed oriented surface in $\E^3$.

Let $K(p)$ be Gaussian curvature at any point $p$ of this surface.



\medskip

{\bf  Theorem} (Gau\ss\,\,-Bonnet)
The integral of Gaussian curvature over the
closed compact oriented surface  $M$ is equal to $2\pi$ multiplied
by the Euler characteristic of the surface $M$
\begin{equation}\label{gaussbonnet}
  {1\over 2\pi}\int_M Kd\sigma=\chi(M)=2(1-\hbox{number of holes})
\end{equation}

In particular for the surface $M$ diffeomorphic to the sphere   $\kappa(M)=2$,
for the surface diffeomorphic to the torus it is equal to $0$.

\smallskip

The value of the integral does not change under continuous deformations of
surface! It is integer number (up to the factor $\pi$) which characterises
topology of the surface.

E.g. consider surface $M$ which is diffeomorphic to the sphere.
If it is sphere of the radius $R$ then curvature is equal to $1\over R^2$,
 area of the sphere is equal to $4\pi R^2$ and left hand side
 is equal to ${4\pi \over 2\pi}=2$.

If surface $M$ is an arbitrary surface diffeomorphic to $M$
then  metrics and curvature depend from point to the point,
Gau\ss\,-Bonnet states that integral nevertheless remains unchanged.

\smallskip

Very simple but impressive corollary:

{\it Let $M$ be surface diffeomorphic to sphere in $\E^3$. Then
there exists at least one point where Gaussian curvature is positive.}

Proof: Suppose it is not right. Then $\int_M K\sqrt {\det g}dudv\leq 0$.
On the other hand according to the Theorem it is equal to $4\pi$. Contradiction.






\bigskip
{\footnotesize
 {\it Proof of Gau\ss-Bonet Theorem}

 Consider triangulation of the surface $M$. Suppose $M$ is covered by $N$ triangles. Then number
 of edges will be $3N/over 2$. If $V$ number of vertices then according to Euler Theorem
                $$
        N-{3N\over 2}+V=V-{N\over 2}=\chi(M).
                $$
Calculate the sum of the angles of all triangles.
On the one hand it is equal to $2\pi V$. On the other hand
according the formula \eqref{theoremofrotationonangle} it is equal to
           $$
   \sum_{i=1}^N\left(\pi+\int_{\triangle_i}K d\sigma\right)=
   \pi N+\sum_{i=1}^N\left(\int_{\triangle_i}K d\sigma\right)=
   N\pi+\int_M Kd\sigma
           $$
We see that $2\pi V=N\pi+\int_M Kd\sigma$, i.e.
            $$
         \int_M Kd\sigma=\pi \left(2V-{N\over 2}\right)=2\pi \chi(M)\hbox{\finish}
            $$
}


%\end{document}
\section {Appendices}



\subsection {$^*$Integrals of motions and geodesics.}

{\footnotesize


We see how useful in Riemannian geometry to use the Lagrangian approach.

To solve and study solutions of Lagrangian equations (in particular geodesics which are solutions of Euler-Lagrange
equations for Lagrangian of free particle) it is very useful to use {integrals of motion}

\subsubsection {$^*$Integral of motion for arbitrary Lagrangian $L(x,\dot x)$}
   Let $L=L(x,\dot x)$ be a Lagrangian, the function of point and velocity vectors on manifold $M$
   (the function on tangent bundle $TM$).

   {\bf Definition}  We say that the function $F=F(q,\dot q)$ on $TM$ is {\it integral of motion}
   for Lagrangian $L=L(x,\dot x)$ if for any curve $q=q(t)$ which is the solution of Euler-Lagrange equations of motions
     the magnitude  $I(t)=F(x(t), \dot x(t))$ is preserved along this curve:
               \begin{equation}\label{integralofmotion}
               F(x(t), \dot x(t))=const\,\,\hbox{if $x(t)$  is a solution of Euler-Lagrange equations\eqref{ELequations1}.}
                 \end{equation}
In other words
               \begin{equation}\label{integralofmotion2}
               {d\over dt}
               \left(F(x(t), \dot x(t))\right)=0\,\,{\rm if}\,\,
               x^i(t)\colon\,\,   {d\over dt}\left({\p L\over \dot \p x^i}\right)-{\p L\over \p x^i}=0\,.
                 \end{equation}

               \subsubsection {$^*$Basic examples of Integrals of motion: Generalised momentum and Energy}

 Let $L(x^i,\dot x^i)$ does not depend on the coordinate $x^1$.
    $L=L(x^2,\dots,x^n, \dot x^1,\dot x^2,\dots, \dot x^n)$. Then  the function
                   $$
       F_1(x,\dot x)={\p L\over \p\dot x^1}
                   $$
is integral of motion. (In the case if $L(x^i,\dot x^i)$ does not depend on the coordinate $x^i$.
    the function $F_i(x,\dot x)={\p L\over \dot \p x^i}$ will be integral of motion.)


Proof is simple. Check the condition \eqref{integralofmotion2}:
Euler-Lagrange equations of motion are:
          $$
          {d\over dt}\left({\p L\over \dot \p x^i}\right)-{\p L\over \p x^i}=0\quad (i=1,2,\dots,n)
          $$
 We see that exactly first equation of motion is
               $$
       {d\over dt}\left({\p L\over \dot \p x^1}\right)= {d\over dt}F_1(q,\dot q)=0
       \quad \hbox{since ${\p L\over \p x^1}=0,$}\,.
               $$


 (if $L(x^i,\dot x^i)$ does not depend on the coordinate $x^i$
    then the function $F_i(x,\dot x)={\p L\over \dot x^1}$ is  integral of motion
    since $i-th$ equation is exactly the condition $\dot F_i=0$.)

    The integral of motion $F_i={\p L\over \dot \p x^i}$ is called sometimes {\it generalised momentum}.


    \m

    Another very important example of integral of motion is: energy.
      \begin{equation}\label{energy}
      E(x,\dot x)=\dot x^i {\p L\over \p \dot x^i}-L\,.
      \end{equation}
    One can check by direct calculation that it is indeed integral of motion.
    Using Euler Lagrange equations
    ${d\over dt}\left({\p L\over \dot \p x^i}\right)-{\p L\over \p x^i}$ we have:
           $$
     {d\over dt}E(x(t), \dot x(t))={d\over dt}\left(\dot x^i {\p L\over \p \dot x^i}-L\right)=
        {\p L\over \p \dot x^i}{d\dot x^i\over dt}+{d\over dt}\left({\p L\over \p\dot x^i}\right)\dot x^i-{dL\over dt}=
           $$
           $$
        {\p L\over \p \dot x^i}{d\dot x^i\over dt}+{\p L\over \p x^i}{dx^i\over dt}-{dL(x,\dot x)\over dt}=
        {dL(x,\dot x)\over dt}-{dL(x,\dot x)\over dt}=0\,.
           $$
\subsubsection {$^*$Integrals of motion for geodesics}

Apply the integral of motions for studying geodesics.

The Lagrangian of "free" particle  $L_{\rm free}={g_{ik}(x)\dot x^i\dot x^k\over 2}$.
  For Lagrangian of free particle solution of Euler-Lagrange equations of motions are geodesics.

  If $F=F(x,\dot x)$ is the integral of motion of the free Lagrangian $L_{\rm free}={g_{ik}(x)\dot x^i\dot x^k\over 2}$
  then the condition \eqref{integralofmotion} means that the magnitude
        $I(t)=F(x^i(t), \dot x^i(t))$ is preserved along the geodesics:
                  \begin{equation}\label{integralofmotionforgeodesic}
               I(t)=F(x^i(t), \dot x^i(t))=const, {\rm i.e.}\,
               {d\over dt}I(t)=0\,\,\,\hbox{if $x^i(t)$  is geodesic.}
                 \end{equation}

  Consider examples of integrals of motion for free Lagrangian, i.e. magnitudes which preserve along the geodesics:

  \m

  {\bf Example 1}  Note that for an arbitrary "free" Lagrangian Energy integral \eqref{energy}
   is an integral of motion:
                       $$
      E=\dot x^i {\p L\over \p \dot x^i}-L=
      \dot x^i {\p \left({g_{pq}(x)\dot x^p\dot x^q\over 2}\right)\over \p \dot x^i}-
      {g_{ik}(x)\dot x^i\dot x^k\over 2}=
                       $$
                    \begin{equation}\label{energy2}
   \dot x^ig_{iq}(x)\dot x^q-{g_{ik}(x)\dot x^i\dot x^k\over 2}={g_{ik}(x)\dot x^i\dot x^k\over 2}\,.
                    \end{equation}
   This is an integral of motion for an arbitrary Riemannian metric. It is preserved on an arbitrary geodesic
                         $$
            {dE(t)\over dt}={d\over dt}\left({1\over 2}
            g_{ik}(x(t))\dot x^i(t)\dot x^k(t)\right)=0\,.
                         $$
   In fact we already know this integral of motion:  Energy \eqref{energy2} is proportional to the
   square of the length of velocity vector:
   \begin{equation}\label{energyandspeed}
   |\v|=\sqrt {g_{ik}(x)\dot x^i\dot x^k}=\sqrt {2E}\,.
   \end{equation}
   We already proved that velocity vector is preserved
   along the geodesic (see the Proposition in the subsection 3.2.1 and its proof \eqref{preservationofthelength}.)


\m


  {\bf Example 2}  Consider Riemannian metric $G=adu^2+bdv^2$ (see also calculations in subsection 2.3.3)
   in the case if $a=a(u)$, $b=b(u)$, i.e. coefficients do not depend on the second coordinate  $v$:
                \begin{equation}\label{intofmotionwhenfirstcoordianteiscyclic}
G=a(u)du^2+b(u)dv^2,\,\, L_{\rm free}={1\over 2}\left(a(u)\dot u^2+b(u)\dot v^2\right)
                \end{equation}
 We see that Lagrangian does not depend on the second coordinate $v$ hence the magnitude
      \begin{equation}\label{cyclicintegra2}
        F={\p L_{\rm free}\over \p \dot v}=b(u)\dot v
      \end{equation}
  is preserved along geodesic. It
   is integral of motion because Euler-Lagrange equation for coordinate $v$ is
    $$
 {d\over dt}{\p L_{\rm free}\over \p\dot v}-{\p L_{\rm free}\over \p v}=
 {d\over dt}{\p L_{\rm free}\over \p\dot v}={d\over dt}F=0\,\,
 \hbox{ since ${\p L_{\rm free}\over \p v}=0$.}\,.
 $$


 \m

 In fact all revolution surfaces which we consider here (cylinder, cone, sphere,...) have Riemannian
 metric of this type.
 Indeed consider further examples.

 {\bf Example (sphere)}

 Sphere of the radius $R$ in $\E^3$. Riemannian metric:
$G=Rd\theta^2+R^2\sin^2\theta d\varphi^2$ and
$L_{\rm free}={1\over 2}\left(R^2\dot \theta^2+R^2\sin^2\theta\dot \varphi^2\right) $
It is the case \eqref{intofmotionwhenfirstcoordianteiscyclic} for $u=\theta,v=\varphi$, $b(u)=R^2\sin^2\theta$
                The integral of motion is
                 $$
           F={\p L_{\rm free}\over \p \dot\varphi}=R^2\sin^2\theta\dot\varphi
                 $$

 \m

{\bf Example (cone)}

Consider cone $\begin{cases}x=ah\cos\varphi\cr y=ah\sin\varphi \cr z=bh\cr\end{cases}$. Riemannian metric:
                          $$
                          G=d(ah\cos\varphi)^2+d(ah\sin\varphi)^2+(dbh)^2=
                          (a^2+b^2)dh^2+a^2h^2d\varphi^2\,.
                          $$
and free Lagrangian
                 $$
L_{\rm free}={(a^2+b^2)\dot h^2+a^2h^2\dot\varphi^2\over 2}\,.
                  $$
The integral of motion is
                 $$
           F={\p L_{\rm free}\over \p \dot\varphi}=a^2h^2\dot\varphi.
                 $$
  \m

{\bf Example (general surface of revolution)}

Consider  a surface  of revolution in $\E^3$:
                   \begin{equation}\label{surfaceoffrevolutiongeneral}
                    \r(h,\varphi)\colon \,\,
                  \begin{cases}
                  x=f(h)\cos\varphi\cr
                  y=f(h)\sin\varphi\cr
                  z=h
                   \end{cases}\qquad
                   (f(h)>0)
                       \end{equation}

(In the case $f(h)=R$ it is cylinder, in the case $f(h)=kh$ it is a cone, in the case $f(h)=\sqrt{R^2-h^2}$
it is a sphere, in the case $f(h)=\sqrt {R^2+h^2}$ it is one-sheeted hyperboloid, in the case
$z=\cos h$ it is catenoid,...)

For the surface of revolution \eqref{surfaceoffrevolutiongeneral}
                   $$
          G=d(f(h)\cos\varphi)^2+d(f(h)\sin\varphi)^2+(dh)^2=
          (f'(h)\cos\varphi dh-f(h)\sin\varphi d\varphi)^2+
                   $$
                   $$
     (f'(h)\sin\varphi dh+f(h)\cos\varphi d\varphi)^2+dh^2=(1+f'^2(h))dh^2+f^2(h)d\varphi^2\,.
                   $$
The "free" Lagrangian of the surface of revolution is
                 $$
L_{\rm free}={\left(1+f'^2(h)\right)\dot h^2+f^2(h)\dot\varphi^2\over 2}\,.
                  $$
and the integral of motion is
                 $$
           F={\p L_{\rm free}\over \p \dot\varphi}=f^2(h)\dot\varphi.
                 $$



\subsubsection {$^*$Using integral of  motions to calculate geodesics}

  Integrals of motions may be very useful to calculate geodesics.
  The equations for geodesics are second order differential equations.  If we know integrals of motions
  they help us to solve these equations.  Consider just an example.

    For Lobachevsky plane the free Lagrangian $L={\dot x^2+\dot y^2\over 2y^2}$. We already
    calculated geodesics in the subsection 3.3.4. Geodesics are solutions of second order Euler-Lagrange equations for
    the Lagrangian $L={\dot x^2+\dot y^2\over 2y^2}$ (see the subsection 3.3.4)
             $$
             \begin{cases}
    {\buildrel \cdot\cdot\over x}-{2\dot x\dot y\over y}=0\cr
    {\buildrel \cdot\cdot\over y}+{\dot x^2\over y}-{\dot y^2\over y}=0\cr
    \end{cases}
                          $$
  It is not so easy to solve these differential equations.

  For Lobachevsky plane we know two integrals of motions:
  \begin{equation}\label{energyforlobacchevskyplaneandsecondintegral}
    E=L={{\dot x^2+\dot y^2\over 2y^2}},\quad {\rm and}\,\,\,
     F={\p L\over \p \dot x}={\dot x\over y^2}\,.
  \end{equation}
         These both integrals preserve in time: if $x(t),y(t)$ is geodesics then
         $$
         \begin{cases}
         F={\dot x(t)\over y(t)^2}\cr
         E={{\dot x(t)^2+\dot y(t)^2\over 2y(t)^2}}=C_2\cr
         \end{cases}\Rightarrow
         \begin{cases}
         \dot x=C_1y^2\cr
         \dot y=\pm\sqrt {2C_2y^2-C_1^2y^4}\cr
         \end{cases}
                 $$
  These are first order differential equations.  It is much easier to solve these equations in general case
  than initial second order differential equations.




 \subsection{ Induced metric on surfaces.}

   Recall here again induced metric (see for detail subsection 1.4)

  If surface $M\colon\,\, \r=\r(u,v)$is embedded in $\E^3$ then induced Riemannian metric
  $G_M$ is defined by the formulae
  \begin{equation}\label{scalarproduct2}
    \langle\X,\Y\rangle=G_M(\X,\Y)=G(\X,\Y)\,,
  \end{equation}
  where $G$ is Euclidean metric in $\E^3$:
           $$
        G_M=dx^2+dy^2+dz^2\big\vert_{\r=\r(u,v)}=
        \sum_{i=1}^3(dx^i)^2\big\vert_{\r=\r(u,v)}=\sum_{i=1}^3\left({\p x^i\over \p u^\a}du^\a\right)^2
                $$
                $$=
        {\p x^i\over \p u^\a} {\p x^i\over\p  u^\beta}du^\a du^\beta\,,
           $$
        i.e.
            $$
      G_M=g_{\a\beta}du^\a ,\,\, {\rm where}\,\, g_{\a\beta}=
      {\p x^i\over \p u^\a} {\p x^i\over \p u^\beta}du^\a du^\beta\,.
           $$
 We use notations $x,y,z$ or $x^i$ ($i=1,2,3$) for Cartesian coordinates in $\E^3$,
 $u,v$ or $u^\a$ ($\a=1,2$) for coordinates on the surface. We usually   omit summation symbol
 over  dummy indices.    For coordinate tangent vectors
           $$
 \underbrace{\p\over \p u_\a}_{\hbox{Internal observer}} =
 \underbrace{\r_\a={\p x^i\over \p u^\a}{\p\over \p x^i}}_{\hbox{External observer}}
           $$
We have already plenty examples in the subsection 1.4. In particular for scalar product
\begin{equation}\label{scalarproduct4}
  g_{\a\beta}=\left\langle{\p\over u_\a},{\p\over u_\beta}\right\rangle=x^i\a x^i\beta\,.
  \langle \r_\a,\r_\beta\rangle\,.
\end{equation}



   \subsubsection {Recalling Weingarten operator}

 Continue to play with formulae \footnote{In some sense differential geometry it is when we write down
 the formulae expressing the geometrical facts, differentiate these formulae then reveal the geometrical meaning
  of the new obtained formulae e.t.c.}.


Recall the Weingarten (shape) operator which acts on tangent vectors:

  \begin{equation}\label{weingoperator1}
    S\X=-\p_\X \n\,,
  \end{equation}
 where we denote by $\n$-unit normal vector field at the points of the surface $M$: $\langle\n,\r_\a\rangle=0$,
 $\langle\n,\r_\a\rangle=1$.

\m

 Now we realise that the derivative $\p_\X \R$ of vector field with respect to another vector field
 is not a well-defined object: we need a connection.
 The formula $\p_\X \R$ in Cartesian coordinates,
 is nothing but the derivative with respect to flat canonical connection:
 If we work only in Cartesian coordinates we do not need to distinguish between
 $\p_\X \R$ and $\nabla^{\rm can.flat}_\X\R $. Sometimes with some abuse of notations we will use
 $\p_\X \R$ instead $\nabla^{\rm can.flat}_\X\R $, but never forget: this can be done only in Cartesian coordinates
 where Christoffel symbols of flat canonical connection vanish:
              $$
      \p_\X \R=\nabla^{\rm can.flat}_\X\R  \quad \hbox{in Cartesian coordinates}\,.
         $$

 So  the rigorous definition of Weingarten operator is
\begin{equation}\label{weingoperator2}
    S\X=-\nabla^{\rm can.flat}_\X \n\,,
  \end{equation}
  but we often use the former one (equation \eqref{weingoperator1})
   just remembering that this can be done only in Cartesian coordinates.

  Recall that the fact that Weingarten operator $S$ maps tangent vectors to tangent vectors follows from the property:
  $\langle\n,\X\rangle=0\Rightarrow \X$ is tangent to the surface.

  Indeed:
     $$
 0= \p_\X\langle\n,\n\rangle=2\langle\p_\X\n,\n\rangle=-2\langle S\X,\n\rangle=0\Rightarrow S\X\,\,
 \hbox {is tangent to the surface}
     $$
     \m
Recall also that normal unit vector is defined up to a sign, $\n\to -\n$. On the other hand if $\n$
is chosen then $\S$ is defined uniquely.


\subsubsection {Second quadratic form}

We define now the new object: {\it second quadratic form}

{\bf Definition}. For two tangent vectors $\X, \Y$
$A(\X,\Y)$ is defined such that
\begin{equation}\label{secondquadraticform1}
    \left(\nabla^{\rm can.flat}_\X \Y\right)_\bot=A(\X,\Y)\n
\end{equation}
i.e. we take orthogonal component of the derivative of $\Y$ with respect to $\X$.

This definition seems to be very vague: to evaluate covariant derivative we have to consider not a
vector $\Y$ at a given point
but  the vector field.
In fact one can see that  $A(\X,\Y)$ does depend only on the value of $\Y$ at the given point.

Indeed it follows from the definition of second quadratic form and from the properties of Weingarten operator that
        $$
         A(\X,\Y)=\left\langle\left(\nabla^{\rm can.flat}_\X \Y\right)_\bot,\n\right\rangle=
    \left\langle\nabla^{\rm can.flat}_\X \Y,\n\right\rangle=
        $$
\begin{equation}\label{properofsecquadrform4}
    \p_\X \langle\Y,\n\rangle-\left\langle \Y,\nabla^{\rm can.flat}_\X\n\right\rangle=\langle S(\X), \Y\rangle
\end{equation}



We proved that second quadratic form depends only on value of vector field $\Y$ at the given poit and we established
the relation between second quadratic form and Weingarten operator.

\m

{\bf Proposition} {\it The second quadratic form $A(\X,\Y)$ is symmetric bilinear form on tangent vectors $\X,\Y$
in a given point.}

       \begin{equation}\label{symmetricityofsecquadrform}
A\colon T_\pt M\otimes T_\pt M\to \R,\quad A(\X,\Y)=A(\Y,\X)=\langle S\X,\Y\rangle\,.
       \end{equation}

 In components
 \begin{equation}\label{compexpressforscquadrform}
    A=A_{\a\beta}du^\a du^\beta=\langle \r_{\a\beta}, \n\rangle={\p^2 x^i\over \p u^\a \p u^\beta}n^i\,.
 \end{equation}
and
\begin{equation}\label{compexpressforweingarten}
    S^\a_\beta=g^{\a\pi}A_{\pi\beta}=g^{\a\pi}x^i_{\pi\beta}n^i\,,
     \end{equation}
i.e.
            $$
         A=GS, S=G^{-1}A\,.
            $$

 {\bf Remark} The normal unit vector field is defined up to a sign.


\subsubsection {Gaussian and mean curvatures}

   Recall that Gaussian curvature
            $$
          K=\det S
            $$
and mean curvature
           $$
         H={\rm Tr \,}S
           $$
It is easy to see that  for Gaussian curvature
           $$
         K=\det S=\det(G^{-1}A)={\det A\over \det G}.
           $$

We know already the geometrical meaning of Gaussian and mean curvatures from the point of view of the External Observer:

Gaussian curvature $K$ equals to the product of principal curvatures, and mean curvartures equals to the sum of principal curvatures.

Now we ask a principal question: what bout internal observer, "aunt" living on the surface?

We will show that Gaussian curvature can be expressed in terms of induced Riemannian metric, i.e. it is an
 internal characteristic of the surface, invariant of isometries.

  It is not the case with mean curvature: cylinder is isometric to the plane but it have
   non-zero mean curvature.

\subsubsection {Examples of calculation of Weingarten operator,
Second quadratic forms, curvatures for cylinder, cone and sphere.}

\m

{\it Cylinder}

\m

  We already calculated induced Riemannian metric on the cylinder (see \eqref {formula forfirstformcyl}).

 Cylinder is given by the equation $x^2+y^2=R^2$. One can consider the following
parameterisation
 of this surface:
\begin{equation}\label{surface11}
  \r(h,\varphi)\colon\quad
  \begin{cases}
  x=R\cos\varphi\\
  y=R\sin\varphi\\
  z=h\\
  \end{cases}\,,\quad   \r_h=
  \begin{pmatrix}
        0\\
        0\\
        1\\
   \end{pmatrix}\,,
\quad
  \r_\varphi=\begin{pmatrix}
        -R\sin\varphi\\
        R\cos\varphi\\
          0\\
   \end{pmatrix}\,,
\end{equation}
$G_{cylinder}=\left(dx^2+dy^2+dz^2\right)\big\vert_{x=a\cos\varphi,y=a\sin\varphi,z=h}=$
        \begin{equation*}\label{firtsquadraticformcylinder(diff)11}
               =(-a\sin\varphi d\varphi)^2+(a\cos\varphi d\varphi)^2+dh^2=a^2d\varphi^2+dh^2\,,\quad
               ||g_{\a\beta}||=
  \begin{pmatrix}
   1 & 0 \\
   0& R^2 \\
   \end{pmatrix}\,.
        \end{equation*}

   Normal unit vector $\n=\pm \begin{pmatrix}
        \cos \varphi\cr
        \sin\varphi\cr
        0\cr
   \end{pmatrix}$. Choose $\n=\begin{pmatrix}
        \cos \varphi\cr
        \sin\varphi\cr
        0\cr
   \end{pmatrix}$. Weingarten operator
     \begin{equation*}\label{weingrartenforcylinder}
        S\p_h=-\nabla^{\rm can.flat}_{\r_h}\n=-\p_{\r_h} \n=-\p_h\begin{pmatrix}
        \cos \varphi\cr
        \sin\varphi\cr
        0\cr
   \end{pmatrix}=0\,,
     \end{equation*}
\begin{equation*}\label{weingrartenforcylinder}
        S\p_\varphi=-\nabla^{\rm can.flat}_{\r_\varphi}\n=-\p_{\r_\varphi} \n=-{\p_\varphi}
        \begin{pmatrix}
        \cos \varphi\cr
        \sin\varphi\cr
        0\cr
   \end{pmatrix}=\begin{pmatrix}
        \sin\varphi \cr
        -\cos\varphi\cr
        0\cr
   \end{pmatrix}=-{\p_\varphi\over R}.
    \end{equation*}
\begin{equation}\label{weingrartenforcylinder2}
           S
   \begin{pmatrix}
        \p_h \cr
        \p_\varphi\cr
   \end{pmatrix}=
   \begin{pmatrix}
         0\cr
        {-\p_\varphi\over R}\cr
   \end{pmatrix},\quad S=\begin{pmatrix}0&0\cr 0 &{-1\over R}\end{pmatrix}\,.
    \end{equation}
Calculate   second quadratic form:  $\r_{hh}=\p_h\r_h=
  \begin{pmatrix}
        0\\
        0\\
        0\\
   \end{pmatrix}\,$, $\r_{h\varphi}=\r_{\varphi h}=$
             $$
             \p_h
       \begin{pmatrix}
        -R\sin\varphi\\
        R\cos\varphi\\
          0\\
   \end{pmatrix}=0,\,\,
   \r_{\varphi \varphi}=\p_\varphi
       \begin{pmatrix}
        -R\sin\varphi\\
        R\cos\varphi\\
          0\\
   \end{pmatrix}=\begin{pmatrix}
        -R\cos\varphi\\
        -R\sin\varphi\\
          0\\
   \end{pmatrix}=-R\n\,.
                      $$
We have
                    \begin{equation}\label{secquadrformforcylinder}
            A_{\a\beta}=\langle\r_{\a\beta},\n\rangle,\,
              A=\begin{pmatrix}\langle\r_{hh},\n\rangle&\langle\r_{h\varphi},\n\rangle\cr
                               \langle\r_{\varphi h},\n\rangle &\langle\r_{\varphi\varphi},\n\rangle\cr
                                   \end{pmatrix}=
                                   \begin{pmatrix}0&0\cr
                                0&-R\cr
                                   \end{pmatrix},
                    \end{equation}
                    $$
       A=GS=\begin{pmatrix}1&0\cr
                                0&R^2\cr
                                   \end{pmatrix}
                                   \begin{pmatrix}0&0\cr
                                0&-{1\over R}\cr
                                   \end{pmatrix}=
                                   \begin{pmatrix}0&0\cr
                                0&-R\cr
                                   \end{pmatrix},
                    $$
For Gaussian and mean curvatures we have
     \begin{equation}\label{Gaussianforcylinder}
        K=\det S={\det A\over \det G}=\det
                              \begin{pmatrix}0&0\cr
                                0&-{1\over R}\cr
                                   \end{pmatrix}=0\,,
     \end{equation}
     and mean curvature
      \begin{equation}\label{meanianforcylinder}
        H={\rm Tr\,} S={\rm Tr\,}
                              \begin{pmatrix}0&0\cr
                                0&-{1\over R}\cr
                                   \end{pmatrix}=-{1\over R}\,.
     \end{equation}
Mean curvature is define up to a sign. If we change $\n\to-\n$ mean curvature $H\to {1\over R}$ and Gaussian curvature
will not change.


\bigskip

{\it Cone}

  We already calculated induced Riemannian metric on the cone (see \eqref{firtsquadraticformforconus(diff)}.

 Cone is given by the equation $x^2+y^2-k^2z^2=0$. One can consider the following
parameterisation
 of this surface:
\begin{equation}\label{surface111}
  \r(h,\varphi)\colon\quad
  \begin{cases}
  x=kh\cos\varphi\\
  y=kh\sin\varphi\\
  z=h\\
  \end{cases}\,,\quad   \r_h=
  \begin{pmatrix}
        k\cos\varphi\\
        k\sin\varphi\\
        1\\
   \end{pmatrix}\,,
\quad
  \r_\varphi=\begin{pmatrix}
        -kh\sin\varphi\\
        kh\cos\varphi\\
          0\\
   \end{pmatrix}\,,
\end{equation}
$G_{cone}=\left(dx^2+dy^2+dz^2\right)\big\vert_{x=kh\cos\varphi,y=kh\sin\varphi,z=h}=$
        \begin{equation*}\label{firtsquadraticformcylinder(diff)11}
               =(-kh\sin\varphi d\varphi+k\cos\varphi dh)^2+
               (kh\cos\varphi d\varphi+k\sin\varphi dh)^2+dh^2=
                      \end{equation*}
             $$
              k^2h^2d\varphi^2+(k^2+1)dh^2,\quad
               ||g_{\a\beta}||=
  \begin{pmatrix}
   k^2+1 & 0 \\
   0& k^2h^2 \\
   \end{pmatrix}\,.
          $$

  One can see that  $\N=
      \begin{pmatrix}
        \cos \varphi\cr
        \sin\varphi\cr
        -k\cr
   \end{pmatrix}$ is orthogonal to the surface: $\N\bot \r_h,\r_\varphi$. Hence
   normal unit vector $\n=\pm
     {1\over {\sqrt {1+k^2}}}
      \begin{pmatrix}
        \cos \varphi\cr
        \sin\varphi\cr
        -k\cr
   \end{pmatrix}$. Choose $\n=
   {1\over {\sqrt {1+k^2}}}
      \begin{pmatrix}
        \cos \varphi\cr
        \sin\varphi\cr
        -k\cr
   \end{pmatrix}$.
     Weingarten operator
     \begin{equation*}\label{weingrartenforcylinder}
        S\p_h=-\nabla^{\rm can.flat}_{\r_h}\n=-\p_{\r_h} \n=
        -\p_h
        \left(
        {1\over {\sqrt {1+k^2}}}
      \begin{pmatrix}
        \cos \varphi\cr
        \sin\varphi\cr
        -k\cr
   \end{pmatrix}
   \right)
        =0\,,
     \end{equation*}
\begin{equation*}\label{weingrartenforcylinder}
        S\p_\varphi=-\nabla^{\rm can.flat}_{\r_\varphi}\n=-\p_{\r_\varphi} \n=
        -\p_\varphi
        \left(
{1\over {\sqrt {1+k^2}}}
      \begin{pmatrix}
        \cos \varphi\cr
        \sin\varphi\cr
        -k\cr
   \end{pmatrix}
   \right)
       =
       \end{equation*}
       \begin{equation*}
       {1\over {\sqrt {1+k^2}}}
       \begin{pmatrix}
        \sin\varphi \cr
        -\cos\varphi\cr
        0\cr
   \end{pmatrix}=-{\p_\varphi\over kh\sqrt {k^2+1}}.
    \end{equation*}
\begin{equation}\label{weingrartenforcylinder2}
           S
   \begin{pmatrix}
        \p_h \cr
        \p_\varphi\cr
   \end{pmatrix}=
   \begin{pmatrix}
         0\cr
        -{\p_\varphi\over kh\sqrt {k^2+1}}\cr
   \end{pmatrix},\quad S=\begin{pmatrix}0&0\cr 0 &{-1\over kh\sqrt {k^2+1}}\end{pmatrix}\,.
    \end{equation}
Calculate   second quadratic form:  $\r_{hh}=\p_h\r_h=
  \begin{pmatrix}
        0\\
        0\\
        0\\
   \end{pmatrix}\,$, $\r_{h\varphi}=\r_{\varphi h}=$
             $$
             \p_h
       \begin{pmatrix}
        -kh\sin\varphi\\
        kh\cos\varphi\\
          0\\
   \end{pmatrix}=\begin{pmatrix}
        -k\sin\varphi\\
        k\cos\varphi\\
          0\\
   \end{pmatrix},\,\,
   \r_{\varphi \varphi}=\p_\varphi
       \begin{pmatrix}
        -kh\sin\varphi\\
        kh\cos\varphi\\
          0\\
   \end{pmatrix}=\begin{pmatrix}
        -kh\cos\varphi\\
        -kh\sin\varphi\\
          0\\
   \end{pmatrix}\,.
                      $$
We have
                    \begin{equation}\label{secquadrformforcylinder}
            A_{\a\beta}=\langle\r_{\a\beta},\n\rangle,\,
              A=\begin{pmatrix}\langle\r_{hh},\n\rangle&\langle\r_{h\varphi},\n\rangle\cr
                               \langle\r_{\varphi h},\n\rangle &\langle\r_{\varphi\varphi},\n\rangle\cr
                                   \end{pmatrix}=
                                   \begin{pmatrix}0&0\cr
                                0&-{kh\over {\sqrt {1+k^2}}}\cr
                                   \end{pmatrix},
                    \end{equation}
                    $$
       A=GS=\begin{pmatrix}k^2+1&0\cr
                                0&k^2h^2\cr
                                   \end{pmatrix}
                                   \begin{pmatrix}0&0\cr
                                0&{-1\over kh\sqrt {k^2+1}}\cr
                                   \end{pmatrix}=
                                   \begin{pmatrix}0&0\cr
                                0&{-kh\over \sqrt {k^2+1}}\cr
                                   \end{pmatrix},
                    $$
For Gaussian and mean curvatures we have
     \begin{equation}\label{Gaussianforcylinder}
        K=\det S={\det A\over \det G}=\det
                              \begin{pmatrix}0&0\cr
                                0&{-1\over kh\sqrt {k^2+1}}\cr
                                   \end{pmatrix}=0\,,
     \end{equation}
     and mean curvature
      \begin{equation}\label{meanianforcylinder}
        H={\rm Tr\,} S={\rm Tr\,}
                              \begin{pmatrix}0&0\cr
                                0&{-1\over kh\sqrt {k^2+1}}\cr
                                   \end{pmatrix}={-1\over kh\sqrt {k^2+1}}\,.
     \end{equation}
Mean curvature is define up to a sign. If we change $\n\to-\n$ mean curvature $H\to {1\over R}$ and Gaussian curvature
will not change.


\bigskip




{\it Sphere}


\medskip


  Sphere is given by the equation $x^2+y^2+z^2=a^2$. Consider the  parameterisation
 of sphere in spherical coordinates
\begin{equation}\label{surfacesphere11}
  \r(\theta,\varphi)\colon\quad
  \begin{cases}
  x=R\sin\theta\cos\varphi\\
  y=R\sin\theta\sin\varphi\\
  z=R\cos\theta\\
  \end{cases}
\end{equation}

\medskip

We already calculated induced Riemannian metric on the sphere (see \eqref{firtsquadraticformforsphere(diff)}).
 Recall that
    $$
  \r_\theta=\begin{pmatrix}
        R\cos\theta\cos\varphi\\
        R\cos\theta\sin\varphi\\
        -R\sin\theta\\
   \end{pmatrix}\,,
\quad
  \r_\varphi=\begin{pmatrix}
        -R\sin\theta\sin\varphi\\
        R\sin\theta\cos\varphi\\
          0\\
   \end{pmatrix}
  $$
 and
            $$
              G_{S^2}=\left(dx^2+dy^2+dz^2\right)\big\vert_{x=R\sin\theta\cos\varphi,y=R\sin\theta\sin\varphi,
              z=R\cos\theta}=
                      $$
                      $$
                      (R\cos\theta\cos\varphi d\theta-R\sin\theta\sin\varphi d\varphi)^2+
                      (R\cos\theta\sin\varphi d\theta+R\sin\theta\cos\varphi d\varphi)^2+
                      $$
                      $$
                      (-R\sin\theta d\theta)^2=
          R^2\cos^2\theta d\theta^2+R^2\sin^2\theta d\varphi^2+R^2\sin^2\theta d\theta^2=
                      $$
        \begin{equation*}\label{firtsquadraticformforsphere(diff)}
           \qquad
             =R^2d\theta^2+R^2\sin^2\theta d\varphi^2\,,\qquad
                        ||g_{\a\beta}||=
   \begin{pmatrix}
   R^2 & 0 \\
   0&  R^2\sin^2\theta \\
   \end{pmatrix}\,.
                       \end{equation*}

For the sphere $\r(\theta,\varphi)$ is orthogonal to the surface.
       Hence
   normal unit vector
   $   \n(\theta,\varphi)=\pm {\r(\theta,\varphi)\over R}=\pm
   \begin{pmatrix}
    \sin\theta\cos\varphi\cr
     \sin\theta\sin\varphi\cr
      \cos\theta\cr
   \end{pmatrix}$.
           Choose $\n= {\r\over R}=
   \begin{pmatrix}
    \sin\theta\cos\varphi\cr
     \sin\theta\sin\varphi\cr
      \cos\theta\cr
   \end{pmatrix}.
     $
     Weingarten operator
     \begin{equation*}\label{weingrartenforcylinder}
        S\p_\theta=-\nabla^{\rm can.flat}_{\r_\theta}\n=-\p_\theta \n=
          -\p_\theta\left({\r\over R}\right)=-{\r_\theta\over R}\,,
       \end{equation*}
       \begin{equation*}\label{weingrartenforcylinder}
       S\p_\varphi=-\nabla^{\rm can.flat}_{\r_\varphi}\n=-\p_\varphi \n=
          -\p_\varphi\left({\r\over R}\right)=-{\r_\varphi\over R}\,.
    \end{equation*}
\begin{equation}\label{weingrartenforcylinder2}
           S
   \begin{pmatrix}
        \p_\theta \cr
        \p_\varphi\cr
   \end{pmatrix}=
   \begin{pmatrix}
         -{\p_\theta\over R}\cr
        -{\p_\varphi\over R}\cr
   \end{pmatrix},\quad S=
        -
   \begin{pmatrix}{1\over R}&0\cr 0 &{1\over R}\end{pmatrix}\,.
    \end{equation}
For second quadratic form:  $\r_{\theta\theta}=\p_\theta\r_\theta=
  \begin{pmatrix}
        -R\sin\theta\cos\varphi\\
        -R\sin\theta\sin\varphi\\
        -R\cos\theta\\
   \end{pmatrix}\,$, $\r_{\theta\varphi}=\r_{\varphi \theta}=$
             $$
             \p_\theta
       \begin{pmatrix}
        -R\sin\theta\sin\varphi\\
        R\sin\theta\cos\varphi\\
          0\\
   \end{pmatrix}=\begin{pmatrix}
        -R\cos\theta\sin\varphi\\
        R\cos\theta\cos\varphi\\
          0\\
   \end{pmatrix},\,\,
   \r_{\varphi \varphi}=\p_\varphi
       \begin{pmatrix}
        -R\sin\theta\sin\varphi\\
          R\sin\theta\cos\varphi\\
          0\\
   \end{pmatrix}=\begin{pmatrix}
        -R\sin\theta\cos\varphi\\
        -R\sin\theta\sin\varphi\\
          0\\
   \end{pmatrix}\,.
                      $$
We have
                    \begin{equation}\label{secquadrformforcylinder}
            A_{\a\beta}=\langle\r_{\a\beta},\n\rangle,\,
              A=\begin{pmatrix}\langle\r_{\theta\theta},\n\rangle&\langle\r_{\theta\varphi},\n\rangle\cr
                               \langle\r_{\varphi \theta},\n\rangle &\langle\r_{\varphi\varphi},\n\rangle\cr
                                   \end{pmatrix}=
                                   \begin{pmatrix}-R&0\cr
                                0&-R\sin^2\theta\cr
                                   \end{pmatrix},
                    \end{equation}
                    $$
       A=GS=\begin{pmatrix}R^2&0\cr
                                0&R^2\sin^2\theta\cr
                                   \end{pmatrix}
                                   \begin{pmatrix}{-1\over R}&0\cr
                                0&{-1\over R}\cr
                                   \end{pmatrix}=
                                       -R
                                  \begin{pmatrix}1&0\cr
                                0&\sin^2\theta\cr
                                   \end{pmatrix},
                    $$
For Gaussian and mean curvatures we have
     \begin{equation}\label{Gaussianforcylinder}
        K=\det S={\det A\over \det G}=\det
                              \begin{pmatrix}-{1\over R}&0\cr
                                0&-{1\over R}\cr
                                   \end{pmatrix}={1\over R^2}\,,
     \end{equation}
     and mean curvature
      \begin{equation}\label{meanianforcylinder}
        H={\rm Tr\,} S={\rm Tr\,}
                              \begin{pmatrix}-{1\over R}&0\cr
                                0&-{1\over R}\cr
                                   \end{pmatrix}=-{2\over R}\,,
     \end{equation}
Mean curvature is define up to a sign. If we change $\n\to-\n$ mean curvature $H\to {1\over R}$ and Gaussian curvature
will not change.


We see that for the sphere  Gaussian curvature is not equal to zero, whilst for cylinder and cone Gaussian curvature
equals to zero.





\end{document}
