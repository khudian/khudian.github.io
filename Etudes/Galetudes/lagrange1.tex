


 \magnification=1200

\def \finish {${\,\,\vrule height1mm depth2mm width 8pt}$}

   \centerline {\bf Lagrange Theorem}


            $$ $$


Let $K_n=K[x_1,x_2,\dots,x_n]$ be a ring of polynomials
 over field $K$ where $x_1,\dots,x_n$ are free variables
 (independent indeterminates)
 and $K(x_1,x_2,\dots,x_n)$ be a corresponding field of fractions.


 Let $G_n[x_1,\dots,x_n]$ be a subring of symmetrical polynomials.
 Consider basic symmetric polynomials
                   $$
           t_k=x_1^k+x_2^k+\dots+x_n^k\,,\quad k=1,2,3,\dots,n\,.
           \eqno (1)
                    $$
 All elements of subring $G_n[x_1,\dots,x_n]$ are polynomials
 on $t_1,\dots,t_n$.


  Every polynomial $F$ has its invariance subgroup $H_F\leq S_n$
  of the symmetric group $S_n$:
  permutation $\sigma\in S_n$ belongs to $H_F$ iff
  polynomial $F$ remains invariant under this permutation:
  $F^\sigma=F$.

\medskip

    {\bf Theorem} (Lagrange): Let $F=F(x_1,\dots,x_n), G=G(x_1,\dots,x_n)$
     be two polynomials in $K[x_1,\dots,x_n]$
    such that $H_F$ is a subgroup of $H_G$. Then polynomial $G$
    can be considered as rational function on symmetric polynomials
    $t_1,\dots,t_n$  and polynomial $F$.
    More formally there exists a fraction
    $P(z,t_1,\dots,t_n)\over Q(z,t_1,\dots,t_n)$
    such that
                           $$
                G(x_1,\dots,x_n)=
                {P(z,t_1,\dots,t_n)\over Q(z,t_1,\dots,t_n)}
                  \big\vert_{z=F(x_1,\dots,x_n),t_k=x_1^k+x_2^k+\dots+x_n^k}
                  \eqno (2)
                $$

     {\it Proof}:


      Expose an explicit formula:
       Consider the following fraction:
                            $$
                   U(z,x_1,\dots,x_n)=
                   {G(x_1,\dots,x_n)\over z-F(x_1,\dots,x_n)}
                   \eqno (3)
                            $$
     Take the average with respect of an action of all permutations of symmetric group
     $S_n$ on indeterminates $x_1,x_2,\dots,x_n$:
                        $$
             U_{\rm average}(z,x_1,\dots,x_n)=
              {1\over n!}\sum_{\sigma\in S_n}U^\sigma(z,x_1,\dots,x_n)=
              {1\over n!}
              \sum_{\sigma\in S_n}
              {G^\sigma(x_1,\dots,x_n)\over z-F^\sigma(x_1,\dots,x_n)}
                          $$

 Because of symmetrization  $U_{\rm average}(z,x_1,\dots,x_n)$
 is a fraction such that its nominator and denominator
 can be considered as polynomials over symmetric polynomials
 $t_1,t_2,\dots,t_n$:
                   $$
 U_{\rm average}(z,x_1,\dots,x_n)=
{P(z,t_1,\dots,t_n)\over Q(z,t_1,\dots,t_n)}
                  \big\vert_{t_k=x_1^k+\dots+x_n^k}.
                   \eqno (4)
                  $$
            where
                          $$
         Q\left(z,t_1,\dots,t_n\right)\big\vert_{t_k=x_1^k+\dots+x_n^k}=
         \prod_{r=1}^l \left(z-F^{\sigma_r}(x_1,\dots,x_k)\right)\,,
                       \eqno (5)
                          $$
 $\{\sigma_1,\dots\sigma_l\}, (l={n!\over |H_F|})$ are any representatives
 of left equivalence classes of the group $S_n$ with respect
  to the subgroup $H_F$
 \footnote{$^*$}{Transformations $\sigma\in H_F$ do not change denominator
$z-F(x_1,\dots,x_n)$ of the fraction. Hence for every $g\in S_n$,
$g=h\sigma_r$, where $h\in GH_F$ and
 $z-F^{h\sigma_r}=z-F^{\sigma_r}$}.

  One can see that
                $$
                G(x_1,\dots,x_n)=
                 {n!\over |H_F|}Res \left({P(z,t_1,\dots,t_n)\over Q(z,t_1,\dots,t_n)}\right)
                      \big\vert_{z=F(x_1,\dots,x_n),t_k=x_1^k+x_2^k+\dots+x_n^k}=
                      \eqno (6)
                       $$
                       $$
 {n!\over |H_F|}\left({P(z,t_1,\dots,t_n)\over Q'(z,t_1,\dots,t_n)}\right)
  \big\vert_{z=F(x_1,\dots,x_n),t_k=x_1^k+x_2^k+\dots+x_n^k}
   \eqno (7)
                         $$
This is just what we claimed in (2).
 To prove this relation we note that


 1)
  every permutation $\sigma\in S_n$
 preserves polynomial $G$ if it preserves polynomial $F$:
                   $$
               F^\sigma=F\Rightarrow  G^\sigma=G,\quad{\rm because}\quad H_F\leq H_G
                   $$

 2) Residue of the function $1\over z-F^\sigma$ in the point $z=F$
 is equal to one if $\sigma\in H_F$ and it is equal to zero if $\sigma\not\in H_F$:
                 $$
               Res \left({1\over z-F^\sigma}\right)\big\vert_{z=F}=
                      \cases
                      {
                      1\quad {\rm if} \sigma\in H_F\cr
                      0\quad
                       {\rm if} \sigma\not\in H_F\cr
                      }
                      $$

  \medskip

 Hence
                  $$
                     {n!\over |H_F|}    Res
                     \left({P(z,t_1,\dots,t_n)\over Q(z,t_1,\dots,t_n)}\right)
                      \big\vert_{z=F(x_1,\dots,x_n),t_k=x_1^k+x_2^k+\dots+x_n^k}
               $$
               $$
              {1\over |H_F|}Res
 \left(\sum_{\sigma\in S_n}
 {G^\sigma(x_1,\dots,x_n)\over z-F^\sigma(x_1,\dots,x_n)}
    \right)\big\vert_{z=F(x_1,\dots,x_n)}=
              $$
              $$
       {1\over |H_F|}
        \sum_{\sigma\in H_F}
       G^\sigma(x_1,\dots,x_n)={1\over |H_F|}
        \sum_{\sigma\in H_F}
       G(x_1,\dots,x_n)=
              G(x_1,\dots,x_n)\hbox{\finish}
              $$


\medskip

{\bf Example}

Consider
                   $$
                   F=x_1^2+x_2^2, G=x_3
                   $$

 Then $H_F=H_G$. Express $G=x_3$ via $F=x_1^2+x_3^2$ and $t_1,t_2,t_3$. According to (2-7):
                   $$
    U_{\rm average}={1\over 6}\sum_{\sigma\in S_3}
            {x_3^\sigma\over z-(x_1^2+x_2^2)^\sigma}=
            $$
               $$
 {1\over 3}\left({x_3\over z-(x_1^2+x_2^2)}+
 {x_2\over z-(x_1^2+x_3^2)}+{x_1\over z-(x_2^2+x_3^2)}\right)=
               $$

For simplicity assume that $x_1,x_2,x_3$ are roots of the equation
$x^3+x-1=0$, i.e. $t_1=x_1+x_2+x_3=0$, $t_2=x_1^2+x_2^2+x_3^2=-2$
and $t_3=x_1^3+x_2^3+x_3^3=3$
(In this case we have simplification:
$x_1^2+x_2^2=-x_3^2-2$ but still calculations are not very quick)
                 $$
            U_{\rm average}={1\over 3}
   \left({x_3\over z+(x_3^2+2)}+
 {x_2\over z+(x_2^2+2)}+{x_1\over z+(x_1^2+2)}\right)=
   {-3z-5\over z^3+4z^2+5z+3}
               $$
and according to (7)
             $$
    x_3=Res \left({-3z-5\over z^3+4z^2+5z+3}\right)\big\vert_{z=x_1^2+x^2_2}=
             \left({-3z-5\over 3z^2+8z+5}\right)\big\vert_{z=x_1^2+x^2_2}=
              $$
              $$
           {-3(x_1^2+x_2^2)-5\over 3(x_1^2+x_2^2)^2+8(x_1^2+x_2^2)+5}
              $$

One can see that indeed it is a fact.

  \bye
