\magnification=1200 \baselineskip=14pt



\def \finish {${\,\,\vrule height1mm depth2mm width 8pt}$}



\def\p {\partial}
\def \D {\Delta_{d{\bf v}}}
\def \Ds  {\Delta^{\#}}
\def\t{\tilde}
\def\s {\sigma}
\def\vare {\varepsilon}
\def\L {\Lambda}
\def\Darboux {$z^A=$  $x^1,\dots,x^n$, $\theta_1,\dots,\theta_n$}
\def\a{\alpha}
\def\O{\Omega}
\def\d{\delta}
\def\dv  {{d{\bf{v}}}}
\def\A {{\cal A}}
\def\R {I\!R}
\def\t {\tilde}
\def\l {\lambda}
\def\e {{\bf e}}
\def\x {{\bf x}}
\def\y {{\bf y}}



{\it 12 March 2016}

             \centerline {\bf
Cubic and quadric equations; Galois theory for pedestrians}

       \centerline{   H.M. Khudaverdian}

{\it This \'etude  is 
written on the base of the book of A. Khovansky
"Galois Theory" and it is inspired 
by the lecture `Galois Lecture' for students 
on 2-nd march 2016 and by the discussion with 
R. Mkrtchyan in December 2015 of
 quantum mechanical interpretation of roots of Lie algebra,}




 The content of this \'etude is the following: Let
  $H$ be an abelian normal 
subgroup of group $S_n$ of permutations of $n$ elements.
 (Instead $S_n$ one may consider an arbitrary Galois group $G$,
but for clarity we consider just a group $S_n$.)
  We suppose that $S_n$ acts on the space of polynomials $\Sigma^{(n)}$
of $n$ variables $x_1,x_2,\dots,x_n$.)
       $$
\Sigma^{(n)}={\bf C}[x_1,\dots,x_n]\,.
       $$


Then we can perform the following constructions.

Consider an arbitrary element $h\in H$ of this group.
The corresponding linear operator acting on space $\Sigma^{(n)}$
is diagonalisable, since $h^N=1$. Moreover all elements
of the group $H$ can be diagonalised 
simultaneously since $H$ is an abelian group.
More precisely this means that 
one can consider the decomposition of
space $\Sigma=\Sigma^{(n)}$ of polynomials on $n$ variables
 on linear subspaces over characters of group $H$:
       $$
\Sigma=\oplus_{\l\in \hat H}\Sigma^{(n)}_{\l}
       $$
such that if $\l\in \hat H$ is an arbitrary
character of $H$, then  an arbitrary polynomial $P\in\Sigma^{(n)}_\l$
is an eignevector of all elements of $h$ with eigenvalues
  $\l(h)$,
        $$
  h P=\lambda(h)P\,.
       $$
(Here $\hat H$ is a dual group of group $H$. it is 
a  group of characters of group $H$
\footnote{$^{1)}$}{Groups
  $\hat H$ and $H$ are both abelian groups with 
same numebr of elements,
but in general they {\it are not isomorphic}.}).
One can say that
all elements of group $H$ are commuting observables,
and they are simultaneously measurable.

Denote by $\Sigma^{(n)}_H$ the subspace of $H$-invariant polynomials
(this is subspace corresponding to character $\lambda\equiv 1$.).
All  characters are taking values in roots of unity, i.e.
for an arbitrary polynomial $P\in \Sigma^{(n)}_\l$,
there exists an  integer $N$ such that the polynomial 
$P^N$ belongs to the space $\Sigma_H$.
 Thus we come to conclusion:

  {\it An arbitrary polynomial in $\Sigma^{(n)}$
is a sum of roots of polynomials in $\Sigma_H$.}


\medskip

{\sl Now concetrate on the question how to calculate 
$H$-invariant polynomials,
ie. polybnomials in $\Sigma_H$.}

Now suppose that $H$ is an invariant subgroup in group $S_n$.
In this case  the smaller group $S_n\backslash H$ acts on the
 space $\Sigma_H$, i.e. $H$-invariant polynomials
 are roots of polynomial with smaller Galois group; if
$S_n$ is Galois group of initial polynbomial, then Galois group
acting on $H$-invariant polynomials becomes  $G= S_n\backslash H$.
  These considerations explain why if Galois group is solvable,
then the roots  of polynomial
are expressed by taking operation of roots\footnote{$^{2)}$}
{here the word `root' I use in two different meanings: `root of polynomial'
and `operation of taking root'.}.
In particular for
  $n=2,3,4$ symmetric groups (groups of all permutations) 
$S_2,S_3,S_4$ are solvable
\footnote{$^{3)}$}{The abelian group is solvable. The group
$G$ is solvable if it possesses abelian normal subgroup
such that factor is solvable. In particular $S_3$
is solvable since $S_3\backslash C_3=S_2$  is abelian,
where $C_3$ is cyclic subgroup.
For $S_4$ one can consider abelian normal subgroup
$KI$ generated by permutations $(12)(34)$ and $(13)(24)$
(see details later in the text).
The factor is group $S_3$. Hence $S-4$ is solvable also.}.
We come to the formulae which express polynomials
in $S_n$ via $S_n$-invariant polynomials for $n=2,3,4,$ i.e.,
solving cubic and quartic equations in radicals.


\bigskip
{\sl We will perform the scheme described above for quadratic, 
cubic and quatric polynomials.}
        \centerline { quadratic equation  $n=2$}


Group $S_2$ is abelian $S_2=\{1,\sigma\}$, $\sigma^2=1$.
  It has two characters:
         $$
     \matrix
         {
\l_I\equiv 1\cr
 \l_{II}\colon\quad
   \l_I(1)=1\,,  \l_{II}(\sigma)=-1\cr
            }\,,\quad
      \hat {S_2}=\{\l_I,\l_{II}\}\,. 
         $$
For an arbitrary polynomial $P\in \Sigma^{(2)}$, $P=P(x_1,x_2)$,
we have
             $$
P=P_I+P_{II}=
\underbrace{P+\sigma P\over 2}_{\hbox{even polynomial}}+
\underbrace{P+\sigma P\over 2}_{\hbox{odd polynomial}}
             $$
($(\s P)(x_1,x_2)=P(x_2,x_1)$),

The decomposition of the space of polynomials is
      $$
\Sigma^{(2)}=\Sigma^{(2)}_{\l_I}+
           \Sigma^{(2)}_{\l_{II}}\,.
      $$ 





If $x_1+x_2=-p$, $x_1x_2=q$ ($x_1,x_2$ are roots of polynomial
$x^2+px+q$) then  every even polynomial 
is $S_2$-invariant, i.e. it
is polynomial
on $p,q$. For every odd polynomial
its square is $S_2$-invariant also,
i.e.   and odd polynomial is square root of polynomial on $p,q$.
In particular for polynomial $P=x_1$ we have
      $$
x_1={x_1+x_2\over 2}
+{x_1-x_2\over 2}=
      {x_1+x_2\over 2}\pm
\sqrt{\left({x_1-x_2\over 2}\right)^2}=
        $$
       $$
      {x_1+x_2\over 2}\pm
\sqrt{\left({x_1+x_2\over 2}\right)^2-x_1x_2}=
   -{p\over 2}+
\sqrt{{p^2\over 4}-q}\,.
      $$


\bigskip

  \centerline{Cubic equation $n=3$}

Group $S_3$ contains  abelian normal subgroup 
$C_3=\{1,s,s^2\}$, where $s=(123)$.


Abelian subgroup $C_3$ has following three characters:
         $$
       \matrix
         {
&\l_0\equiv 1\cr 
&\l_{I}\colon\quad  \l_{I}(1)=1\,, \l_{I}(s)=\vare\,,\l_{I}(s^2)=\vare^2\cr
&\l_{II}\colon\quad  \l_{II}(1)=1\,, 
  \l_{II}(s)=\vare^2\,,\l_{II}(s^2)=\vare\cr
          }\,, \quad 
\hbox{where $\vare=e^{2\pi i\over 3}$.}\,,\quad
       \,,
         $$
that is the group $\hat C_3$ of characters is 
$\hat {C_3}=\{\l_0,\l_I,\l_{II}\}$.

 For an arbitrary polynomial $P\in \Sigma^{(3)}$, $P=P(x_1,x_2,x_3)$
we have
             $$
P=P_0+P_I+P_{II}=
\underbrace{P+ (s P)+(s^2 P)\over 3}_{\hbox{eigenvalues $(1,1,1)$}}+
\underbrace{P+ \vare^2 (s P)+\vare (s^2 P)\over 3}_
           {\hbox{eigenvalues $(1,\vare,\vare^2)$}}+
\underbrace{P+ \vare s P+\vare^2 (s^2 P)\over 3}
  _{\hbox{eigenvalues $(1,\vare^2,\vare)$}}\,
             $$
In details:
 $(sP)(x_1,x_2,x_3)=P(x_2,x_3,x_1)$,
 the polynomials $P_I, P_{II}$ are eigenvectors such that
           $$
         \matrix
            {
 sP_I=\l_I(s)P_I=\vare P_I\,,
 s^2P_I=\l_I(s^2)P_I=\vare^2 P_I\cr
 sP_{II}=\l_{II }(s)P_I=\vare^2 P_{II}\,,
 s^2P_{II}=\l_{II}(s^2)P_{II}=\vare P_{II}\cr
        }
           $$
The decomposition of spaces is:
      $$
\Sigma^{(3)}=\Sigma^{(3)}_{\l_{0}}+
             \Sigma^{(3)}_{\l_{I}}+
              \Sigma^{(3)}_{\l_{II}}\,.
      $$ 
The subspace $\Sigma_{\l_0}$ is subspace of $C_3$-invariant polynomials.


 The cube of every polynomial in $\Sigma^{(3)}_{I}$ or in
$\Sigma^{(3)}_{II}$ is $C_3$-invariant polynomial. Hence 
every polynomial can be expressed via $C_3$-invariant polynomials
with use of operation of taking cubic roots.

  Now concetratae on  $C_3$-invariant polynomials.
  On the space $\Sigma^{(3)}_{C_3}$ 
of $C_3$-invariant polynomials acts factor-group
        $$
      S_3\backslash C_3=S_2
        $$
 i.e. $C_3$ invariant polynomials are roots of quadratic equation!

Now if we consider polynomial $P=x_1$ we come to the formula for cubic roots.



Perform calulations

Suppose that $x_1+x_2+x_3=-a$, $x_1x_2+x_1x_3+x_2x_3=p$
and $x_1x_2x_3=-q$ i.e.
$x_1,x_2,x_3$ are roots of polynomial
$x^3+ax^2+px+q$.   According to decomposition formula  
 we have:
        $$
x_1=({x_1})_0+({x_1})_I+({x_1})_{II}=
\underbrace{x_1+ x_2+x_3\over 3}_{\hbox{eigenvalue $1$}}+
\underbrace{x_1+ \vare^2 x_2+\vare x_3\over 3}_{\hbox{eigenvalue $\vare$}}+
\underbrace{x_1+ \vare x_2+\vare^2 x_3\over 3}_{\hbox{eigenvalue $\vare^2$}}+
        $$
(We write down here eigenvalue of operator $s$.)
	The first expression is obviously not only $C_3$-invariant
but it is  $S_3$-invariant also:
$({x_1})_0={x_1+ x_2+x_3\over 3}=-{a\over 3}$.
Later  for simplicity without loss of generality 
we assume later than $a=x_1+x_2+x_3=0$
(changing $x_i\mapsto x_i-{a\over 3}$).

Denote $w_I=({x_1})_{I}$ and
$w_{II}=({x_2})_{II}$.
The cubes of expressions $w_I=({x_1})_{I}$ and
$w_{II}=({x_2})_{II}$ are eigenvectors with eigenvalue $1$, hence
they are $C_3$-invariant. Hence the group
  $S_3\backslash C_3=S_2$ acts on these numbers, i.e. they are roots
of quadratic equation:
  $[(12)]w^3_I=w^3_{II}$.

$C_3$-invariant 
polynomails $w^3_{I}+w^3_{II}$ and $w^3_{I}w^3_{II}$
are invariant with respect to the action of factorgroup 
$S_2=S_3\backslash C_3$, i.e. these polynomials 
are $S_3$ invariant polynomials, i.e. they are expressed 
via coefficients:
 we have after long but simple calculations that
       $$
w_I^3+w_{II}^3=
\left({x_1+ \vare^2 x_2+\vare x_3\over 3}\right)^3+
\left({x_1+ \vare x_2+\vare^2 x_3\over 3}\right)^3=-q
        $$
and    $$
w_I^3\cdot w_{II}^3=
\left({x_1+ \vare^2 x_2+\vare x_3\over 3}\right)^3
\left({x_1+ \vare x_2+\vare^2 x_3\over 3}\right)^3=-27p^6
        $$

Hence

       $$
x_1=w_0+w_{I}+w_{II}=
 \root 3\of {w_1}+\root 3\of {w_2}=
  \root 3\of {-{q\over 2}+\sqrt{{q^2\over 4}+{p^3\over 27}}}
  +\root 3\of {-{q\over 2}-\sqrt{{q^2\over 4}+{p^3\over 27}}}
       \eqno (\dagger)
       $$



{\bf Remark}
The question what branch of cubic root to choose can be answered
if we note that $w_I w_{II}$ is $S_3$ invariant under the action of $S_3$.







\bigskip

  \centerline{Quartic  equations $n=4$}

First explain why and how we choose ableian subbgroup in $S_4$.

Consider platonic body, tetrahedron $A_1A_2A_3A_4$. On vertices
of this tetrahedron acts group $S_4$.


Let

\noindent 
$E_1$ be a middle point of the segment $A_1A_2$,

\noindent 
$F_1$ be a middle point of the segment $A_3A_4$

\noindent $E_2$ be a middle point of the segment $A_1A_3$

\noindent 
$F_2$ be a middle point of the segment $A_2A_4$

\noindent $E_3$ be a middle point of the segment $A_1A_4$

\noindent
$F_3$ be a middle point of the segment $A_2A_3$
    
  Consider the cross  formed by segments 
$l_1=E_1F_1, l_2=E_2F_2,l_3=E_3F_3$,
and consider the subgroup of all 
permutations of vertices of the tetrahedron, 
such that the cross remains intact:
 They will be permuttions
       $a=(12)(34)$, $b=(13(24)$ and permutation
$ab=(14)(23)$. We come to abelian group:
           $$
KI=\{1,a,b,ab\}
           $$
 It is normal subgroup 
since it preserves the cross $l_1l_2l_3$ in tetraedron $A_1A_2A_3A_4$
Factorgroup
 $S_4\backslash KI$ acts
on the cross. It is group of permutations of edges of CROSS,
i.e. it is $S_3$. 
 We come to 
              $$
        S_4\backslash KI=S_3\,.
            $$
 Since we know that group $S_3$ is solvable ($S_3\backslash C_3=C_2$), 
hence
$S_4$ is also solvable. 
Now perform calculations according our scheme.

Abelian subgroup $KI$ of $S_4$ 
has following four characters:
         $$
       \matrix
         {
&\l_0\equiv 1\cr 
&\l_{I}\colon\quad  \l_{I}(1)=1\,, 
\l_{I}(a)=1\,,\l_{I}(b)=-1\,,\l_I(ab)=-1\cr
&\l_{II}\colon\quad  
   \l_{II}(1)=1\,, 
  \l_{II}(a)=-1\,,\l_{II}(b)=1\,,\l_{II}(ab)=-1\cr
&\l_{III}\colon\quad  
   \l_{III}(1)=1\,, 
  \l_{III}(a)=-1\,,\l_{III}(b)=-1\,,\l_{III}(ab)=1\cr
          }\,, \qquad 
\hbox{since $a^2=b^2=1$.}\,,
         $$
i.e. group of characters of $KI$ is 
$\hat {KI}=\{\l_0,\l_{I},\l_{II},\l_{III}\}$.
Respectively
for an arbitrary polynomial of roots,
$P\in \Sigma^{(4)}$, $P=P(x_1,x_2,x_3,x_4)$
we have
             $$
P=P_0+P_I+P_{II}+P_{III}=
          $$
          $$
\underbrace{P+ (a P)+(b P)+(ab P)\over 4}_{\hbox{eigenvalues $(1,1,1,1)$}}+
\underbrace{P+ (a P)+(b P)+(ab P)\over 4} _
           {\hbox{eigenvalues  $(1,1,-1,-1)$}}+
         $$
         $$
\underbrace{P-(a P)+(b P)-(ab P)\over 4} _
           {\hbox{eigenvalues  $(1,1,-1,-1)$}}+
\underbrace{P-(a P)-(b P)+(ab P)\over 4} _
           {\hbox{eigenvalues  $(1,-1,-1,-1)$}}+
                 \,
             $$
In details:

 $(aP)(x_1,x_2,x_3,x_4)=P(x_2,x_1,x_4,x_3)$,

\noindent  $(bP)(x_1,x_2,x_3,x_4)=P(x_2,x_1,x_4,x_3)$,
 
\noindent $(bP)(x_1,x_2,x_3,x_4)=P(x_3,x_4,x_1,x_2)$,
            $$
         \matrix
            {
       aP_0=\l_0(a)P_0=P_0\,, bP_0=\l_0(b)P_0\,,
        abP_0=\l_0(ab)P_0=P_0\cr
       aP_{I}=\l_{I}(a)P_{I}=P_{I}\,, bP_{I}=\l_{I}(b)P_{I}=-P_{I}\,,
        abP_I=\l_{I}(ab)P_I=-P_I\cr
       aP_{II}=\l_{II}(a)P_{II}=-P_{I}\,, 
       bP_{II}=\l_{II}(b)P_{II}=P_{II}\,,
        abP_{II}=\l_{II}(ab)P_{II}=-P_{II}\cr
       aP_{III}=\l_{III}(a)P_{III}=-P_{III}\,, 
       bP_{III}=\l_{III}(b)P_{III}=-P_{III}\,,
      abP_{III}=\l_{III}(ab)P_{III}=P_I\cr
        }\,.
           $$
Polynomial $P_0$ is $KI$-invariant polynomial,
all other polynomials are not $KI$ invariants
but their squares are. 
The decomposition of spaces is:
      $$
\Sigma^{(4)}=\Sigma^{(4)}_{\l_{0}}+
             \Sigma^{(4)}_{\l_{I}}+
              \Sigma^{(4)}_{\l_{II}}+
              \Sigma^{(4)}_{\l_{III}}\,.
      $$ 
The subspace $\Sigma_{0}$ is subspace of $K4$-invariant polynomials.


 The square of every polynomial in $\Sigma^{(4)}_{I}$ or in
$\Sigma^{(4)}_{II}$ or in
$\Sigma^{(4)}_{III}$ is $KI$-invariant polynomial. 
Hence we see that
every polynomial can be expressed via $KI$-invariant polynomials
with use of operation of quadratic roots 
$\sqrt{}$.

  On the space of $KI$-invariant polynomials acts group
        $$
      S_4\backslash C_3=S_3
        $$
 i.e. $KI$ invariant polynomials are roots of cubic polynomials.!

Now if we consider polynomial $P=x_1$ we come to the formula for 
roots of quartic polynomials.



Perform calculations

Suppose that $x_1+x_2+x_3+x_4=-a$, $x_1x_2+x_1x_3+x_2x_3+\dots=p$
and $x_1x_2x_3+dots=-q$, $x_1x_2x_3x_4=r$ i.e.
$x_1,x_2,x_3$ are roots of polynomial
$x^4+ax^3+p2+qx+r$.   According to decomposition formula  
 we have:
        $$
x_1=({x_1})_0+({x_1})_I+({x_1})_{II}+({x_1})_{III}=
        $$
        $$
\underbrace{x_1+ x_2+x_3+x_4\over 4}_{\hbox{all eigenvalues $1$}}+
\underbrace{x_1+ x_2-x_3-x_4\over 4}_{\hbox{eigenvalues $(1,1,-1,-1)$}}+
          $$
        $$
\underbrace{x_1- x_2+x_3-x_4\over 4}_{\hbox{ eigenvalues $(1,-1,1,-1)$}}+
\underbrace{x_1- x_2-x_3+x_4\over 4}_{\hbox{eigenvalues $(1,-1,-1,1)$}}
        $$
Denote by
       $$
u_0={{x_1+ x_2+x_3+x_4\over 4}}\,,\quad
u_{I}={{x_1+ x_2-x_3-x_4\over 4}}\,,
       $$
       $$
u_{II}={{x_1-x_2+x_3-x_4\over 4}}\,,\quad
u_{III}={{x_1-x_2-x_3+x_4\over 4}}\,.
       $$
Polynomial $w_0$ is not only $KI$-invariant it is $S_4$-invariant--
$u_0=-a$.   Squares of all other polynomials are $KI$-invarianbt
polynomials,i.e.
on polynomials $v_{I}=u^2_{I}, v_{II}=u^2_{II}, v_{III}=u^2_{III}$
acts the factor group $S_4/KI=S_3$.
 hence they are roots of cubic polynomial
(with coefficeints which are polynomials on $a,p.q.r$).

We see finally that root $x_1$ is expressed via $u_0,u_I,u_{II},
u_{III}$ via square root operations, and these numbers
being roots of cubic equation
are expressed via coefficients of polynomials by taking
 operation of suqre and cube roots.\finish 

\bye
