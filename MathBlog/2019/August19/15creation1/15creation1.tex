

\magnification=1200
\baselineskip=14pt
\def\vare {\varepsilon}
\def\t {\tilde}
\def\a {\alpha}
\def\K {{\bf K}}
\def\N {{\bf N}}
\def\C {{\bf C}}
\def\L {{\cal L}}
\def\E {{\bf E}}
\def\s {{\sigma}}
\def\S {{\cal S}}
\def\SS {{\Sigma}}
\def\p{\partial}
\def\vare{{\varepsilon}}
\def\Q {{\bf Q}}
\def\D {{\cal D}}
\def\G {{\Gamma}}
\def\Z {{\bf Z}}
\def\R  {{\bf R}}
\def\l {\lambda}
\def\ll {{\bf l}}
\def\degree {{\bf {\rm degree}\,\,}}
\def \finish {${\,\,\vrule height1mm depth2mm width 8pt}$}
\def \m {\medskip}
\def\p {\partial}
\def\r {{\bf r}}
\def\pt {{\bf p}}
\def\v {{\bf v}}
\def\n {{\bf n}}
\def\t {{\bf t}}
\def\b {{\bf b}}
\def\c {{\bf c }}
\def\e{{\bf e}}
\def\f{{\bf f}}
\def\ac {{\bf a}}
\def \X   {{\bf X}}
\def \Y   {{\bf Y}}
\def \x   {{\bf x}}
\def \y   {{\bf y}}
\def\w {{\omega}}
\def \Tr  {{\rm Tr\,}}
\def\dim {{\rm dim\,\,}}
% I wrote this file on 16 January
\def\t {{\tilde}} 
\def\dist {{\hbox{\tt "distance"}}}
\def  \dim {{\rm dim\,}}
\def  \Im  {{\rm Im\,}}
\def  \ker {{\rm ker\,}}


\def \Cl {\hbox{\tt Cliff}}
\def\F {\cal F}

\centerline {\bf Creation and annihilation operators}

Let $L$ be Hilbert space, and $M$ a space with meausre
such that $L=L^2(M)$.

   We define
    generalised operator function
           $$
  a(\xi)=
      $$
     $$
 \pmatrix
   {
&0  &\delta (y,\xi)   &0      &0 &\dots \cr
&0  &0  &\sqrt 2 \delta (x_1,y_1)\delta(y_2,\xi)   &0
       &0   &\dots\cr
&0   &0  & 0 &\sqrt 3  
\delta (x_1,y_1)\delta(x_2,y_2)\delta (y_3,\xi)
        &\dots\cr
&\dots
&\dots
&\dots
&\dots
&\dots
\cr
   }
           $$

and its adjoint
           $$
  a^*(\xi)=
      $$
     $$
 \pmatrix
   {
&0  &0  &0  &\dots  \cr
&\delta(x,\xi)  &0 &0  &\dots \cr
&0 &\sqrt 2\delta(x_1,y_1)\delta(x_2,\xi) 
 &0 &\dots\cr
&0  &0 &\sqrt 3
  \delta(x_1,y_1)
  \delta(x_2,y_2)
  \delta(x_3,\xi)
   &\dots \cr
&\dots &\dots& \dots&\dots\cr
    }
           $$
If $f=f(\xi)$ is function in $L$ then
       $$
a_f=\int a(\xi)f(\xi)d\xi=
       $$
  $$
 \pmatrix
   {
&0  &f(y)   &0      &0 &\dots \cr
&0   &\sqrt 2 \delta (x_1,y_1)f(y_2)   &0
       &0   &\dots\cr
&0   &0  &\sqrt 3  
\delta (x_1,y_1)\delta(x_2,y_2)f(y_3)
 &0        &\dots\cr
&\dots
&\dots
&\dots
&\dots
&\dots
&\dots
\cr
   }
           $$
and
     $$
a^*_f=\int a^*(\xi)f(\xi)d\xi=
       $$
     $$
 \pmatrix
   {
&0  &0  &0  &\dots  \cr
&f(x)  &0 &0  &\dots \cr
&0 &\sqrt 2\delta(x_1,y_1)f(x_2) 
 &0 &\dots\cr
&0  &0 &\sqrt 3
  \delta(x_1,y_1)
  \delta(x_2,y_2)
  f(x_3)
   &\dots \cr
&\dots &\dots& \dots&\dots\cr
    }
           $$
and  
     $$
a_f\Psi=\pmatrix 
     {
 \int K_1(y)f(y)dy\cr
 \sqrt 2 \int K_2(x_1,y)f(y)dy\cr
 \sqrt 3 \int K_3(x_1,x_2,y)f(y)dy\cr
   \dots\cr   
     }\,\,,
a^*_f\Psi=\pmatrix 
     {
     0\cr
    f(x_1)K_0\cr
 \sqrt 2  K_1(x_1)f(x_2)\cr
 \sqrt 3  K_2(x_1,x_2)f(x_3)\cr
   \dots\cr   
     }
       $$
for the vector  $\Psi$
       $$
    \Psi=
   \pmatrix
  {
  K_0\cr
  K_1(x_1)\cr
  K_2(x_1,x_2)\cr
  K_3(x_1,x_2,x_3)\cr
 \dots \cr
    }\,, 
     $$
in  $\F$. 


Now project this on the subspaces
of symmetric and antisymmetric functions.

Let $P_B$ be 
the projection of $\F$ on subspace $\F_B$
of symmetric states, and 
 $P_F$ be 
the projection of $\F$ on subspace $\F_F$
of antisymmetric states. 

{\tt We define now creation and annihilation 
operators}
      $$
a_B(f)=P_Ba_fP_B\,,\,
a^*_B(f)=P_Ba^*_fP_B\,,\,
a_F(f)=P_Fa_fP_F\,,\,
a^*_F(f)=P_Fa^*_fP_F\,,\,
      $$
One can see that
          $$
\left[a_B(f),a_B(g)\right]=
\left[a^*_B(f),a^*_B(g)\right]=0\,,\quad
{\rm and}
\,\,
\left[a_B(f),a_B^*(g)\right]=\int f(x)g(x)dx
          $$


Compare with operators
    $$
  a|N>=\sqrt {N}|N-1>
  a^+|N>=\sqrt {N+1}|N+1>
    $$
Calculate commutators for bosons.


We have that
     $$
 a_B(f)a_B^*(g)
   \pmatrix
       {
  K_0\cr
  K_1(x_1)\cr
  K_2(x_1,x_2)\cr
  K_3(x_1,x_2,x_3)\cr
  K_4(x_1,x_2,x_3,x_4)\cr
 \dots \cr
    }
    =a_B(f)P_B
   \pmatrix
       {
      0\cr
  K_0g(x_1)\cr
  \sqrt 2K_1(x_1)g(x_2)\cr
  \sqrt 3 K_2(x_1,x_2)g(x_3)\cr
  2 K_3(x_1,x_2,x_3)g(x_4)\cr
  \sqrt 5 K_4(x_1,x_2,x_3,x_4)g(x_5)\cr
 \dots \cr
    }
    =
      $$
      $$
   a_B(f)
   \pmatrix
  {
    0\cr
  g(x_1)K_0\cr
  {\sqrt 2} 
 \left(
{K_1(x_1)g(x_2)+K_1(x_2)g(x_1)\over 2}
  \right)\cr
  {\sqrt 3} 
 \left(
     {
 K_2(x_1,x_2)g(x_3)+
 K_2(x_1,x_3)g(x_2)+
 K_2(x_3,x_2)g(x_1)
   \over 3}
     \right)\cr
  {\sqrt 4} 
 \left(
     {
 K_3(x_1,x_2,x_3)g(x_4)+
 K_3(x_1,x_2,x_4)g(x_3)
 K_3(x_1,x_4,x_3)g(x_2)+
 K_3(x_4,x_2,x_3)g(x_1)+
   \over 4}
     \right)\cr
\dots \cr
    }=
     $$
     $$
   \pmatrix
  {
  K_0\int f(y)g(y)dy\cr
 \sqrt 2 {\sqrt 2} 
 \int \left(
{K_1(x_1)g(y)+K_1(y)g(x_1)\over 2}
  \right)f(y)dy\cr
  \sqrt 3{\sqrt 3} 
      \int 
 \left(
     {
 K_2(x_1,x_2)g(y)+
 K_2(x_1,y)g(x_2)+
 K_2(y,x_2)g(x_1)
   \over 3}
     \right)
     f(y)dy\cr
  \sqrt 4{\sqrt 4} 
        \int
 \left(
     {
 K_3(x_1,x_2,x_3)g(y)+
 K_3(x_1,x_2,y)g(x_3)
 K_3(x_1,y,x_3)g(x_2)+
 K_3(y,x_2,x_3)g(x_1)+
   \over 4}
     \right)f(y)dy\cr
\dots \cr
    }=
   $$
     $$
   \left(
   \matrix
  {
  K_0\int f(y)g(y)dy\cr
 K_1(x_1)\int g(y)f(y)dy
   +\cr
 K_2(x_1,x_2)\int g(y) f(y)dy+
 \cr
 K_3(x_1,x_2,x_3)\int g(y)f(y)dy+
     \cr
\dots \cr
   }
   \right.\,,
   $$
     $$
  \left. 
  \matrix
  {
  \cr
   g(x_1)\int K_1(y)f(y)dy\cr
 g(x_2)\int K_2(x_1,y)f(y)dy+
 g(x_1)\int K_2(x_2,y)f(y)dy
 \cr
 g(x_3)\int K_3(x_1,x_2,y)f(y)dy+
 g(x_1)\int K_3(x_3,x_2,y)f(y)dy+
 g(x_2)\int K_3(x_1,x_3,y)f(y)dy+
     \cr
\dots \cr}
  \right)
     $$
 
and
  $$
 a^*_B(g)a_B(f)
   \pmatrix
       {
  K_0\cr
  K_1(x_1)\cr
  K_2(x_1,x_2)\cr
  K_3(x_1,x_2,x_3)\cr
  K_4(x_1,x_2,x_3,x_4)\cr
 \dots \cr
    }=
  a_B^*(g)\pmatrix
       {
  \int K_1(y)f(y)dy\cr
  \sqrt 2\int K_2(x_1,y)f(y)\cr
 \sqrt 3\int  K_3(x_1,x_2,y)f(y)dy\cr
 2\int K_4(x_1,x_2,x_3,y)f(y)dy\cr
 \dots \cr
    }=
  $$
   $$
P_B     \pmatrix {
        0\cr
  g(x_1)\int K_1(y)f(y)dy\cr
  \sqrt 2\sqrt 2\int K_2(x_1,y)f(y)g(x_2)\cr
 \sqrt 3\sqrt 3\int  K_3(x_1,x_2,y)f(y)dyg(x_3)\cr
2\cdot  2\int K_4(x_1,x_2,x_3,y)f(y)dy g(x_4)\cr
 \dots \cr
    }=
    $$
        $$
\left( 
   \matrix {
             0\cr
  g(x_1)\int K_1(y)f(y)dy\cr
\int K_2(x_1,y)f(y)g(x_2)+
\int K_2(x_2,y)f(y)g(x_1)
             \cr
 \int  K_3(x_1,x_2,y)f(y)dyg(x_3)+
 \int  K_3(x_3,x_2,y)f(y)dyg(x_1)+
 \cr
  \int K_4(x_1,x_2,x_3,y)f(y)dy g(x_4)+
  \int K_4(x_4,x_2,x_3,y)f(y)dy g(x_1)+
  \cr
 \dots \cr
    }\right.
    $$
 $$
\left.
    \matrix {
             0\cr
       0\cr
            0 \cr
 \int  K_3(x_1,x_3,y)f(y)dyg(x_2)\cr
  \int K_4(x_1,x_4,x_3,y)f(y)dy g(x_2)+
  \int K_4(x_1,x_2,x_4,y)f(y)dy g(x_3)\cr
 \dots \cr
    }\right)\,.
    $$
Thus we see that
     $$
a_B(f)a_B^*(g)-
a_B^*(g)a_B(f)=\int g(y)f(y)dy\,.
    $$
\centerline {\bf Useful formulae}

Let $\Phi$ be so called vacuum vector, i.e.
         $$
\Phi=\pmatrix {1\cr 0\cr\cr\dots\cr}
         $$
Then for every vector
    $$
\Psi=\pmatrix {
               K_0\cr 
           K_1(x_1)\cr 
          K_2(x_1,x_2)\cr
          K_3(x_1,x_2,x_3)\cr
          K_4(x_1,x_2,x_3,x_4)\cr
              \dots\cr  
              }=
         $$
       $$
        \left(
\sum_{n=0}^\infty {1\over \sqrt {n!}}
       \int
       K_n(x_1,x_2,\dots,x_n)
                a_B^*(x_1)
                a_B^*(x_2)
               \dots 
               a_B^*(x_n)
            dx^1dx^2\dots dx^n
               \right)\Phi_0\,.
    $$
The same formula holds for fermionic case.
\bye



