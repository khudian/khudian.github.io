

\magnification=1200
\baselineskip=14pt

\def\A {{\bf A}} 
\def\B {{\cal B}}
\def\C {{\bf C}}
\def\CC {{\cal C}}
\def\Cl {{\tt \hbox{Cliff}}}
\def\E {{\bf E}}
\def\EE {{\cal E}}
\def\F {{\cal F}}
\def\FF {{\cal F}}
\def\G {\Gamma}
\def\GG {{\cal G}}
\def\H {{\bf H}}
\def\K {{\bf K}}
\def\L {{\cal L}}
\def\M {{\cal M}}
\def\N {{\bf N}}
\def\R {{\bf R}}
\def\Sb {{\bf S}}
\def\SS {{\cal S}}
\def\Tr {{\rm Tr\,}}
\def\V {{\cal V}}
\def\X {{\bf X}}
\def\XX {{\cal X}}
\def\Y {{\bf Y}}
\def\Z {{\bf Z}}

\def\a {\alpha}
\def\ac {{\bf a}}
\def\b {{\bf b}}
\def\bs {{\bf s}}
\def\c {{\bf c}}
\def\d {\delta}
\def\dist {{\tt \hbox{distance}}}
\def\e {{\bf e}}
\def\f {{\bf f}}
\def\finish {\blacksquare}
\def\g {{\bf g}}
\def\grad {{\rm grad\,}}
\def\h {\hbar}
\def\k {{\bf k}}
\def\l {{\bf l}}
\def\m {{\bf m}}
\def\n {{\bf n}}
\def\p {\partial}
\def\pb {{\bf p}}
\def\pt {{\bf pt}}
\def\q {{\bf q}}
\def\r {{\bf r}}
\def\s {\sigma}
\def\t {{\bf t}}
\def\tS {{\tilde \Sigma}}
\def\td {\tilde}
\def\v {{\bf v}}
\def\vare {\varepsilon}
\def\x {{\bf x}}
\def\y {{\bf y}}
\def\w {\omega}


\centerline{\bf One beautiful way to calculate one
action}


{\it To calculate the action of particle
in homogeneous field: $H={p^2\over 2m}-mgx$ 
is not difficult, however calculations are not too
enjoyable.
We will calculate here this action using just
symmetires.}


   Let
     $$
    W(t_1,x_1;t_2,x_2)
     $$
be an action of the theory.
   It obeys the following property:
        $$
    W(t_1,x_1;t_2,x_2)=-
    W(t_2,x_2;t_1,x_1)\,.
        \eqno (1)
        $$
We calculated this action straightforwardly
(see the blog on July 2019).  
Calculate here this action using just this symmetry,
Hamilton-Jacobi equation
 and some physical intuition.
              
Recall that for free particle
    $$
W_{\rm free}={m(x_2-x_1)^2\over 2(t_2-t_1)}\,.
    $$
Hence
for  a particle  in homogeneous field,  
        $$
W=W_{free}(t_1,x_1;t_2,x_2)+gW'(t_1,x_1;t_2,x_2)\,,
        $$
where $g$ is acceleration.

We use the following fact: Let  a function has dimension
("razmernostj") of action and 
 $$
F=mg^\alpha(x_1+x_2)^\beta(x_2-x_1)^\gamma
t^\delta\,.
       $$
Tehn
       $$ 
[F]=[mght]=kg\cdot
{m\over sec^2}\cdot m\cdot { sec}\,.
 \eqno (2)
 $$
The condition (1) implies that
        $$
\gamma+\delta=2k+1
         $$
and condition (2) implies that
         $$
[F]=kg\cdot m^{\a+\beta+\gamma} \cdot sec^{\delta-2\a}=
kg\cdot m^2 \cdot sec^{-1}\,,
         $$
thus we see that
              $$
\cases{
\gamma+\delta=2k+1\cr
   \a+\beta+\gamma=2\cr
\delta-2\a=-1
     }
          $$
Using this we look for the action of particle in
homogeneous filed as
     $$
W=\underbrace {
{m(x_2-x_1)^2\over2(t_2-t_1)}
     }
_{\hbox{free particle}}+
   c_1mg(x_1+x_2)(t_2-t_1)+c_2mg^2(t_2-t_1)^3\,.
     $$
(This expression transforms in a right way and has right
dimension.)

\medskip

Calculate the constants $c_1$ and $c_2$.
Use that  $W$ obeys
Hamilton-Jacobi equation for Hamiltonian $H={p^2\over
2m}-mgx$:
         $$
        0=
{ \p W(x_1,t_1;t_2,x_2)\over \p t_2}+
      \left(
               {
             \left(
{ \p W(x_1,t_1;t_2,x_2)\over \p x_2}
    \right)^2
     \over
        2m}-mgx_2
      \right)=
         $$
         $$
-{m(x_2-x_1)^2\over2(t_2-t_1)^2}
   +c_1mg(x_1+x_2)+3c_2mg^2t^2+
        {\left(
     {m(x_2-x_1)\over (t_2-t_1)}+
    c_1mg(t_2-t_1)      
    \right)^2
       \over 2m}
   -mgx_2=
         $$
          $$
  =2c_1mgx_2-mgx_2+\left(3c_2+{c_1^2\over
2}\right)mg^2(t_2-t_1)^2=0\,.
          $$
Thus
        $$
c_1={1\over 2}\,,\quad
      c_2=-{1\over 24}
     $$
  and
       $$
W=W=\underbrace {
{m(x_2-x_1)^2\over2(t_2-t_1)}
     }
_{\hbox{free particle}}+
   {1\over 2}mg(x_1+x_2)(t_2-t_1)-{1\over 24}mg^2(t_2-t_1)^3\,.
      $$
\bye
