

  Few days ago Yurij Bazlov told me about one very beautiful problem
   were Fermat (small)  Theorem appears.

   The problem is following. 

Calculate the number 
of subsets of the set $\{1,2,3,\dots, p\}$
such that the number $p$ divides the sum of elements in this subset. 
Suppose that $p$ is a prime number.


 This problem  was given to schoolkids on something like 
International Olympiad.

In this problem the Fermat (small) theorem is unexpectably appears.

   Yurij solved this problem in a very beautiful way.

I spent the Saturday aolving this problem; as a result I failed
to produce the coursework in a time, however,
 I solved it {\tt Paris il vaut bien une messe!}.

My solution is the following:

   Consider polynomial
           $$
  P_p(x)=(1+x)(1+x^2)(1+x^3)\dots (1+x^p)
           $$
where $x$ is undeterminate, and show that
           $$
  P_p(x)=1+C_p{x^{p}-1\over x-1}\,.
           $$

\bye
\bye
