

     \centerline  {\bf Удивительное рядом }

     Is it mystery or not, but
the Lagrangian in projective space $L(x^i,v^i)$ 
in homogeneous coordinates
is gauge invariant with respect  to local reparameterisation:
                     $$
     L(\lambda(t)x^i, \lambda v^i+\dot\lambda x^i)=L(x^i,v^i)
                     $$

 $$




   Я, мучительно обдумывая, как устроен Лагранюиан
на Грассманиане (скалярное произведение) вдруг заметил,
что уже для проективной прямой
          $$
   L={v^2\over r^2}-{(rv)^2\over r^4}
           $$   
есть калибровочная симметрия
         $$
     x^i\mapsto \lambda(t) x^i         $$
($x^i$ однородные координаты)

 и соответствующее тождество Нетер:
              $$
           x^i{\p L\over v^i}\equiv 0
              $$


\bye
