

 %This is file of my planning talk in Bialoveza June--July 2016
% this is for sheffield I do it on the base of bialoveza.  July 2016
\documentclass{beamer}
\usepackage{mathptmx}
\usepackage{helvet}

\def\vare {\varepsilon}
\def\A {{\cal A}}
\def\t {\tilde}
\def\a {\alpha}
\def\F {{\cal F}}
\def\RR {{\bf R}}
\def\K {{\bf K}}
\def\A {{\bf A}}
\def\B {{\bf B}}
\def\V {{\cal V}}
\def\s {{\sigma}}
%\def\S {{\Sigma}}
\def\ss {{\bf s}}
\def\bs {{\bf s}}
\def\w {{\omega}}
\def\p{\partial}
\def\vare{{\varepsilon}}
\def\Q {{\bf Q}}
\def\D {{\cal D}}
\def\G {{\Gamma}}
\def\C {{\bf C}}
\def\LL {{\cal L}}
\def\L {{\Lambda}}
\def\Z {{\bf Z}}
\def\U  {{\cal U}}
\def\H {{\cal H}}
\def\R  {{\bf R}}
\def\S  {{\bf S}}
\def\E  {{\bf E}}
\def\l {\lambda}
\def\M {{\mathfrak M}}
\def\g {\gamma}
\def\degree {{\bf {\rm degree}\,\,}}
\def \finish {${\,\,\vrule height1mm depth2mm width 8pt}$}
\def \m {\medskip}
\def\p {\partial}
\def\r {{\bf r}}
\def\v {{\bf v}}
\def\n {{\bf n}}
\def\t {{\bf t}}
\def\b {{\bf b}}
\def\c {{\bf c }}
\def\e{{\bf e}}
\def\ac {{\bf a}}
\def \X   {{\bf X}}
\def \Y   {{\bf Y}}
\def \x   {{\bf x}}
\def \y   {{\bf y}}
\def \G{{\cal G}}
\def\Ber {{\rm Ber\,}}
\def\hD {{\hat\Delta}}
\def\hl {{\hat w}}
\def\Tr {{\rm Tr\,}}
\def \spectr {\lambda_1,\dots,\lambda_p;\mu_1,\dots,\mu_q}
\def \finish {${\,\,\vrule height1mm depth2mm width 8pt}$}
\def\R {{\rm Res \,}}
\def\pr {{\,\,'}}
\def\brho {{\mathbf \rho}}
\def\srho {{\underbar {\hbox{$\rho$}}}}
\def\Dv {{{\cal D}\bf v}}
\mode<presentation>{\usetheme{Montpellier}}


\title[ Higher Koszul brackets and thick] 
% (optional, use only with long paper titles)
{\textbf{Thick morphisms and higher  Koszul brackets}}






\author[Hovhannes Khudaverdian] 
% (optional, use only with lots of authors)
{Hovhannes Khudaverdian }




\institute[University of Manchester, Manchester, UK] 
% (optional, but mostly needed)
{University of Manchester, Manchester, UK}

\date{GAP XIV\\
  s\'eminaire itin\'erant ``G\'eom\'etrie et Physique''           \\
 8 August---12 August, Sheffield, UK \\ 

\medskip

{\sl The talk is based on the work with Ted Voronov}

}


 

  


\begin{document}

\begin{frame}
  \titlepage
\end{frame}
\begin{frame}{Contents}%[pausesections]
  \tableofcontents
\end{frame}

\begin{frame}{Papers that talk is based on are}
[1] H.M.Khudaverdian, Th. Voronov {\it Higher Poisson 
             brackets and differential forms}, 2008a
 In: Geometric Methods in Physics. AIP Conference Proceedings 1079, 
American Institute of Physics, Melville, New York, 2008, 203-215.,
       arXiv: 0808.3406
\medskip

[2] Th. Voronov, {\it Nonlinear pullback on functions and
a formal category extending the category of supermanifolds]},
          arXiv: 1409.6475

\medskip


[3] Th. Voronov, {\it Microformal geometry},
      arXiv: 1411.6720

\end{frame}


\section {Abstracts}


\begin{frame}{Abstract...}
            For an arbitrary manifold $M$, we consider supermanifolds 
$\Pi TM$  and $\Pi T^*M$, where $\Pi$ is the parity reversion functor.
The space  $\Pi T^*M$ possesses canonical odd Schouten bracket and space 
$\Pi TM$ posseses canonical de Rham differential $d$.  An arbitrary even 
function $P$ on $\Pi T^*M$  such that $[P,P]=0$ induces a 
homotopy Poisson  bracket on $M$, a differential, $d_P$ on $\Pi T^*M$, 
and higher Koszul brackets on $\Pi TM$. (If $P$ is fiberwise quadratic, 
then we arrive at standard structures of Poisson
geometry.)  Using the language of $Q$-manifolds and in particular 
of Lie algebroids, we study the interplay between canonical
structures and structures depending on $P$. Then using just 
recently invented theory of thick morphisms we construct  
a non-linear map between the $L_{\infty}$ algebra of functions 
on $\Pi TM$ with higher Koszul brackets and the Lie algebra of functions 
on $\Pi T^*M$ with the canonical odd Schouten bracket.







\end{frame}
\begin{frame}
\end{frame}

\section {Poisson manifold and....}
\begin{frame}{Poisson manifold}

 Let $M$ be Poisson manifold with Poisson tensor $P=P^{ab}\p_b\wedge \p_a$
               $$
\{f,g\}=\{f,g\}_P={\p f\over \p x^a}P^{ab}{\p g\over \p x^b}\,.
               $$
               $$
\{\{f,g\},h\}+\{\{g,h\},f\}+\{\{h,f\},g\}=0\,,
        $$
        $$
      \Updownarrow
        $$
        $$
    P^{ar}\p_r P^{bc}+
    P^{br}\p_r P^{ca}+
    P^{cr}\p_r P^{ab}=0\,.
               $$
If $P$ is non-degenerate, then $\w=(P^{-1})_{ab}dx^a\wedge dx^b$
is closed non-degenerate form defining symplectic structure on $M$.

\end{frame}
\begin{frame}{Differentials}
$d$---de Rham differential, $d\colon \Omega^k(M)\to \Omega^{k+1}(M)$, 
          $$
d^2=0\,, df={\p f\over \p x^a}dx^a\,,\qquad
  d(\w\wedge\rho)=d\w\wedge \rho+(-1)^{p(\w)}\w\wedge d\rho\,,
     $$
$d_P$---Lichnerowicz- Poisson differential,
  $d_P\colon \mathfrak{A}^k(M)\to \mathfrak{A}^{k+1}(M)$, 
 
         $$
d_Pf={\p f\over \p x^b}P^{ba}{\p\over \p x^a}\,\,
   (\hbox{for a function $f=f(x)$})\,,d_P^2=0\,,
         $$
         $$
 d_P P=0\leftrightarrow \hbox {Jacobi identity for odd Poisson bracket 
$[\,,\,]$}
         $$


\end{frame}

\begin{frame}{Differential forms and multivector fields}

$\mathfrak {A}^*$ space of multivector fields on $M$,\\
$\Omega^*$ space of differential forms on $M$,
            $$
          \begin{matrix}
     &\mathfrak{A}^k(M)&{\buildrel d_P\over \longrightarrow} 
       &\mathfrak{A}^{k+1}(M)\cr
        &\uparrow& &\uparrow\cr 
     &\Omega^k(M)&{\buildrel d\over \longrightarrow} 
       &\Omega^{k+1}(M)\cr
        \end{matrix}
                  $$
      \end{frame}
\begin{frame}{Differential forms and multivector fields}

          $$
      \begin{matrix}
\hbox{$\mathfrak {A}^*$--- multivector fields on $M$=
                     functions on $\Pi T^*M$}\cr
\hbox{$\Omega^*$--- differential forms on $M$=
                     functions on $\Pi TM$},\cr
       \end{matrix}
         $$
            $$
          \begin{matrix}
     &\mathfrak{A}^k(M)&{\buildrel d_P\over \longrightarrow} 
       &\mathfrak{A}^{k+1}(M)\cr
        &\uparrow& &\uparrow\cr 
     &\Omega^k(M)&{\buildrel d\over \longrightarrow} 
       &\Omega^{k+1}(M)\cr
        \end{matrix}
        \qquad
           \begin{matrix}
     &C(\Pi T^*M)&{\buildrel d_P\over \longrightarrow} 
       &C(\Pi T^*M)\cr
        &\uparrow& &\uparrow\cr 
     &C(\Pi TM)&{\buildrel d\over \longrightarrow} 
       &C(\Pi TM)\cr
        \end{matrix}
            $$
            $$
d\w(x,\xi)=\xi^a{\p\over \p x^a} \w(x,\xi)\,,
 d_P F(x,\theta)=\left[P,F\right]_1\,,
           $$ 
$\left[P,F\right]_1$-canonical odd Poisson bracket on $\Pi T^*M$. 
\end{frame}
\begin{frame}
$x^a=(x^1,\dots,x^n)$--- coordinates on $M$ 

$(x^a,\xi^b)=(x^1,\dots,x^n;\xi^1,\dots,\xi^n)$, 
    ---coordinates on $\Pi TM$ 
          $$
p(\xi^a)=p(x^a)+1, x^{a'}=x^{a'}(x^a)\to \xi^{a'}=\xi^a{\p x^{a'}\over \p x^a}\,.
\qquad (dx^a\leftrightarrow \xi^a)\,.
           $$` 
  Respectively

$(x^a,\theta_b)=(x^1,\dots,x^n;\theta_1,\dots,\theta_n)$, 
    ---coordinates on $\Pi T^*M$ 
          $$
p(\theta_a)=p(x^a)+1, x^{a'}=x^{a'}(x^a)\to 
                           \theta_{a'}=
                           \theta_a{\p x^{a}\over \p x^{a'}}\,.
\qquad ({\p_a}\leftrightarrow \theta_a)\,.
           $$` 
%{\bf Example}
\begin{example} 
  $$
\Omega^*\ni \w=l_adx^a+r_{ab}dx^a\wedge dx^b\leftrightarrow 
  \w(x,\xi)=l_a\xi^a+r_{ab}\xi^a\xi^b\in C(\Pi TM)
  $$
 $$
\mathfrak{A}^*\ni F=X^a\p_a+M^{ab}\p_a\wedge\p_b 
    \leftrightarrow 
  F(x,\theta)=X^a\theta_a+M^{ab}\theta_a\theta_b\in C(\Pi T^*M)\,.
  $$
 \end{example}
\end{frame}

\begin{frame}{Canonical odd Poisson bracket} 

           $$
         \begin{matrix}
\hbox{$F,G$ multivector fields}\cr
 [F,G]\, \hbox{Schouten commutator}\cr 
  \end{matrix}\,,
      \qquad
         \begin{matrix}
\hbox{$F,G$ functions on $\Pi T^*M$}\cr
 [F,G]\, \hbox{odd Poisson bracket}\cr 
  \end{matrix}\,,
 $$
       $$
     \begin{matrix}
\X=X^a\p_a, [\X, F]=\mathfrak{L}_\X F\cr
  P=P^{ab}\p_a\wedge \p_b\,,\,
      [P,F]=d_P F
      \end{matrix}\,,\qquad
         \begin{matrix}
[\X, F]=[X^a\theta_a, F(x,\theta)]\cr
  d_PF=[P,F]=[P^{ab}\theta_a\theta_b, F(x,\theta)]\cr
      \end{matrix}
        $$
       $$
  \left[F(x,\theta),G(x,\theta)\right]=
     {\p F(x,\theta)\over \p x^a}
     {\p G(x,\theta)\over \p \theta_a}
         +(-1)^{p(F)}
     {\p F(x,\theta)\over \p \theta_a}
     {\p G(x,\theta)\over \p x^a}\,.
       $$
               $$
       \hbox{Names are}\quad
            \begin{matrix}
      &\hbox{odd Poisson bracket}\cr
      &\hbox{Schouten bracket}\cr
      & \hbox{ Buttin bracket}\cr
       &\hbox{ anti-bracket}\cr
           \end{matrix}
               $$
      
\end{frame}
\begin{frame}{Koszul bracket on differential forms}
       $$
      \begin{matrix}
        \varphi_P\colon \Pi T^*M
           \rightarrow \Pi TM\cr
        \varphi_P^*\colon   C(\Pi T^*M)
                 \leftarrow C(\Pi TM)\cr
         \end{matrix}\,,
            \qquad
      \xi^a=P^{ab}\theta_b\,{\rm or}\,\, dx^a=P^{ab}\p_b
       $$
From bracket $[\,,\,]$ on functions  
to Koszul bracket on diff. forms
       $$
     [\w,\s]_P=({\varphi_P^*})^{-1}
   \left(\left[\varphi_P^*(\w),\varphi_P^*(\s)\right]_P\right)\,.
       $$
       $$
 [f,g]_P=0\,,\,[f,dg]_P=(-1)^{p(f)}\{f,g\}_P\,,\,
      [df,dg]_P=(-1)^{p(f)}d\left(\{f,g\}_P\right)
      $$
 This formula survives the limit if $P$ is degenerate.
\end{frame}

\begin{frame}{Question}
       $$
\hbox{We have}\qquad  \Pi T^*M
 {\buildrel \varphi_P\over\longrightarrow} \Pi TM
       $$
What happens if 
even function $P={1\over 2}P^{ab}(x,\theta)\theta_a\theta_b$ 
is replaced by an arbitrary even function
   $P=P(x,\theta)$ which obeys the 
master-equation
          $$
[P,P]=2{\p P(x,\theta)\over \p x^a}{\p P(x,\theta)\over \p \theta^a}=0\,.
          $$  
(In the case $P={1\over 2}P^{ab}(x,\theta)\theta_a\theta_b$
master-equation is just Jacobi identity for Poisson bracket 
$\{\,,\,\}_P$
on $M$.) 
\end{frame}

\section{Higher brackets}
\begin{frame}{Master-Hamiltonian $\to$ brackets (I-st case)}

\centerline  {$M$---(super)manifold. (coordinates `$x=x^a$)}

\smallskip

Odd  Hamiltonian $Q(x,p)$ on $ T^*M$, ($p=p_b$ fibre coordinates) 
\\
defines homotopy odd Poisson (Schouten)  brackets on $M$---
   collection 
$\left\{\{\}_Q, \{\,,\,\}_Q, \{\,,\,,\,\}_Q,\dots\right\}$ 
of brackets on $M$:
                  $$
   \{f\}_Q=(Q,f)\big\vert_{p=0}\,,\quad
   \{f,g\}_Q=\left(\left(Q,f\right),g\right)\big\vert_{p=0}\,,
                 $$
                 $$
      \{f_1,f_2,\dots,f_n\}_Q=
\left(\left(\dots(Q,f_1),f_2\right),\dots,f_n\right)
        \big\vert_{p=0}
                 $$
$(\,,\,)$---canonical even Poisson
bracket on $T^*M$.\\
$(Q,Q)=0$---Jacobi identity, 
\medskip
  
We come to usual odd Poisson bracket if Hamiltonian is quadratic in
momenta, $Q=Q^{ab}p_ap_b$.


\end{frame}
\begin{frame}{Master-Hamiltonian $\to$ brackets (II-nd case)}


Even  Hamiltonian $H(x,\theta)$ on 
$\Pi T^*M$, ($\theta=\theta_b$ fibre coordinates) \\ 
defines  homotopy Poisson brackets on $M$---\\
  collection 
$\left\{\{\}_H, \{\,,\,\}_H, \{\,,\,,\,\}_H,\dots\right\}$ 
of brackets on $M$:
                 $$
   \{f\}_H=[H,f]\big\vert_{\theta=0}\,,\quad
   \{f,g\}_H=\left[\left[H,f\right],g\right]\big\vert_{\theta=0}\,,
                 $$
                 $$
      \{f_1,f_2,\dots,f_n\}_H=\left[\left[\dotsc\left[H,f_1\right],f_2\right],
\dotsc,f_n\right]\big\vert_{\theta=0}
                 $$
$[\,,\,]$---canonical odd Poisson
bracket on $\Pi T^*M$\\
$[H,H]=0$--- Jacobi identity              


We come to usual even Poisson bracket if Hamiltonian is quadratic in
momenta, $H=H^{ab}\theta_a\theta_b$.

    
           \end{frame}
\begin{frame}{Mackenzie-Xu symplectomorphism}
   $E\to B$---vector bundle.   Canonical symplectomorphism 
 (MX-symplectomorphism)
               $$
     T^*E\leftrightarrow T^*E^*
               $$
Local coordinates  
               $$
             \underbrace
                {
     \overbrace{x^\mu, u^i}^{\hbox {\footnotesize coord. on}\, E}; 
           p_\mu, p_j
                }
              _{\hbox {\footnotesize coord. on}\, T^*E}\,,
   \qquad\qquad
             \underbrace
                {
     \overbrace{y^\nu, u_i}^{\hbox {\footnotesize coord. on}\, E^*}; 
           q_\mu, p^k
                }
              _{\hbox {\footnotesize coord. on}\, T^*E^*}\,.
               $$
Then $\kappa\colon T^*E\to T^*E^*$ is such that
           $$
\kappa^*(y^\mu)=x^\mu\,,\,\,
\kappa^*(u_i)=p_i\,,\,\,
\kappa^*(q_\mu)=-p_\mu\,,\,\,
 \kappa(p^i)=u^i\,. 
           $$
\end{frame}
\begin{frame}{Canonical odd Poisson bracket on $\Pi T^*M$}
Consider an odd Hamiltonian  $Q=p_a\eta^a$ 
on tangent bundle $T^*(\Pi T^*M)$ to $\Pi T^*M$.
                       $$
 {\rm coordinates}\,\qquad
  \underbrace {
    \overbrace{x^a,\theta_b}^{\hbox{ $\Pi T^*M$}}; 
 p_a,\eta^b 
       }_{T^*(\Pi T^*M)}
                       $$
Odd Hamiltonian $Q=p_a\eta^a$ is quadratic in momenta.

   It generates an odd canonical Poisson bracket $[\,,\,]$ 
on $\Pi T^*M$:
        $\,[f,g]=[f,g]_P=\left(\left(Q,f\right),h\right)=$  
                  $$
=\left(\eta^a{\p f\over \p x^a}+p_a{\p g\over \p \theta_a} ,g\right)
={\p f\over \p x^a}{\p g\over \p \theta_a}+
 {\p g\over \p \theta_a}{\p f\over \p x^a}
                   $$

$(\,,\,)$ is canonical Poisson bracket on $T^*(\Pi T^*M)$.

\end{frame}
\begin{frame} Consider MX symplectomorphism
  $T^*(\Pi T^*M)\leftrightarrow T^*(\Pi TM)$:
       $$
   \underbrace {
    \overbrace{x^a,\theta_b}^{\hbox{ $\Pi T^*M$}}; 
  p_a,\eta^b 
       }_{T^*(\Pi T^*M)}
     \,\leftrightarrow\,
        \underbrace {
    \overbrace{x^a,\xi^b}^{\hbox{ $\Pi TM$}}; 
 q_a,\pi_b 
       }_{T^*(\Pi TM)}
       $$
   $p_a\leftrightarrow -q_a$, $\theta_b\leftrightarrow \pi_\b$,
   $\eta^a\leftrightarrow \xi^a$, 
 
Odd Hamiltonian $Q=p_a\eta^a$
$\leftrightarrow$  odd Hamiltonian 
       $ K=q_a\xi^a$. 
      $$
 [\w]=(K,\w)=\xi^a{\p \w\over \p x^a}=d\w\,,(\w(x,\xi)\to \w(x,dx)).
      $$
(all higher brackets vanish)

Odd homtopy bracket is nothing but de Rham differential.
\end{frame}


\begin{frame}{Lichnerowicz differential $d_P$}
For even function $P=P(x,\theta)$ ($[P,P]=0$)
        $$
d_P F=[P,F]  \,. (d_P^2=0)
        $$
Consider  $$
Q_P=(P,Q)=(P,p_a\eta^a)=
p_a{\p P(x,\theta)\over \p \theta_a}+
\eta^a{\p P(x,\theta)\over \p x^a}
        $$
This is Hamiltonian linear in momenta. It produces
degenerate homotopy bracket---Lichnerowicz differential:. 
        $$
[F]=(Q_P,F)=((P,Q),F)=[P,F]=d_P F. 
       $$ 
(all higher brackets vanish)
     \end{frame}

\begin{frame}{Lichnerowic differential $\to$ Higher Koszul brackets}
  Under MX symplectomorphism, Hamiltonian 
     $$
Q_P(x,\theta,p,\eta)=
p_a{\p P(x,\theta)\over \p \theta_aa}+
\eta^a{\p P(x,\theta)\over \p x^a}
\,\,\hbox {on $T^*(\Pi T^*M)$}
      $$
 transforms to Hamiltonian
      $$
K_P(x,\xi,q,\pi)=
q_a{\p P(x,\pi)\over \p \pi_aa}+
\eta^a{\p P(x,\pi)\over \p x^a}
 \,\, {\rm on}\,  T^*(\Pi TM) \,.   $$
This Hamiltonian defines homotopy Schouten bracket
on $\Pi TM$ (Higher Koszul bracket on differential forms)
\end{frame}
\begin{frame}{Higher Koszul brackets on $M$}
 
Odd Hamiltonian  $K_P$ on  $T^*(\Pi TM)$ 
defines  homotopy odd Poisson bracket 
(higher Koszul bracket) on  $\Pi TM$,
            $$
        [F_1,F_2,\dots,F_n]_P=
  \left[\dotsc\left[K_P,F_1\right],\dots,F_p\right]\big\vert_{\Pi TM}\,,\qquad
          \big\vert_{\Pi TM}=\big\vert_{p=\pi=0}\,.
           $$
           $$
   F=F(x,\xi)=f(x)+\xi^af_a(x)+\dots,    (df=\xi^a\p_a f)\,,
                     $$
              $$
     [f]_P=0\,,  [f_1,f_2,\dots,f_k]_P=0
          $$
          $$
        [f_1,df_2,\dots,df_n]=\{f_1,f_2,\dots,f_n\}\,,
           $$
          $$
        [df_1,df_2,\dots,df_n]=d\{f_1,f_2,\dots,f_n\}\,,
           $$


In the same way as for classical case ($P=P^{ab}\theta_b\theta_a$)
       
\end{frame}


\begin{frame}{Recall the classical case $P=P^{ab}\theta_b\theta_a$}
     
          $$
      \begin{matrix}
\hbox{$\mathfrak {A}^*$--- multivector fields on $M$=
                     functions on $\Pi T^*M$}\cr
\hbox{$\Omega^*$--- differential forms on $M$=
                     functions on $\Pi TM$},\cr
       \end{matrix}
         $$
            $$
          \begin{matrix}
     &\mathfrak{A}^k(M)&{\buildrel d_P\over \longrightarrow} 
       &\mathfrak{A}^{k+1}(M)\cr
        &\uparrow& &\uparrow\cr 
     &\Omega^k(M)&{\buildrel d\over \longrightarrow} 
       &\Omega^{k+1}(M)\cr
        \end{matrix}
        \qquad
           \begin{matrix}
     &C(\Pi T^*M)&{\buildrel d_P\over \longrightarrow} 
       &C(\Pi T^*M)\cr
        &\uparrow& &\uparrow\cr 
     &C(\Pi TM)&{\buildrel d\over \longrightarrow} 
       &C(\Pi TM)\cr
        \end{matrix}
            $$
            $$
\varphi_P\colon \xi^a=P^{ab}\theta_b\,,\quad
\varphi_P^*\colon C(\Pi TM)\to C(\Pi T^*M)
           $$ 
Then     
       $$
     \varphi_P^*(d\w)=d_P(\varphi_P^*\w)\,.
            $$
This relation survives for an arbitrary  $P=P(x,\theta)$
($[P,P]=0$.)
\end{frame}
\begin{frame}{Two Hamiltonians}
         $$
      \varphi_P\,\,\Pi TM\to\Pi T^*M\colon\,\,
 \xi^a={1\over 2}{\p P(x,\theta)\over \p \theta_a}\,.
         $$
         $$
     \varphi_P^*(d\w)=d_P(\varphi_P^*\w)\,.,\qquad
(Q_P,\varphi_P^*\w)=\varphi_P^*\left((K,\varphi)\right)\,.
         $$
          $$
       \begin{matrix}
    T^*(\Pi T^*M) &\longrightarrow & T^*(\Pi TM)\cr
             Q=p_a\eta^a& \longrightarrow &  K=\eta^aq_a\cr
        \hbox{canonical odd bracket} &\longrightarrow&
         \hbox{de Rham differential on $\Pi TM$ }\cr
     Q_P=(P,Q)& \longrightarrow &  K_P\cr
        \hbox{Lichnerowicz diff. on $\Pi T^*M$} &\longrightarrow&
         \hbox{Higher Koszul bracket on $\Pi TM$ }\cr 
       \end{matrix}
        $$
         $$
         $$
Map $\varphi_P$ intertwines Lichnerowicz and de Rham differentials,
i.e. Hamiltonians $Q_P$ and $K$.



\end{frame}
\begin{frame}{Question}
Map $\varphi_P$ intertwines Lichnerowicz and de Rham differentials,
i.e. Hamiltonians $Q_P$ and $K$.


\bigskip

How look a map which intertwines Hamiltonians, $Q$ and $K_P$,
i.e. a map which intewins canonical Schouten bracket and higher 
Koszul brackets???


\end{frame}

\begin{frame}{Usual Poisson bracket}
     $P(x,\theta)=P^{ab}(x)\theta_a\theta_b$
   even function on $\Pi T^*M$ quadratic on $\theta$ defines
usual Poisson bracket on $M$:
for $f,g\in C(M)$ 
                 $$
\{f,g\}=\{f(x),g(x)\}_P=[[P,f],g]=
{\p \over \p \theta^b}
\left({\p P(x,\theta)\over \p \theta^a}{\p f\over \p x^a}\right)
    {\p g\over \p x^b}=
               $$
               $$
{\p f\over \p x^a}P^{ab}{\p g\over \p x^b}\,.
                 $$ 
$$
\hbox{Jacobi identity}\,
:0=[P,P]=2{\p P\over \p x^a}{\p P\over \p \theta_a}=
  4\p_a P^{bc}P^{ad}\theta_b\theta_c\theta_d
                $$
i.e.
                $$
     P^{da}\p_a P^{bc}+P^{ba}\p_a P^{cd}+P^{ca}\p_a P^{db}=0\,.
                $$
\end{frame}
\begin{frame}{Higher Poisson brackets on $M$}
Even  (non-quadratic in momenta) Hamiltonian in $\Pi T^*M$,  
$H=P(x,\theta)$, ( $[P,P]=0$ Jacobi identity)
defines homotopy Poisson brackets, higher even brackets:
           $$
        \{f_1,f_2,\dots,f_n\}_P=
  \left[\dotsc\left[P,f_1\right],\dots,f_p\right]\big\vert_M\,,\qquad
          \big\vert_M=\big\vert_{\theta=0}\,.
           $$
If   
       $$
P=P^a\theta_a+{1\over 2}P^{ab}\theta_b\theta_a+
     {1\over 6}P^{abc}\theta_c\theta_b\theta_a+\dots
            $$
then  
          $$
\{x^a\}_P=P^a\,,\,\{x^a,x^b\}=P^{ab}\,,\,\{x^a,x^b,x^c\}=P^{abc}\,\dots\,,
            $$
\end{frame}




\begin{frame}{From $\Pi T^*M$ to $\Pi TM$.  }
\begin{theorem}
 There is a natural odd linear map
           $
       C(\Pi T^*M)\to   C(T^*(\Pi TM)) 
           $\\ 
that takes canonical odd Poisson bracket on $\Pi T^*M$ to
canonical even  Poisson bracket on $T^*(\Pi TM)$ 
\end{theorem}

\begin{corollary}
   $$
 \begin{matrix}
\hbox{even  Hamiltonian}\cr
  \hbox{$P=P(x,\theta)$ on $\Pi T^*M$ }\cr
\hbox{defining higher even}\cr
  \hbox{ Poisson bracket on $M$}
\end{matrix}
   \longrightarrow
 \begin{matrix}
\hbox{odd  Hamiltonian}\cr
 \hbox{$K=K_P(x,\xi;p,\pi)$ on $T^*(\Pi TM)$} \cr
\hbox{defining higher Koszul}\cr
  \hbox{bracket on $\Pi T^*M$}
\end{matrix}
  $$
\end{corollary}


\end{frame}

\begin{frame}{Recall: Master-Hamiltonian $\to$ homotopy brackets }


\centerline  {$M$---(super)manifold. (coordinates $x=x^a$)}

\smallskip

Odd  Hamiltonian $Q(x,p)$ on $ T^*M$, ($p=p_b$ fibre coordinates) 
\\
defines homotopy odd Poisson (Schouten)  brackets on $M$---
   collection 
$\left\{\{\}_Q, \{\,,\,\}_Q, \{\,,\,,\,\}_Q,\dots\right\}$ 
of brackets on $M$:
                  $$
   \{f\}_Q=(H,f)\big\vert_{p=0}\,,\quad
   \{f,g\}_H=\left(\left(H,f\right),g\right)\big\vert_{p=0}\,,
                 $$
                 $$
      \{f_1,f_2,\dots,f_n\}_Q=
(\dots(Q,f_1),f_2),\dots,)\big\vert_{p_b=0}
                 $$
$(\,,\,)$---canonical even Poisson
bracket on $T^*M$.\\
$(Q,Q)=0$---Jacobi identity, 
\medskip
  
We come to usual odd Poisson bracket if Hamiltonian is quadratic in
momenta, $Q=Q^{ab}p_ap_b$.


\end{frame}

\begin{frame} {Homotopy bracket on $M$ $\to$ $L_\infty$-algebra of functions
on $M$ $\to$ $Q$-manifold}

\centerline  {$M$ an arbitrary (super)manifold}

\smallskip

 Let $Q=Q(x,p)$ be an odd  
Hamiltonian in $T^*M$, and Jacobi identity 
$(Q,Q)=0$ is obeyed.

The odd Hamiltonian $Q$  defines homotopy odd Poisson (homotopy Schouten)
bracket on $M$.  

iConisder the following Hamilton-Jacobi vector field
             $$
\X_Q\colon \,C(T^*M)\ni f\to f+\vare   Q\left(x^a, p_b={\p f(x)
\over \p x^b}\right)\,,
          $$
            $$
 \X_Q=\int_M dx Q\left(x^a, {\p f(x)
\over \p x^b}\right){\delta\over \delta f(x)}\,, 
\X_Q^2={1\over 2}[\X_Q,\X_Q]=0\,.
            $$
$\X_Q$ is homological vector field on infinite-dimensional space
$\M=C(M)$ of functions on manifold $M$.

\end{frame}

\begin{frame}
           $$
          \begin{matrix}
\hbox
 {Homotopy Schouten structure on functions on $M$}\cr
       \hbox{defined by odd Hamiltonian $Q$}\cr
          \end{matrix}
          $$
          $$
         \downarrow
          $$
         $$
       \begin{matrix}
  \hbox{ $Q$-manifold $(\M, \X_Q)$, $L_\infty$ algebra}\cr
 \hbox{$\M=C(M)$ and $\X_Q$ is Hamilton Jacobi field of $Q$}\cr
       \end{matrix}
         $$



\end{frame}
\begin{frame}

\centerline {$P=P(x,\theta)$, $[P,P]=0$.}

                 $$
              \begin{matrix}
                     \Pi T^*M-(x^a,\theta_b)\cr
%\hbox {Homolog.vector field on $\Pi T^*M$}\cr
 \hbox {Odd Poisson canonical bracket}\cr
   \hbox {Hamiltonian $Q=p_a\xi^a$}\cr
        \hbox{on $T(\Pi T^*M)-(x^a,\theta_b; p_a,\xi ^a)$}
                \end{matrix}
                      \rightarrow
                     \begin{matrix}
                     \Pi TM-(x^a,\xi^b)\cr
    \hbox{Odd homotopy Koszul bracket}\cr
    \hbox{Hamiltonian $K_P=\xi^a{\p P\over \p x^a}+
         p_a{\p P\over \p \theta_a}$}\cr
\hbox{on $T^*(\Pi TM)-(y^a,\xi^b; p_a,\theta_b)$}   
                \end{matrix}
                 $$

   
%$A\Rrightarrow B$     $A^* \Lleftarrow B^*$

      $$
      \begin{matrix}
      \hbox{$Q$-manifold, $L_\infty$ algebra}\cr
  \M_1=C(\Pi T^*M)\,, \X_1=\X_Q\cr
             \end{matrix}
           \qquad
            \begin{matrix}
      \hbox{$Q$-manifold, $L_\infty$ algebra}\cr
      \M_2=C(\Pi TM)\,, \X_2=\X_{K_p}\cr
                 \end{matrix}
         $$
Does there exist $L_\infty$-morphism $(\M_2,\X_2)\to (\M_1,\X_1)$, i.e.
map $\M_2\to \M_1$ (may be non-linear)
which intertwines homological vector fields $\X_1,\X_2$?

\begin{theorem}
   Yes, it does.
\end{theorem}



\end{frame}
\begin{frame} {Special case, $P={1\over 2}P^{ab}\theta_b\theta_a$}  

In this case  the map 
       $$
 \Pi T^*M\to \Pi TM\colon\quad  
\xi^a={\p P\over\p \theta^a}=P^{ab}(x)\theta_b\,,
       $$
is linear in fibres.  
Morphism of $Q$-manifolds
               $$
 C(\Pi T^*M)\leftarrow C(\Pi TM)
           $$ 
is its pull-back.

These linear maps intertwine differentials $d$ and $d_P$, 
Hamiltonians $Q$ and $K_P$  and their homological vector fields
  $\X_Q$ and $\X_{K_p}$
on infinite-dimensional spaces of functions.

  


\end{frame}
\begin{frame}
It is more tricky if 
$P(x,\theta)$ is an arbitrary function 
(solution of master-equation $[P,P]=0$.
The map  
        $$
 \Pi T^*M\to \Pi TM\colon\quad  
\xi^a={\p P(x,\theta)\over\p \theta^a}
       $$
is in general non-linear map.


Does there exist morphism of $Q$-manifolds 
 $(\M_2,\X_2)=(C(\Pi TM),\X_{K_P})\to 
  (\M_1,\X_1)=(C(\Pi T^*M), \X_Q)$?

In other words does there exist a  (non-linear) map 
$C(\Pi TM)\to C(\Pi T^*M)$? which intertwines 
canonical odd Poisson bracket $[\,,\,]$ on $\Pi T^*M$
and homotopy Koszul brackets 
$[\,\,]_P, [\,\,,\,\,]_P, [\,\,,\,\,,\,\,]_P.\dots$ on $\Pi TM$?

\end{frame}
\begin{frame}{Recall: Two Hamiltonians}
         $$
      \varphi_P\,\,\Pi TM\to\Pi T^*M\colon\,\,
 \xi^a={1\over 2}{\p P(x,\theta)\over \p \theta_a}\,.
         $$
         $$
     \varphi_P^*(d\w)=d_P(\varphi_P^*\w)\,.,\qquad
(Q_P,\varphi_P^*\w)=\varphi_P^*\left((K,\varphi)\right)\,.
         $$
          $$
       \begin{matrix}
    T^*(\Pi T^*M) &\longrightarrow & T^*(\Pi TM)\cr
             Q=p_a\eta^a& \longrightarrow &  K=\eta^aq_a\cr
        \hbox{canonical odd bracket} &\longrightarrow&
         \hbox{de Rham differential on $\Pi TM$ }\cr
     Q_P=(P,Q)& \longrightarrow &  K_P\cr
        \hbox{Lichnerowicz diff. on $\Pi T^*M$} &\longrightarrow&
         \hbox{Higher Koszul bracket on $\Pi TM$ }\cr 
       \end{matrix}
        $$
         $$
         $$
Map $\varphi_P$ intertwines Lichnerowicz and de Rham differentials,
i.e. Hamiltonians $Q_P$ and $K$.



\end{frame}
\begin{frame}{Recall:Question}
Map $\varphi_P$ intertwines Lichnerowicz and de Rham differentials,
i.e. Hamiltonians $Q_P$ and $K$.


\bigskip

How look a map which intertwines Hamiltonians $Q$ and $K_P$?\\
 a map which intertwines canonical Schouten bracket and higher 
Koszul brackets???


\end{frame}


\begin{frame} {Answer}
 
         Morphism $\varphi_P\colon\, \Pi T^*M \to \Pi TM$
   intertwines Hamiltonians $Q_p$ and $K$      

           We try to construct a `morphism', (sort of morphism)
                             $$
        \Phi\colon\, \Pi T^*M \to \Pi TM\,,
                       $$
   which intertwines Hamiltonians $Q$ and $K_P$\\
 
\medskip

  $\Phi=\phi_P^*$ is {\it thick morphism} which is 
adjoint to morphism $\varphi_P$.      
\end{frame}

\section {Thick morphisms}

\begin{frame}{Definition of thick morphism. (T.Voronov)}
\centerline  {$M_1,M_2$--two (super)manifolds}

 $x^i$--coordinates on $M_1$, $y^a$--coordinates on $M_2$\\
\medskip

   \centerline {Consider symplectic manifold $T^*M_1\times (-T^*M_2)$}
   \centerline {equipped with canonical symplectic structure }
                $$
   \w=\w_1-\w_2=\underbrace{dp_i\wedge dx^i}_
        {\hbox{coord. on $T^*M_1$}}-
        \underbrace{dq_a\wedge dy^a}_
        {\hbox{coord. on $T^*M_2$}}
                $$
    \centerline {function $S=S(x,q)$} \\defines Lagrangian surface  
$\L_S\subset T^*M_1\times (-T^*M_2)$:
           $$
  \L_S=\left\{(x,p,y,q)\colon\quad 
                     p_i={\p S(x,q)\over \p x^i}\,,
                     y^b={\p S(x,q)\over \p q_b}\right\}
           $$
   
\end{frame}
\begin{frame}{Lagrangian surface---canonical relation---thick morphism}

     Lagr. surf. $\L_S$ is canon. relation
  $\Phi_s$ in $T^*M_1\times (-T^*M_2)$
         $$
(x^i,p_j)\sim_S (y^a,q_b)\leftrightarrow\,
 (x^i,p_j,y^a,q_b)\in \L_S\,,\, (\Phi_S=\sim_S)\,.
         $$
        
                     $$
 \Phi=\Phi_s \,\hbox{is a thick morphism}\,\,     
   M_1\Rrightarrow M_2
                     $$
   It defines pull-back $\Phi_S^*$ of functions
                   $$
\Phi_S^*\colon \, \M_2=C(M_2)\rightarrow \M_1=C(M_1)\,,
                   $$
such that for every function $g=g(y)\in \M_2$,
                   $$
f=f(x)=(\Phi_S^* g)(x)\colon \L_f=\Phi_S\circ \L_g\,,
                   $$
where $\L_f, \L_g$ are Lagrangian surfaces, graphs of
$df,dg$ in $T^*M_1, T^*M_2$.
 
\end{frame}
\begin{frame}{Explicit expression}
      $$
f(x)=(\Phi_S^* g)(x)=g(y)+S(x,q)-y^aq_a\,
      $$ 
where $y^a$ and $q_a$ are defined from the equations
             $$
      y^a={\p S(x,q)\over \p q_a}\,,\quad
      q_a={\p g(y)\over \p y^a}\,
             $$
  We see that $\L_f=\Phi_S\circ \L_g$ since
       $$
p_i={\p f\over \p x^i}=
{\p\over \p x^i}
 \left(g(y)+S(x,q)-y^aq_a\right)={\p S(x,q)\over \p x^i}\,.
       $$
\end{frame}

\begin{frame}{Properties of thick morphism}
\begin{example}
   Generating function $S=S^a(x)q_a$
            $$
    (\Phi_S^*g)(x)=g(y)+S(x,q)-y^aq_a=g(y)+
   \underbrace{(S^a(x)-y^a)}_{\hbox{vanishes}}q_a=g(S^a(x))
            $$ 
  Thick morphism $M_1{\buildrel \Phi_s\over\Rrightarrow} M_2$
 is usual morphism $M_1{\buildrel y^a=S^a(x)\over \rightarrow}M_2$.
\end{example}
In general case if  $S(x,q)=S(x)+S^a(x)q_aq_b+S^{ab}(x)q_aq_b+\dots$
     $$
(\Phi_S^* g)(x)=S(x)+\left(g(y)+S^{ab}(x)
                             {\p g(y)\over \p y^a }
                             {\p g(y)\over \p y^b}+\dots\right)_
                           {y^a=S^a(x)}
     $$
is non-linear pull-back. 
  
\end{frame}

\begin{frame} {Why it is important. Voronov's Theorem and Corollary}
\begin{theorem}

 Let $\Phi_S\colon M_1\Rrightarrow M_2$ be a thick morphism.\\

   Let $Q_1, Q_2$ be $\Phi_S$ related  Hamilt. on $T^*M_1, T^*M_2$: 
 :
             $$
     Q_1\left(x^i,p_j={\p S(x,q)\over \p x^j}\right)\equiv
     Q_2\left(y^a={\p S(x,q)\over \p q_a}, q_b\right)\,.
             $$
Then Hamilton-Jacobi vector fields $\X_{Q_1}$, $\X_{Q_2}$
on spaces $\M_1,\M_2$ of functions are 
related by non-linear pull back 
$\Phi_S^*$ 
\end{theorem}

\end{frame}
\begin{frame}
\begin{corollary}
 Let $\Phi_S\colon M_1\Rrightarrow M_2$ be a thick morphism.

If odd Hamiltonians  $Q_1.Q_2$ are $\Phi_S$ related and
              $$
(Q_1,Q_1)=(Q_2)=Q_2)=0\,,
              $$
then non-linear pull-back
        $$
  \Phi_S^* \colon\, \M_2\rightarrow \M_1
        $$ 
defines $L_\infty$-morphism of $L_\infty$ algebras $(\M_1, \X_{Q_1})$
and $(\M_2,\X_{Q_2})$.
\end{corollary}


\end{frame}
\begin{frame}{Revenons \`a   nos moutons}
         $$
      \varphi_P\,\,\Pi TM\to\Pi T^*M\colon\,\,
 \xi^a={1\over 2}{\p P(x,\theta)\over \p \theta_a}\,.
         $$
         $$
     \varphi_P^*(d\w)=d_P(\varphi_P^*\w)\,.,\qquad
(Q_P,\varphi_P^*\w)=\varphi_P^*\left((K,\varphi)\right)\,.
         $$
          $$
       \begin{matrix}
    T^*(\Pi T^*M) &\longrightarrow & T^*(\Pi TM)\cr
             Q=p_a\eta^a& \longrightarrow &  K=\eta^aq_a\cr
        \hbox{canonical odd bracket} &\longrightarrow&
         \hbox{de Rham differential on $\Pi TM$ }\cr
     Q_P=(P,Q)& \longrightarrow &  K_P\cr
        \hbox{Lichnerowicz diff. on $\Pi T^*M$} &\longrightarrow&
         \hbox{Higher Koszul bracket on $\Pi TM$ }\cr 
       \end{matrix}
        $$
         $$
         $$
Morphism (usual) $\varphi_P$ intertwines 
Hamilt. $Q_P$ and $K$\\

??? thick morphism $\phi_S$
which intertwines Hamilt. $Q$ and $K_P$.
        $$
  \hbox{$Q_P$ and $K$ are $\varphi_P$ related}\,,\,
  \hbox{$Q$ and $K_P$ will be  $\phi_P$ related}
        $$


\end{frame}

\begin{frame}{Thick morphism---generalisation of adjoint}
              $$
  \begin{matrix}
  \hbox{$E\to M$ vector bundle} && \hbox{ $E^*\to M$ its dual}\cr
\Phi\colon E\Rrightarrow E^* &\leftrightarrow& 
\Phi^+\colon E\Rrightarrow E^*\cr
 \hbox{$L=L_S$ Lagr. surf.defining $\Phi$}
      &\leftrightarrow&    
 \hbox{$L^*=L_{S^*}$ Lagr. surf.defining $\Phi$}`
     \end{matrix}
          $$
\centerline{
These Lagrangian surfaces belong to $T^*E\times (-T^*E^*$.}  
 $$
  \begin{matrix}
   T^*E\times (-T^*E^*)
       \,& {\buildrel 
  \hbox{MX symplectom.}\over \leftrightarrow}\,& 
   T^*E\times (-T^*E^*)\cr
      S &\leftrightarrow&  S^*\cr
      \L=\L_S &\leftrightarrow& \L^*=\L_{S^*}\cr
     \end{matrix}
          $$
If  $\Phi$ is linear map in fibres, then $\Phi^*$ is just its adjoint.
\end{frame}
\begin{frame}{Return again to our case}
         $$
      \varphi_P\,\,\Pi TM\to\Pi T^*M\colon\,\,
 \xi^a={1\over 2}{\p P(x,\theta)\over \p \theta_a}\,.
         $$
         $$
     \varphi_P^*(d\w)=d_P(\varphi_P^*\w)\,.,\qquad
(Q_P,\varphi_P^*\w)=\varphi_P^*\left((K,\varphi)\right)\,.
         $$
        $$
       \begin{matrix}
    T^*(\Pi T^*M) &\longrightarrow & T^*(\Pi TM)\cr
             Q=p_a\eta^a& \longrightarrow &  K=\eta^aq_a\cr
        \hbox{canonical odd bracket} &\longrightarrow&
         \hbox{de Rham differential on $\Pi TM$ }\cr
     Q_P=(P,Q)& \longrightarrow &  K_P\cr
        \hbox{Lichnerowicz diff. on $\Pi T^*M$} &\longrightarrow&
         \hbox{Higher Koszul bracket on $\Pi TM$ }\cr 
       \end{matrix}
        $$
         $$
         $$
Morphism (usual) $\varphi_P$ intertwines 
Hamilt. $Q_P$ and $K$\\

??? thick morphism $\phi_S$ ?
which intertwines Hamilt. $Q$ and $K_P$.
        $$
  \hbox{$Q_P$ and $K$ are $\varphi_P$ related}\,,\,
  \hbox{$Q$ and $K_P$ will be  $\phi_P$ related}
        $$




\end{frame}
\begin{frame}{Solution}
           $$
      \varphi_P\colon\quad\,\,\Pi TM\to\Pi T^*M\colon\,\,
 \xi^a={1\over 2}{\p P(x,\theta)\over \p \theta_a}\,.
         $$
  

      Let $\phi$ be a thick morphism adjoint to morphism $\varphi_P$.

We know that $\varphi_P$ intertwines $Q_P$ and $K$

We have  that Mackenzie-Xu symplectomorphism transforms:
             $$
\begin{matrix}
   \varphi_P\leftrightarrow \phi\cr
    K\leftrightarrow Q\cr
    Q_P\leftrightarrow K_P\cr
\end{matrix}
        $$
  
  Hence adjoint thick morphism $\phi$ intertwines $Q$ and $K_P$.
             

The pull-back $\phi^*\colon C(\Pi T^*M)\leftarrow C(\Pi TM)$
is non-linear map on space of functions which transfors
homotopy Koszul brcket to canonical Schouten bracket. 

\end{frame}

\begin{frame}
          $$ $$
        \centerline { Thank you}
           $$ $$
\end{frame}


\begin{frame}{Papers that talk is based on}
[1] H.M.Khudaverdian, Th. Voronov {\it Higher Poisson 
             brackets and differential forms}, 2008a
 In: Geometric Methods in Physics. AIP Conference Proceedings 1079, 
American Institute of Physics, Melville, New York, 2008, 203-215.,
       arXiv: 0808.3406
\medskip

[2] Th. Voronov, {\it Nonlinear pullback on functions and
a formal category extending the category of supermanifolds]},
          arXiv: 1409.6475

\medskip


[3] Th. Voronov, {\it Microformal geometry},
      arXiv: 1411.6720

\end{frame}




\end{document}
