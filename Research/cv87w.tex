
% File name: cv.tex
% This is a LaTeX 2e file.
% Last time changed: October 05, 2008

  \documentclass[12pt]{article}

 \topmargin=0in
 \textheight=227mm

 \usepackage{array}
 \newcommand{\punkt}{\par\medskip\noindent}
 \newcommand{\razdel}{\par\bigskip\noindent}
 \newcommand{\ots}{\hspace*{2em}}
 \newcommand{\pha}{{$\phantom{\mbox{1996-}}$}}

\def\m {\medskip}

 \begin{document}

 \begin{center}
 \Large \bf CURRICULUM VITAE
 \end{center}
 \punkt
 \begin{center}
 \large \bf 1. PERSONAL RECORD.
 \end{center}
   \punkt
   Full name: {\bf Dr. Hovhannes Khudaverdyan}\\
        (in most of scientific articles after 2000 
`H. M.Khudaverdian', before 2000: `O.M.Khudaverdian')
   \punkt
  Nationality: British, Armenian \\
Date and place of birth: 28 May 1955, Yerevan (Armenia)\\

\noindent Marital status: Married with two adult sons.

 \noindent Place of residence: 33 Highfield Road, Prestwich, Manchester M25 3AQ.

\noindent Languages: Armenian and Russian mother tongues, English fluent, French competent

\punkt
 Phone: \parbox[t]{6cm}{ (0161) 200 8975}
\par\smallskip\noindent
FAX: (0161) 200 3669
   \par\smallskip\noindent
 E-mail:  \parbox[t]{5in}{\tt khudian@manchester.ac.uk}


  \razdel
   \begin{center}
 \large \bf 2. EDUCATION AND QUALIFICATIONS.
   \end{center}
   \punkt
 {\bf Ph.D. in Theoretical and Mathematical Physics.}         1982\\
   \ots Thesis: \parbox[t]{3.8in}{\it Multiplicative and Additive Functionals;
                their Role in  Quantum Field Theory}
\par\smallskip\noindent
   \ots Advisor: Professor A.S. Schwarz.\\
   \ots Referees: Professor V.I. Ogievetsky, Professor Yu.I.
   Manin
   \smallskip\\
    1978--1981  \quad\parbox[t]{5in}{{\it Postgraduate student}\\Department
               of  Theoretical Nuclear Physics\\ Moscow Physical
  Engineering Institute (MPEI). }

 \punkt
 {\bf M.S. in Physics (with Honors).}  1978.\\
   \ots              Thesis:  \parbox[t]{5in}
{On Anomalies in Quantum Field Theory.}\\
    \ots         Advisor: Professor A.S. Schwarz.
                                 \smallskip\\
  1975--1978 \quad \parbox[t]{5in}{Department of Theoretical Nuclear
                                  Physics\\
                Moscow Physical Engineering Institute.}
\smallskip\\
  1972--1975 \quad \parbox[t]{5in}{Department of Physics\\
               Yerevan State University.}




   \newpage
   \razdel
   \begin{center}
   \large \bf 3. PROFESSIONAL EXPERIENCE.
   \end{center}
 \punkt
 {\bf Current position:}

\smallskip\noindent
Since August 2005 Senior Lecturer in Pure Mathematics, University of
Manchester\\



\punkt
 {\bf Previous positions:}



 \smallskip\noindent
 \begin{tabular}{ll}

 2001---2005
        &  Lecturer in Pure Mathematics, UMIST\\
         &  (University of Manchester from 2004)\\


1996---2005
&Senior Researcher,\\
 &Laboratory of Computing Technique and Automation\\
 &Joint Institute for Nuclear Research (Dubna, Russia)\\


\smallskip\noindent
              Since 1985
          & Senior Researcher,
         Department of Theoretical  Physics \\
        & Yerevan State University (Yerevan, Armenia)\\




   1981--1985    &     Junior Researcher,
              Laboratory of Theoretical Physics\\
         &          Yerevan Physics Institute (Yerevan, Armenia)\\
%   1978--1981   &  Research Fellow,\\
 %             &    Department of Theoretical Nuclear
  %                                Physics\\
   %           &  Moscow Physical Engineering Institute (Moscow, Russia)
 \end{tabular}

\punkt
 {\bf Invited guest  positions:}




 \begin{tabular}{ll}

December 2015--January 2016
        & Guest Professor
         Max-Planck-Institut f\"ur Mathematik (Bonn)\\



October-Novembre  2015
        & Guest Professor
         in I.H.E.S. (Paris)\\

January 2012
        & Guest Professor
         in I.H.E.S. (Paris)\\



October-November 2011
        & Guest Professor
         Max-Planck-Institut f\"ur Mathematik (Bonn)\\


         2000
        & Academic Visitor, UMIST \\

October--November  1999
        & Guest Professor
         Max-Planck-Institut f\"ur Mathematik (Bonn)\\




   Autumn 1994
        &  Advanced Institute of
       Basic Sciences, (Zanjan, Iran)\\

    1989 &     Visiting Researcher,
          Department of Theoretical Physics\\
        &  Geneva University (Geneva, Switzerland)\\

 \end{tabular}




\punkt {\bf  Research grants:}



 \smallskip\noindent
 \begin{tabular}{cl}

  10-2010 and 02-2012; LMS (scheme 2) grant & supporting a visitor to UK\\


     10-2007 and 07-2008 & Royal Society grants for conferences\\
    2000-2001  & EPSRC\\
    1996--1998, &  INTAS-RFFI: European Union and\\
                &  Russian Foundation for Basic Research
    \smallskip\\
    1993--1995  & International Science Foundation
  \end{tabular}


\medskip

   {\bf Main field of scientific interest: Mathematical Physics
   and Differential Geometry}. In particulkar  my interests 
  include Quantum Field Theory (mathematical aspects),
   supermanifold geometry and its application in physics and
  in other areas of mathematics, Poisson brackets, Lie algebroids,
geometry of differential operators.


 \punkt
 {\bf Talks on seminars (after 2000):}
 \par\smallskip\noindent
 \begin{tabular}{ll}

October  2017   &  Colloquim. Department of Mathematiques, 
Sheffield 
University \\


  January  2017   &  Seminar in Geometry.
 Haifa University, Izrael\\




November  2016   &  Seminar in Mathem. Physics.
 Institute of Appl. Math. and 
and Mech, Lviv, Ukraine \\



March  2016   &  Colloquim. Department of Mathematiques, Durham 
University \\


 January 2016   &Seminar in Mathematical Physics, Max Planck Institute, 
                      Bonn \\

 November 2015   & Colloquim \&
           S\'eminaire. Laboratoire 
        de Math\'ematiques de Reims 
             Universit\'e \\
 November 2015   & S\'eminaire ``Groupes de Lie et espaces des modules''.
             Universit\'e de Gen\`eve\\
 October 2015   & S\'eminaire de physique math\'ematique.
           IHES (Bures sur Yvette, Paris)\\
 
October 2015   & S\'eminaire de g\'eom\'etrie et physique math\'ematique.
           Universit\'e Paris Diderot (VII)\\
 
October 2015   &  Colloquim. D\'epartment de Mathematiques, Universit\'e 
d'Angers (France)\\


 September 2014 &Mathematical Physics Seminar in Department of Mathematics MGU
                                       (Russia)\\ 

 May  2014   &Mathematical Physics  Seminar in UCL Davis University (USA)\\
 
November 2011   &Geometry  Seminar in Liege University (Belgium)\\

November 2011   &Seminar in Mathematical Physics, Max Planck Institute, 
                      Bonn \\


 Novembre 2010   &Seminar Geometrique, Physique et Symetries
              Luminy (Marseille, France)\\
 
   Novembre  2010   &Geometry  Seminar in Lyon (France)\\
 
January 2010   &Analysis and Geometry  Seminar in Newcastle University\\

December 2009   &Mathematics Seminar in Valencienne University (France)\\

December 2009   &Mathematical Physics Seminar in Lille University (France)\\

October 2009   &Geometry and Topology  Seminar in Aberdeen University\\

May 2008   &Mathematical Physics Seminar in Bordeaux University (France)\\

April 2008   &W.Brauder Special topology seminar in Princeton University (USA)\\

January 2008 & S.P.~Novikov Geometry and Topology Seminar,\\
   &  V.A.~Steklov
 Mathematical Institute (Moscow)\\

November 2007   &Applied Mathematics  Seminar in Brunel University\\

January 2007  &All-Moscow seminar "Globus"\\

December 2006  &Geometry seminar in Independent University\\

 January 2006   &Geometry Seminar in Edinbuirg University \\


 September 2004   &I.R.Shafarevitch Algebra  Seminar, Steklov Institute (Moscow)\\

 February 2004   &Seminar on Mathematical Physics, University of Loughborough\\

 December 2003   & Analysis/Geometry Seminar, King's College\\
 October 2003,   & Bristol Pure Mathematics Seminar\\
 October 2003,   & Liverpool Pure Mathematics Colloquium\\
 September 2003  & I.V. Tamm Department  of Theoretical Physics, Moscow\\
 April 2003, & Geometry of differential equations Seminar, Moscow, IU\\
 July 2002 & International Workshop Quantum Gravity and Superstrings, Dubna\\
  November 2001, &  Geometry with Brackets and Quantization, Warwick \\
  December 2000 & Geometry and Theoretical Physics Seminar, King's College London\\
  November 2000 & N.~Hitchin Geometry and Analysis Seminar, Oxford University\\
  November 2000    & Departmental Colloquium, Sheffield University \\
 July   2000       & International Workshop on Quantization, Warwick\\
 April  2000      & Mathematical Physics Seminar, Loughborough University \\
 February 2000 & I.V.~Tamm Department of Theoretical
         Physics, \\
 %May 1997        & P.N. Lebedev Physical Institute (Moscow)\\
 %November 1999 & Theoretical Physics Seminar, Leiden University \\
%July 1999 & S.P.~Novikov Geometry and Topology Seminar,\\
 %  &  V.A.~Steklov
% Mathematical Institute (Moscow)\\
%July 1999 & Annual conference ``Supersymmetry and Quantum
%Symmetries" (Dubna)\\
%July 1998 &  \\
%May 1999 &  International workshop QFTHEP-99 (Moscow)\\
%November 1997 & Geometry of Nonlinear PDEs Seminar,\\
 %     &Department of Mathematics,
 %      Moscow State University\\
%August 1997 & International conference\\
%   &``Secondary Calculus and Cohomological Physics"
\end{tabular}

 \punkt
 {\bf General audience lectures:}

Lecture ``Krammer rule; from Determinants to Berezinians'' for students
  and staff of Moscow Highest School of Economy

         October 28, 2014, Moscow 






 \punkt
 {\bf Invited talks on meetings (after 2000):}

\noindent Invited Lecturer on 
FAR/ANSEF-ICTP  Summer School in 
Theoretical Physics in Armenia. 
{\it Lecture course
  "Geometry of Differential oeprators"}
Yerevan  21 August 2017---25 August 2017.

\m

\noindent GAP XIV s\'eminaire itin\'erant 
``G\'eometrie et Physique''

\noindent 8---12 August, 2016,Sheffield University

\m

\noindent Workshop on Higher Geometry and Field Theory
             9--11 December, 2015, Luxembourg University\\


\noindent Conference ``Integrability in Algebra Geometry and 
        Physics. New Trends''\\
  dedicated to  Sasha Veselov's 60th birthday\\
 13-17 July 2015 @ Congressi Stefano Franscini, Suisse, 14 July

\m

\noindent Conference ``Integrability and All That''\\
 dedicated to  Sasha Veselov's 60th birthday\\
8\,--\,10   May 2015, Loughborough , 10 May


\m



\noindent Workshop ``Geometry, Topology and Integrability''
    
  October  20--26, 2014, Moscow, Skolkovo 


\noindent International workshops 
 ``Supersymmetry in Integrable Systems - SIS'' 

 SIS 14, September 11-13, 2014, Dubna, Russia
 

 SIS 11, August 1-4, 2011,  Hannover, Germany, ({\it Closing talk}) 

 
 SIS 10, August 24-28, 2010, Yerevan, Armenia, 

\m

\noindent Workshop ``Homological Methods in Algebra,Geometry 
and Physics''
    
  July 23---25, 2014, London 

\m


\noindent  International conference 
``The Modern Physics of Compact Stars and Relativistic Gravity''
 
      September 18---21, 2013, Yerevan, Armenia {\it Plenary talk} 

\m

\noindent  International conference  "Algebraic topology and 
abelian functions" in honor of Professor Buchstaber, 

     June 18---22, 2013, Moscow
  
\m

\noindent Mini-workshop on pseudogroups and differential equations

            13---16 March, 2013, Tromso, Norway 
\m

\noindent Workshops on Geometry 
    in Physics. Bialowieza, Poland,

        June 2016 ({\it Invited  talk}) 
        
         July 2011 ({\it Opening plenary talk}) 
     
        July 2008 (plenary talk)       
    
        July 2006 (plenary talk)
      
       July 2003 (plenary talk) 



  
\noindent Lectures for students at School attached to 
the conference--July 2012,2013        

\m




\noindent Workshop on Equivariant Quantisation 
     
           September 11 - 13, 2011, Luxembourg 

\m
 
\noindent Third International Conference on Geometry and 
     Quantization GEOQUANT

       September 7 - 11, 2009, Luxembourg,  ({\it Closing talk}) 

\m

\noindent  International workshops "Supersymmetry and Quantum
Symmetries. 

  August 2015, July 2009, July 2007, and July 2005, Dubna, Russia 


\m



\noindent  XVIII International Colloquium
"Integrable Systems and Quantum Symmetries"

  June 18---20,2009, Prague, Czech Republic 


\m

\noindent One-day Geometry workshop Lyon---Manchester---Loughborough
 

            March 13,  2009, Loughborough 



\m


\noindent Conference "Conformal Filed Theory and Integrability",

       October 2007, Nor-Amberd, Armenia {\it Plenary talk}

\m

\noindent "Integrable Day" in Loughborough

        November 2006, Loughborough  



\m


\noindent  10-th International Conference Symmetry Methods in
Physics.
   
        13--20 August, 2003, Yerevan 

\m

\noindent  International workshop "Quantum Gravity and Superstrings".
 
       11---18 July, 2002, Dubna, Russia,  {\it Plenary Talk.}

\m

\noindent   LMS Northern Meeting and international workshop "Quantization, deformations,
and new homological and categorical methods in mathematical
physics". 

       
   7-13 July 2001, Manchester 
 
   \punkt
 {\bf Reviewing:}\smallskip Mathematical Reviews (1994--1996), Zentralblatt (2000---2003)

  \noindent {\bf Refereeing:}\smallskip Lett.Math.Phys and Journal of Math. Physics.

  \noindent {\bf Editorial work:} member of editorial board 
 of the Armenian Journal of Mathematics
  (since 2008).



%\razdel



   \begin{center}
   \large \bf 4. TEACHING.
   \end{center}



  During my life from 1978 till now I  taught different courses in Moscow,  Yerevan and Manchester in
  Mathematical and Theoretical Physics and
  in Mathematics: Electrodynamics, Quantum Mechanics, Differential Geometry,
Applications of Differential Geometry in Theoretical Physics, Group Theory, Homological methods in Physics,
  Galois Theory, Elements of Functional analysis e.t.c.
   Now I  teach the new courses: the course 
  "Introduction in Geometry" for second years 
students and the course "Riemannian Geometry" for 
third and forth year students.




\m



 \centerline {\bf  Supervision of research students}:


  Last 12 years I share my PHD students with Ted Voronov.
   I supervised these years 8 PhD students. Five of them already 
finished their thesis.

 One of my former PhD students (Armen Nersessian) nowdays is the leading researcher
 in mathematical physics in Yerevan, Armenia.


   I regularly supervised projects of undergraduate and master students in
  Galois Theory and Differential Geometry and I supervised
  dissertations and projects of MSC students. 


  I was four times internal examiner,
 three times the external examiner 
(Aberdeen 2010, Loughborough 2012, 2013)
and once the member of juri 
(international rapporteur in Paris-Diderot, 2016)
  on PHD oral examinations.

 I make regularly projects with foundation year students.

 \begin{center}
   \large \bf 5. Social Activity.
   \end{center}


  1. From 2001 I am helping on the regular basis in organising
  weekly sessions of Manchester  Geometry Seminar

   2. From 2003 till 2006 I was responsible for organising staff development proccess
   in our Department.

   3. From 2007 till 2010 I was  a member of promotion 
   commitee in the School of Mathematics.

  4. I am actively interesting in the process of mathematical education
    of children in UK. I prepared the talk "Euler Theorem for polyhedra"
    which I give during interview days for sixthformers.
   In March 2005 I participated in the conference
  "Where will the next generation of UK mathematicians come from?"


       I had four lectures for general audience 
      in the School of Mathematics
    
        February 2008 ``Tubes formula''

       November 2012    ``Mean algebraico-arithmetic''

       November 2013    
`` Chebyshev approximation and Helly's Theorem''

       March  2016
``Galois theory for pedestrians; explicit formulae
for cubic and quartic equations''

  5. Now I am fire marshal in our School.
   \begin{center}
    \large \bf 6. PUBLICATIONS.
   \end{center}

\medskip
    1. A.V. Gayduk, O.M. Khudaverdian, A.S. Schwarz.
{\it Multiplicative Functionals
   on Curves, Additive Functionals on Surfaces; their significance in QFT.}
     In  Proceedings: ``Group Theor. Methods in Physics",
      vol. 2, p. 201-205, Zvenigorod 1979.

\medskip
   2. O.M. Khudaverdian, A.S. Schwarz.
{\it A Few Comments on the String Representation
      of Gauge Fields.}  Phys. Lett. v. 91B (1980), p. 107-110.

\medskip
   3. O.M. Khudaverdian, A.S. Schwarz.
  {\it Additive and Multiplicative Functionals.}
             Preprint ITEP--3--1980.

\medskip
   4. O.M. Khudaverdian, A.S. Schwarz.
   {\it Multiplicative Functionals and  Gauge fields.} Theor. Math. Phys.
  (in Russian), v. 46 (1981) p. 187-198
                 (transl. into English: p. 124-132).

\medskip
  5. O.M. Khudaverdian, A.S. Schwarz, Yu.S. Tyupkin.
{\it Integral invariants for Supercanonical Transformations.}
 Lett. Math. Phys., v. 5 (1981), p. 517-522.

\medskip
  6. A.V. Gayduk, O.M. Khudaverdian, A.S. Schwarz.
{\it Integration over Surfaces in  Superspace.}
 Theor. Math. Phys. (in Russian), v. 52 (1982), p. 375-383
           (transl. into English: p. 862-868).

\medskip
  7. O.M. Khudaverdian, A.S. Schwarz.
 {\it Normal Gauge in Supergravity.}
           Theor. Math. Phys., v. 57 (1983), p. 354-362
           (transl. into English: p. 1189-1195).

\medskip
  8. O.M. Khudaverdian, A.V. Rosly, A.S. Schwarz.
  {\it Supergravity and Complex Geometry.}  In  book:
 News of Science and Technics.
   Modern Problems of Mathematics. v. 9, 1986,
   p. 247-284 (in Russian)
   (transl. into English: ``Encyclopedia
  of Modern Mathematics", Springer-Verlag).

\medskip
  9. O.M. Khudaverdian, R.L. Mkrtchian, L.A. Zurabian.
{\it On the Axial Anomalies
    in External Tensor Fields.}
 Theor. Math. Phys. v. 71 (1987) p. 393-401.

\medskip
 10. O.M. Khudaverdian, A.P. Nersessian.
{\it Formulation of Hamiltonian Mechanics
          with Even and Odd Poisson Bracket.}
 Preprint EFT 1031--81(87), Yerevan (1987).

\medskip
 11. O.M. Khudaverdian, R.L. Mkrtchian.
 {\it Integral Invariants of Buttin Bracket.}
              Lett. Math. Phys. v. 18 (1989),  p. 229-231
    (Preprint  EFI--918--69--86- Yerevan (1986)).

\medskip
  12. O.M. Khudaverdian, A.P. Nersessian.
{\it Superspaces with Odd and Even
   Canonical Two-Forms and the Strange Superalgebra.}
 Izv. Acad. Nauk Arm. SSR,
   v. 24, No. 6, (1989) p. 288-294 (transl. into English: Soviet Journal
   of Contemp. Phys., v. 24 No.6, p. 22-27).

\medskip
  13. O.M. Khudaverdian, A.P. Nersessian.
{\it The Supergeneralization of $CP(N)$ as Reduced Phase Space
  of Super-Hamiltonian Systems.}
 Izv. Acad. Nauk Arm. SSR,
 v. 25, No. 6, (1990) p. 330-337  (transl. into English: Soviet Journal
   of Contemp. Phys., v. 25 No.6.)

\medskip
 14. O.M. Khudaverdian.
{\it Geometry of Superspace with Even and Odd Brackets.}
    J. Math. Phys. v. 32 (1991) p. 1934-1937
  (Preprint of the Geneva University, UGVA--DPT 1989/05--613).

\medskip
 15. O.M. Khudaverdian, A.P. Nersessian.
{\it Canonical Poisson Brackets of Different
     Grading and Strange Superalgebras.}
 J. Math. Phys. v. 32 (1991) p. 1938-1941
  (Preprint of the Geneva University, UGVA--DPT 1989/05--614).

\medskip
 16. O.M. Khudaverdian, A.P. Nersessian.
  {\it Even and Odd Symplectic and  K\"ahlerian
  Structures on Projective Superspaces.}
 J. Math. Phys. v. 34 (1993), p. 5533-5548.

\medskip
 17. O.M. Khudaverdian, A.P. Nersessian.
 {\it On Geometry of Batalin-Vilkovisky
   Formalism.}  Mod. Phys. Lett. A, v. 8 (1993), No. 25, p. 2377-2385.

\medskip
 18. O.M. Khudaverdian.
{\it Algebras with Operator and Campbell-Hausdorff Formula.}
    Lett. Math. Phys., v. 35 (1995), pp.27-38.

\medskip
 19. O.M. Khudaverdian.
 {\it Batalin-Vilkovisky Formalism and Odd Symplectic
 Geometry.}
 In: Proceedings of International Workshop
 ``Geometry and Integrable Systems",
 P.N.Pyatov and S.N.Solodukhin, eds.
 Word Scientific Publishing Co., 1996, p. 144-181.

\medskip
 20. O.M. Khudaverdian, A.P. Nersessian.
 {\it Batalin-Vilkovisky Formalism and
  Integration Theory on Manifolds.}
  J. Math. Phys., v. 37 (1996), p. 3713-3724.

\medskip
 21.  O.M. Khudaverdian, D.A. Sahakyan.
 {\it Cohomological Aspects of Noether Theorem
 for Lagrangians of Classical Mechanics.}
   Proceedings of the conference
   ``Secondary calculus and Cohomological Physics",
    Moscow, 1997, in
   Electronic Proceedings of EMIS,
   http://www.emis.proceedings/SCCP97

\medskip
 22.   O.M. Khudaverdian. {\it Algebraic and Geometric Aspects
  of Constrained Systems.} In  survey:
   ``Collaboration JINR--YSU (1992--1997)", JINR E-98-12, Dubna (1998).

\medskip
 23. O.M. Khudaverdian. {\it Odd Invariant Semidensity and
Divergence-like Operators on  Odd Symplectic Superspace.}
 Comm. Math. Phys., v. 198 (1998), p. 591-606.

\medskip
  24. O.M. Khudaverdian, D.A. Sahakyan.
  {\it Double Complexes and Cohomological Hierarchy
  in the Space of Weakly Invariant Lagrangians of Mechanics.}
  Acta Applicandae Mathematicae., v. 56 (2/3), (1999), p. 181-215.

\medskip
  25. O.M. Khudaverdian.
  {\it Delta-Operator on Semidensities and Integral
  Invariants in the Batalin-Vilkovisky Geometry.}
   Preprint of Max-Planck-Institut f\"ur Mathematik,
   MPI-135 (1999), Bonn.

\medskip
  26. O.M.~Khudaverdian.
{\it Evolution of oscillator wave function and Fourier
transformation.} In: ``Symmetries and Intergrable Systems",
collected papers.  A.N.~Sissakian, ed., Dubna, 2000, p. 269--272.


\medskip

 27. H.M.~Khudaverdian, T.Voronov
  {\it On complexes related with calculus of variations.},
  { J. Geom. Phys.} 44 (2-3) (2002),
  221-250 %\texttt{arXiv:math.DG/0105223}

\medskip

   28.  H.M.~Khudaverdian
   {\it Laplacians in odd symplectic geometry.}---
  In {\it Quantization, Poisson Brackets and Beyond},
  Theodore Voronov, ed.,

  \noindent { Contemp. Math.}, Vol. 315, Amer. Math. Soc.,
    Providence, RI, 2002, pp. 199-212.





\medskip

29.  H.M.~Khudaverdian, T.Voronov
  {\it On Odd Laplace operators.}.

 \noindent { Lett. Math. Phys.} 62
(2002), 127-142 %\texttt{arXiv:math.DG/0205202}

\medskip


30.  H.M.~Khudaverdian, T.Voronov {\it Geometry of
differential operators, and odd Laplace operators.} { Russian Math.
Surveys} 58 (2003) %\texttt{arXiv:math.DG/0301236}


\medskip
 31. H.M.~Khudaverdian.
      {\it Semidensities on odd symplectic supermanifold.},
   { Comm. Math. Phys.}, v. 247 (2004), pp. 353-390
   %\texttt{arXiv:math.DG/0012256}


\medskip

32.  H.M.~Khudaverdian, T.Voronov
On odd Laplace operators. II.
In book: \textit{Geometry, Topology and Mathematical Physics.
S.~P.~Novikov's seminar: 2002 - 2003}, V.~M.~Buchstaber,
I.~M.~Krichever, eds., \textit{Amer. Math. Soc. Transl.} (2),
Vol.~212, 2004, pp.179--205 %\texttt{arXiv:math.DG/0212311} .



\medskip


33.  H.M.~Khudaverdian, T.Voronov {\it Geometry of
differential operators, odd Laplacians, and homotopy algebras\,} { Journal of Nonlinear
Math. Phys.} {\bf 11}, Supplement  (2004), pp. 217--227.  arXiv:math.DG/0402292

\medskip

34. H.M.~Khudaverdian, T.Voronov. {\it New facts about Berezinians.}
In book: Supersymmetries and Quantum Symmetries 2005. Proceedings of
International Workshop, Joint Institute of Nuclear Research, Dubna,
27-31 July 2005, E. Ivanov and B. Zupnik, eds., Dubna, 2006,
393-398, arXiv:math-ph/0512031.


\medskip
35. H.M.~Khudaverdian, T.Voronov.
  {\it Berezinians, Exterior Powers and Recurrent
     Sequences.}--- {Lett. Math. Phys.} (Berezin memorial volume), 74
(2005), 201-228 (arXiv:math.DG/0309188)



\medskip

36. H.M.~Khudaverdian, T.Voronov. {\it On generalized symmetric powers
and a generalization of Kolmogorov-Gelfand-Buchstaber-Rees theory}. 
Russian Mathematical Survey, {\bf 62} (3), 623---625,  2007,
arXiv:math.RA/0612072.

37. H.M. Khudaverdian.  {\it Tube formula, Berezinians and Dwork formula.}
   Journal of Geometry and Symmetry in Physics, v10, 2007, pp.29--40,
  arXiv:math.-phys.0707.1893


\medskip

38. H.M.~Khudaverdian, T.Voronov. {\it
  Operators on superspaces and generalizations of the 
Gelfand-Kolmogorov theorem.}
In: XXVI Workshop on Geometric Methods in Physics. Bialowieza, Poland, 1 - 7 July 2007.
AIP Conference Proceedings 956, Melville, New York, 2007, p. 149-155. arXiv:0709.4402 [math-ph].



\medskip


39. H.M.~Khudaverdian, T.Voronov. {\it  Differential forms and odd
symplectic geometry.}
       Amer. Math. Soc.  Transl (2) Vol 224, 2008 pp.159---171


\medskip

40. H.M.~Khudaverdian, T.Voronov. {\it Higher Poisson brackets and
differential forms.} In: Geometric Methods in Physics. AIP
Conference Proceedings 1079, American Institute of Physics,
Melville, New York, 2009, 203-215. arXiv:0808.3406v2 [math-ph].

\medskip

41. A. Borovik, O.M. Khudaverdian. {\it Merkator projection, logarithm and...(in Russian)}
  Matematiqeskoe Prosvewe- nie no. 14 (2010), 58--82.


\medskip

42. H.M.~Khudaverdian, T.Voronov. {\it  
A short proof of Buchstaber---Rees Theorem.}
       Phil. Trans. R. Soc. A. {\bf 369} (1939) (2011),
     pp.1334--1345



%43. H.M.~Khudaverdian, T.Voronov.   {\it Second order 
%operators on the algebra of densities
%and a groupoid of connections}   MAX Planck Institut 
%fur Mathematik. Preprint 2011 (73)
%(math-archive---. arXiv:1112.5379 [).
%   It is published in the book:  ``Astrophysics, Gravitation 
%and Quantum Physics''.
%   {\it A volume in honour of academician Edvard Chubaryan.
%   Editor A.A. Saharian, Yerevan State University Publishing House,
%    Yerevan, 2012. ISBN 978-5-8084-1579-9.
%  Number of pages 294.
% (The pages of this  paper are 171-205.)

\m

43. H.M.~Khudaverdian, T.Voronov. {\it Geometry of differential operators
of second order, the algebra of densities, and groupoids} J. Geom. Phys.
 {\bf 64} (February 2013), 31--53, math-arXiv:1210.0784.
%(See also 
% {\it Second order operators on the algebra of densities
%and a groupoid of connections}   
%MAX Planck Institut fur Mathematik. Preprint 2011 (73)
%   It is published in the book:  ``
%Astrophysics, Gravitation and Quantum Physics''.
%  {\it A volume in honour of academician Edvard Chubaryan.
%   Editor A.A. Saharian, Yerevan State University Publishing House,
%    Yerevan, 2012. ISBN 978-5-8084-1579-9, pp.171---205.

\m
 
44 A.Biggs, H.M.Khudaverdian
``Operator pencil passing through a given operator'' 
 J. Math.Phys. {\bf 54}, 123503 (2013), math-arXive:1301.6625 

\m

45.  H.M.Khudaverdian, T.Voronov
``Geometric constructions on the algebra of densities'' 
Amer.Math. Soc. Trans, volume {\bf 254}, 2014

math-arXiv:1310.0784. 
 
\m

46. H.M.Khudaverdian ``Kaluza-Klein theory revisited: 
   projective structures and 
differential operators on algebra of densities'',
Journal of Physics: Conference Series {\bf 496}, 2014,
 012032
   math-arXive  1312-5208 


\m 


     47. A.Biggs, H.M. Khudaverdian 
``Operator pencils on the algebra of densities''
 Proceedings of Steklov Institute, 2014, volume 286, pp. 33--54
  (in Russian pp.40--64)      

\m

48, H.M.Khudaverdian, M.Peddie  ``Odd laplacians; geometrical meaning
   of potential, and modular class'',
      math-arXive  1509-05586 
(Accepted for publication in {\it Lett. in Math.Phys.})
\m

49  H.M.Khudaverdian, R.Mkrtchyan  ``Universal volume of groups 
   and anomaly of Vogel's symmetry'',
      math-arXive 1602.00337
(Accepted for publication in {\it Lett. in Math.Phys.})
\m


 

 50  
H.M.Khudaverdian, R.L.Mkrtchyan
 [pdf, ps, other]
``Diophantine equations, Platonic solids, McKay correspondence, 
equivelar maps and Vogel's universality'', math-arXiv:1604.06062 
(Accepted for publication in {\it Geometry and Physics})
\m

51 H.M.Khudaverdian, M.Peddie 
 ``Odd Symmetric Tensors, and an Analogue of the Levi-Civita 
Connection for Odd Symplectic Supermanifold'',
   Yerevan University Journal. Physics $\&$ Mathematics,
3, p. 25---31, 2016, math-arXive: 1607.03439 


 \end{document}


https://ticket.iop.org/atomreg?digest=fe1968e0a5c9d5284e8b6c82962ec904&urn=249458&expires=2010-05-01&skin=ej&exists=1&return=http://atom.iop.org/
